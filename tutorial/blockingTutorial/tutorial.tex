\chapter{ROSE Tutorial}

\label{RoseExamples:RoseTutorial}

    This chapter lays out a set of progressively
complex example programs which serve as a tutorial for the use of ROSE.
Each example is described and can be alternatively assigned as
an exercise for the reader.

\fixme{Add more tutorial examples}



\chapter{Rewrite Tutorial}
\label{RoseExamples:RewriteTutorial}

This chapter will explain how to use the rewrite system with three 
example programs each performing more complex transformations. The 
last program will be able to
perform a blocking transformation on a simple loop nest, however, missing
the analysis that would be usually required to automatically perform a
similar transformation.

\section{A very simple rewrite traversal}

The first example program ''rewriteExample1.C'' shows a minimal preprocessor, that
performs a very simple rewrite operation. Hence, it is similar to the 
''ROSE/ExamplePreprocessors/DocumentedExamples/AstRewriteExamples/astRewriteExample1.C'' code. It 
sets up a traversal using synthesized and inherited attributes, with a main function
that initializes the frontend and starts the rewrite traversal.

In contrast to the other rewrite example, this program performs a rewrite operation
in \emph{evaluateRewriteSynthesizedAttribute} function using the high level rewrite interface. 
Whenever a \emph{SgVariableDeclaration}
node is found, the preprocessor adds a line after the declaration, containing the comment
''// declaration found!''. Note that the \emph{insert} function of the high level interface
is a member function of the synthesized attribute. As first parameter the current node is
passed, the second parameter contains the source code to insert as a string. The third 
and fourth parameter use enumeration values of the \emph{HighLevelCollectionTypedefs}
to specify where to insert the code relative to the current node. In this case, the line
after the current statement in the local scope is chosen.

When you compile this test code, and run it on ''rewriteTestprog1\_small.c'' by typing:
\begin{verbatim}
./rewriteExample1 rewriteTestprog1_small.c
\end{verbatim}
a new file ''rose\_rewriteTestprog1\_small.c'' will be created, that will have the additional
comment after the ''int i'' declaration.


\section{Replacing variable reference expressions}

Although there are different ways to perform a blocking transformation, the one
presented in this tutorial will leave the declaration of the old loop variables 
unmodified. Thus it is necessary to have the loop kernel use the loop variables
of the added blocked loops. As a first step, this tutorial will present a preprocessor
to replace the references to a variable in the corresponding scope.

The basic layout of the code is similar to that of the first example, only
that the synthesized attributes are used to store and pass along information about the AST,
and the \emph{evaluateRewriteSynthesizedAttribute} needs to perform more tasks. Now,
when it comes across a SgVariableDeclaration, it checks if this is really the ''int i''
declaration we are looking for, and if yes, it adds a new variable ''int newInt'' after
the old declaration. This ''newInt'' variable now should replace all following occurrences
of ''i''. 

To achieve this, the declaration
of the \emph{MySynthesizedAttribute} has become more complex. It needs to store the information whether
the variable to be replaced has been found. The synthesized attributes are used to 
pass information to higher nodes in the AST -- this is necessary in this case, as it is
only possible to replace whole statments (and not single expressions). So once
a statement containing a reference to the variable that should be replaced is found,
it has to be replaced by the same expression referencing the new variable. In this
case this is done by using the \emph{AstRestructure::unparserReplace} function. This
adds a string to an expression, that will be used if \emph{unparseToString} is called
on this node. Hence, it does not yet modify the AST, but can be used to easily 
replace the variable references in the following way: each reference expression
to the old ''i'' is changed to the string ''newInt'' using the \emph{AstRestructure::unparserReplace}
function. If this is done somewhere in a statement, the \emph{setVarRefFound} flag
if the MySynthesizedAttribute is set, and on the next \emph{SgStatement} node,
the whole statement is replaced by it's unparsed string. Note that before this
replace call, the AST is not modified. Only new strings will be added as
attributes to the AST. Upon calling replace, the AST nodes for the modified source
code (using the ''newInt'' variable) will be created, and inserted instead 
of the old subtree. Note that it is necessary to implement \emph{operator=} and
\emph{operator+=} for the synthesized attribute, so that it passes along all needed
variables (''bool varRefFound'' in this case).

Another addition is ''mReplaceVariable'' as member variable of the ''MyTraversal'' class.
This variable is used keep track of the scopes, in which the added variable is valid. 
For this, the ''evaluateRewriteInheritedAttribute'' function is used, to increase ''mReplaceVariable''
for every scope that is found after inserting ''newInt''. For every scope that is left
during a call to ''evaluateRewriteSynthesizedAttribute'', it is decreased again. Running
this preprocessor on the test code ''rewriteTestprog1\_small.c'' should result in a modified
file with the new ''newInt'' variable used instead of the old ''int i''. To test the
scope feature just mentioned, run the preprocessor on ''rewriteTestprog2\_scopes.c''. This
test code contains several variables and scopes, but only the references to ''int i'' in
the correct scope will be replaced by the preprocessor.

\section{A simple blocking transformation}

The last example program ''rewriteExample3.C'' is now capable of performing a blocking
transformation on a loop nest. This is done in several steps. First, the inherited attribute
is used to count the encountered nested loops by storing a
vector containing information about these loops. Here the ''forBlockInfo'' struct contains all necessary
information about a loop -- it's position in the AST, information about the loop variable,
and whether it was blocked or not. A list of all loops is also stored in the traversal object 
(in contrast to the inherited attribute, which only stores the vector of nested loops), to 
simplify the replacing of loop variable reference expressions.

The loop kernel is found, when a loop does not contain any other loops. For this,
a node query is used. Once the loop kernel of a loop nest is
found, which means that the current AST-node is the scope of the loop body, 
some simple validation of the loops is done. The transformation of this tutorial
only works for loops using ''++'' as increment, and ''<='' as test expression. Inside
of the loop kernel, all loop variable reference expressions are replaced, as described
in the previous section. When the loop body is encountered in the ''evaluateRewriteSynthesizedAttribute''
function again, the blocked loops are inserted between the old loop nest and
the loop kernel. This is done in the ''insertTransformedLoopBody'' function. It traverses
the vector of found loops from the inherited attribute, and builds blocked loops, each
with a new variable to test the upper bound of the loop.

The last thing to do is to modify the original ''++'' increment, to match the
size of the blocks. This is done in the ''V\_SgForStatement'' block of
the switch statement in ''evaluateRewriteSynthesizedAttribute''. To test this preprocessor,
compile it, and run it on ''rewriteTestprog3\_loop.c''. This test program contains
three nested loops, that will be correctly blocked by the preprocessor. A
more complex example can be found in ''rewriteTestprog4\_moreloops.c''. In this case,
only those loop nests matching the requirements of this tutorial preprocessor
will be blocked.

