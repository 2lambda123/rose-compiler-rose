\chapter{Graph Processing Tutorial}

\section{Traversal Tutorial}
ROSE can collect and analyze paths in both source and binary CFGs. At moment it doesn't attempt to save paths because if
you save them directly the space necessary is extremely large, as paths grow $2^n$ with successive if statements and even faster
when for loops are involved. Currently a path can only cannot complete the same loop twice. However it is possible for a graph
such that 1 -> 2 , 2->3, 3->1, 3->5, has paths, 1,2,3,1,2,3,5 and 1,2,3,5 because the loop 1,2,3,1 is not repeated.

The tutorial example works as such:
\begin{figure}[!h]
{\indent
{\mySmallFontSize

% Do this when processing latex to generate non-html (not using latex2html)
\begin{latexonly}
   \lstinputlisting[language=C++]{\TOPSRCDIR/tutorial/sourceTraversalTutorial.C}
\end{latexonly}

% Do this when processing latex to build html (using latex2html)
\begin{htmlonly}
   \verbatiminput{\TOPSRCDIR/tutorial/sourceTraversalTutorial.C}
\end{htmlonly}

% end of scope in font size
}
% End of scope in indentation
}
\caption{Source CFG Traversal Example}
\label{Tutorial:sourceTraversalTutorial}
\end{figure}

\begin{figure}[!h]
{\indent
{\mySmallFontSize

% Do this when processing latex to generate non-html (not using latex2html)
\begin{latexonly}
   \lstinputlisting[language=C++]{\TOPSRCDIR/tutorial/binaryTraversalTutorial.C}
\end{latexonly}

% Do this when processing latex to build html (using latex2html)
\begin{htmlonly}
   \verbatiminput{\TOPSRCDIR/tutorial/binaryTraversalTutorial.C}
\end{htmlonly}

% end of scope in font size
}
% End of scope in indentation
}
\caption{Binary CFG Traversal Example}
\label{Tutorial:binaryTraversalTutorial}
\end{figure}
