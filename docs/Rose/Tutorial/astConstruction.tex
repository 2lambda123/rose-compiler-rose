\clearpage
\chapter{AST Construction}
AST construction is a fundamental operation needed for building ROSE
source-to-source translators. Several levels of interfaces are available in
ROSE for users to build AST from scratch. 
High level interfaces are recommended to use whenever possible for their
simplicity. Low level interfaces can give users the maximum freedom to
manipulate some details in AST trees.


This chapter uses several examples to demonstrate how to 
create AST fragments for common language
constructs (such as variable declarations, functions, function calls, etc.)
  and how to insert them into an existing AST tree.
More examples of constructing AST using high level interfaces can be found at
\textit{rose/tests/roseTests/astInterfaceTests}. 
The source files of the high level interfaces are located in
\textit{rose/src/frontend/SageIII/sageInterface}.

\section{Variable Declarations}

\paragraph{What To Learn} Two examples are given to show how to construct a SAGE III 
AST subtree for a variable declaration and its insertion into the existing AST tree.
\begin{itemize}
\item Example 1. Building a variable declaration using the high level AST construction and 
manipulation interfaces defined in namespace SageBuilder and SageInterface. 

Figure~\ref{Tutorial:exampleAddVariableDeclaration2} shows the high level
construction of an AST fragment (a variable declaration) and its insertion 
into the AST at the top of each block. buildVariableDeclaration() takes the name and type
to build a variable declaration node. prependStatement() inserts the declaration at the top
of a basic block node. Details for parent and scope pointers, symbol
tables, source file position information 
and so on are handled transparently.

%------------------- begin figure---------------------
\begin{figure}[!hbp]
{\indent
{\mySmallFontSize
% Do this when processing latex to generate non-html (not using latex2html)
\begin{latexonly}
   \lstinputlisting{\TutorialExampleDirectory/addVariableDeclaration2.C}
\end{latexonly}

% Do this when processing latex to build html (using latex2html)
\begin{htmlonly}
   \verbatiminput{\TutorialExampleDirectory/addVariableDeclaration2.C}
\end{htmlonly}

% end of scope in font size
}
% End of scope in indentation
}
\caption{AST construction and insertion for a variable using the high level interfaces}
\label{Tutorial:exampleAddVariableDeclaration2}
\end{figure}

%------------------- end figure---------------------

%\newpage
\item
Example 2. Building the variable declaration using low level member functions of SAGE III node classes.

Figure~\ref{Tutorial:exampleAddVariableDeclaration} shows the low level
construction of the same AST fragment (for the same variable declaration) and its insertion 
into the AST at the top of each block. SgNode constructors and their member functions are used.
Side effects for scope, parent pointers and symbol tables have to be handled by programmers explicitly.

%Note that the code does not handle symbol table issues.
%\fixme{The input code should use the LowLevelRewrite classes instead of the still lower level AST nodes member functions directly.}

%\fixme{Both this section and the next section (adding a function declaration)
%       and the section(s) specific to the string based rewrite mechanism
%       could be subsections of a section on AST Rewrite Mechanisms.}

\begin{figure}[!hbp]
{\indent
{\mySmallFontSize
% Do this when processing latex to generate non-html (not using latex2html)
\begin{latexonly}
   \lstinputlisting{\TutorialExampleDirectory/addVariableDeclaration.C}
\end{latexonly}

% Do this when processing latex to build html (using latex2html)
\begin{htmlonly}
   \verbatiminput{\TutorialExampleDirectory/addVariableDeclaration.C}
\end{htmlonly}

% end of scope in font size
}
% End of scope in indentation
}
\caption{Example source code to read an input program and add a new variable 
         declaration at the top of each block.}
\label{Tutorial:exampleAddVariableDeclaration}
\end{figure}
\end{itemize}


Figure~\ref{Tutorial:exampleInputCode_AddVariableDeclaration} shows the
input code used to test the translator. 
Figure~\ref{Tutorial:exampleOutput_AddVariableDeclaration} shows the resulting output.

%-----------------------input code-------------------------
\begin{figure}[!h]
{\indent
{\mySmallFontSize


% Do this when processing latex to generate non-html (not using latex2html)
\begin{latexonly}
   \lstinputlisting{\TutorialExampleDirectory/inputCode_AddVariableDeclaration.C}
\end{latexonly}

% Do this when processing latex to build html (using latex2html)
\begin{htmlonly}
   \verbatiminput{\TutorialExampleDirectory/inputCode_AddVariableDeclaration.C}
\end{htmlonly}

% end of scope in font size
}
% End of scope in indentation
}
\caption{Example source code used as input to the translators adding new variable.}
\label{Tutorial:exampleInputCode_AddVariableDeclaration}
\end{figure}

%-----------------------output code-------------------------
\begin{figure}[!h]
{\indent
{\mySmallFontSize

% Do this when processing latex to generate non-html (not using latex2html)
\begin{latexonly}
   \lstinputlisting{\TutorialExampleBuildDirectory/rose_inputCode_AddVariableDeclaration.C}
\end{latexonly}

% Do this when processing latex to build html (using latex2html)
\begin{htmlonly}
   \verbatiminput{\TutorialExampleBuildDirectory/rose_inputCode_AddVariableDeclaration.C}
\end{htmlonly}

% end of scope in font size
}
% End of scope in indentation
}
\caption{Output of input to the translators adding new variable.}
\label{Tutorial:exampleOutput_AddVariableDeclaration}
\end{figure}


\clearpage
\section{Expressions}
Figure~\ref{Tutorial:exampleAddExpression} shows a translator
using the high level AST builder interface to add an assignment statement right
before the last statement in a main() function.  

   Figure~\ref{Tutorial:exampleInputCode_AddExpression} shows the
input code used to test the translator.
Figure~\ref{Tutorial:exampleOutput_AddExpression} shows the resulting output.

%----------------------- translator--------------------------       
\begin{figure}[!ht]
{\indent
{\mySmallFontSize
% Do this when processing latex to generate non-html (not using latex2html)
\begin{latexonly}
   \lstinputlisting{\TutorialExampleDirectory/addExpression.C}
\end{latexonly}
% Do this when processing latex to build html (using latex2html)
\begin{htmlonly}
   \verbatiminput{\TutorialExampleDirectory/addExpression.C}
\end{htmlonly}
}
}
\caption{Example translator to add expressions}
\label{Tutorial:exampleAddExpression}
\end{figure}

%----------------------- input code--------------------------       
\begin{figure}[!ht]
{\indent
{\mySmallFontSize
\begin{latexonly}
   \lstinputlisting{\TutorialExampleDirectory/inputCode_AddExpression.C}
\end{latexonly}
\begin{htmlonly}
   \verbatiminput{\TutorialExampleDirectory/inputCode_AddExpression.C}
\end{htmlonly}
}
}
\caption{Example source code used as input}
\label{Tutorial:exampleInputCode_AddExpression}
\end{figure}

%----------------------- output code--------------------------       
\begin{figure}[!ht]
{\indent
{\mySmallFontSize
% Do this when processing latex to generate non-html (not using latex2html)
\begin{latexonly}
   \lstinputlisting{\TutorialExampleBuildDirectory/rose_inputCode_AddExpression.C}
\end{latexonly}
% Do this when processing latex to build html (using latex2html)
\begin{htmlonly}
   \verbatiminput{\TutorialExampleBuildDirectory/rose_inputCode_AddExpression.C}
\end{htmlonly}

% end of scope in font size
}
% End of scope in indentation
}
\caption{Output of the input}
\label{Tutorial:exampleOutput_AddExpression}
\end{figure}

\clearpage
\section{Assignment Statements}
Figure~\ref{Tutorial:exampleAddAssign} shows a translator
using the high level AST builder interface to add an assignment statement right
before the last statement in a main() function.  

   Figure~\ref{Tutorial:exampleInputCode_addAssignmentStmt} shows the
input code used to test the translator.
Figure~\ref{Tutorial:exampleOutput_addAssignmentStmt} shows the resulting output.

%\fixme{Consider making subsections for this section, so that we can present the high-level
%       and mid-level AST rewrite mechanisms. Fix input example to work correctly with
%       functions returning values (this should work).}
%----------------------- translator--------------------------       
\begin{figure}[!ht]
{\indent
{\mySmallFontSize


% Do this when processing latex to generate non-html (not using latex2html)
\begin{latexonly}
   \lstinputlisting{\TutorialExampleDirectory/addAssignmentStmt.C}
\end{latexonly}

% Do this when processing latex to build html (using latex2html)
\begin{htmlonly}
   \verbatiminput{\TutorialExampleDirectory/addAssignmentStmt.C}
\end{htmlonly}

% end of scope in font size
}
% End of scope in indentation
}
\caption{Example source code to add an assignment statement}
\label{Tutorial:exampleAddAssign}
\end{figure}

%----------------------- input code--------------------------       
\begin{figure}[!ht]
{\indent
{\mySmallFontSize
% Do this when processing latex to generate non-html (not using latex2html)
\begin{latexonly}
   \lstinputlisting{\TutorialExampleDirectory/inputCode_AddAssignmentStmt.C}
\end{latexonly}

% Do this when processing latex to build html (using latex2html)
\begin{htmlonly}
   \verbatiminput{\TutorialExampleDirectory/inputCode_AddAssignmentStmt.C}
\end{htmlonly}

% end of scope in font size
}
% End of scope in indentation
}
\caption{Example source code used as input}
\label{Tutorial:exampleInputCode_addAssignmentStmt}
\end{figure}

%----------------------- output code--------------------------       
\begin{figure}[!ht]
{\indent
{\mySmallFontSize


% Do this when processing latex to generate non-html (not using latex2html)
\begin{latexonly}
   \lstinputlisting{\TutorialExampleBuildDirectory/rose_inputCode_AddAssignmentStmt.C}
\end{latexonly}

% Do this when processing latex to build html (using latex2html)
\begin{htmlonly}
   \verbatiminput{\TutorialExampleBuildDirectory/rose_inputCode_AddAssignmentStmt.C}
\end{htmlonly}

% end of scope in font size
}
% End of scope in indentation
}
\caption{Output of the input}
\label{Tutorial:exampleOutput_addAssignmentStmt}
\end{figure}

\clearpage
\section{Functions}
This section shows how to add a function at the top of a global scope in a
file. Again, examples for both high level and low level constructions of AST are given.
\begin{itemize}
\item Figure~\ref{Tutorial:exampleAddFunctionDeclaration2} shows the high
level construction of a defining function (a function  with a function
    body). Scope information is passed to builder functions explicitly when
it is needed.
%---------------------example 2. ----------------------
\begin{figure}[!h]
{\indent
{\mySmallestFontSize
% Do this when processing latex to generate non-html (not using latex2html)
\begin{latexonly}
  \lstinputlisting{\TutorialExampleDirectory/addFunctionDeclaration2.C}
\end{latexonly}

% Do this when processing latex to build html (using latex2html)
\begin{htmlonly}
   \verbatiminput{\TutorialExampleDirectory/addFunctionDeclaration2.C}
\end{htmlonly}

% end of scope in font size
}
% End of scope in indentation
}
\caption{Addition of function to global scope using high level interfaces}
\label{Tutorial:exampleAddFunctionDeclaration2}
\end{figure}


\item Figure~\ref{Tutorial:exampleAddFunctionDeclaration3} shows almost the
same high level construction of the defining function, but with an
additional scope stack. 
 Scope information is passed to builder functions implicitly when
it is needed.
%---------------------example 3. ----------------------
\begin{figure}[!h]
{\indent
{\mySmallestFontSize
% Do this when processing latex to generate non-html (not using latex2html)
\begin{latexonly}
  \lstinputlisting{\TutorialExampleDirectory/addFunctionDeclaration3.C}
\end{latexonly}

% Do this when processing latex to build html (using latex2html)
\begin{htmlonly}
   \verbatiminput{\TutorialExampleDirectory/addFunctionDeclaration3.C}
\end{htmlonly}

% end of scope in font size
}
% End of scope in indentation
}
\caption{Addition of function to global scope using high level interfaces
  and a scope stack}
\label{Tutorial:exampleAddFunctionDeclaration3}
\end{figure}


\item  The low level construction of the AST fragment of the same function 
declaration and its insertion 
is separated into two portions and shown in two figures 
(Figure~\ref{Tutorial:exampleAddFunctionDeclarationA} and
Figure~\ref{Tutorial:exampleAddFunctionDeclarationB} ).
%---------------------example 1. part 1----------------------
\begin{figure}[!h]
{\indent
{\mySmallestFontSize
% Do this when processing latex to generate non-html (not using latex2html)
\begin{latexonly}
%  \lstinputlisting{\TutorialExampleDirectory/addFunctionDeclaration.C}
   \lstinputlisting{\TutorialExampleBuildDirectory/addFunctionDeclaration.aa}
\end{latexonly}

% Do this when processing latex to build html (using latex2html)
\begin{htmlonly}
   \verbatiminput{\TutorialExampleDirectory/addFunctionDeclaration.C}
\end{htmlonly}

% end of scope in font size
}
% End of scope in indentation
}
\caption{Example source code shows addition of function to global scope (part 1).}
\label{Tutorial:exampleAddFunctionDeclarationA}
\end{figure}

%--------------------------example 1. part 2----------------------
\begin{figure}[!h]
{\indent
{\mySmallestFontSize
% Do this when processing latex to generate non-html (not using latex2html)
\begin{latexonly}
%  \lstinputlisting{\TutorialExampleDirectory/addFunctionDeclaration.C}
   \lstinputlisting{\TutorialExampleBuildDirectory/addFunctionDeclaration.ab}
\end{latexonly}

% Do this when processing latex to build html (using latex2html)
\begin{htmlonly}
   \verbatiminput{\TutorialExampleDirectory/addFunctionDeclaration.C}
\end{htmlonly}

% end of scope in font size
}
% End of scope in indentation
}
\caption{Example source code shows addition of function to global scope (part 2).}
\label{Tutorial:exampleAddFunctionDeclarationB}
\end{figure}


\end{itemize}

Figure~\ref{Tutorial:exampleInputCode_AddFunctionDeclaration} and 
Figure~\ref{Tutorial:exampleOutput_AddFunctionDeclaration} 
give the input code and output result for the
translators above.

%---------------------- input code--------------------------
\begin{figure}[!h]
{\indent
{\mySmallFontSize


% Do this when processing latex to generate non-html (not using latex2html)
\begin{latexonly}
   \lstinputlisting{\TutorialExampleDirectory/inputCode_AddFunctionDeclaration.C}
\end{latexonly}

% Do this when processing latex to build html (using latex2html)
\begin{htmlonly}
   \verbatiminput{\TutorialExampleDirectory/inputCode_AddFunctionDeclaration.C}
\end{htmlonly}

% end of scope in font size
}
% End of scope in indentation
}
\caption{Example source code used as input to translator adding new function.}
\label{Tutorial:exampleInputCode_AddFunctionDeclaration}
\end{figure}

%-----------------------output code----------------------------------
\begin{figure}[!h]
{\indent
{\mySmallFontSize


% Do this when processing latex to generate non-html (not using latex2html)
\begin{latexonly}
   \lstinputlisting{\TutorialExampleBuildDirectory/rose_inputCode_AddFunctionDeclaration.C}
\end{latexonly}

% Do this when processing latex to build html (using latex2html)
\begin{htmlonly}
   \verbatiminput{\TutorialExampleBuildDirectory/rose_inputCode_AddFunctionDeclaration.C}
\end{htmlonly}

% end of scope in font size
}
% End of scope in indentation
}
\caption{Output of input to translator adding new function.}
\label{Tutorial:exampleOutput_AddFunctionDeclaration}
\end{figure}


\clearpage
\section{Function Calls}
Adding functions calls is a typical task for instrumentation translator. 
\begin{itemize}
\item Figure~\ref{Tutorial:exampleInstrumentationTranslator} shows the use of
the AST string based rewrite mechanism to add function calls to the top and bottom of 
each block within the AST.

\item Figure~\ref{Tutorial:exampleAddFunctionCalls} shows the use of
the AST builder interface to do the same instrumentation work.
\end{itemize}
% PHL (2/3/2015): inputCode_InstrumentationTranslator.C was removed by TV
%   Figure~\ref{Tutorial:exampleInputCode_InstrumentationTranslator} shows the
%input code used to get the translator.
%Figure~\ref{Tutorial:exampleOutput_InstrumentationTranslator} shows the resulting output.

Another example shows how to add a function call at the end of each
function body. A utility function, \textit{instrumentEndOfFunction()}, from SageInterface name space is used. 
The interface tries to locate all return statements of a target function 
 and rewriting return expressions with side effects, if there are any. 
 Figure~\ref{Tutorial:instrumentEndOfFunction} shows the translator code. 
 Figure~\ref{Tutorial:inputCode_instrumentEndOfFunction} shows the input
 code. The instrumented code is shown in
 Figure~\ref{Tutorial:rose_inputCode_instrumentEndOfFunction}.

 
%\fixme{Consider making subsections for this section, so that we can present the high-level
%       and mid-level AST rewrite mechanisms. Fix input example to work correctly with
%       functions returning values (this should work).}
%----------------------- translator--------------------------       
\begin{figure}[!ht]
{\indent
{\mySmallFontSize


% Do this when processing latex to generate non-html (not using latex2html)
\begin{latexonly}
   \lstinputlisting{\TutorialExampleDirectory/instrumentationExample.C}
\end{latexonly}

% Do this when processing latex to build html (using latex2html)
\begin{htmlonly}
   \verbatiminput{\TutorialExampleDirectory/instrumentationExample.C}
\end{htmlonly}

% end of scope in font size
}
% End of scope in indentation
}
\caption{Example source code to instrument any input program.}
\label{Tutorial:exampleInstrumentationTranslator}
\end{figure}
%----------------------- translator 2--------------------------       
\begin{figure}[!ht]
{\indent
{\mySmallFontSize


% Do this when processing latex to generate non-html (not using latex2html)
\begin{latexonly}
   \lstinputlisting{\TutorialExampleDirectory/addFunctionCalls.C}
\end{latexonly}

% Do this when processing latex to build html (using latex2html)
\begin{htmlonly}
   \verbatiminput{\TutorialExampleDirectory/addFunctionCalls.C}
\end{htmlonly}

% end of scope in font size
}
% End of scope in indentation
}
\caption{Example source code using the high level interfaces}
\label{Tutorial:exampleAddFunctionCalls}
\end{figure}


% PHL (2/3/2015): inputCode_InstrumentationTranslator.C was removed by TV
%%----------------------- input code--------------------------       
%\begin{figure}[!ht]
%{\indent
%{\mySmallFontSize
%
%
%% Do this when processing latex to generate non-html (not using latex2html)
%\begin{latexonly}
%   \lstinputlisting{\TutorialExampleDirectory/inputCode_InstrumentationTranslator.C}
%\end{latexonly}
%
%% Do this when processing latex to build html (using latex2html)
%\begin{htmlonly}
%   \verbatiminput{\TutorialExampleDirectory/inputCode_InstrumentationTranslator.C}
%\end{htmlonly}
%
%% end of scope in font size
%}
%% End of scope in indentation
%}
%\caption{Example source code used as input to instrumenting translator.}
%\label{Tutorial:exampleInputCode_InstrumentationTranslator}
%\end{figure}
%
%%----------------------- output code--------------------------       
%\begin{figure}[!ht]
%{\indent
%{\mySmallFontSize
%
%
%% Do this when processing latex to generate non-html (not using latex2html)
%\begin{latexonly}
%   \lstinputlisting{\TutorialExampleBuildDirectory/rose_inputCode_InstrumentationTranslator.C}
%\end{latexonly}
%
%% Do this when processing latex to build html (using latex2html)
%\begin{htmlonly}
%   \verbatiminput{\TutorialExampleBuildDirectory/rose_inputCode_InstrumentationTranslator.C}
%\end{htmlonly}
%
%% end of scope in font size
%}
%% End of scope in indentation
%}
%
%
%\caption{Output of input to instrumenting translator.}
%\label{Tutorial:exampleOutput_InstrumentationTranslator}
%\end{figure}
%
%
%%------------------end of function example
%----------------------- translator 3--------------------------       
\begin{figure}[!ht]
{\indent
{\mySmallFontSize

% Do this when processing latex to generate non-html (not using latex2html)
\begin{latexonly}
   \lstinputlisting{\TutorialExampleDirectory/instrumentEndOfFunction.C}
\end{latexonly}

% Do this when processing latex to build html (using latex2html)
\begin{htmlonly}
   \verbatiminput{\TutorialExampleDirectory/instrumentEndOfFunction.C}
\end{htmlonly}

% end of scope in font size
}
% End of scope in indentation
}
\caption{Example source code instrumenting end of functions}
\label{Tutorial:instrumentEndOfFunction}
\end{figure}


%----------------------- input code 3--------------------------       
\begin{figure}[!ht]
{\indent
{\mySmallFontSize

% Do this when processing latex to generate non-html (not using latex2html)
\begin{latexonly}
   \lstinputlisting{\TutorialExampleDirectory/inputCode_instrumentEndOfFunction.C}
\end{latexonly}

% Do this when processing latex to build html (using latex2html)
\begin{htmlonly}
   \verbatiminput{\TutorialExampleDirectory/inputCode_instrumentEndOfFunction.C}
\end{htmlonly}

% end of scope in font size
}
% End of scope in indentation
}
\caption{Example input code of the instrumenting translator for end of
functions.}
\label{Tutorial:inputCode_instrumentEndOfFunction}
\end{figure}

%----------------------- output code--------------------------       
\begin{figure}[!ht]
{\indent
{\mySmallFontSize


% Do this when processing latex to generate non-html (not using latex2html)
\begin{latexonly}
   \lstinputlisting{\TutorialExampleBuildDirectory/rose_inputCode_instrumentEndOfFunction.C}
\end{latexonly}

% Do this when processing latex to build html (using latex2html)
\begin{htmlonly}
   \verbatiminput{\TutorialExampleBuildDirectory/rose_inputCode_instrumentEndOfFunction.C}
\end{htmlonly}

% end of scope in font size
}
% End of scope in indentation
}

\caption{Output of instrumenting translator for end of functions.}
\label{Tutorial:rose_inputCode_instrumentEndOfFunction}
\end{figure}

\section{Creating a 'struct' for Global Variables}
\fixme{TODO: This tutorial uses low level AST manipulation. We should have a more
concise version using SageInterface and SageBuilder functions.}
   This is an example written to support the Charm++ tool. This translator
extracts global variables from the program and builds a structure to hold them.
The support is part of a number of requirements associated with using Charm++
and AMPI.

   Figure~\ref{Tutorial:exampleGlobalVariableHandling} shows repackaging of global
variables within an application into a struct. All reference to the global variables
are also transformed to reference the original variable indirectly through the structure.
This processing is part of preprocessing to use Charm++.  

   This example shows the low level handling directly at the level of the IR.

\begin{figure}[!h]
{\indent
{\mySmallestFontSize


% Do this when processing latex to generate non-html (not using latex2html)
\begin{latexonly}
%  \lstinputlisting{\TutorialExampleDirectory/addFunctionDeclaration.C}
   \lstinputlisting{\TutorialExampleBuildDirectory/CharmSupport.aa}
\end{latexonly}

% Do this when processing latex to build html (using latex2html)
\begin{htmlonly}
   \verbatiminput{\TutorialExampleDirectory/CharmSupport.C}
\end{htmlonly}

% end of scope in font size
}
% End of scope in indentation
}
\caption{Example source code shows repackaging of global variables to a struct (part 1).}
\label{Tutorial:exampleGlobalVariableHandling}
\end{figure}

\begin{figure}[!h]
{\indent
{\mySmallestFontSize


% Do this when processing latex to generate non-html (not using latex2html)
\begin{latexonly}
   \lstinputlisting{\TutorialExampleBuildDirectory/CharmSupport.ab}
\end{latexonly}

% Do this when processing latex to build html (using latex2html)
\begin{htmlonly}
%   \verbatiminput{\TutorialExampleDirectory/addFunctionDeclaration.C}
\end{htmlonly}

% end of scope in font size
}
% End of scope in indentation
}
\caption{Example source code shows repackaging of global variables to a struct (part 2).}
\label{Tutorial:exampleGlobalVariableHandling2}
\end{figure}

\begin{figure}[!h]
{\indent
{\mySmallestFontSize


% Do this when processing latex to generate non-html (not using latex2html)
\begin{latexonly}
   \lstinputlisting{\TutorialExampleBuildDirectory/CharmSupport.ac}
\end{latexonly}

% Do this when processing latex to build html (using latex2html)
\begin{htmlonly}
%   \verbatiminput{\TutorialExampleDirectory/addFunctionDeclaration.C}
\end{htmlonly}

% end of scope in font size
}
% End of scope in indentation
}
\caption{Example source code shows repackaging of global variables to a struct (part 3).}
\label{Tutorial:exampleGlobalVariableHandling3}
\end{figure}

\begin{figure}[!h]
{\indent
{\mySmallestFontSize


% Do this when processing latex to generate non-html (not using latex2html)
\begin{latexonly}
   \lstinputlisting{\TutorialExampleBuildDirectory/CharmSupport.ad}
\end{latexonly}

% Do this when processing latex to build html (using latex2html)
\begin{htmlonly}
%   \verbatiminput{\TutorialExampleDirectory/addFunctionDeclaration.C}
\end{htmlonly}

% end of scope in font size
}
% End of scope in indentation
}
\caption{Example source code shows repackaging of global variables to a struct (part 4).}
\label{Tutorial:exampleGlobalVariableHandling4}
\end{figure}

\begin{figure}[!h]
{\indent
{\mySmallestFontSize


% Do this when processing latex to generate non-html (not using latex2html)
\begin{latexonly}
   \lstinputlisting{\TutorialExampleBuildDirectory/CharmSupport.ae}
\end{latexonly}

% Do this when processing latex to build html (using latex2html)
\begin{htmlonly}
%   \verbatiminput{\TutorialExampleDirectory/addFunctionDeclaration.C}
\end{htmlonly}

% end of scope in font size
}
% End of scope in indentation
}
\caption{Example source code shows repackaging of global variables to a struct (part 5).}
\label{Tutorial:exampleGlobalVariableHandling5}
\end{figure}

\begin{figure}[!h]
{\indent
{\mySmallFontSize


% Do this when processing latex to generate non-html (not using latex2html)
\begin{latexonly}
   \lstinputlisting{\TutorialExampleDirectory/inputCode_ExampleCharmSupport.C}
\end{latexonly}

% Do this when processing latex to build html (using latex2html)
\begin{htmlonly}
   \verbatiminput{\TutorialExampleDirectory/inputCode_ExampleCharmSupport.C}
\end{htmlonly}

% end of scope in font size
}
% End of scope in indentation
}
\caption{Example source code used as input to translator adding new function.}
\label{Tutorial:exampleInputCode_AddFunctionDeclaration}
\end{figure}

\begin{figure}[!h]
{\indent
{\mySmallFontSize


% Do this when processing latex to generate non-html (not using latex2html)
\begin{latexonly}
   \lstinputlisting{\TutorialExampleBuildDirectory/rose_inputCode_ExampleCharmSupport.C}
\end{latexonly}

% Do this when processing latex to build html (using latex2html)
\begin{htmlonly}
   \verbatiminput{\TutorialExampleBuildDirectory/rose_inputCode_ExampleCharmSupport.C}
\end{htmlonly}

% end of scope in font size
}
% End of scope in indentation
}
\caption{Output of input to translator adding new function.}
\label{Tutorial:exampleOutput_AddFunctionDeclaration}
\end{figure}


