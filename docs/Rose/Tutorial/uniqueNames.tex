\chapter{Generating Unique Names for Declarations}

     There are many instances where a unique name must be generated for
either a function or variable declaration.  ROSE defines a mechanism
to make the generation of unique names from all SgDeclarationStatment
IR nodes and the SgInitializedName IR node.  This simplifies ROSE based
applications that require this sort of mechanism.  Our experience has found
that a significant number of tools require such a mechanism and that it
correct implementation can have subtle complex points, thus we have provided
one as part of ROSE.

     The specific translator described in this chapter traverses an AST and outputs the
unique names that can be generated for each declaration showing the use of the 
unique name generation mechanism.  This tool is intended as an example of how to 
generate unique names using ROSE. Not all IR nodes can be used to generate a unique
name. The generated names are unique under the following rules:
\begin{enumerate}
   \item Any two generated names are the same iff the declarations are the same. \\
   Declaration can be the same across files or within the same file.  Declarations that
   are the same can have different location in the same file (be represented multiple
   times) or be in different files. Language constructs that are the same must follow 
   the One-time Definition Rule (ODR) across files.
   \item Declarations in different unnamed scopes (e.g. for loop bodies) will generate 
   different names.
   \item Names are the same when generated buy different ROSE tools. \\
   Pointer values could be used to generate unique names of all IR nodes, but this would
   work only within a single invocation of the ROSE based tool.  Generated names are 
   not based on internal pointer values and are thus insensitive to pointer values.
   Generated names of the same declaration are thus the same even if generated from
   different tools.  This allows multiple ROSE tools to inter-operate.
\end{enumerate}

This unique name generation mechanism is only applicable to specific IR nodes, specifically:
\begin{itemize}
   \item SgInitializedName
   \item SgDeclarationStatement IR nodes:
   \begin{itemize}
      \item Obvious IR nodes supported:
      \begin{itemize}
         \item SgClassDeclaration
         \item SgFunctionDeclaration
         \item SgEnumDeclaration
         \item SgNamespaceDeclarationStatement
         \item SgTypedefDeclaration
      \end{itemize}
      \item Less obvious IR nodes not supported (support for these would not make sense):
      \begin{itemize}
         \item SgAsmStmt
         \item SgCtorInitializerList
         \item SgFunctionParameterList
         \item SgNamespaceAliasDeclarationStatement
         \item SgPragmaDeclaration
         \item SgTemplateDeclaration (can this have a mangled name?)
         \item SgTemplateInstantiationDirectiveStatement
         \item SgUsingDeclarationStatement
         \item SgUsingDirectiveStatement
         \item SgVariableDeclaration \\
               Note that the SgVariableDeclaration contains a list of SgInitializedName
               nodes and the mangled names are best queried from each SgInitializedName 
               instead of the SgVariableDeclaration.
         \item SgVariableDefinition
      \end{itemize}
   \end{itemize}
   \item Un-named scopes \\
         A number of scopes are un-names and so there is an opportunity to generate
    non-unique names from declarations in such scopes.  To fix this we generate names for
    each un-named scope to guarantee uniqueness.  Nodes handled are:
   \begin{itemize}
      \item SgForStatement 
      \item SgBasicBlock
      \item SgIfStmt
      \item get the complete list ...
   \end{itemize}
\end{itemize}
Other language constructs can generate unique names as well, but their
name could be invalid after certains transformation that move it 
structurally within the generated source code.


\section{Example Code Showing Generation of Unique Names}

\begin{figure}[!h]
{\indent
{\mySmallFontSize

% Do this when processing latex to generate non-html (not using latex2html)
\begin{latexonly}
   \lstinputlisting{\TutorialExampleDirectory/generatingUniqueNamesFromDeclaration.C}
\end{latexonly}

% Do this when processing latex to build html (using latex2html)
\begin{htmlonly}
   \verbatiminput{\TutorialExampleDirectory/generatingUniqueNamesFromDeclaration.C}
\end{htmlonly}

% end of scope in font size
}
% End of scope in indentation
}
\caption{Example source code showing the output of mangled name. The string
         represents the code associated with the subtree of the target IR node.}
\label{Tutorial:uniqueNameGeneration}
\end{figure}


\section{Input For Examples Showing Unique Name Generation for Variables}

Figure~\ref{Tutorial:uniqueNameGeneration},
shows an example translator demonstrating the generation of unique names from 
declarations in the AST.  For each SgInitializedName we generate the
mangled name.  Figure~\ref{Tutorial:exampleInputCode_UniqueVariableNameGeneration} 
shows the input code and 
figure~\ref{Tutorial:exampleOutput_UniqueVariableNameGeneration}
shows the generated output from the translator (the mangled names from
the AST associated with the input application).

\begin{figure}[!h]
{\indent
{\mySmallFontSize

% Do this when processing latex to generate non-html (not using latex2html)
\begin{latexonly}
   \lstinputlisting{\TutorialExampleDirectory/inputCode_generatingUniqueNamesFromDeclaration.C}
\end{latexonly}

% Do this when processing latex to build html (using latex2html)
\begin{htmlonly}
   \verbatiminput{\TutorialExampleDirectory/inputCode_generatingUniqueNamesFromDeclaration.C}
\end{htmlonly}

% end of scope in font size
}
% End of scope in indentation
}
\caption{Example source code used as input to program in
         codes showing debugging techniques shown in this section.}
\label{Tutorial:exampleInputCode_UniqueVariableNameGeneration}
\end{figure}



% Output file name for unique name generation
% generatingUniqueNamesFromDeclaration.out



\section{Example Output Showing Unique Variable Names}

\begin{figure}[!h]
{\indent
{\mySmallestFontSize

% Do this when processing latex to generate non-html (not using latex2html)
\begin{latexonly}
   \lstinputlisting{\TutorialExampleBuildDirectory/generatingUniqueNamesFromDeclaration.out}
\end{latexonly}

% Do this when processing latex to build html (using latex2html)
\begin{htmlonly}
   \verbatiminput{\TutorialExampleBuildDirectory/generatingUniqueNamesFromDeclaration.out}
\end{htmlonly}

% end of scope in font size
}
% End of scope in indentation
}
\caption{Output of input code using generatingUniqueNamesFromDeclaration.C}
\label{Tutorial:exampleOutput_UniqueVariableNameGeneration}
\end{figure}


\section{Input For Examples Showing Unique Name Generation for Functions}

Figure~\ref{Tutorial:uniqueNameGeneration},
shows an example translator demonstrating the generation of unique names from 
declarations in the AST.  For each SgInitializedName we generate the
mangled name.  Figure~\ref{Tutorial:exampleInputCode_UniqueFunctionNameGeneration} 
shows the input code and 
figure~\ref{Tutorial:exampleOutput_UniqueFunctionNameGeneration}
shows the generated output from the translator (the mangled names from
the AST associated with the input application).

\begin{figure}[!h]
{\indent
{\mySmallFontSize


% Do this when processing latex to generate non-html (not using latex2html)
\begin{latexonly}
   \lstinputlisting{\TutorialExampleDirectory/inputCode_generatingUniqueNamesFromDeclaration2.C}
\end{latexonly}

% Do this when processing latex to build html (using latex2html)
\begin{htmlonly}
   \verbatiminput{\TutorialExampleDirectory/inputCode_generatingUniqueNamesFromDeclaration2.C}
\end{htmlonly}

% end of scope in font size
}
% End of scope in indentation
}
\caption{Example source code used as input to program in
         codes showing debugging techniques shown in this section.}
\label{Tutorial:exampleInputCode_UniqueFunctionNameGeneration}
\end{figure}


\section{Example Output Showing Unique Function Names}

\begin{figure}[!h]
{\indent
{\mySmallestFontSize


% Do this when processing latex to generate non-html (not using latex2html)
\begin{latexonly}
   \lstinputlisting{\TutorialExampleBuildDirectory/generatingUniqueNamesFromDeclaration2.out}
\end{latexonly}

% Do this when processing latex to build html (using latex2html)
\begin{htmlonly}
   \verbatiminput{\TutorialExampleBuildDirectory/generatingUniqueNamesFromDeclaration2.out}
\end{htmlonly}

% end of scope in font size
}
% End of scope in indentation
}
\caption{Output of input code using generatingUniqueNamesFromDeclaration.C}
\label{Tutorial:exampleOutput_UniqueFunctionNameGeneration}
\end{figure}

