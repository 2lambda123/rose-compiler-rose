\chapter{Binary Construction}

ROSE is normally used in such a way that a file (source code or
binary) is parsed to construct an AST, then operations are performed
on the AST, and the modified AST is unparsed to create a new source or
binary file.  However, it is also possible to construct an AST
explicitly without parsing and then use that AST to generate the
output.  The AST construction interface for binary files was designed
so that working files could be created simply, while still providing
methods to control the finer details of the resulting file.

The example in this chapter shows how to construct a statically linked
ELF executable containing a small ``.text'' section that simply causes
the process to exit with a specific non-zero value.

\section{Constructors}

The AST node constructors are designed to construct the tree from the
root down, and thus generally take the parent node as an
argument. Nodes that refer to other nodes as prerequisites also take
those prerequisites as arguments. For instance, an ELF Relocation
Section is a child of the ELF File Header but also needs an ELF Symbol
Table and therefore takes both objects as constructor arguments.

\section{Read-Only Data Members}

When two or more file formats have a similar notion then that notion
is represented in a base class. However, part of the information may
continue to survive in the specialized class. In these situations
modifications to the specilized data will probably be overridden by
the generic values from the base class.  For instance, all formats
have a notion of byte order which is represented in the base class
SgAsmGenericHeader as little- or big-endian (an enumeration
constant). The ELF specification provides an 8-bit unsigned field to
store the byte order and therefore has potentially more than two
possibilities. Any value assigned to the ELF-specific byte order will
likely be overwritten by the generic byte order before the AST is
unparsed.

A similar situation arises with section offsets, sizes, memory
mappings, permissions, etc. The SgAsmGenericSection class represents
ELF Sections, ELF Segments, PE Objects, and other contiguous regions
of a file and has methods for obtaining/setting these values. In
addition, many of the formats have some sort of table that describes
these sections and which also contains similar information (e.g., the
ELF Segment Table, a.k.a., the ELF Program Header Table). As above,
the generic representation of these notions (stored in
SgAsmGenericSection) override the format-specific values (stored in
SgAsmElfSegmentEntry).

ROSETTA doesn't make a distinction between data members that can be
user-modified and data members that should be modified only by the
parser. Therefore it is up to the user to be aware that certain data
members will have their values computed or copied from other locations
in the AST during the unparsing phase.

\section{Constructing the Executable File Container}

All executable files are stored as children of an SgAsmGenericFile
node. The children are file format headers (SgAsmGenericHeader) such
as an ELF File Header (SgAsmElfFileHeader). This design allows a
single executable file to potentially contain more than one executable
and is used by formats like Windows-PE where the file contains a DOS
File Header as well as a PE File Header.

For the purposes of this example the SgAsmGenericFile node will serve
as the root of the AST and therefore we do not need to specify a
parent in the constructor argument list.

\begin{verbatim}
SgAsmGenericFile *ef = new SgAsmGenericFile;
\end{verbatim}

\section{Constructing the ELF File Header}

The ELF File Header is the first thing in the file, always at offset
zero. File headers are always children of an SgAsmGenericFile which is
specified as the constructor argument.

The section constructors (a file header is a kind of section) always
create the new section so it begins at the current end-of-file and
contains at least one byte. This ensures that each section has a
unique starting address, which will be important when file memory is
actually allocated and sections need to be moved around--the allocator
needs to know the relative positions of the sections in order to
correctly relocate them.

If we were parsing an ELF file we would usually use ROSE's frontend()
method. However, one can also parse the file by first constructing the
SgAsmElfFileHeader and then invoking its parse() method, which parses
the ELF File Header and everything that can be reached from that
header.

We use the typical 0x400000 as the virtual address of the main LOAD
segment, which occupies the first part of the file up through the end
of the ``.text'' section (see below). ELF File Headers don't actually
store a base address, so instead of assigning one to the
SgAsmElfFileHeader we'll leave the header's base address at the
default zero and add base\_va explicitly whenever we need to.

\begin{verbatim}
SgAsmElfFileHeader *fhdr = new SgAsmElfFileHeader(ef);
fhdr->get_exec_format()->set_word_size(8);                  /* default is 32-bit; we want 64-bit */
fhdr->set_isa(SgAsmExecutableFileFormat::ISA_X8664_Family); /* instruction set architecture; default is ISA_IA32_386 */
rose_addr_t base_va = 0x400000;                             /* base virtual address */
\end{verbatim}

\section{Constructing the ELF Segment Table}

ELF executable files always have an ELF Segment Table (also called the
ELF Program Header Table), which usually appears immediately after the
ELF File Header. The ELF Segment Table describes contiguous regions of
the file that should be memory mapped by the loader. ELF Segments
don't have names--names are imparted to the segment by virtue of the
segment also being described by the ELF Section Table, which we'll
create later.

Being a contiguous region of the file, an ELF Segment Table
(SgAsmElfSegmentTable) is derived from SgAsmGenericSection. All
non-header sections have a header as a parent, which we supply as an
argument to the constructor. Since all ELF Segments will be children of
the ELF File Header rather than children of the ELF Segment Table, we
could define the ELF Segment Table at the end rather than here. But
defining it now is an easy way to get it located in its usuall
location immediately after the ELF File Header.

\begin{verbatim}
SgAsmElfSegmentTable *segtab = new SgAsmElfSegmentTable(fhdr);
\end{verbatim}

\section{Constructing the .text Section}

ROSE doesn't treat a ``.text'' section as being anything particularly
special--it's just a regular SgAsmElfSection, which derives from
SgAsmGenericSection. However, in this example, we want to make sure
that our ``.text'' section gets initialized with some
instructions. The easiest way to do that is to specialize
SgAsmElfSection and override or augment a few of the virtual
functions.

We need to override two functions. First, the calculate\_sizes()
function should return the size we need to store the
instructions. We'll treat the instructions as an array of entries each
entry being one byte of the instruction stream. In other words, each
``entry'' is one byte in length consisting of one required byte and no
optional bytes.

We need to also override the unparse() method since the base class
will just fill the ``.text'' section with zeros. The
SgAsmGenericSection::write method we use will write the instructions
starting at the first byte of the section.

Finally, we need to augment the reallocate() method. This method is
reponsible for allocating space in the file for the section and
performing any other necessary pre-unparsing actions. We don't need to
allocate space since the base class's method will take care of that in
conjuction with our version of calculate\_sizes(), but we do need to
set a special ELF flag (SHF\_ALLOC) in the ELF Segment Table entry for
this section. There's a few ways to accomplish this. We do it this way
because the ELF Section Table Entry is not created until later and we
want to demonstrate how to keep all .text-related code in close
proximity.

\begin{verbatim}
class TextSection : public SgAsmElfSection {
public:
    TextSection(SgAsmElfFileHeader *fhdr, size_t ins_size, const unsigned char *ins_bytes)
	: SgAsmElfSection(fhdr), ins_size(ins_size), ins_bytes(ins_bytes)
	{}
    virtual rose_addr_t calculate_sizes(size_t *entsize, size_t *required, size_t *optional, size_t *entcount) const {
	if (entsize)  *entsize  = 1;                        /* an "entry" is one byte of instruction */
	if (required) *required = 1;                        /* each "entry" is also stored in one byte of the file */
	if (optional) *optional = 0;                        /* there is no extra data per instruction byte */
	if (entcount) *entcount = ins_size;                 /* number of "entries" is the total instruction bytes */
	return ins_size;                                    /* return value is section size required */
    }
    virtual bool reallocate() {
	bool retval = SgAsmElfSection::reallocate();        /* returns true if size or position of any section changed */
	SgAsmElfSectionTableEntry *ste = get_section_entry();
	ste->set_sh_flags(ste->get_sh_flags() | 0x02); /* set the SHF_ALLOC bit */
	return retval;
    }
    virtual void unparse(std::ostream &f) const {
	write(f, 0, ins_size, ins_bytes);                   /* Write the instructions at offset zero in section */
    }
    size_t ins_size;
    const unsigned char *ins_bytes;
};
\end{verbatim}

The section constructors and reallocators don't worry about alignment
issues--they always allocate from the next available byte. However,
instructions typically need to satisfy some alignment constraints. We
can easily adjust the file offset chosen by the constructor, but we
also need to tell the reallocator about the alignment constraint. Even
if we didn't ever resize the ``.text'' section the reallocator could
be called for some other section in such a way that it needs to move
the ``.text'' section to a new file offset.

For the purpose of this tutorial we want to be very picky about the
location of the ELF Segment Table. We want it to immediately follow
the ELF File Header without any intervening bytes of padding. At the
current time, the ELF File Header has a size of one byte and will
eventually be reallocated. When we reallocate the header the
subsequent sections will need to be shifted to higher file
offsets. When this happens, the allocator shifts them all by the same
amount taking care to satisfy all alignment constraints, which means
that an alignment constraint of byte bytes on the ``.text'' section
will induce a similar alignment on the ELF Segment Table. Since we
don't want that, the best practice is to call reallocate() now, before
we create the ``.text'' section.

\begin{verbatim}
ef->reallocate();                                           /* Give existing sections a chance to allocate file space */
static const unsigned char instructions[] = {0xb8, 0x01, 0x00, 0x00, 0x00, 0xbb, 0x56, 0x00, 0x00, 0x00, 0xcd, 0x80};
SgAsmElfSection *text = new TextSection(fhdr, NELMTS(instructions), instructions);
text->set_purpose(SgAsmGenericSection::SP_PROGRAM);         /* Program-supplied text/data/etc. */
text->set_offset(alignUp(text->get_offset(), 4));           /* Align on an 8-byte boundary */
text->set_file_alignment(4);                                /* Tell reallocator about alignment constraint */
text->set_mapped_alignment(4);                              /* Alignment constraint for memory mapping */
text->set_mapped_rva(base_va+text->get_offset());           /* Mapped address is based on file offset */
text->set_mapped_size(text->get_size());                    /* Mapped size is same as file size */
text->set_mapped_rperm(true);                               /* Readable */
text->set_mapped_wperm(false);                              /* Not writable */
text->set_mapped_xperm(true);                               /* Executable */
\end{verbatim}

At this point the text section doesn't have a name. We want to name it
``.text'' and we want those characters to eventually be stored in the
ELF file in a string table which we'll provide later. In ELF, section
names are represented by the section's entry in the ELF Section Table
as an offset into an ELF String Table for a NUL-terminated ASCII
string. ROSE manages strings using the SgAsmGenericString class, which
has two subclasses: one for strings that aren't stored in the
executable file (SgAsmBasicString) and one for strings that are stored
in the file (SgAsmStoredString). Both are capable of string an
std::string value and querying its byte offset (although
SgAsmBasicString::get\_offset() will always return
SgAsmGenericString::unallocated).  Since we haven't added the
``.text'' section to the ELF Section Table yet the new section has an
SgAsmBasicString name. We can assign a string to the name now and the
string will be allocated in the ELF file when we've provided further
information.

\begin{verbatim}
text->get_name()->set_string(".text");
\end{verbatim}

The ELF File Header needs to know the virtual address at which to
start running the program. In ROSE, virtual addresses can be attached
to a specific section so that if the section is ever moved the address
is automatically updated. Some formats allow more than one entry
address which is why the method is called add\_entry\_rva() rather than
set\_entry\_rva(). ELF, however, only allows one entry address.

\begin{verbatim}
rose_rva_t entry_rva(text->get_mapped_rva(), text);
fhdr->add_entry_rva(entry_rva);
\end{verbatim}

\section{Constructing a LOAD Segment}

ELF Segments define parts of an executable file that should be mapped
into memory by the loader.  A program will typically have a LOAD
segment that begins at the first byte of the file and continues
through the last instruction (in our case, the end of the ``.text''
section) and which is mapped to virtual address 0x400000.

We've already created the ELF Segment Table, so all we need to do now
is create an ELF Segment and add it to the ELF Segment Table. ELF
Segments, like ELF Sections, are represented by SgAsmElfSection. An
SgAsmElfSection is an ELF Section if it has an entry in the ELF
Section Table, and/or it's an ELF Segment if it has an entry in the
ELF Segment Table. The methods get\_section\_entry() and
get\_segment\_entry() retrieve the actual entries in those tables.

Recall that the constructor creates new sections located at the
current end-of-file and containing one byte. Our LOAD segment needs to
have a different offset and size.

\begin{verbatim}
SgAsmElfSection *seg1 = new SgAsmElfSection(fhdr);          /* ELF Segments are represented by SgAsmElfSection */
seg1->get_name()->set_string("LOAD");                       /* Segment names aren't saved (but useful for debugging) */
seg1->set_offset(0);                                        /* Starts at beginning of file */
seg1->set_size(text->get_offset() + text->get_size());      /* Extends to end of .text section */
seg1->set_mapped_rva(base_va);                              /* Typically mapped by loader to this memory address */
seg1->set_mapped_size(seg1->get_size());                    /* Make mapped size match size in the file */
seg1->set_mapped_rperm(true);                               /* Readable */
seg1->set_mapped_wperm(false);                              /* Not writable */
seg1->set_mapped_xperm(true);                               /* Executable */
segtab->add_section(seg1);                                  /* Add definition to ELF Segment Table */
\end{verbatim}

\section{Constructing a PAX Segment}

This documentation shows how to construct a generic ELF Segment,
giving it a particular file offset and size. ELF Segments don't have
names stored in the file, but we can assign a name to the AST node to
aid in debugging--it just won't be written out. When parsing an ELF
file, segment names are generated based on the type stored in the
entry of the ELF Segment Table. For a PAX segment we want this type to
be PT\_PAX\_FLAGS (the default is PT\_LOAD).

\begin{verbatim}
SgAsmElfSection *pax = new SgAsmElfSection(fhdr);
pax->get_name()->set_string("PAX Flags");                   /* Name just for debugging */
pax->set_offset(0);                                         /* Override address to be at zero rather than EOF */
pax->set_size(0);                                           /* Override size to be zero rather than one byte */
segtab->add_section(pax);                                   /* Add definition to ELF Segment Table */
pax->get_segment_entry()->set_type(SgAsmElfSegmentTableEntry::PT_PAX_FLAGS);
\end{verbatim}

\section{Constructing a String Table}

An ELF String Table always corresponds to a single ELF Section of
class SgAsmElfStringSection and thus you'll often see the term ``ELF
String Section'' used interchangeably with ``ELF String Table'' even
though they're two unique but closely tied classes internally.

When the ELF String Section is created a corresponding ELF String
Table is also created under the covers. Since string tables manage
their own memory in reponse to the strings they contain, one should
never adjust the size of the ELF String Section (it's actually fine to
enlarge the section and the new space will become free space in the
string table).

ELF files typically have multiple string tables so that section names
are in a different section than symbol names, etc. In this tutorial
we'll create the section names string table, typically called
``.shstrtab'', but use it for all string storage.

\begin{verbatim}
SgAsmElfStringSection *shstrtab = new SgAsmElfStringSection(fhdr);
shstrtab->get_name()->set_string(".shstrtab");
\end{verbatim}

\section{Constructing an ELF Section Table}

We do this last because we want the ELF Section Table to appear at the
end of the file and this is the easiest way to achieve that. There's
really not much one needs to do to create the ELF Section Table other
than provide the ELF File Header as a parent and supply a string
table.

The string table we created above isn't activated until we assign it
to the ELF Section Table. The first SgAsmElfStringSection added to the
SgAsmElfSectionTable becomes the string table for storing section
names. It is permissible to add other sections to the table before
adding the string table.

\begin{verbatim}
SgAsmElfSectionTable *sectab = new SgAsmElfSectionTable(fhdr);
sectab->add_section(text);                                  /* Add the .text section */
sectab->add_section(shstrtab);                              /* Add the string table to store section names. */
\end{verbatim}

\section{Allocating Space}

Prior to calling unparse(), we need to make sure that all sections
have a chance to allocate space for themselves, and perform any other
operations necessary. It's not always possible to determine sizes at
an earlier time, and most constructors would have declared sizes of
only one byte.

The reallocate() method is defined in the SgAsmGenericFile class since
it operates over the entire collection of sections simultaneously. In
other words, if a section needs to grow then all the sections located
after it in the file need to be shifted to higher file offsets.

\begin{verbatim}
ef->reallocate();
\end{verbatim}

The reallocate() method has a shortcoming (as of 2008-12-19) in that
it might not correctly update memory mappings in the case when the
mapping for a section is inferred from the mapping of a containing
segment. This can happen in our example since the ``.text'' section's
memory mapping is a function of the LOAD Segment mapping. The
work-around is to adjust mappings for these situations and then call
reallocate() one final time. This final reallocate() call won't move
any sections, but should always be the last thing to call before
unparsing() (it gives sections a chance to update data dependencies
which is not possible during unparse() due to its const nature).

\begin{verbatim}
text->set_mapped_rva(seg1->get_mapped_rva()+(text->get_offset()-seg1->get_offset()));
ef->reallocate(); /*won't resize or move things this time since we didn't modify much since the last call to reallocate()*/
\end{verbatim}

\section{Produce a Debugging Dump}

A debugging dump can be made with the following code. This dump will
not be identical to the one produced by parsing and dumping the
resulting file since we never parsed a file (a dump contains some
information that's parser-specific).

\begin{verbatim}
ef->dump(stdout);
SgAsmGenericSectionPtrList all = ef->get_sections(true);
for (size_t i=0; i<all.size(); i++) {
    fprintf(stdout, "Section %zu:\n", i);
    all[i]->dump(stdout, "    ", -1);
}
\end{verbatim}

\section{Produce the Executable File}

The executable file is produced by unparsing the AST.

\begin{verbatim}
std::ofstream f("a.out");
ef->unparse(f);
\end{verbatim}

Note that the resulting file will not be created with execute
permission--that must be added manually.
