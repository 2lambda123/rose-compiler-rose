\chapter{Simple AST Graph Generator}
\label{Tutorial:chapterASTGraphGenerator}

\paragraph{What To Learn From This Example}
This example shows how to generate a DOT file to visualize a simplified AST 
% (filtered out symbol tables and others for simplicity) 
from any input program.
\vspace{0.25in}

   DOT is a graphics file format from the AT\&T GraphViz project used to 
visualize moderate sized graphs.  It is one of the visualization tools used within the
ROSE project.  More inforamtion can be readily found at {\em www.graphviz.org/}.
We have found the {\em zgrviewer} to be an especially useful program for 
visualizing graphs generated in the DOT file format (see 
chapter~\ref{Tutorial:chapterGeneralASTGraphGeneration} for more information 
on zgrviewer).

Each node of the graph in figure~\ref{tutorial:exampleOutputCodeGraph} shows
a node of the Intermediate Representation (IR); the graphs demonstrates that the 
AST is formally a tree.  Each edge shows the connection
of the IR nodes in memory. The generated graph shows the connection of different 
IR nodes that form the AST for an input program source code.  Binary executables
can similarly be vizualized using DOT files.
The generation of such graphs is appropriate for small 
input programs, chapter~\ref{Tutorial:chapter_AST_PDF_Generator} shows a mechanism 
using PDF files that is more appropriate to larger programs (e.g. 100K lines of code).
More information about generation of specialized AST graphs can be found in 
\ref{Tutorial:chapterGeneralASTGraphGeneration} and custom graph generation in
\ref{Tutorial:chapterCustomGraphs}.

Note that a similar utility program named \textit{dotGenerator} already exists within
ROSE/exampleTranslators/DOTGenerator. 
%(and includes a utility to output an alternative PDF representation (suitable for larger ASTs) as well). 
It is also installed to \textit{ROSE\_INS/bin}.

\begin{figure}[!h]
{\indent
{\mySmallFontSize

% Do this when processing latex to generate non-html (not using latex2html)
\begin{latexonly}
   \lstinputlisting{\TutorialExampleDirectory/ASTGraphGenerator.C}
\end{latexonly}

% Do this when processing latex to build html (using latex2html)
\begin{htmlonly}
   \verbatiminput{\TutorialExampleDirectory/ASTGraphGenerator.C}
\end{htmlonly}

% end of scope in font size
}
% End of scope in indentation
}
\caption{Example source code to read an input program and generate an AST graph.}
\label{Tutorial:exampleASTGraphGenerator}
\end{figure}

The program in figure~\ref{Tutorial:exampleASTGraphGenerator} calls 
an internal ROSE function that traverses the AST and generates 
an ASCII file in {\tt dot} format.
Figure~\ref{Tutorial:exampleInputCode_ASTGraphGenerator} shows an input
code which is processed to generate a graph of the AST, generating a 
{\tt dot} file.   The {\tt dot} file is then processed
using {\tt dot} to generate a postscript file~\ref{tutorial:exampleOutputCodeGraph}
(within the {\tt Makefile}).
Figure~\ref{tutorial:exampleOutputCodeGraph}
(\TutorialExampleBuildDirectory/test.ps) can be found in the compile 
tree (in the tutorial directory) and viewed directly using ghostview 
or any postscript viewer to see more detail.


\begin{figure}[!h]
{\indent
{\mySmallFontSize

% Do this when processing latex to generate non-html (not using latex2html)
\begin{latexonly}
   \lstinputlisting{\TutorialExampleDirectory/inputCode_ASTGraphGenerator.C}
\end{latexonly}

% Do this when processing latex to build html (using latex2html)
\begin{htmlonly}
   \verbatiminput{\TutorialExampleDirectory/inputCode_ASTGraphGenerator.C}
\end{htmlonly}

% end of scope in font size
}
% End of scope in indentation
}
\caption{Example source code used as input to generate the AST graph.}
\label{Tutorial:exampleInputCode_ASTGraphGenerator}
\end{figure}

\begin{figure}
%\centerline{\epsfig{file=\TutorialExampleBuildDirectory/inputCode_ASTGraphGenerator.ps,
%                    height=1.3\linewidth,width=1.0\linewidth,angle=0}}
\includegraphics[scale=0.75]{\TutorialExampleBuildDirectory/inputCode_ASTGraphGenerator}
\caption{AST representing the source code file: inputCode\_ASTGraphGenerator.C.}
\label{tutorial:exampleOutputCodeGraph}
\end{figure}

   Figure~\ref{tutorial:exampleOutputCodeGraph} displays the individual
C++ nodes in ROSE's intermediate representation (IR).  Each circle represents
a single IR node, the name of the C++ construct appears in the center of the
circle, with the edge numbers of the traversal on top and the number of
child nodes appearing below.  Internal processing to build the graph generates
unique values for each IR node, a pointer address, which is displays at the bottom
of each circle.  The IR nodes are connected for form a tree, and abstract syntax
tree (AST). Each IR node is a C++ class, see SAGE III reference for details,
the edges represent the values of data members in the class (pointers which connect
the IR nodes to other IR nodes).  The edges are labeled with the names of the 
data members in the classes representing the IR nodes.

\fixme{Use this first example as a chance to explain the use of the
       header files (config.h and rose.h) and the code to build the 
       SgProject object.}



%------------------------------------------------------------------------
\chapter{AST Whole Graph Generator}
\label{Tutorial:chapterASTWholeGraphGenerator}

\paragraph{What To Learn From This Example}
This example shows how to generate and visualize the AST from any input program.
This view of the AST includes all additional IR nodes and edges that form attributes to
the AST, as a result this graph is not a tree.  These graphs are more complex but
show significantly more detail about the AST and its additional edges and attributes.
Each node of the graph in figure~\ref{tutorial:exampleOutputCodeWholeGraph} shows
a node of the Intermediate Representation (IR).  Each edge shows the connection
of the IR nodes in memory. The generated graph shows the connection of different 
IR nodes to form the AST and its additional attributes (e.g types, modifiers, etc).  
The generation of such graphs is appropriate for very small 
input programs, chapter~\ref{Tutorial:chapter_AST_PDF_Generator} shows a mechanism 
using PDF files that is more appropriate to larger programs (e.g. 100K lines of code).
More information about generation of specialized AST graphs can be found in 
\ref{Tutorial:chapterGeneralASTGraphGeneration} and custom graph generation in
\ref{Tutorial:chapterCustomGraphs}.

Again, a utility program, called
\textit{dotGeneratorWholeASTGraph} is provided within ROSE to generate
detailed dot graph for input code. It is available from
\textit{ROSE\_BUILD/exampleTranslators/DOTGenerator} or
\textit{ROSE\_INS/bin}. A set of options is available to further customize
what types of AST nodes to be shown or hidden. Please consult the screen output of
\textit{dotGeneratorWholeASTGraph -help} for details.

   {\em Viewing these dot files is best done using:} {\bf zgrviewer} at 
{\tt http://zvtm.sourceforge.net/zgrviewer.html}.  This tool permits zooming
in and out and viewing isolated parts of even very large graphs. {\bf Zgrviewer} permits 
a more natural way of understanding the AST and its additional IR nodes than the 
pdf file displayed in these pages.  The few lines of code used to generate the
graphs can be used on any input code to better understand how the AST represents
different languages and their constructs.

\begin{figure}[!h]
{\indent
{\mySmallFontSize

% Do this when processing latex to generate non-html (not using latex2html)
\begin{latexonly}
   \lstinputlisting{\TutorialExampleDirectory/wholeASTGraphGenerator.C}
\end{latexonly}

% Do this when processing latex to build html (using latex2html)
\begin{htmlonly}
   \verbatiminput{\TutorialExampleDirectory/wholeASTGraphGenerator.C}
\end{htmlonly}

% end of scope in font size
}
% End of scope in indentation
}
\caption{Example source code to read an input program and generate a {\em whole} AST graph.}
\label{Tutorial:exampleASTWholeGraphGenerator}
\end{figure}

The program in figure~\ref{Tutorial:exampleASTWholeGraphGenerator} calls 
an internal ROSE function that traverses the AST and generates 
an ASCII file in {\tt dot} format.
Figure~\ref{Tutorial:exampleInputCode_ASTWholeGraphGenerator_small} shows an tiny input
code which is processed to generate a graph of the AST with its attributes, generating a 
{\tt dot} file.   The {\tt dot} file is then processed
using {\tt dot} to generate a pdf file~\ref{tutorial:exampleOutputCodeWholeGraph_small}
(within the {\tt Makefile}).
Note that a similar utility program already exists within ROSE/exampleTranslators
(and includes a utility to output an alternative PDF representation 
(suitable for larger ASTs) as well).  Figure~\ref{tutorial:exampleOutputCodeWholeGraph}
(\TutorialExampleBuildDirectory/test.ps) can be found in the compile 
tree (in the tutorial directory) and viewed directly using 
any pdf or dot viewer to see more detail ({\bf zgrviewer} working with
the dot file directly is strongly advised).

   Note that AST's can get very large, and that the additional IR nodes required to
represent the types, modifiers, etc, can generate visually complex graphs. ROSE
contains the mechanisms to traverse these graphs and do analysis on them.  In
one case the number of IR nodes exceeded 27 million, an analysis was done through
a traversal of the graph in 10 seconds on a desktop x86 machine (the memory requirements
were 6 Gig).  ROSE organizes the IR in ways that permit analysis of programs that can 
represent rather large ASTs.


% ************************************************************
%                     Small Graph Example
% ************************************************************

\begin{figure}[!h]
{\indent
{\mySmallFontSize

% Do this when processing latex to generate non-html (not using latex2html)
\begin{latexonly}
   \lstinputlisting{\TutorialExampleDirectory/inputCode_wholeAST_1.C}
\end{latexonly}

% Do this when processing latex to build html (using latex2html)
\begin{htmlonly}
   \verbatiminput{\TutorialExampleDirectory/inputCode_wholeAST_1.C}
\end{htmlonly}

% end of scope in font size
}
% End of scope in indentation
}
\caption{Example tiny source code used as input to generate the small AST graph with attributes.}
\label{Tutorial:exampleInputCode_ASTGraphGenerator_small}
\end{figure}

\begin{figure}
\includegraphics[scale=0.95]{\TutorialExampleBuildDirectory/inputCode_ASTWholeGraphGenerator_small}
\caption{AST representing the tiny source code file: inputCode\_wholeAST\_1.C. This graphs
    shows types, symbols, and other attributes that are defined on the {\em attributed} AST.}
\label{tutorial:exampleOutputCodeWholeGraph_small}
\end{figure}

   Figure~\ref{tutorial:exampleOutputCodeWholeGraph_small} displays the individual
C++ nodes in ROSE's intermediate representation (IR).  Colors and shapes are used to 
represent different types or IR nodes. Although only visible using {\bf zgrviewer} 
the name of the C++ construct appears in the center of each node in the graph, with 
the names of the data members in each IR node as edge labels. Unique pointer values
are includes and printed next to the IR node name.  These graphs are the single best
way to develop an intuitive understanding how language constructs are organized
in the AST.  In these graphs, the color yellow is used for types (SgType IR nodes),
the color green is used for expressions (SgExpression IR nodes), and statements
are a number of different colors and shapes to make them more recognizable.

% ************************************************************
%                     Large Graph Example
% ************************************************************

\begin{figure}[!h]
{\indent
{\mySmallFontSize

% Do this when processing latex to generate non-html (not using latex2html)
\begin{latexonly}
   \lstinputlisting{\TutorialExampleDirectory/inputCode_wholeAST_2.C}
\end{latexonly}

% Do this when processing latex to build html (using latex2html)
\begin{htmlonly}
   \verbatiminput{\TutorialExampleDirectory/inputCode_wholeAST_2.C}
\end{htmlonly}

% end of scope in font size
}
% End of scope in indentation
}
\caption{Example source code used as input to generate a larger AST graph with attributes.}
\label{Tutorial:exampleInputCode_ASTGraphGenerator_large}
\end{figure}

\begin{figure}
\includegraphics[scale=0.95]{\TutorialExampleBuildDirectory/inputCode_ASTWholeGraphGenerator_large}
\caption{AST representing the small source code file: inputCode\_wholeAST\_2.C. This
    graph shows the significantly greater internal complexity of a slightly larger 
    input source code.}
\label{tutorial:exampleOutputCodeWholeGraph_large}
\end{figure}

   Figure~\ref{tutorial:exampleOutputCodeWholeGraph_large} shows a graph similar to the
previous graph but larger and more complex because it is from a larger code. Larger
graphs of this sort are still very useful in understanding how more significant
language constructs are organized and reference each other in the AST.  Tools
such as {\bf zgrviewer} are essential to reviewing and understanding these
graphs. Although such graphs can be visualized, in practice this is only useful
for debugging small codes in the construction of custom analysis and transformation
tools. The graphs for real million line applications would never be visualized.
Using ROSE one can build automated tools to operate on the AST for large scale
applications where visualization would not be possible.

