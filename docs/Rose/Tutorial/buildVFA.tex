\chapter{Dataflow Analysis based Virtual Function Analysis}

C++ Virtual function provides polymorphism to the developer but makes it difficult for compilers to do optimizations. 
Virtual functions are usually resolved at runtime from the vtable.
It's very difficult for a compiler to know which functions will be called at compile time.
ROSE provides a flow sensitive dataflow analysis based approach to cut down the set of possible function calls. 
The code for Virtual Function Analysis is located in \textit{src/midend/programAnalysis/VirtualFunctionAnalysis/VirtualFunctionAnalysis.h}. 
It also
provides a mechanism to resolve any function calls. It's a whole program analysis and supposed to be expensive.
It memorizes all the resolved function calls for any call site, so that subsequent calls are resolved  faster.

Figure~\ref{Tutorial:exampleVFA} shows the code required to generate
the pruned call graph.  Using the input code shown in
figure~\ref{Tutorial:exampleInputCode_VFA} Call Graph Analysis generates call graph shown in figure~\ref{Tutorial:exampleVFA_OrigGraph}.
Executing dataflow analysis to resolve virtual function calls resulted in the 
 figure~\ref{Tutorial:exampleVFA_Graph}.

\begin{figure}[!h]
{\indent
{\mySmallFontSize


% Do this when processing latex to generate non-html (not using latex2html)
\begin{latexonly}
   \lstinputlisting{\TutorialExampleDirectory/buildVFA.C}
\end{latexonly}

% Do this when processing latex to build html (using latex2html)
\begin{htmlonly}
   \verbatiminput{\TutorialExampleDirectory/buildVFA.C}
\end{htmlonly}

% end of scope in font size
}
% End of scope in indentation
}
\caption{Source code to perform virtual function analysis}
\label{Tutorial:exampleVFA}
\end{figure}


\begin{figure}[!h]
{\indent
{\mySmallFontSize


% Do this when processing latex to generate non-html (not using latex2html)
\begin{latexonly}
   \lstinputlisting{\TutorialExampleDirectory/inputCode_VFA.C}
\end{latexonly}

% Do this when processing latex to build html (using latex2html)
\begin{htmlonly}
   \verbatiminput{\TutorialExampleDirectory/inputCode_VFA.C}
\end{htmlonly}

% end of scope in font size
}
% End of scope in indentation
}
\label{Tutorial:exampleInputCode_VFA}
\caption{Example source code used as input for Virtual Function Analysis.}
\end{figure}

\begin{figure}
\includegraphics[scale=0.7]{\TutorialExampleBuildDirectory/vfa_orig}
\caption{Call graph generated by Call Graph Analysis for input code in inputCode\_vfa.C.}
\label{Tutorial:exampleVFA_OrigGraph}
\end{figure}


\begin{figure}
% \centerline{\epsfig{file=\TutorialExampleBuildDirectory/callGraph.ps,
%                    height=1.3\linewidth,width=1.0\linewidth,angle=0}}
\includegraphics[scale=0.7]{\TutorialExampleBuildDirectory/vfa}
\caption{Call graph resulted from Virtual Function Analysis for input code in inputCode\_vfa.C.}
\label{Tutorial:exampleVFA_Graph}
\end{figure}

%   Figure~\ref{Tutorial:exampleVFA} shows the call graph for the
%function in the input code in figure~\ref{Tutorial:exampleInputCode_VFA}.



