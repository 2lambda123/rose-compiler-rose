%==========================================================================
\chapter{Function Outlining using the AST Outliner}
\label{chap:outliner}
%==========================================================================

\emph{Function outlining} is the process of replacing a block of
consecutive statements with a function call to a new function
containing those statements. Conceptually, outlining the inverse of
inlining (Chapter~\ref{chap:inliner}). This chapter shows how to use
the basic experimental outliner implementation included in the ROSE projects
directory.

There are two basic ways to use the outliner. The first is a
``user-level'' method, in which you may use a special pragma to mark
outline targets in the input program, and then call a high-level
driver routine to process these pragmas. The second method is to call
``low-level'' outlining routines that operate directly on AST nodes.
After a brief example of what the outliner can do and a discussion
of its limits
(Sections~\ref{sec:Outliner:example}--\ref{sec:Outliner:limits}),
we discuss each of these methods in
Sections~\ref{sec:Outliner:basic} and~\ref{sec:Outliner:direct},
respectively.

%==========================================================================
\section{An Outlining Example}
\label{sec:Outliner:example}

Figure~\ref{fig:Outliner:input1} shows a small program with a
pragma marking the outline target, a nested for loop, and
Figure~\ref{fig:Outliner:output1} shows the result. The outliner
extracts the loop and inserts it into the body of a new function, and
inserts a call to that function. The outlined code's input and output
variables are wrapped up as parameters to this function. We make the
following observations about this output.

\textbf{Placement and forward declarations}. The function itself is
placed, by default, at the end of the input file to guarantee that it
has access to all of the same declarations that were available at the
outline target site. The outliner inserts any necessary forward
declarations as well, including any necessary \texttt{friend}
declarations if the outline target appeared in a class member
function.

\textbf{Calling convention}. The outliner generates a C-callable
function (\texttt{extern ``C''}, with pointer arguments). This design
choice is motivated by our need to use the outliner to extract code
into external, dynamically loadable library modules.

%--------------------------------------------------------------------------
\begin{figure}[!b]
{\indent
{\mySmallFontSize
\begin{latexonly}
   \lstinputlisting[language=C++,numbers=left,firstnumber=1,stepnumber=5]{\TutorialExampleDirectory/outliner/inputCode_OutlineLoop.cc}
\end{latexonly}
\begin{htmlonly}
   \verbatiminput{\TutorialExampleDirectory/outliner/inputCode_OutlineLoop.cc}
\end{htmlonly}

% end of scope in font size
}
% End of scope in indentation
}
\caption{inputCode\_OutlineLoop.cc: Sample input program. The \texttt{\#pragma} directive marks the nested for loop for outlining.}
\label{fig:Outliner:input1}
\end{figure}
%--------------------------------------------------------------------------

%--------------------------------------------------------------------------
\begin{figure}[!h]
{\indent
{\mySmallFontSize
\begin{latexonly}
   \lstinputlisting[language=C++,numbers=left,firstnumber=1,stepnumber=5]{\TutorialExampleBuildDirectory/outliner/rose_outlined-inputCode_OutlineLoop.cc}
\end{latexonly}
\begin{htmlonly}
   \verbatiminput{\TutorialExampleBuildDirectory/outliner/rose_outlined-inputCode_OutlineLoop.cc}
\end{htmlonly}

% end of scope in font size
}
% End of scope in indentation
}
\caption{rose\_outlined-inputCode\_OutlineLoop.cc: The nested for loop of Figure~\ref{fig:Outliner:input1} has been outlined.}
\label{fig:Outliner:output1}
\end{figure}
%--------------------------------------------------------------------------

%==========================================================================
\section{Limitations of the Outliner}
\label{sec:Outliner:limits}

The main limitation of the outliner implementation is that it can only
outline single SgStatement nodes. However, since an SgStatement node
may be a block (\emph{i.e.}, an SgBasicBlock node), a ``single
statement'' may actually comprise a sequence of complex statements.

The rationale for restricting to single SgStatement nodes is to avoid
subtly changing the program's semantics when outlining code. Consider
the following example, in which we wish to outline the middle 3 lines
of executable code.
%
\lstset{language=C++,numbers=left,firstnumber=1,stepnumber=2}
\begin{lstlisting}
  int x = 5;
// START outlining here.
  foo (x);
  Object y (x);
  y.foo ();
// STOP outlining here.
  y.bar ();
\end{lstlisting}
%
This example raises a number of issues. How should an outliner handle
the declaration of \texttt{y}, which constructs an object in local
scope? It cannot just cut-and-paste the declaration of \texttt{y} to
the body of the new outlined function because that will change its
scope and lifetime, rendering the call to \texttt{y.bar()}
impossible. Additionally, it may be unsafe to move the declaration of
\texttt{y} so that it precedes the outlined region because the
constructor call may have side-effects that could affect the execution
of \texttt{foo(x)}. It is possible to heap-allocate \texttt{y} inside
the body of the outlined function so that it can be returned to the
caller and later freed, but it is not clear if changing \texttt{y}
from a stack-allocated variable to a heap-allocated one will always be
acceptable, particularly if the developer of the original program has,
for instance, implemented a customized memory allocator. Restricting
outlining to well-defined SgStatement objects avoids these issues.  It
is possible to build a ``higher-level'' outliner that extends the
outliner's basic infrastructure to handle these and other issues.

The outliner cannot outline all possible SgStatement nodes.
However, the outliner interface provides a routine,
\texttt{outliner::isOutlineable(s)}, for testing whether an
SgStatement object \texttt{s} is known to satisfy the outliner's
preconditions (see Section~\ref{sec:Outliner:direct} for details).

%==========================================================================
\section{User-Directed Outlining \emph{via} Pragmas}
\label{sec:Outliner:basic}

Figure~\ref{fig:Outliner:basic} shows the basic translator,
\texttt{outline}, that produces Figure~\ref{fig:Outliner:output1}
from Figure~\ref{fig:Outliner:input1}. This translator extends the
identity translator with an include directive on line 5 of
Figure~\ref{fig:Outliner:basic}, and a call to the outliner on
line 16. All outliner routines live in the \texttt{Outliner}
namespace. Here, the call to \texttt{Outliner::outlineAll (proj)}
on line 16 traverses the AST, looks for \texttt{\#pragma
rose\_outline} directives, outlines the SgStatement objects to which
each pragma is attached, and returns the number of outlined objects.

\textbf{A slightly lower-level outlining primitive}. The
\texttt{Outliner::outlineAll()} routine is a wrapper around calls
to a simpler routine, \texttt{Outliner::outline()}, that operates
on pragmas:
%
\begin{lstlisting}
  Outliner::Result  Outliner::outline (SgPragmaDeclaration* s);
\end{lstlisting}
%
Given a pragma statement AST node \texttt{s}, this routine checks if
\texttt{s} is a \texttt{rose\_outline} directive, and if so, outlines
the statement with which \texttt{s} is associated. It returns a
\texttt{Outliner::Result} object, which is simply a structure that
points to (a) the newly generated outlined function and (b) the
statement corresponding to the new function call (\emph{i.e.}, the
outlined function call-site). See \texttt{Outliner.hh} or the ROSE
Programmer's Reference for more details.

\textbf{The \texttt{Outliner::outlineAll()} wrapper}. The advantage
of using the wrapper instead of the lower-level primitive is that the
wrapper processes the pragmas in an order that ensures the outlining
can be performed correctly in-place. This order is neither a preorder
nor a postorder traversal, but in fact a ``reverse'' preorder
traversal; refer to the wrapper's documentation for an explanation.

%--------------------------------------------------------------------------
\begin{figure}[!h]
{\indent
{\mySmallFontSize
\begin{latexonly}
   \lstinputlisting[language=C++,numbers=left,firstnumber=1,stepnumber=5]{\TutorialExampleDirectory/outliner/outline.cc}
\end{latexonly}
\begin{htmlonly}
   \verbatiminput{\TutorialExampleDirectory/outliner/outline.cc}
\end{htmlonly}

% end of scope in font size
}
% End of scope in indentation
}
\caption{outline.cc: A basic outlining translator, which generates
Figure~\ref{fig:Outliner:output1} from
Figure~\ref{fig:Outliner:input1}. This outliner relies on the
high-level driver, \texttt{Outliner::outlineAll()}, which scans the
AST for outlining pragma directives (\texttt{\#pragma rose\_outline})
that mark outline targets.}
\label{fig:Outliner:basic}
\end{figure}
%--------------------------------------------------------------------------

%==========================================================================
\section{Calling Outliner Directly on AST Nodes}
\label{sec:Outliner:direct}

The preceding examples rely on the outliner's \texttt{\#pragma}
interface to identify outline targets. In this section, we show how
to call the outliner directly on SgStatement nodes from within your
translator.

Figure~\ref{fig:Outliner:direct} shows an example translator that
finds all \texttt{if} statements and outlines them. A sample input
appears in Figure~\ref{fig:Outliner:input3}, with the corresponding
output shown in Figure~\ref{fig:Outliner:output3}. Notice that
valid preprocessor control structure is accounted for and preserved in
the output.

The translator has two distinct phases. The first phase selects all
\emph{outlineable} if-statements, using the
\texttt{CollectOutlineableIfs} helper class. This class produces a
list that stores the targets in an order appropriate for outlining
them in-place. The second phase iterates over the list of statements
and outlines each one. The rest of this section explains these phases,
as well as various aspects of the sample input and output.

%--------------------------------------------------------------------------
\begin{figure}[!h]
{\indent
{\mySmallFontSize
\begin{latexonly}
   \lstinputlisting[language=C++,numbers=left,firstnumber=1,stepnumber=5]{\TutorialExampleDirectory/outliner/outlineIfs.cc}
\end{latexonly}
\begin{htmlonly}
   \verbatiminput{\TutorialExampleDirectory/outliner/outlineIfs.cc}
\end{htmlonly}

% end of scope in font size
}
% End of scope in indentation
}
\caption{outlineIfs.cc: A lower-level outlining translator, which
calls \texttt{Outliner::outline()} directly on SgStatement
nodes. This particular translator outlines all SgIfStmt nodes.}
\label{fig:Outliner:direct}
\end{figure}
%--------------------------------------------------------------------------

\subsection{Selecting the \emph{outlineable} \texttt{if} statements}
\label{sec:Outliner:direct:translator}

Line~45 of Figure~\ref{fig:Outliner:direct} builds a list,
\texttt{ifs} (declared on line~44), of outlineable if-statements.  The
helper class, \texttt{CollectOutlineableIfs} in lines~12--35,
implements a traversal to build this list. Notice that a node is
inserted into the target list only if it satisfies the outliner's
preconditions; this check is the call to
\texttt{Outliner::isOutlineable()} on line~28.

The function \texttt{Outliner::isOutlineable()} also accepts an
optional second boolean parameter (not shown). When this parameter is
true and the statement cannot be outlined, the check will print an
explanatory message to standard error. Such messages are useful for
discovering why the outliner will not outline a particular
statement. The default value of this parameter is \texttt{false}.

%--------------------------------------------------------------------------
\begin{figure}[!h]
{\indent
{\mySmallFontSize
\begin{latexonly}
   \lstinputlisting[language=C++,numbers=left,firstnumber=1,stepnumber=5]{\TutorialExampleDirectory/outliner/inputCode_Ifs.cc}
\end{latexonly}
\begin{htmlonly}
   \verbatiminput{\TutorialExampleDirectory/outliner/inputCode_Ifs.cc}
\end{htmlonly}

% end of scope in font size
}
% End of scope in indentation
}
\caption{inputCode\_Ifs.cc: Sample input program, without explicit
outline targets specified using \texttt{\#pragma rose\_outline}, as in
Figures~\ref{fig:Outliner:input1}
and~\ref{fig:Outliner:input2}.}
\label{fig:Outliner:input3}
\end{figure}
%--------------------------------------------------------------------------

\subsection{Properly ordering statements for in-place outlining}
\label{sec:Outliner:direct:ordering}

Each call to \texttt{Outliner::outline(*i)} on line~50 of
Figure~\ref{fig:Outliner:direct} outlines a target if-statement
\texttt{*i} in \texttt{if\_targets}. However, in order for these
statements to be outlined in-place, it is essential to outline the
statements in the proper order.

The postorder traversal implemented by the helper class,
\texttt{CollectOutlineableIfs}, produces the correct ordering.  To see
why, consider the following example code:
%
\begin{lstlisting}
if (a) // [1]
{
  if (b) foo (); // [2]
}
else if (c) // [3]
{
  if (d) bar (); // [4]
}
\end{lstlisting}
%
The corresponding AST is (roughly)
\begin{verbatim}
       SgIfStmt:[1]
      /            \
     /              \
SgIfStmt:[2]    SgIfStmt:[3]
                     | 
                SgIfStmt:[4]
\end{verbatim}
%
The postorder traversal---2, 4, 3, 1---ensures that child
if-statements are outlined before their parents.

%--------------------------------------------------------------------------
\begin{figure}[!h]
{\indent
{\mySmallFontSize
\begin{latexonly}
   \lstinputlisting[language=C++,numbers=left,firstnumber=1,stepnumber=5]{\TutorialExampleBuildDirectory/outliner/rose_inputCode_Ifs.cc}
\end{latexonly}
\begin{htmlonly}
   \verbatiminput{\TutorialExampleBuildDirectory/outliner/rose_inputCode_Ifs.cc}
\end{htmlonly}

% end of scope in font size
}
% End of scope in indentation
}
\caption{rose\_inputCode\_Ifs.cc: Figure~\ref{fig:Outliner:input3},
after outlining using the translator in
Figure~\ref{fig:Outliner:direct}.}
\label{fig:Outliner:output3}
\end{figure}
%--------------------------------------------------------------------------

%==========================================================================
\section{Outliner's Preprocessing Phase}
\label{sec:Outliner:preproc}

Internally, the outliner implementation itself has two distinct
phases. The first is a \emph{preprocessing phase}, in which an
arbitrary outlineable target is placed into a canonical form that is
relatively simple to extract. The second phase then creates the
outlined function, replacing the original target with a call to the
outlined function. It is possible to run just the preprocessing phase,
which is useful for understanding or even debugging the outliner
implementation.

%--------------------------------------------------------------------------
\begin{figure}[!h]
{\indent
{\mySmallFontSize
\begin{latexonly}
   \lstinputlisting[language=C++,numbers=left,firstnumber=1,stepnumber=5]{\TutorialExampleDirectory/outliner/outlinePreproc.cc}
\end{latexonly}
\begin{htmlonly}
   \verbatiminput{\TutorialExampleDirectory/outliner/outlinePreproc.cc}
\end{htmlonly}

% end of scope in font size
}
% End of scope in indentation
}
\caption{outlinePreproc.cc: The basic translator of
Figure~\ref{fig:Outliner:basic}, modified to execute the
Outliner's preprocessing phase only.  In particular, the original
call to \texttt{Outliner::outlineAll()} has been replaced by a call
to \texttt{Outliner::preprocessAll()}.}
\label{fig:Outliner:preproc}
\end{figure}
%--------------------------------------------------------------------------

To call just the preprocessor, simply replace a call to
\texttt{Outliner::outlineAll(s)} or
\texttt{Outliner::outline(s)} with a call to
\texttt{Outliner::preprocessAll(s)} or
\texttt{Outliner::preprocess(s)}, respectively. The translator in
Figure~\ref{fig:Outliner:preproc} modifies the translator in
Figure~\ref{fig:Outliner:basic} in this way to create a
preprocessing-only translator.

The preprocessing phase consists of a sequence of initial analyses and
transformations that the outliner performs in order to put the
outline target into a particular canonical form. Roughly speaking,
this form is an enclosing SgBasicBlock node, possibly preceded or
followed by start-up and tear-down code. Running just the
preprocessing phase on Figure~\ref{fig:Outliner:input1} produces
the output in Figure~\ref{fig:Outliner:preproc1}. In this example,
the original loop is now enclosed in two additional SgBasicBlocks
(Figure~\ref{fig:Outliner:preproc1}, lines 24--35), the outermost
of which contains a declaration that shadows the object's
\texttt{this} pointer, replacing all local references to \texttt{this}
with the new shadow pointer. In this case, this initial transformation
is used by the main underlying outliner implementation to explicitly
identify all references to the possibly implicit references to
\texttt{this}.

%--------------------------------------------------------------------------
\begin{figure}[!h]
{\indent
{\mySmallFontSize
\begin{latexonly}
   \lstinputlisting[language=C++,numbers=left,firstnumber=1,stepnumber=5]{\TutorialExampleBuildDirectory/outliner/rose_outlined_pp-inputCode_OutlineLoop.cc}
\end{latexonly}
\begin{htmlonly}
   \verbatiminput{\TutorialExampleBuildDirectory/outliner/rose_outlined_pp-inputCode_OutlineLoop.cc}
\end{htmlonly}

% end of scope in font size
}
% End of scope in indentation
}
\caption{rose\_outlined\_pp-inputCode\_OutlineLoop.cc:
Figure~\ref{fig:Outliner:input1} after outline preprocessing only,
\emph{i.e.}, specifying \texttt{-rose:outline:preproc-only} as an
option to the translator of Figure~\ref{fig:Outliner:basic}.}
\label{fig:Outliner:preproc1}
\end{figure}
%--------------------------------------------------------------------------

The preprocessing phase is more interesting in the presence of
non-local control flow outside the outline target. Consider
Figure~\ref{fig:Outliner:input2}, in which the outline target
contains two \texttt{break} statements, which require jumping to a
regions of code outside the target. We show the preprocessed code in
Figure~\ref{fig:Outliner:preproc2}. The original non-local jumps
are first transformed into assignments to a flag,
\texttt{EXIT\_TAKEN\_\_} (lines 18--20 and 26--29), and then relocated
to a subsequent block of code (lines 38--53) with their execution
controlled by the value of the flag. The final outlined result appears
in Figure~\ref{fig:Outliner:output2}; the initial preprocessing
simplifies this final step of extracting the outline target.

%--------------------------------------------------------------------------
\begin{figure}[!h]
{\indent
{\mySmallFontSize
\begin{latexonly}
   \lstinputlisting[language=C++,numbers=left,firstnumber=1,stepnumber=5]{\TutorialExampleDirectory/outliner/inputCode_OutlineNonLocalJumps.cc}
\end{latexonly}
\begin{htmlonly}
   \verbatiminput{\TutorialExampleDirectory/outliner/inputCode_OutlineNonLocalJumps.cc}
\end{htmlonly}

% end of scope in font size
}
% End of scope in indentation
}
\caption{inputCode\_OutlineNonLocalJumps.cc: Sample input program,
with an outlining target that contains two non-local jumps (here,
\texttt{break} statements).}
\label{fig:Outliner:input2}
\end{figure}
%--------------------------------------------------------------------------

%--------------------------------------------------------------------------
\begin{figure}[!h]
{\indent
{\mySmallFontSize
\begin{latexonly}
   \lstinputlisting[language=C++,numbers=left,firstnumber=1,stepnumber=5]{\TutorialExampleBuildDirectory/outliner/rose_outlined_pp-inputCode_OutlineNonLocalJumps.cc}
\end{latexonly}
\begin{htmlonly}
   \verbatiminput{\TutorialExampleBuildDirectory/outliner/rose_outlined_pp-inputCode_OutlineNonLocalJumps.cc}
\end{htmlonly}

% end of scope in font size
}
% End of scope in indentation
}
\caption{rose\_outlined\_pp-inputCode\_OutlineNonLocalJumps.cc: The
non-local jump example of Figure~\ref{fig:Outliner:input2} after
outliner preprocessing, but before the actual outlining.  The
non-local jump is handled by an additional flag,
\texttt{EXIT\_TAKEN\_\_}, which indicates what non-local jump is to be
taken.}
\label{fig:Outliner:preproc2}
\end{figure}
%--------------------------------------------------------------------------

%--------------------------------------------------------------------------
\begin{figure}[!h]
{\indent
{\mySmallFontSize
\begin{latexonly}
   \lstinputlisting[language=C++,numbers=left,firstnumber=1,stepnumber=5]{\TutorialExampleBuildDirectory/outliner/rose_outlined-inputCode_OutlineNonLocalJumps.cc}
\end{latexonly}
\begin{htmlonly}
   \verbatiminput{\TutorialExampleBuildDirectory/outliner/rose_outlined-inputCode_OutlineNonLocalJumps.cc}
\end{htmlonly}

% end of scope in font size
}
% End of scope in indentation
}
\caption{rose\_outlined-inputCode\_OutlineNonLocalJumps.cc:
Figure~\ref{fig:Outliner:input2} after outlining.}
\label{fig:Outliner:output2}
\end{figure}
%--------------------------------------------------------------------------

% eof
