\chapter{Virtual CFG}

The ROSE virtual control flow graph interface provides a higher level of
detail than ROSE's other control flow graph interfaces.  It expresses
control flow even within expressions, and handles short-circuited logical
and conditional operators properly\footnote{It assumes operands of
expressions are computed in left-to-right order, unlike the actual language
semantics, however.}.  The interface is referred to as ``virtual'' because
no explicit graph is ever created: only the particular CFG nodes and edges
used in a given program ever exist.  CFG nodes and edges are value classes
(they are copied around by value, reducing the need for explicit memory
management).

A CFG node consists of two components: an AST node pointer, and an index of
a particular CFG node within that AST node.  There can be several CFG nodes
corresponding to a given AST node, and thus the AST node pointers cannot be
used to index CFG nodes.  The particular index values for the different AST
node types are explained in Section~\ref{cfg_index_values}.

\section{Important functions}

The main body of the virtual CFG interface is in \lstinline{virtualCFG.h};
the source code is in \lstinline{src/frontend/SageIII/virtualCFG/} and is
linked into \lstinline{librose}.  The filtered CFG interface explained
below is in \lstinline{filteredCFG.h}, and functions for converting the CFG
to a graph in Dot format are in \lstinline{cfgToDot.h}.

Two functions provide the basic way of converting from AST nodes to CFG
nodes.  Each \lstinline{SgNode} has two methods,
\lstinline{cfgForBeginning()} and \lstinline{cfgForEnd()}, to generate the
corresponding CFG nodes.  These functions require that the AST node is
either an expression, a statement, or a \lstinline{SgInitializedName}.  The
beginning node represents the point in the control flow immediately before
the construct starts to execute, and the ending node represents the point
immediately after the construct has finished executing.  Note that these
two nodes do not dominate the other CFG nodes in the construct due to
\lstinline{goto} statements and labels.

\subsection{Node methods}

\subsection{Edge methods}

\section{Drawing a graph of the CFG}

FIXME, add example

\includegraphics{\TutorialExampleBuildDirectory/vcfg.pdf}

\section{Index values}
\label{cfg_index_values}

\section{Robustness to AST changes}

\section{Limitations}

Although workable for intraprocedural analysis of C code, the virtual CFG
code has several limitations for other languages and uses.

\subsection{Fortran support}

The virtual control flow graph includes support for many Fortran
constructs, but that support is fairly limited and not well tested.  It is
not recommended for production use.

\subsection{Exception handling}

The virtual CFG interface does not support control flow due to exceptions
or the \lstinline{setjmp}/\lstinline{longjmp} constructs.
It does, however, support \lstinline{break}, \lstinline{continue},
\lstinline{goto}, and early returns from functions.

\subsection{Interprocedural control flow analysis}

A limited form of interprocedural control flow analysis is supported.  That
feature is enabled with a global variable named
\lstinline{interproceduralControlFlowGraph}.  Setting that variable to
\lstinline{true} changes the out edges of function calls and the in edges
of nodes just after function calls when the function references are known.
Although this causes interprocedural behavior, it also leads to a mismatch
between the in and out edges between certain pairs of nodes.  Solving this
problem would require a precomputed call graph for the program, which
defeats the goal of not requiring any precomputed or cached information to
traverse the control flow graph.

\section{Node filtering}

\subsection{``Interesting'' node filter}

\subsection{Arbitrary filtering}

% \section{Control flow graph on binaries}
