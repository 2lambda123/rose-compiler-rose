\chapter{{\em TAU} Instrumentation}

  Tau is a performance analysis tool from University of 
Oregon.  They have mechanisms for automating instrumentation
of the source code's text file directly, but these can be
problematic in the present of macros. We present an example
of instrumentation combined with code generation to provide a more
robust means of instrumentation for source code.  This work is
preliminary and depends upon two separate mechanisms for the
rewrite of the AST (one high level using strings and one low level
representing a more direct handling of the AST at the IR level).

\section{Input For Examples Showing Information using Tau}

   Figure~\ref{Tutorial:exampleInputCode_tauInstrumenter}
shows the example input used for demonstration of Tau instrumentation.

% \commentout{

\begin{figure}[!h]
{\indent
{\mySmallFontSize

% Do this when processing latex to generate non-html (not using latex2html)
\begin{latexonly}
   \lstinputlisting{\TutorialExampleDirectory/inputCode_tauInstrumenter.C}
\end{latexonly}

% Do this when processing latex to build html (using latex2html)
\begin{htmlonly}
   \verbatiminput{\TutorialExampleDirectory/inputCode_tauInstrumenter.C}
\end{htmlonly}

% end of scope in font size
}
% End of scope in indentation
}
\caption{Example source code used as input to program in
         codes used in this chapter.}
\label{Tutorial:exampleInputCode_tauInstrumenter}
\end{figure}

% }

% \commentout{

\section{Generating the code representing any IR node}

    Figure~\ref{Tutorial:example_tauInstrumenter}
shows a code that traverses each IR node and for a 
SgInitializedName of SgStatement output the scope information.
The input code is shown in figure \ref{Tutorial:example_tauInstrumenter},
the output of this code is shown in 
figure~\ref{Tutorial:exampleOutput_tauInstrumenter}.


\begin{figure}[!h]
{\indent
{\mySmallFontSize


% Do this when processing latex to generate non-html (not using latex2html)
\begin{latexonly}
   \lstinputlisting{\TutorialExampleDirectory/tauInstrumenter.C}
\end{latexonly}

% Do this when processing latex to build html (using latex2html)
\begin{htmlonly}
   \verbatiminput{\TutorialExampleDirectory/tauInstrumenter.C}
\end{htmlonly}

% end of scope in font size
}
% End of scope in indentation
}
\caption{Example source code showing how to instrument using Tau. }
\label{Tutorial:example_tauInstrumenter}
\end{figure}




\begin{figure}[!h]
{\indent
{\mySmallFontSize


% Do this when processing latex to generate non-html (not using latex2html)
\begin{latexonly}
   \lstinputlisting{\TutorialExampleBuildDirectory/rose_inputCode_tauInstrumenter.C}
\end{latexonly}

% Do this when processing latex to build html (using latex2html)
\begin{htmlonly}
   \verbatiminput{\TutorialExampleBuildDirectory/rose_inputCode_tauInstrumenter.C}
\end{htmlonly}

% end of scope in font size
}
% End of scope in indentation
}
\caption{Output of input code using tauInstrumenter.C}
\label{Tutorial:exampleOutput_tauInstrumenter}
\end{figure}

% }
