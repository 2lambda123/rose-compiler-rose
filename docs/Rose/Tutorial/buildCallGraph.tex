\chapter{Generating the Call Graph (CG)}

\begin{figure}[!h]
{\indent
{\mySmallFontSize

\label{Tutorial:exampleBuildCG}

% Do this when processing latex to generate non-html (not using latex2html)
\begin{latexonly}
   \lstinputlisting{\TutorialExampleDirectory/buildCallGraph.C}
\end{latexonly}

% Do this when processing latex to build html (using latex2html)
\begin{htmlonly}
   \verbatiminput{\TutorialExampleDirectory/buildCallGraph.C}
\end{htmlonly}

% end of scope in font size
}
% End of scope in indentation
}
\caption{Example source code showing visualization of call graph.}
\end{figure}

The formal definition of a call graph is: 
\newline\newline
'A diagram that identifies the modules in a system or computer program and shows which modules call one another.' IEEE 
\newline\newline
A call graph shows all function call paths of an arbitrary code. These paths are found by following all 
function calls in a function, where a function in the graph is represented by a node and each possible function call by
an edge (arrow). To make a call graph this process is  redone for every called function until all edges are followed
and there are no ungraphed functions. ROSE has an in-build mechanism for generating call graphs. 

ROSE provides support for generating call graphs, as defined in
\textit{src/midend/programAnalysis/CallGraphAnalysis/CallGraph.h}.
   Figure~\ref{Tutorial:exampleBuildCG} shows the code required to generate
the call graph for each function of an application.  Using the input code shown in
figure~\ref{Tutorial:exampleInputCode_BuildCG} the first function's call graph is
shown in figure~\ref{Tutorial:exampleBuildCGGraph}.
A standalone tool named \textit{buildCallGraph} is installed under
\textit{ROSE\_INSTALL/bin} so users can use it to generate call graphs in
dot format.

\begin{figure}[!h]
{\indent
{\mySmallFontSize

\label{Tutorial:exampleInputCode_BuildCG}

% Do this when processing latex to generate non-html (not using latex2html)
\begin{latexonly}
   \lstinputlisting{\TutorialExampleDirectory/inputCode_BuildCG.C}
\end{latexonly}

% Do this when processing latex to build html (using latex2html)
\begin{htmlonly}
   \verbatiminput{\TutorialExampleDirectory/inputCode_BuildCG.C}
\end{htmlonly}

% end of scope in font size
}
% End of scope in indentation
}
\caption{Example source code used as input to build call graph.}
\end{figure}


\begin{figure}
% \centerline{\epsfig{file=\TutorialExampleBuildDirectory/callGraph.ps,
%                    height=1.3\linewidth,width=1.0\linewidth,angle=0}}
\includegraphics[scale=0.7]{\TutorialExampleBuildDirectory/callGraph}
\caption{Call graph for function in input code file: inputCode\_BuildCG.C.}
\label{Tutorial:exampleBuildCGGraph}
\end{figure}

%   Figure~\ref{Tutorial:exampleBuildCGGraph} shows the call graph for the
%function in the input code in figure~\ref{Tutorial:exampleInputCode_BuildCG}.



