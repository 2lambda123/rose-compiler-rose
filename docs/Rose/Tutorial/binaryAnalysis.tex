\chapter{Binary Analysis}

This chapter discusses the capabilities of ROSE to read and analyze software binaries.

In the following we use a small example that demonstrates various features of Binary-Rose.
The source code of our binary example is:

\begin{figure}[!h]
{\indent
{\mySmallFontSize


% Do this when processing latex to generate non-html (not using latex2html)
\begin{latexonly}
   \lstinputlisting{\TutorialExampleDirectory/bin_example.C}
\end{latexonly}

% Do this when processing latex to build html (using latex2html)
\begin{htmlonly}
   \verbatiminput{\TutorialExampleDirectory/bin_example.C}
\end{htmlonly}

% end of scope in font size
}
% End of scope in indentation
}
\caption{Example source code.}
\label{Tutorial:examplesourcecode}
\end{figure}


\section{Loading binaries}

Binary-ROSE is currently based on two front-ends: \emph{objdump} and \emph{IdaPro-mysql}.

\subsection{objdump}

The following code reads in a binary and creates a binary ROSE AST:

{\mySmallFontSize
\begin{verbatim}
     SgProject* project = frontend(argc,argv);
     ROSE_ASSERT (project != NULL);
\end{verbatim}
}


\subsection{IdaPro-mysql}

A binary processed by IdaPro needs to be processed into a MySQL database (DB). With that DB,
the following lines create a binary ROSE AST:


{\mySmallFontSize
\begin{verbatim}
  // create RoseBin object with DB specific information 
  RoseBin* roseBin = new RoseBin(def_host_name,
                                def_user_name,
                                def_password,
                                def_db_name);

  RoseBin_Arch::arch=RoseBin_Arch::bit32;
  RoseBin_OS::os_sys=RoseBin_OS::linux_op;
  RoseBin_OS_VER::os_ver=RoseBin_OS_VER::linux_26;
  RoseBin_Def::RoseAssemblyLanguage=x86;
  // connect to the DB
  roseBin->connect_DB(socket);
  // query the DB to retrieve all data
  SgAsmNode* globalBlock = roseBin->retrieve_DB_IDAPRO();
  // close the DB
  roseBin->close_DB();

  // traverse the AST and test it
  roseBin->test();
\end{verbatim}
}




\section{The AST}

The binary AST can now be unparsed to represent assembly instructions, cf. Figure ~\ref{Tutorial:examplesourcecode2}:

{\mySmallFontSize
\begin{verbatim}
  SgAsmFile* file = project->get_file(0).get_binaryFile();
  RoseBin_unparse* unparser = new RoseBin_unparse();
  unparser->init(file->get_global_block(), fileName);
  unparser->unparse();
\end{verbatim}
}


\begin{figure}[!h]
{\indent
{\mySmallFontSize


% Do this when processing latex to generate non-html (not using latex2html)
\begin{latexonly}
 % \lstinputlisting{\TutorialExampleBuildDirectory/rose_bin_example.s}
   \lstinputlisting{\TutorialExampleBuildDirectory/rose_bin_example.s.aa}
\end{latexonly}

% Do this when processing latex to build html (using latex2html)
\begin{htmlonly}
 % \verbatiminput{\TutorialExampleBuildDirectory/rose_bin_example.s}
   \verbatiminput{\TutorialExampleBuildDirectory/rose_bin_example.s.aa}
\end{htmlonly}

% end of scope in font size
}
% End of scope in indentation
}
\caption{Assembly code.}
\label{Tutorial:examplesourcecode2}
\end{figure}


To visualize the binary AST, we can use the following lines to
write the AST out to a .dot format:

{\mySmallFontSize
\begin{verbatim}
  string filename=''_binary_tree.dot'';
  AST_BIN_Traversal* trav = new AST_BIN_Traversal();
  trav->run(file->get_global_block(), filename);
\end{verbatim}
}


\section{The ControlFlowGraph}

\begin{figure}
%\centerline{\epsfig{file=\TutorialExampleBuildDirectory/controlFlowGraph.ps,
%                    height=1.3\linewidth,width=1.0\linewidth,angle=0}}
\includegraphics[scale=0.7]{\TutorialExampleBuildDirectory/binaryAnalysis_pic1}
\caption{Controlflow graph for example program.}
\label{Tutorial:examplefig1}
\end{figure}

Based on a control flow traversal of the binary AST, a separate control flow graph
is created that can be used for further analyses:


{\mySmallFontSize
\begin{verbatim}
  // forward or backward analysis ?
  bool forward = true;
  // when creating a visual representation, visualized edges?
  bool edges = true;
  // visualize multiple edges or merge edges between same nodes to one edge?
  bool mergedEdges = false;
  // create DotGraph Object or/and GmlGraph Object
  RoseBin_DotGraph* dotGraph = new RoseBin_DotGraph();
  RoseBin_GMLGraph* gmlGraph = new RoseBin_GMLGraph();
  char* cfgFileNameDot = ``cfg.dot'';
  char* cfgFileNameGml = ``gml.dot'';
  RoseBin_ControlFlowAnalysis* cfganalysis = new RoseBin_ControlFlowAnalysis(
                                             file->get_global_block(), forward, NULL, edges);
  cfganalysis->run(dotGraph, cfgFileNameDot, mergedEdges);
  cfganalysis->run(gmlGraph, cfgFileNameGml, mergedEdges);
\end{verbatim}
}

The control flow graph of our example is represented in Figure ~\ref{Tutorial:examplefig1}.
The graph shows different functions represented purple, containing various instructions.
Instructions within a function are represented within the same box. Common instructions are
represented yellow. Green instructions are jumps and pink instructions calls and returns.  
Respectively, blue edges are call relationships and red edges return relationships.
Black edges represent plain controlflow from one instruction to the next.



\section{DataFlow Analysis}

\begin{figure}
%\centerline{\epsfig{file=\TutorialExampleBuildDirectory/controlFlowGraph.ps,
%                    height=1.3\linewidth,width=1.0\linewidth,angle=0}}
\includegraphics[scale=0.7]{\TutorialExampleBuildDirectory/binaryAnalysis_pic2}
\caption{Dataflowflow graph for example program.}
\label{Tutorial:examplefig2}
\end{figure}

Based on the control flow many forms of dataflow analysis may be performed.
The code to perform the dataflow analysis looks as follows:

{\mySmallFontSize
\begin{verbatim}
  string dfgFileName = "dfg.dot";
  forward = true;
  bool printEdges = true;
  // choose between a interprocedural and intraprocedural dataflow analysis
  bool interprocedural = true;
  RoseBin_DataFlowAnalysis* dfanalysis = new RoseBin_DataFlowAnalysis(
                                         file->get_global_block(), forward, NULL);
  dfanalysis->init(interprocedural, printEdges);
  dfanalysis->run(dotGraph, dfgFileName, mergedEdges);
\end{verbatim}
}

Dataflow analyses available are:
 
\subsection{Def-Use Analysis}

Definition-Usage is one way to compute dataflow information about a binary program.
Figure ~\ref{Tutorial:examplefig2} shows a typical dataflow graph with additional 
definition and usage information visualized at the edges.

\subsection{Variable Analysis}

This analysis helps to detect different types within a binary.
Currently, we use this analysis to detect interrupt calls and their parameters together with the def-use analysis.

This allows us to track back the value of parameters to the calls, such as eax and therefore 
determine whether a interrupt call is for instance a write or read.

Another feature is the buffer overflow analysis. By traversing the CFG, we can detect buffer overflows.
The black node in ~\ref{Tutorial:examplefig2} shows such a buffer overlow, cf. source code above.








