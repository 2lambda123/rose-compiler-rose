\chapter{Building custom graphs}
\label{Tutorial:chapterCustomGraphs}

\paragraph{What To Learn From This Example}
This example shows how to generate custom graphs (typically used for analysis results).

     The mechanisms used internally to build different graphs of program data
are also made externally available. This section shows how new graphs of 
program information can be built or existing graphs customized.
More information about generation of specialized AST graphs to support debugging 
can be found in \ref{Tutorial:chapterGeneralASTGraphGeneration}.

Figure~\ref{tutorial:exampleCustomGraph} shows a graph representing the
highest levels of the ROSE directory tree built using the code shown
in figure~\ref{Tutorial:exampleCustomGraphCode}.  This example is build
explicitly in the example code, the script {\em lsdot} in the 
{\em ROSE/scripts/} directory builds such a graph of the directory tree 
automatically for any directory structure and is used to present the ROSE 
source directory tree in the {\em ROSE User Manual}.

% \fixme{Add examples from the ROSE/projects/HighLevelGrammars subproject.}

\begin{figure}[!h]
% Use the includegraphics command from the graphicx package with pdflatex.
%\centerline{\epsfig{file=\TutorialExampleBuildDirectory/customGraph.ps,
%                    height=0.3\linewidth,width=1.0\linewidth,angle=0}}
\includegraphics[scale=0.4]{\TutorialExampleBuildDirectory/customGraph}
\caption{Graph of top level of ROSE directory  tree.}
\label{tutorial:exampleCustomGraph}
\end{figure}


% {\bf {\em This tutorial example is not finished yet! }}

\begin{figure}[!h]
{\indent
{\mySmallFontSize

% Do this when processing latex to generate non-html (not using latex2html)
\begin{latexonly}
   \lstinputlisting{\TutorialExampleDirectory/customGraphGeneration.C}
\end{latexonly}

% Do this when processing latex to build html (using latex2html)
\begin{htmlonly}
   \verbatiminput{\TutorialExampleDirectory/customGraphGeneration.C}
\end{htmlonly}

% end of scope in font size
}
% End of scope in indentation
}
\caption{Example source code to read an input program and generate an AST graph.}
\label{Tutorial:exampleCustomGraphCode}
\end{figure}


Figure~\ref{tutorial:exampleCustomGraphWithFiltering} shows the same graph
but with filtering to tailor the graph by removing nodes.  The removal of any
node from the graph automatically removes all edged pointing to that node removed
and all edges pointing away from the node being removed.  This mechanism is
provided as a way to customized graphs automatically generated using ROSE
(e.g. call graphs, control flow graphs, etc.).

\begin{figure}[!h]
% Use the includegraphics command from the graphicx package with pdflatex.
%\centerline{\epsfig{file=\TutorialExampleBuildDirectory/customGraphWithFiltering.ps,
%                    height=0.3\linewidth,width=1.0\linewidth,angle=0}}
% \includegraphics[trim=0.0in 0in 0in 0in,scale=0.5]{\TutorialExampleBuildDirectory/customGraphWithFiltering.png}
% \epsfig{file=\TutorialExampleBuildDirectory/customGraphWithFiltering.eps}
% \epsfig{\TutorialExampleBuildDirectory/customGraphWithFiltering.eps}
%\includegraphics{\TutorialExampleBuildDirectory/customGraphWithFiltering}
\includegraphics[trim=0.0in 0.0in 0in 0in,scale=0.4]{\TutorialExampleBuildDirectory/customGraphWithFiltering}
\caption{Graph of top level of ROSE directory tree with filtering of subtree.}
\label{tutorial:exampleCustomGraphWithFiltering}
\end{figure}

