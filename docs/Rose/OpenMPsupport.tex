\chapter{OpenMP Support}
\label{chap::ompsupport}

%-------------------------------------------------------------
%-------------------------------------------------------------
\section{Introduction}
ROSE supports the OpenMP 3.0 specifications~\cite{OpenMP3.0}. 
OpenMP is a popular shared-memory parallel programming model which extends serial programming languages like C/C++ and Fortran 77/90 to include additional parallel semantics. 
The extensions OpenMP provides contains compiler directives, user level runtime routines and environment variables. 
Depending on the languages (C/C++ or Fortran), the OpenMP support in ROSE includes parsing OpenMP directive, generating dedicated AST nodes for them, and finally translating OpenMP program into multithreaded programs targeting the GCC GOMP OpenMP runtime library. 
We also have an implementation of automatic parallelization using OpenMP, which automatically introduces OpenMP directive into sequential C/C++ applications.

%-------------------------------------------------------------
%-------------------------------------------------------------
\section{Command Line Options}
Like most other OpenMP implementations, ROSE's OpenMP support has to be explicitly turned on via command line options.
By running any ROSE-based translator (e.g. identityTranslator) with \textit{-help}, 
the OpenMP-related command line options can be displayed as follows: 
{\scriptsize
\begin{verbatim}
  -rose:OpenMP, -rose:openmp
                          follow OpenMP 3.0 specification for C/C++ and Fortran, perform one of the following actions:
  -rose:OpenMP:parse_only, -rose:openmp:parse_only
                          parse OpenMP directives to OmpAttributes, no further actions (default behavior now)
  -rose:OpenMP:ast_only, -rose:openmp:ast_only
                          on top of -rose:openmp:parse_only, build OpenMP AST nodes from OmpAttributes, no further actions
  -rose:OpenMP:lowering, -rose:openmp:lowering
                          on top of -rose:openmp:ast_only, transform AST with OpenMP nodes into multithreaded code 
                          targeting GCC GOMP runtime library
\end{verbatim}
}
%-------------------------------------------------------------
%-------------------------------------------------------------
\section{Entry Point and Top Level Function}
ROSE processes OpenMP right after preprocessing information is processed within \lstinline{SgFile::callFrontEnd()}.
This order is necessary since preprocessing information may existing within OpenMP directives, such as macro calls or named constants defined using \lstinline{#define}.
ROSE needs to be aware of such preprocessing information when processing OpenMP.

The top level function for processing OpenMP is named \lstinline{processOpenMP()}, as shown below:
\lstset{language=C,basicstyle=\scriptsize}
\begin{lstlisting}
void processOpenMP(SgSourceFile *sageFilePtr)
{
  ROSE_ASSERT(sageFilePtr != NULL);
  // skip if no OpenMP directives
  if (sageFilePtr->get_openmp() == false)
    return;

  // parse OpenMP directives and attach OmpAttributeList to relevant SgNode
  attachOmpAttributeInfo(sageFilePtr);

  // stop here if only OpenMP parsing is requested
  if (sageFilePtr->get_openmp_parse_only())
    return;

  //Build OpenMP AST nodes based on parsing results
  build_OpenMP_AST(sageFilePtr);

  // Stop here if only OpenMP AST construction is requested
  if (sageFilePtr->get_openmp_ast_only())
    return;

  // Translate to multithreaded code targeting GOMP
  lower_omp(sageFilePtr);
}
\end{lstlisting}
\lstset{language=C,basicstyle=\small}

%-------------------------------------------------------------
%-------------------------------------------------------------
\section{Parsing OpenMP Directives}
Since the EDG C/C++ frontend (as of version 4.0 and earlier versions) does not support OpenMP, 
ROSE has its own OpenMP directive parsers. 
The parsers recognize all OpenMP directives as defined in OpenMP 3.0.
Persistent AST attributes (\emph{OmpAttribute} as defined in \textit{src/frontend/SageIII/OmpAttribute.h}) are generated as the results of parsing and attached to relevant AST nodes. 
The attributes serve as a light-weight representation for OpenMP directives since they only incur minimum change to existing ROSE AST.
% and can be handy in some special occasions. 

The source files of the OpenMP directive parser for C/C++ are located in \textit{rose/src/frontend/SageIII/ }.
They include \textit{omplexer.ll} (for Lex) and \textit{ompparser.yy} (for Yacc).
The Fortran OpenMP parser is hand crafted and has only one source file \textit{src/frontend/SageIII/ompFortranParser.C}. 

A rich set of C/C++/Fortran OpenMP tests is located in \textit{src/tests/CompileTests/OpenMP\_tests} to test the ROSE OpenMP parsers.
%-------------------------------------------------------------
%-------------------------------------------------------------
\section{Generating AST with OpenMP Nodes}
This phase converts the ROSE AST attached with OmpAttribute instances into AST with OpenMP-specific nodes.
The introduction of OpenMP-specific ROSE AST nodes can help reuse all existing ROSE AST traversal, query, and other manipulation interfaces, thus facilitating analysis and translation of OpenMP program.

All OpenMP AST nodes have names starting with \textit{SgOmp}. 
Directives are treated as statement-like nodes, which in turn can have a list of clause nodes.
A list of some example OpenMP constructs and their corresponding ROSE AST nodes are given below:

{\scriptsize
\begin{verbatim}
omp atomic              SgOmpAtomicStatement
omp barrier             SgOmpBarrierStatement
omp critical            SgOmpCriticalStatement
omp parallel            SgOmpParallelStatement
omp for                 SgOmpForStatement
omp task                SgOmpTaskStatement
omp sections            SgOmpSectionsStatement
omp flush               SgOmpFlushStatement
omp taskwait            SgOmpTaskwaitStatement
omp threadprivate       SgOmpThreadprivateStatement

reduction               SgOmpReductionClause
schedule                SgOmpScheduleClause
private                 SgOmpPrivateClause
firstprivate            SgOmpFirstprivateClause
lastprivate             SgOmpLastprivateClause
nowait                  SgOmpNowaitClause
copyin                  SgOmpCopyinClause
collapse                SgOmpCollapseClause
untied                  SgOmpUntiedClause
ordered                 SgOmpOrderedClause
\end{verbatim}
}
 
Please refer to the ROSE Web Reference\footnote{\url{http://www.rosecompiler.org/ROSE_HTML_Reference/index.html}} for details about these nodes and their class hierarchy. 
%-------------------------------------------------------------
%-------------------------------------------------------------
\section{Translating OpenMP Directives}
Using a command line like \textit{identityTranslator -rose:openmp:lower inputcode.c}),
ROSE can translate OpenMP 3.0 programs into multithreaded code targeting the GCC GOMP OpenMP runtime library.
If the path to GOMP (preferably the one shipped with GCC 4.3 for OpenMP task support) is specified (using a configure option \textit{--with-gomp\_omp\_runtime\_library=/home/liao6/opt/gcc-svn/lib/}),
the generated multithreaded code can be automatically linked to the GOMP library to generate final executables after compilation. 

The major source file of the OpenMP translation is \textit{src/midend/ompLowering/omp\_lowering.cpp}. 
Basically, it applies the following algorithm to each input source file using OpenMP:
\begin{enumerate}
\item  Use a top-down AST traversal to make implicit data-sharing attribute explicit, including implicit private variables for loop constructs and implicit firstprivate variables for task constructs. 
\item  Uses a bottom-up AST traversal to locate OpenMP nodes and performance necessary translation for each type of them.
\begin{enumerate}
\item Handle OpenMP-specific variables, such as private, firstprivate, lastprivate and reduction variables used by the target construct, if any.
\item For parallel (\lstinline{omp parallel}) and task (\lstinline{omp task}) constructs, call the general-purpose source-to-source AST outliner~\cite{LiaoEffective2009} to generate tasks and replace the original code block with GOMP runtime calls.
\item For loop constructs, normalize target loops and generate code to calculate iteration chunks for each thread
\item Translation for other constructs are omitted since they are relatively simpler cases. 
\end{enumerate}
\end{enumerate}

\subsection{Variable Handling}
The separation of OpenMP variable handling from the rest of translation is key to the successful reuse of a general-purpose outliner within an OpenMP 3.0 implementation. 
After OpenMP variable handling, a structured code block of a parallel or task construct becomes much easier to be handled by the outliner.

Variable handling is implemented in \lstinline{OmpSupport::transOmpVariables()}. 
It translates all OpenMP clauses with variable lists, such as private, firstprivate, lastprivate, reduction, etc.
Assume \lstinline{bb} is a structured block affected by the variable clauses, the algorithm for handling OpenMP variables is given below:
\begin{enumerate}
\item Collect all variables used in clauses with variable lists
\item For each variable, do the following:
\begin{enumerate}
\item Prepend a local declaration statement for the variable to the beginning of \lstinline{bb}.
\item Insert an assignment statement after the declaration statement to initialize the local copy (e.g. for firstprivate and reduction variables).
\item Replace all references to the variable within \lstinline{bb} with references to its local copy.
\item Append an assignment statement to save the value of the local copy to its global counter part (e.g. for reduction and lastprivate variables)
\end{enumerate}
\end{enumerate}
Please note that a variable can be associated with more than one clauses, such as firstprivate and lastprivate. 


%-------------------------------------------------------------
\subsection{Parallel Regions}

Translation of a simple OpenMP parallel region \lstinline{#pragma omp parallel} without variable uses is demonstrated in
Figure~\ref{Manual:omp:hello1-trans} for an input code shown in
Figure~\ref{Manual:omp:hello1}.

\lstset{language=C,basicstyle=\scriptsize}
\begin{figure}[htbp]
{\indent
  {\mySmallFontSize
    \begin{latexonly}
    \lstinputlisting{\TopSourceDirectory/tests/CompileTests/OpenMP_tests/hello-1.c}
    \end{latexonly}
    \begin{htmlonly}
    \verbatiminput{\TopSourceDirectory/tests/CompileTests/OpenMP_tests/hello-1.c}
    \end{htmlonly}
  }
}
\caption{Example of a simple parallel region}
\label{Manual:omp:hello1}
\end{figure}

\begin{figure}[htbp]
{\indent
  {\mySmallFontSize
    \begin{latexonly}
    \lstinputlisting{\TopBuildDirectory/tests/roseTests/ompLoweringTests/rose_hello-1.c}
    \end{latexonly}
    \begin{htmlonly}
    \verbatiminput{\TopBuildDirectory/tests/roseTests/ompLoweringTests/rose_hello-1.c}
    \end{htmlonly}
  }
}
\caption{Translation of a simple parallel region}
\label{Manual:omp:hello1-trans}
\end{figure}

Translation of a complex OpenMP parallel region with variable references is demonstrated in
Figure~\ref{Manual:omp:preduction-trans} for an input code shown in
Figure~\ref{Manual:omp:preduction}. Note the handling of shared, and reduction variables during the translation.

\lstset{language=C,basicstyle=\scriptsize}
\begin{figure}[htbp]
{\indent
  {\mySmallFontSize
    \begin{latexonly}
    \lstinputlisting{\TopSourceDirectory/tests/CompileTests/OpenMP_tests/parallel-reduction.c}
    \end{latexonly}
    \begin{htmlonly}
    \verbatiminput{\TopSourceDirectory/tests/CompileTests/OpenMP_tests/parallel-reduction.c}
    \end{htmlonly}
  }
}
\caption{Example of a complex parallel region}
\label{Manual:omp:preduction}
\end{figure}

\begin{figure}[htbp]
{\indent
  {\mySmallFontSize
    \begin{latexonly}
    \lstinputlisting{\TopBuildDirectory/tests/roseTests/ompLoweringTests/rose_parallel-reduction.c}
    \end{latexonly}
    \begin{htmlonly}
    \verbatiminput{\TopBuildDirectory/tests/roseTests/ompLoweringTests/rose_parallel-reduction.c}
    \end{htmlonly}
  }
}
\caption{Translation of a complex parallel region}
\label{Manual:omp:preduction-trans}
\end{figure}

%-------------------------------------------------------------
\clearpage
\subsection{Loop Constructs}
Translation of a loop construct is given in
Figure~\ref{Manual:omp:ompfor-trans} for an input code shown in
Figure~\ref{Manual:omp:ompfor}. 
Note that GOMP does not provide a runtime function to calculate iteration chunks for the default schedule policy. 
Compilers have to generate code to calculate it instead. 
Constant folding is used to simplify some expressions with constant values. 

\lstset{language=C,basicstyle=\scriptsize}
\begin{figure}[htbp]
{\indent
  {\mySmallFontSize
    \begin{latexonly}
    \lstinputlisting{\TopSourceDirectory/tests/CompileTests/OpenMP_tests/ompfor.c}
    \end{latexonly}
    \begin{htmlonly}
    \verbatiminput{\TopSourceDirectory/tests/CompileTests/OpenMP_tests/ompfor.c}
    \end{htmlonly}
  }
}
\caption{Example of a loop construct}
\label{Manual:omp:ompfor}
\end{figure}

\begin{figure}[htbp]
{\indent
  {\mySmallFontSize
    \begin{latexonly}
    \lstinputlisting{\TopBuildDirectory/tests/roseTests/ompLoweringTests/rose_ompfor.c}
    \end{latexonly}
    \begin{htmlonly}
    \verbatiminput{\TopBuildDirectory/tests/roseTests/ompLoweringTests/rose_ompfor.c}
    \end{htmlonly}
  }
}
\caption{Translation of a loop construct}
\label{Manual:omp:ompfor-trans}
\end{figure}

Calculating iteration chunks for a loop with a decremental iteration space is shown in 
Figure~\ref{Manual:omp:ompfor5-trans} for an input code given in
Figure~\ref{Manual:omp:ompfor5}. 

\lstset{language=C,basicstyle=\scriptsize}
\begin{figure}[htbp]
{\indent
  {\mySmallFontSize
    \begin{latexonly}
    \lstinputlisting{\TopSourceDirectory/tests/CompileTests/OpenMP_tests/ompfor5.c}
    \end{latexonly}
    \begin{htmlonly}
    \verbatiminput{\TopSourceDirectory/tests/CompileTests/OpenMP_tests/ompfor5.c}
    \end{htmlonly}
  }
}
\caption{Example of an OpenMP loop with a decremental iteration space}
\label{Manual:omp:ompfor5}
\end{figure}

\begin{figure}[htbp]
{\indent
  {\mySmallFontSize
    \begin{latexonly}
    \lstinputlisting{\TopBuildDirectory/tests/roseTests/ompLoweringTests/rose_ompfor5.c}
    \end{latexonly}
    \begin{htmlonly}
    \verbatiminput{\TopBuildDirectory/tests/roseTests/ompLoweringTests/rose_ompfor5.c}
    \end{htmlonly}
  }
}
\caption{Translation of the loop with a decremental iteration space}
\label{Manual:omp:ompfor5-trans}
\end{figure}

GOMP provides loop scheduling runtime functions for loop constructs with known chunk sizes or with an ordered clause. 
Figure~\ref{Manual:omp:ompfor4-trans} show the translation of a dynamic scheduled loop,  for an input code given in
Figure~\ref{Manual:omp:ompfor4}. 

\lstset{language=C,basicstyle=\scriptsize}
\begin{figure}[htbp]
{\indent
  {\mySmallFontSize
    \begin{latexonly}
    \lstinputlisting{\TopSourceDirectory/tests/CompileTests/OpenMP_tests/ompfor4.c}
    \end{latexonly}
    \begin{htmlonly}
    \verbatiminput{\TopSourceDirectory/tests/CompileTests/OpenMP_tests/ompfor4.c}
    \end{htmlonly}
  }
}
\caption{Example of an OpenMP loop with a dynamic schedule}
\label{Manual:omp:ompfor4}
\end{figure}

\begin{figure}[htbp]
{\indent
  {\mySmallFontSize
    \begin{latexonly}
    \lstinputlisting{\TopBuildDirectory/tests/roseTests/ompLoweringTests/rose_ompfor4.c}
    \end{latexonly}
    \begin{htmlonly}
    \verbatiminput{\TopBuildDirectory/tests/roseTests/ompLoweringTests/rose_ompfor4.c}
    \end{htmlonly}
  }
}
\caption{Translation of the loop with a dynamic schedule}
\label{Manual:omp:ompfor4-trans}
\end{figure}

%-------------------------------------------------------------
\clearpage
\subsection{Threadprivate}
GCC uses thread local storage (TLS) to implement OpenMP threadprivate variables. 
No special support is needed from the runtime library's point of view. 
The translation is very simple: add the keyword \lstinline{__thread} in front of the original declaration for a variable declared as threadprivate and then remove the OpenMP pragma. 

Figure~\ref{Manual:omp:threadprivate-trans} shows the translation result for a test input code (Figure~\ref{Manual:omp:threadprivate}). It also demonstrates the handling of loop constructs using the ordered clause.

\lstset{language=C,basicstyle=\scriptsize}
\begin{figure}[htbp]
{\indent
  {\mySmallFontSize
    \begin{latexonly}
    \lstinputlisting{\TopSourceDirectory/tests/CompileTests/OpenMP_tests/threadprivate.c}
    \end{latexonly}
    \begin{htmlonly}
    \verbatiminput{\TopSourceDirectory/tests/CompileTests/OpenMP_tests/threadprivate.c}
    \end{htmlonly}
  }
}
\caption{Example using threadprivate}
\label{Manual:omp:threadprivate}
\end{figure}

\begin{figure}[htbp]
{\indent
  {\mySmallFontSize
    \begin{latexonly}
    \lstinputlisting{\TopBuildDirectory/tests/roseTests/ompLoweringTests/rose_threadprivate.c}
    \end{latexonly}
    \begin{htmlonly}
    \verbatiminput{\TopBuildDirectory/tests/roseTests/ompLoweringTests/rose_threadprivate.c}
    \end{htmlonly}
  }
}
\caption{Translation of threadprivate}
\label{Manual:omp:threadprivate-trans}
\end{figure}


%-------------------------------------------------------------
\clearpage
\subsection{Task Constructs}
The translation of task constructs are similar to the translation of parallel constructs. They share the same AST outliner to generate outlined functions for tasks.

Figure~\ref{Manual:omp:task_untied3-trans} shows the translation of untied task constructs(input code given in
Figure~\ref{Manual:omp:task_untied3}). 

\lstset{language=C,basicstyle=\scriptsize}
\begin{figure}[htbp]
{\indent
  {\mySmallFontSize
    \begin{latexonly}
    \lstinputlisting{\TopSourceDirectory/tests/CompileTests/OpenMP_tests/task_untied3.c}
    \end{latexonly}
    \begin{htmlonly}
    \verbatiminput{\TopSourceDirectory/tests/CompileTests/OpenMP_tests/task_untied3.c}
    \end{htmlonly}
  }
}
\caption{Example of untied tasks}
\label{Manual:omp:task_untied3}
\end{figure}

\begin{figure}[htbp]
{\indent
  {\mySmallFontSize
    \begin{latexonly}
    \lstinputlisting{\TopBuildDirectory/tests/roseTests/ompLoweringTests/rose_task_untied3.c}
    \end{latexonly}
    \begin{htmlonly}
    \verbatiminput{\TopBuildDirectory/tests/roseTests/ompLoweringTests/rose_task_untied3.c}
    \end{htmlonly}
  }
}
\caption{Translation of the untied tasks}
\label{Manual:omp:task_untied3-trans}
\end{figure}

Figure~\ref{Manual:omp:task_wait-trans} shows the translation of task constructs used with taskwait( an input code given in
Figure~\ref{Manual:omp:task_wait}). 

\lstset{language=C,basicstyle=\scriptsize}
\begin{figure}[htbp]
{\indent
  {\mySmallFontSize
    \begin{latexonly}
    \lstinputlisting{\TopSourceDirectory/tests/CompileTests/OpenMP_tests/task_wait.c}
    \end{latexonly}
    \begin{htmlonly}
    \verbatiminput{\TopSourceDirectory/tests/CompileTests/OpenMP_tests/task_wait.c}
    \end{htmlonly}
  }
}
\caption{Example of tasks with taskwait}
\label{Manual:omp:taskwait}
\end{figure}

\begin{figure}[htbp]
{\indent
  {\mySmallFontSize
    \begin{latexonly}
    \lstinputlisting{\TopBuildDirectory/tests/roseTests/ompLoweringTests/rose_task_wait.c}
    \end{latexonly}
    \begin{htmlonly}
    \verbatiminput{\TopBuildDirectory/tests/roseTests/ompLoweringTests/rose_task_wait.c}
    \end{htmlonly}
  }
}
\caption{Translation of the taskwait example}
\label{Manual:omp:task_wait-trans}
\end{figure}

%-------------------------------------------------------------
%-------------------------------------------------------------
\clearpage
\section{Automatic Parallelization}
ROSE has an implementation of automatic parallelization using OpenMP.
The implementation is also used to explore semantics-aware automatic parallelization, as
described in one of our paper~\cite{LiaoExtending2009}.
\fixme{write more here}
%-------------------------------------------------------------
%-------------------------------------------------------------
%\clearpage
%\section{Appendix}
%\subsection{Old OpenMP 2.0 Implementation}
%
%\subsection{Investigating GCC's OpenMP translation}
