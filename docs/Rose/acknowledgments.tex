
% \chapter*{Acknowledgments}
\chapter*{Acknowledgments}

% \begin{center}
% ********  To be removed later  *************  \newline
% \end{center}
% 
%  Purpose:
% \begin{itemize}
%    \item A. Thank everyone
% \end{itemize}
% \begin{center}
% *********************  \newline
% \end{center}
% \vspace{0.25in}
% 

   The Intermediate Representation (IR) used in ROSE is called SAGE III.
SAGE III is something that we have built based on SAGE II, which was never 
completed or widely distributed. SAGE II was based on SAGE++, the improvements 
over SAGE++ were significant. SAGE II was the first version of SAGE to use the 
Edison Design Group (EDG) frontend.  We want to thank the original
developers of SAGE++ and SAGE II for their work, from which we learned a lot
through use of their object-oriented IR interface.

   We chose the name SAGE III to give sufficient credit to the original developers
of SAGE++ and SAGE II, who also suggested that we call what we are doing SAGE III.  
ROSE, of course, builds on SAGE III and adds numerous additional mechanisms, 
including:
\begin{itemize}
    \item Loop Optimizations (called by ROSE users)
    \item Abstrat Syntax Tree (AST) Attributes (tree decoration)
    \item A family of AST Traversals (as used with Attribute Grammars)
    \item AST Rewrite mechanisms
    \item AST Query Mechanisms
    \item C and C++ code generation from SAGE III
    \item AST File I\/O
    \item Significant robustness for C, C99, and C++ (handles large DOE applications)
    \item AST Visualization
    \item and more ...
\end{itemize}

   SAGE III is an automatically generated body of software that uses ROSETTA, a tool we 
wrote and distribute with ROSE. ROSETTA is an IR generator that, as its largest and most
sophisticated test, generates SAGE III.  The connection code that was used to translate 
EDG's AST to SAGE II was derived loosely from the EDG C++ source generator and has
formed the basis of the SAGE III translator from EDG to SAGE III's IR. Under this license
we exclude the EDG source code and the translation from the EDG AST in distributions 
and make available only a binary of those parts with use EDG (front-end AST translation),
and the source to all of ROSE (which does not depend on EDG). No part of the EDG work is 
visible to the user of ROSE.  We can make the EDG source available only to those who have 
the free EDG research license.  We want to thank the developers at Edison Design Group
(EDG) for making their work so widely available under their research license program.

   Markus Schordan was the first post-doctorate researcher on the ROSE project; he made 
significant contributions while employed at Lawrence Livermore National Laboratory (LLNL), 
including the AST traversal mechanism.  We continue to work with Markus, who is now at 
Vienna University of Technology as an Associate Professor. We were also fortunate to 
leverage a significant portion of Qing Yi's thesis work (under Ken Kennedy) 
% on loop optimizations as part of her post-doc at Lawrence Livermore National Laboratory with the ROSE project, 
and we would like to thank her for that work and the work she did as a post-doc at Lawrence Livermore National Laboratory. 
We continue to work with her, although she is now at the University of Texas at San Antonio.

     There are many additioanal people to thank for our current status in 
% where we are currently within 
the ROSE project:
\begin{itemize}
     \item Contributing Collaborators: \\
           Markus Schordan (Vienna University of Technology),
           Rich Vuduc (Georgia Tech), and
           Qing Yi (University of Texas at San Antonio)
     \item Post-docs (including former post-docs): \\
           Chunhua Liao (from University of Houston),
           Thomas Panas (from Vaxjo University, Sweden),
           Markus Schordan (from University of Klagenfurt, Austria),
           Rich Vuduc (from University of California at Berkeley), and
           Jeremiah Willcock (from Indiana University),
           Qing Yi (from Rice University)
     \item Students: \\
           Gergo Barany (Technical University of Vienna),
           Michael Byrd (University of California at Davis),
           Gabriel Coutinho (Imperial College London),
           Peter Collingbough (Imperial College London),
           Valentin David (University of Bergen, Norway),
           Jochen Haerdtlein (University of Erlanger, Germany),
           Vera Hauge (University of Oslo, Norway),
           Christian Iwainsky (University of Erlanger, Germany),
           Lingxiao Jiang (University of California at Davis),
           Alin Jula (Texas A\&M),
           Han Kim (University of California at San Diago),
           Milind Kulkarni (Cornell University),
           Markus Kowarschik (University of Erlanger, Germany),
           Gary Lee (University of California at Berkeley and Purdue University),
           Chunhua Liao (University of Houston),
           Ghassan Misherghi. (University of California at Davis),
           Peter Pirkelbauer (Texas A\&M),
           Bobby Philip (University of Colorado),
           Radu Popovici (Cornell University),
           Robert Preissl (xxx Austria),
           Andreas Saebjornsen (University of Oslo, Norway),
           Sunjeev Sikand (University of California at San Diago),
           Andy Stone (Colorado State University at Fort Collins),
           Danny Thorne (University of Kentucky), 
           Nils Thuerey (University of Erlanger, Germany), 
           Ramakrishna Upadrasta (Colorado State University at Fort Collins),
           Christian Wiess(Munich University of Technology, Germany), 
           Jeremiah Willcock (Indiana University),
           Brian White (Cornell University),
           Gary Yuan (University of California at Davis), and
           Yuan Zhao (Rice University).
     \item Friendly Users: \\
           Paul Hovland (Argonne National Laboratory),
           Brian McCandless (Lawrence Livermore National Laboratory),
           Brian Miller (Lawrence Livermore National Laboratory),
           Boyana Norris (Argonne National Laboratory),
           Jacob Sorensen (University of California at San Diago),
           Michelle Strout (Colorado State University),
           Bronis de Supinski (Lawrence Livermore National Laboratory),
           Chadd Williams (University of Maryland),
           Beata Winnicka (Argonne National Laboratory),
           Ramakrisna xxx (Colorado State University at Fort Collins), and
           Andy Yoo (Lawrence Livermore National Laboratory)
     \item Support: \\
           Steve Ashby,
           David Brown,
           Bill Henshaw,
           Bronis de Supinski, and
           CASC management
     \item Funding: \\
           Fred Johnson (Department of Energy, DOE) and
           Mary Zosel (Lawrence Livermore National Laboratory)
\end{itemize}

\fixme{Check spelling of student names.}

To be clear, nobody is to blame for the poor state of the current version of the 
ROSE documentation (but myself).

% At some point we should add Michelle
% Michelle Strout (Colorado State University)
% and perhaps Jacob
% Jacob Sorenson (UCSD),

% including Kei, Fede, Bobby, Bill, and David. And nobody to blame for the 
% poor state of the current version of the ROSE documentation (but myself).



