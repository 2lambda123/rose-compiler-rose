\chapter{Traversals (part 1)}

  This chapter is about getting to use the AST traversals.
First with the simple traversals (equivalent to the visitor
pattern) and then with more useful traversals.

% *******************************************************
%                  Source Code Analysis
% *******************************************************
\section{Source Code Analysis}
   This subsection is resticted to analysis of the input code
and generating information about what is in the input source code.

\subsection{Function Detection}
   Write the ROSE based analysis tool to detect functions in the input source code
and report the location in the file and the number of lines of code in each function.
Use a traversal to do this exercise. See the tutorial for example code if required.

% *******************************************************
%               Source Code Transformation
% *******************************************************
\section{Source Code Transformations}
   This subsection deals with transformations on the 
input code and the generation of new code.

\subsection{Fixup Uninitialized Variables}
   Write the code required to use ROSE to define a translator
that takes in any input C/C++/Fortran code and detect uninitialized
variables and defines an appropriate initializer (where possible;
structure initialization could be considered only more difficult).
{\em (Extra points for structure initialization).}
Hint: Use a post-order traversal.




% *******************************************************
%                  Binary Code Analysis
% *******************************************************
\section{Binary Code Analysis}
   This subsection deals with analysis of the binary and the
output of information about the binary.

\subsection{Global Variable References}
   Generate a translator that identifies all references to the data segment.
Report there location and the offset into the data segment.



% *******************************************************
%               Binary Code Transformation
% *******************************************************
\section{Binary Code Transformations}
   This subsection tests your ability to rewrite parts of the binary
AST and regenerate the binary.  Note that depending on the transformation
the resulting binary may or may not be executable.

\subsection{Add a new Section}
   Write a tool to add a new section to the binary (at the end of the
binary) and write your name into the section.



