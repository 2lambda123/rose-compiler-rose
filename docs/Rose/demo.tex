\chapter{ ROSE Demo Introduction }
\label{demo:chapter_start}

\section{Brief Introduction to ROSE}
   ROSE is a framework for building compiler-based tools that can 
automate the analysis and transformation of software (source code or binaries).
For Source code we support: C, C++, and Fortran.  Initial support for PHP
is also in place.  Numerous language versions are supported: C89, C99, C++98,
initial support for C++0x, and for Fortran (F66,F77,F90,F95,F2003).  Also
OpenMP and UPC are supported.  Unlike most compilers, ROSE preserves a faithful 
representation of the original source code to support source-to-source (in the 
same language); so the internal compiler's internal representation (IR) is at a 
high level, but parts of it are shared between different languages (to avoid 
unnecessary redundancy).

   ROSE is part of research funded by DOE to support the optimization of High Performance
Computing (HPC) applications.  As a result ROSE support a number of forms of analysis
and optimization well suited for general program optimizations (loop optimizations and
more general optimizations).

   ROSE also handles binaries, specifically the disassembly, representation, analysis 
and limited transformations.  Binary formats handled in ROSE are: ELF (typically 
Linux) and PE (MS Windows), and also the required alternative MS Windows formats
included for compatibility: NE (older Windows format), LE (OS/2), and MS-DOS (DOS).
Instruction sets supported by the binary research in ROSE are: x86 (Common PC's),
ARM (commercial embedded software; cell phones, printers, etc.),
Power-PC (other embedded software; automobiles, network hardware).
% See Wikipedia for more examples of whare the different processors are used.
% \fixme{Should we list where the PowerPC chips are used?}

   ROSE supports a number of forms of program analysis and attempts to unify how 
analysis can be done on both source code and binaries.

\subsection{Brief Introduction to Compass}
   Compass is an infrastructure built using ROSE but which is specific to static
analysis of source code and binaries via the specification of rules and the detection
of violations of those rules.  It is not meant for optimization of software but as a 
result is simpler to use for the narrower requirements of general static analysis.

\subsection{Requirements}

   To run the demos one requires a rather specific installation of ROSE that
uses Qt, SqLite, zgrviewer, etc.  This subsection references the documentation
that outlines the requirements of ROSE. The details of the requirements are
in the ROSE User Manual (Getting Started : ROSE Installation : section 2.2).

\section{Outline of ROSE Demo Guide}

% This is a guide to running the ROSE demo and a description of what it presents.

% \section{Description of Demo}

This demo is assembled from a collection of separate demos of specific parts of
ROSE (and is tailored to the requirements binary analysis).  Later will will include
more demos as part of a more uniform representation of source code analysis research in
ROSE.

{\em Note: We present many representations of data in the demo, they are present to explain
what is available to users to support automated analysis of software (source code 
and binaries). The point is not that users would pour over these visualizations of data 
as a way to discover details about software.  In many cases the visual representations
presented in this demo are only useful for internal debugging of specific parts of ROSE.
The Compass tool is the only tool 
that we encourage users to try to use directly.}

The ROSE Demo Guide is separated into two significant parts:
\begin{enumerate}
   \item {Chapter 2} \\
      This is a description of a hunt for a specific vulnerability, using ROSE, in binary.
      It is part of an ongoing project specific to this vulnerability and the development
      of a Compass checker to detect it in arbitrary binaries.

   \item {Chapters 3 through 14} \\
      This is a collection of different demonstrations of parts of the ROSE infrastructure
    implemented to support
    general analysis of binaries of different file formats (e.g. Linux and MS Windows) and
    on different processor instruction sets (e.g. x86, Power-PC, and ARM). The
    demonstrations cover a wide range of topics specific to binary analysis using ROSE.
    These demos represent research work over the last two years.

\end{enumerate}

\section{Outline of ROSE Demonstration}

The individual demonstrations are presented in an order to suite the initial review of ROSE:
\begin{enumerate}

 % This is Robb's most wonderful example using of ROSE.
   \item {\bf demo\_1: Detecting Unsafe Signal Handlers in Binaries} \\
      This flaw is a subset of race conditions occurring in signal handler calls which is concerned
   primarily with memory corruption caused by calls to non-reentrant functions in signal handlers.
   Non-reentrant functions are functions that cannot safely be called, interrupted, and then recalled
   before the first call has finished without resulting in memory corruption. The function call syslog()
   is an example of this. In order to perform its functionality, it allocates a small amount of memory as
   "scratch space." If syslog() is suspended by a signal call and the signal handler calls syslog(), the
   memory used by both of these functions enters an undefined, and possibly, exploitable state.
   This test on the binary is similar to a test on the source code present in Compass
   (Asynchronous Signal Handler).

   The details of this demo are represented in chapter~\ref{chapter:asynSafeSignalHandleInBinary}.

   \item {\bf demo\_7: Compass on binary file (text mode)} \\
      Compass can be run on both source code and binaries to detect the violation of
   rules that can be on implemented by users. This demo show the results of running 
   Compass on a binary file and the output in text mode.


   \item {\bf demo\_8: Compass on source code using GUI interface} \\
      There are two tests that demonstrate compass, the first is a source code and
   the second is a binary; (demo\_8a and demo\_8b, respectively).


   \item {\bf demo\_9: Show construction of Compass checker (using gen\_checker.sh)} \\
      This demo shows the construction of Compass checker (using gen\_checker.sh).
   The constructed checker has default behavior, but is a valid check that can be filled in with
   vulnerability specific analysis, using existing analysis in ROSE, etc. The final
   edited files are tar'd up as a directory and submitted to Compass for verification
   (both automatic and manual verification of the checker is required) and then inserted into
   Compass.


   \item {\bf demo\_10: Translation of Binary to Source Code (Power-PC example)} \\
      The demo shows the translation of a binary into a low level representation to support general 
   instruction set independent analysis.  Also demonstrated disassemble and semantic analysis of 
   Power-PC instruction set.  The semantic analysis of binaries is supported for x86 and Power-PC
   (but not yet for ARM).


   \item {\bf demo\_11: Demo of Binary File Compare (Binary File Format)} \\
      This demo evaluates the differences between two binaries generated from the same source
   file, one with debugging information (compiled using {\em -g} option) and the other
   without any debugging information.


   \item {\bf demo\_12: Demo of Binary File Compare (Instructions)} \\
      This demo evaluates the differences between two binaries generated from the same
   source file, one without optimization and the other with optimization (compiled using
   {\em -O3} option) and the other without any debugging information.


   \item {\bf demo\_13: Compass on Large Projects} \\
      This demo shows how a data base is built (before running Compass) and then used with
   Compass to define a project with many files (this example shows the use of only a dozen
   files).


   \item {\bf demo\_13: Static Binary Rewriting} \\
      This demo shows the addition of a named section to an existing binary file.  The to
   support this the section table of names is edited and an additional name added, this
   causes the entire binary to be offset by the length of the name (and it's null
   terminator).  A new section is added to the binary with some random data. All the 
   stored offsets in the binary must be recomputed and reset as part of the final
   regeneration of the binary as and executable with extra data.

% template for more demos
%   \item {\bf xxx} \\

\end{enumerate}



\chapter{demo\_1: Detecting Unsafe Signal Handlers in Binaries}
\label{chapter:asynSafeSignalHandleInBinary}

Reference: Mitre Common Weakness Enumeration (CWE) 479\\
{\bf http://cwe.mitre.org/data/definitions/479.html}

This flaw is a subset of race conditions occurring in signal handler calls which is
concerned primarily with memory corruption caused by calls to non-reentrant functions
in signal handlers. Non-reentrant functions are functions that cannot safely be
called, interrupted, and then recalled before the first call has finished without
resulting in memory corruption. The function call syslog() is an example of this. In
order to perform its functionality, it allocates a small amount of memory as "scratch
space." If syslog() is suspended by a signal call and the signal handler calls
syslog(), the memory used by both of these functions enters an undefined, and
possibly, exploitable state.

We've written an analysis tool that evaluates a binary executable (not source code),
performing the following actions:
\begin{enumerate}
   \item Detects calls to dynamically-linked C library functions that register signal handlers and attempts to determine the entry points of all potential signal handlers.
   \item Builds a function call graph for each potential signal handler.
   \item Finds all potential calls to dynamically-linked functions and determines the names of those functions.
   \item Consults a list of known async-signal-safe standard C functions and reports warnings for calls to functions not on that list.
\end{enumerate}

We deviate from CWE-479 in these ways:
\begin{itemize}
   \item We use a function call graph to detect both direct and indirect calls to unsafe functions.
   \item We use the somewhat more strict �async-signal-safe� rather than �reentrant� condition.
\end{itemize}

We don't yet handle:
\begin{itemize}
   \item Registering signal handlers with sigaction
   \item Calls to signal with a non-constant signal-handler argument.
   \item Both of these cases are relatively common and will require an implementation of constant propagation in ROSE. See notes at the end of this document for more info.
\end{itemize}

\begin{verbatim}
Example code:
 1  /* Signal handler set directly using signal() */
 2  
 3  #include <signal.h>
 4  #include <stdio.h>
 5  #include <stdlib.h>
 6  #include <unistd.h>
 7  
 8  int ncaught;
 9  
10  void
11  bar() {
12      puts("Hello, world.\n");    /* this unsafe call should be reported as foo->bar->puts */
13  }
14  
15  void
16  foo() {
17      bar();
18      _exit(0);                   /* this is a safe call */
19  }
20  
21  void
22  handler(int signo)
23  {
24      ncaught++;
25      foo();
26      foo();                      /* only one set of warnings is emitted */
27      signal(SIGUSR1, handler);   /* re-arm handler, but check handler() once, not twice */
28  }
29  
30  int
31  main() 
32  {
33      signal(SIGUSR1, handler);
34      return 0;
35  }
\end{verbatim}
Listing 1: developersScratchSpace/Robb/DynamicLinking/test1.c

The handler is registered at line 33 and is called handler. The signal handler makes three
calls to dynamically linked standard library functions: signal at line 27, {\tt \_exit} at
line 18, and puts at line 12.  The puts call is of interest because it is not
async-signal-safe. Furthermore, it is somewhat difficult to detect since it's not
called directly by the signal handler.

Disassembly of the non-optimized, non-stripped example code shows a very simple,
straightforward translation to assembly. Function boundaries are easily obtained,
names are available from the symbol table, and the functions and function calls follow
a regular pattern.


\begin{verbatim}
0000000000400590 <bar>:
  400590:       55                      push   %rbp
  400591:       48 89 e5                mov    %rsp,%rbp
  400594:       bf f8 06 40 00          mov    $0x4006f8,%edi
  400599:       e8 0a ff ff ff          callq  4004a8 <puts@plt>
  40059e:       c9                      leaveq 
  40059f:       c3                      retq   
00000000004005a0 <foo>:
  4005a0:       55                      push   %rbp
  4005a1:       48 89 e5                mov    %rsp,%rbp
  4005a4:       b8 00 00 00 00          mov    $0x0,%eax
  4005a9:       e8 e2 ff ff ff          callq  400590 <bar>
  4005ae:       bf 00 00 00 00          mov    $0x0,%edi
  4005b3:       e8 10 ff ff ff          callq  4004c8 <_exit@plt>
00000000004005b8 <handler>:
  4005b8:       55                      push   %rbp
  4005b9:       48 89 e5                mov    %rsp,%rbp
  4005bc:       48 83 ec 10             sub    $0x10,%rsp
  4005c0:       89 7d fc                mov    %edi,-0x4(%rbp)
  4005c3:       8b 05 73 0a 20 00       mov    0x200a73(%rip),%eax        # 60103c <ncaught>
  4005c9:       ff c0                   inc    %eax
  4005cb:       89 05 6b 0a 20 00       mov    %eax,0x200a6b(%rip)        # 60103c <ncaught>
  4005d1:       b8 00 00 00 00          mov    $0x0,%eax
  4005d6:       e8 c5 ff ff ff          callq  4005a0 <foo>
  4005db:       b8 00 00 00 00          mov    $0x0,%eax
  4005e0:       e8 bb ff ff ff          callq  4005a0 <foo>
  4005e5:       be b8 05 40 00          mov    $0x4005b8,%esi
  4005ea:       bf 0a 00 00 00          mov    $0xa,%edi
  4005ef:       e8 e4 fe ff ff          callq  4004d8 <signal@plt>
  4005f4:       c9                      leaveq 
  4005f5:       c3                      retq   
00000000004005f6 <main>:
  4005f6:       55                      push   %rbp
  4005f7:       48 89 e5                mov    %rsp,%rbp
  4005fa:       be b8 05 40 00          mov    $0x4005b8,%esi
  4005ff:       bf 0a 00 00 00          mov    $0xa,%edi
  400604:       e8 cf fe ff ff          callq  4004d8 <signal@plt>
  400609:       b8 00 00 00 00          mov    $0x0,%eax
  40060e:       c9                      leaveq 
  40060f:       c3                      retq   
\end{verbatim}
Listing 2: Compiled as gcc -O0 test1.c

The current (work-in-progress) output from the checker for the unoptimized, non-stripped executable is:

\begin{verbatim}
(rose-trunk) matzke@work DynamicLinking $ ./binaryAsyncSignalHandler a.out
/lt-binaryAsyncSignalHandler: disassembling...
/lt-binaryAsyncSignalHandler: scanning for ELF File Header...
0x0040049e: jmp instruction doesn't have a value expression.
0x004004a8: jmp instruction doesn't have a value expression.
0x004004b8: jmp instruction doesn't have a value expression.
0x004004c8: jmp instruction doesn't have a value expression.
0x004004d8: jmp instruction doesn't have a value expression.
0x0040052c: call instruction doesn't have a value expression.
0x0040054e: call instruction doesn't have a value expression.
0x00400589: jmp instruction doesn't have a value expression.
0x0040067c: call instruction doesn't have a value expression.
0x004006cb: call instruction doesn't have a value expression.
/lt-binaryAsyncSignalHandler: found 8 function definitions
0x00400480-0x004004f0: defn=0x7f9f8ff48010 sym=0x7f9f900771d0 <_init>
0x004004f0-0x00400590: defn=0x7f9f8ff480a0 sym=0x7f9f90076720 <_start>
0x00400590-0x004005a0: defn=0x7f9f8ff48130 sym=0x7f9f900767b0 <bar>
0x004005a0-0x004005b8: defn=0x7f9f8ff481c0 sym=0x7f9f90076cc0 <foo>
0x004005b8-0x004005f6: defn=0x7f9f8ff48250 sym=0x7f9f90076840 <handler>
0x004005f6-0x00400610: defn=0x7f9f8ff482e0 sym=0x7f9f90077140 <main>
0x00400610-0x00400612: defn=0x7f9f8ff48370 sym=0x7f9f90076690 <__libc_csu_fini>
0x00400620-0x004006aa: defn=0x7f9f8ff48400 sym=0x7f9f90076e70 <__libc_csu_init>
/lt-binaryAsyncSignalHandler: scanning instructions in 0x004004f0 to 0x004006e4 (.text section)
/lt-binaryAsyncSignalHandler: found 5 dynamic calls
0x004005ef call signal()
0x00400604 call signal()
/lt-binaryAsyncSignalHandler: found 1 potential signal handler
0x004005b8: signal handler handler()
  possible unsafe call to foo->bar->puts
\end{verbatim}
Listing 3: Checker output for unoptimized, non-stripped executable

The main thing to note is the final line of output, indicating that the unsafe function
puts is called from bar, which was called by foo, which was called by the signal
handler, handler.

Stripping the executable of its symbol tables (strip a.out) is a very common occurrence
for installed software. A stripped executable is more difficult to analyze because one
cannot determine function boundaries by looking at symbol table entries. We are very
close to being able to determine function boundaries using alternative methods,
including analyzing branch relationships between basic blocks.

The optimized, non-stripped executable:
\begin{verbatim}
0000000000400590 <main>:
  400590:       48 83 ec 08             sub    $0x8,%rsp
  400594:       be e0 05 40 00          mov    $0x4005e0,%esi
  400599:       bf 0a 00 00 00          mov    $0xa,%edi
  40059e:       e8 35 ff ff ff          callq  4004d8 <signal@plt>
  4005a3:       31 c0                   xor    %eax,%eax
  4005a5:       48 83 c4 08             add    $0x8,%rsp
  4005a9:       c3                      retq   
  4005aa:       66 0f 1f 44 00 00       nopw   0x0(%rax,%rax,1)
00000000004005b0 <foo>:
  4005b0:       bf e8 06 40 00          mov    $0x4006e8,%edi
  4005b5:       48 83 ec 08             sub    $0x8,%rsp
  4005b9:       e8 ea fe ff ff          callq  4004a8 <puts@plt>
  4005be:       31 ff                   xor    %edi,%edi
  4005c0:       e8 03 ff ff ff          callq  4004c8 <_exit@plt>
  4005c5:       66 66 2e 0f 1f 84 00    nopw   %cs:0x0(%rax,%rax,1)
  4005cc:       00 00 00 00 
00000000004005d0 <bar>:
  4005d0:       bf e8 06 40 00          mov    $0x4006e8,%edi
  4005d5:       e9 ce fe ff ff          jmpq   4004a8 <puts@plt>
  4005da:       66 0f 1f 44 00 00       nopw   0x0(%rax,%rax,1)
00000000004005e0 <handler>:
  4005e0:       48 83 ec 08             sub    $0x8,%rsp
  4005e4:       bf e8 06 40 00          mov    $0x4006e8,%edi
  4005e9:       ff 05 4d 0a 20 00       incl   0x200a4d(%rip)        # 60103c <ncaught>
  4005ef:       e8 b4 fe ff ff          callq  4004a8 <puts@plt>
  4005f4:       31 ff                   xor    %edi,%edi
  4005f6:       e8 cd fe ff ff          callq  4004c8 <_exit@plt>
  4005fb:       90                      nop    
  4005fc:       90                      nop    
  4005fd:       90                      nop    
  4005fe:       90                      nop    
  4005ff:       90                      nop    
\end{verbatim}
Listing 4: Compiled as gcc -O3 test1.c

Some things to note about the optimized version are:
\begin{itemize}
   \item The foo and bar functions are in-lined into handler, but remain as stand alone functions due to the fact that they're externally visible (i.e., not declared static)
   \item Function prologues and epilogues no longer follow a simple pattern. In fact, the bar function has neither.
   \item Function calls don't always use the call instruction. The call to puts at 0x405d5 is such an example: since bar didn't adjust the stack frame, the return in puts will return to bar's caller.
   \item Function calls that are known to not return, at first glance appear to fall off the end of the function. For example, the call at 0x4005c0 doesn't ever return, thus foo doesn't need to contain a function epilogue.
\end{itemize}

Again, we can detect function boundaries by consulting the symbol table, and thus build a function call graph. Here's the analysis output:
\begin{verbatim}
(rose-trunk) matzke@work DynamicLinking $ ./binaryAsyncSignalHandler a.out
/lt-binaryAsyncSignalHandler: disassembling...
/lt-binaryAsyncSignalHandler: scanning for ELF File Header...
0x0040049e: jmp instruction doesn't have a value expression.
0x004004a8: jmp instruction doesn't have a value expression.
0x004004b8: jmp instruction doesn't have a value expression.
0x004004c8: jmp instruction doesn't have a value expression.
0x004004d8: jmp instruction doesn't have a value expression.
0x0040052c: call instruction doesn't have a value expression.
0x0040054e: call instruction doesn't have a value expression.
0x00400589: jmp instruction doesn't have a value expression.
0x0040066c: call instruction doesn't have a value expression.
0x004006bb: call instruction doesn't have a value expression.
/lt-binaryAsyncSignalHandler: found 6 function definitions
0x00400480-0x004004f0: defn=0x7f4028efb010 sym=0x7f402902a1d0 <_init>
0x004004f0-0x00400590: defn=0x7f4028efb0a0 sym=0x7f4029029720 <_start>
0x00400590-0x004005aa: defn=0x7f4028efb130 sym=0x7f402902a140 <main>
0x004005e0-0x004005fb: defn=0x7f4028efb1c0 sym=0x7f4029029840 <handler>
0x00400600-0x00400602: defn=0x7f4028efb250 sym=0x7f4029029690 <__libc_csu_fini>
0x00400610-0x0040069a: defn=0x7f4028efb2e0 sym=0x7f4029029e70 <__libc_csu_init>
/lt-binaryAsyncSignalHandler: scanning instructions in 0x004004f0 to 0x004006d4 (.text section)
/lt-binaryAsyncSignalHandler: found 6 dynamic calls
0x0040059e call signal()
/lt-binaryAsyncSignalHandler: found 1 potential signal handler
0x004005e0: signal handler handler()
  possible unsafe call to puts
\end{verbatim}
Listing 5: Checker output for optimized, non-stripped executable

Note that the call site for puts is reported as being in handler rather than bar. The puts call in bar is not reported at all since it cannot be reached from the signal handler.

Future work:
\begin{itemize}
   \item Better detection of function boundaries when symbol tables are not present in the executable. This work should be completed in the next couple days.
   \item Ability to track constant propagation to handle the case where the argument to signal is supplied in function arguments. This is a common occurrence since signal semantics are somewhat variable across platforms and source code accommodates that by using a wrapper functions whose action is determined at compile time through macros and conditional compilation. By runtime the wrapper has usually degenerated to something fairly simple:
      \begin{verbatim}
         void signal_wrapper(int signum, sighandler_t func) {
         signal(signum, func);
         }
      \end{verbatim}
   \item Ability to detect signal handlers registered with sigaction. This is important because most modern codes use this more standardized call rather than signal. What makes this difficult for analysis is that the signal handler is supplied as a pointer in a pointed-to struct.
   \item Ability to detect signal registration in statically linked and statically linked stripped
    executables. In these cases there is no obvious call to signal or sigaction but we
    should be able to detect them via code signatures or by looking for the actual Linux
    syscall that those two functions eventually execute.
\end{itemize}

% This chapter organization permits us to later put in figures to represent screen shots
% of the different things seen in the demo (later).

% This chapter organization permits us to later put in figures to represent screen shots
% of the different things seen in the demo (later).

\chapter{demo\_8: Compass on source code and binary using GUI interface}
\label{chapter:compassOnSourceCodeAndBinaryInGuiMode}
   There are two tests that demonstrate compass, the first is a source code and
the second is a binary; (demo\_8a and demo\_8b, respectively).


\chapter{demo\_9: Show construction of Compass checker (using gen\_checker.sh)}
\label{chapter:buildCompassChecker}

   This demo shows the construction of Compass checker (using gen\_checker.sh).
The constructed checker has default behavior, but is a valid check that can be filled in with
vulnerability specific analysis, using existing analysis in ROSE, etc. The final
edited files are tar'd up as a directory and submitted to Compass for verification
(both automatic and manual verification of the checker is required) and then inserted into
Compass.

The command to build a checker in any directory is:
\begin{verbatim}
  <path to ROSE directory>/projects/compass/src/compass_scripts/gen_checker.sh buffer overflow checker
\end{verbatim}
This command will build a subdirectory with name {\bf bufferOverflowChecker} and 
populate that directory with the files:
\begin{verbatim}
tux270.llnl.gov{dquinlan}359: ls -l bufferOverflowChecker/
total 136
-rw-rw-r--  1 dquinlan dquinlan  3895 Jan 19 18:27 bufferOverflowChecker.C
-rw-rw-r--  1 dquinlan dquinlan  1480 Jan 19 18:27 bufferOverflowCheckerDocs.tex
-rw-rw-r--  1 dquinlan dquinlan   282 Jan 19 18:27 bufferOverflowChecker.inc
-r--r--r--  1 dquinlan dquinlan   252 Jan 19 18:27 bufferOverflowCheckerMain.C
-rw-rw-r--  1 dquinlan dquinlan    35 Jan 19 18:27 bufferOverflowCheckerTest1.C
-rw-rw-r--  1 dquinlan dquinlan 33085 Jan 19 18:27 compass.C
-rw-rw-r--  1 dquinlan dquinlan 15415 Jan 19 18:27 compass.h
-rw-rw-r--  1 dquinlan dquinlan   131 Jan 19 18:27 compass_parameters
-rw-rw-r--  1 dquinlan dquinlan  1541 Jan 19 18:27 compassTestMain.C
-rw-rw-r--  1 dquinlan dquinlan     0 Jan 19 18:27 instantiate_prerequisites.h
-rw-rw-r--  1 dquinlan dquinlan  2459 Jan 19 18:27 Makefile
-rw-rw-r--  1 dquinlan dquinlan   147 Jan 19 18:27 prerequisites.h
tux270.llnl.gov{dquinlan}360:
\end{verbatim}

Using this directory of files as a base, a checker can be constructed by the following
steps:
\begin{enumerate}

   \item Edit the {\bf bufferOverflowChecker.C} \\
      This file defines the implementation of the checker, it has a skeleton of a checked in
   place, but the checker does nothing but traverse the AST.  Modifications to this
   checker permit the implementation of a rule to be used by the rest of Compass to 
   detect violations (or the rule) in source code or binaries.

   \item {Fix up the {\bf Makefile}} \\
      Of these files, the {\bf Makefile} must be edited to fix up the correct paths
   to the installation directory for ROSE and any additional paths for optional
   features should be fixed up.

   \item {Add documentation} \\
   The documentation should be edited to reflect the details of the checker just built.

   \item {Add test code}
   A test code is required to show a violation of the checker, the checker must
   correctly detect the violation of the rule (this is tested internally within Compass).

\end{enumerate}

    The checker can be developed and tested in this directory, the Makefile supports
the compilation and testing rules required to simplify these steps.  When the checker is
finished, it may be submitted to Compass for verification (there are both automated
tests and a manual check required before the new checker can be accepted).

    To submit the checker to Compass, the whole directory should be tar'd up and
submitted using the Compass checker submission script.
\begin{verbatim}
  <path to ROSE directory>/projects/compass/src/compass_scripts/submit_checker.sh bufferOverflowChecker.tar.gz
\end{verbatim}





\chapter{demo\_10: Translation of Binary to Source Code (Power-PC example)}
\label{chapter:binaryTranslationToSource}

Removed (non-working).

\chapter{demo\_11: Demo of Binary File Compare}
\label{chapter:binaryFileCompare}

% In directory:
% /home/dquinlan/ROSE/ROSE_CompileTree/svn-LINUX-64bit-4.2.2/developersScratchSpace/Dan/astEquivalence_tests
% Run "make testBinaryEquivalence"

   This demo evaluates the differences between to binaries generated from the same source
file, one with debugging information (compiled using {\em -g} option) and the other
without any debugging information.


The C source code is:
\begin{verbatim}
int this_is_a_variable = 0x01234567;

int
main()
   {
     int another_variable = this_is_a_variable + 7;
     int y,z;

     int x = 7;
     y = x;
     z = y;

     return 0;
   }
\end{verbatim}

The are built into a binary and compared using both AST comparison on the source code AST
and the binary AST.  The results are shown using the edit graph, from which one can
read off the additions and deletions required to convert one into another.

\begin{verbatim}
tux270.llnl.gov{dquinlan}157: ls -l testProgram_?
-rwxrwxr-x  1 dquinlan overture 7967 Jan 16 18:44 testProgram_1
-rwxrwxr-x  1 dquinlan overture 6551 Jan 16 18:44 testProgram_2
\end{verbatim}

   This test looks only at the binary file structure and not the instructions.

   This test uses a classic tree edit distance algorithm (written and optimized by
the ROSE team) to determine the minimum additions and deletion to make any to files
(typically different files) the same. The demo generates a file called "output.dot"
which can be visualized using zgrviewer.  the generated file is a large graph which
take 20 minutes to layout using dot, but shows the details edits required to translate
one file into the other. 

   There are a number of novel aspects to this work, first it operates on the
file structure (instead of the byte stream, which would be meaningless).
Second it is cheaper as a result of operating on the file structure,
the file structure can be compared for equivalence separately and this provides
for a number of level of granularity in comparing any two binary files.  
The instruction sequence can also be compared using different levels of
normalization (however, only a single normalization is currently implemented in ROSE).
This represents ongoing research work and is of interest because of the ways it can be
done with different level of fidelity and how it addresses the canonical order
$n^2$ complexity of conventional tree edit distance algorithms.



\chapter{demo\_12: Compass on Large Projects}
\label{chapter:compassOnLargeProjects}

% In directory:
% /home/dquinlan/ROSE/ROSE_CompileTree/svn-LINUX-64bit-4.2.2/projects/compass/tools/compass/gui
% Run "make demo"

   This demo shows how a data base is built (before running Compass) and then used with
   Compass to define a project with many files (this example shows the use of only a dozen
   files).
   The database stores the command-line require to compile source code.  The compass is
   run using the database as input.  Click "regenerate" and "refresh" to display the 
   results.  The "regenerate" button will run compass on all the items in the 
   database, this can take a little while (a few seconds in the demo on a fast machine).
   Then the "refresh" button will sort and display the results.  Clicking on a specific
   rule (from the list in the upper left window), will cause the locations of violations 
   of that rule to be displayed in the upper center window.  Clicking on an entry in the 
   upper center window causes the lower windows to load that specific file and highlight
   the line in the file where the violation occurred.  Currently both binaries and source
   code work in Compass, but the GUI handles the source code better, this will become 
   more uniform in time.


\chapter{demo\_13: Static Binary Rewriting}
\label{chapter:staticBinaryRewriting}


% In directory:
% /home/dquinlan/ROSE/ROSE_CompileTree/svn-LINUX-64bit-4.2.2/developersScratchSpace/Dan/translator_tests
% Run "make demo"

   This demo shows the addition of a named section to an existing binary file.  The to
support this the section table of names is edited and an additional name added, this
causes the entire binary to be offset by the length of the name (and it's null
terminator).  A new section is added to the binary with some random data. All the 
stored offsets in the binary must be recomputed and reset as part of the final
regeneration of the binary as and executable with extra data.

   The binary equivalence test is used to test the original and the modified binary.
The edit distance it isolated to a single point in the AST (excerpt of output).

\begin{verbatim}
editOperation = substitution from_label = 10562 to_label = 10727 
editOperation = substitution from_label = 10397 to_label = 10562 
editOperation = substitution from_label = 10232 to_label = 10397 
editOperation = substitution from_label = 10067 to_label = 10232 
editOperation = insertion from_label = 10066 to_label = 10067 
editOperation = insertion from_label = 10065 to_label = 10066 
editOperation = substitution from_label = 9900 to_label = 10065 
editOperation = substitution from_label = 9735 to_label = 9900 
editOperation = substitution from_label = 9570 to_label = 9735 
editOperation = substitution from_label = 9405 to_label = 9570
\end{verbatim}

   The dot file visualization representing the edit graph is 6Meg and too large to 
layout using dot (Graph-Viz).

