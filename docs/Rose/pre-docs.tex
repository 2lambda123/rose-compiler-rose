\documentclass{article}
\usepackage{listings}

\begin{document}
\author{Jeremiah Willcock}
\title{Partial redundancy elimination documentation}
\maketitle

\newcommand{\Cpp}{C\texttt{++}}

The implementation of partial redundancy elimination (PRE) in ROSE is in
\texttt{src/midend/programTransformation/partialRedundancyElimination}.
The version in ROSE is a straightforward implementation of the algorithm in
Paleri et al~\cite{paleri}.  \Cpp{} statements are considered to be the
basic unit for PRE, and conversion to triple form (as done in the paper) is
not performed.  The standard control flow graph structure in ROSE is used
for control flow analysis, and so its limitations also apply to the PRE
implementation.

In order to perform partial redundancy elimination on a
given function definition, ROSE first generates a control flow graph of the
function.  It then adds information to each edge of the control flow graph
showing where to insert statements into the source code so that they will
be executed during every transition along that edge.  Currently, not all
edges can be handled because of limitations in the API used by the control
flow graph implementation (not enough information about how control
structures correspond to graph structures is given to the graph
implementation).  Most edges are properly annotated, however.

Other than the creation of the control flow graph, the rest of PRE is done
for one expression at a time.  This could be optimized in the future to use
bit vectors to do multiple expressions in one iteration through the source
code.

For a given expression, PRE proceeds by examining each of the control flow
graph nodes (basic blocks) to find computations and modifications to the
value of the expression.  Finding modifications is done by looking at all
variables used in the expression and finding statements which modify any of
them.  A more sophisticated alias analysis should be used for this than
what is currently present.  Also, this pass removes redundant computations
within a single basic block.

Next, a series of analysis variables is computed on the basic blocks of the
control flow graph.  These are named as in~\cite{paleri}, and that paper
should be consulted to understand their functions.  The worklist algorithm
is used to compute fix-points of systems of boolean equations during this
computation.

Finally, the analysis recommends changes to the program, and so these
changes are made.  These include replacing computations of the selected
expression with uses of a cache variable and updating the cache variable
with new values of the expression.  These are done in a straightforward
way, and match the paper fairly closely.

After this is done for each expression in the program (with some
expressions skipped because they are too simple or have side effects), all
partial redundancies have been removed.  One caveat is that loop-invariant
expressions will only be moved if the loop will definitely execute at least
one iteration (for example, by using a do-while structure).  This is
because PRE is a conservative analysis that will not insert any
computations into the program, and moving a loop-invariant expression out
of a loop could add an extra computation if the loop executes zero
iterations.  Finite differencing can be used for some kinds of
loop-invariant code motion in cases like this.

\bibliography{docs}
\bibliographystyle{abbrv}

\end{document}
