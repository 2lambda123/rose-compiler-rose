\chapter{SAGE III Intermediate Representation}
\label{SageIII}

There are many details that this chapter on SAGE will present.

\fixme{incomplete-documentation}


\section{History of SAGE}
    We chose to develop and use SAGE III, originally 
developed as SAGE++ by Dennis Gannon and others at University 
of Indiana, and then SAGE II by Dennis at IU and Carl Kesselman at ISI,
and others.  Because SAGE III is a reimplementation of the similar
object-oriented IR API, their work gave us a significant head 
start in the development of ROSE (and an understanding of 
object-oriented IRs).

\subsection{Differences Between SAGE++ and SAGE II}
     SAGE++ was the first version of SAGE and it provided
support for C, a subset of C++ (C++ evolved quite a bit early on
and was a moving target), and F90. SAGE II introduced the use
of the EDG front-end, and dropped the handling of Fortran, 
but its work was incomplete.

\subsection{Difference Between SAGE II and SAGE III}
   The SAGE III IR is now completely generated using the ROSETTA
IR generator tool (a source-code generation tool) which we
developed to support our work within ROSE.  Initial versions 
of SAGE II were well done, but not complete.  Numerous 
details were addressed in the work on SAGE II as part of its 
preparation for use within ROSE.  We are very thankful to the 
initial developers of SAGE II for all their work.  Sage III
hopefully fulfills on a number of the goals behind their work.
SAGE III continues to use the EDG frontend and has updated
the versions of EDG in use (over SAGE II) and separated
out the EDG work so that the connection of SAGE III to EDG
is easier to maintain and update in the future with new versions 
of EDG.

\subsection{Differences Between SAGE III and ROSE}
   ROSE uses SAGE III internally and adds numerous, more sophisticated mechanisms.
For example, ROSE adds:
\begin{itemize}
    \item Attribute mechanisms for use within traversals (ideas borrowed 
          from attribute grammars).
    \item A sophisticated AST rewrite mechanism to simplify the development 
          of transformations.
    \item A more sophisticated persistent attribute mechanism.
    \item Loop analysis and optimization (loop fusion, fission, blocking, etc.)
    \item Operators for conversion of AST subtrees to strings, and of strings 
          to AST fragments.
    \item Database support for global analysis.
    \item C++ Template support.
    \item Fast binary AST File I\/O.
    \item An AST merge mechanism for supporting whole program analysis
          (across hundreds of files).
    \item Complete language support for C, C99, UPC, C++, Fortran 66, Fortran 77, 
          Fortran 90/95, and Fortran 2003.
    \item AST visualizations (program visualization for debugging).
    \item ROSE User Manual and ROSE Tutorial Documentation.
    \item Full IR documentation via Doxygen (web pages).
    \item Web site with software and svn repository access.
    \item And lots more, ...
\end{itemize}

\section {Comments Handling}

    Comments are placed into the SAGE III AST using a separate pass over the source file.
EDG does not preserve comments at this time, and we felt it was important to preserve them
within the unparsed (generated) output of the source-to-source mechanism that ROSE defines.
Comment processing can also be addressed using the AST Rewrite Mechanism, though
the order of how the comments appear in the code is determined by the order of 
invocation of the AST {\tt insert()} function with a comment as the input string.
Internally, the comments annotate the AST (tree decoration) so that AST queries may use 
the comments at will.

\section {C Preprocessor ({\tt cpp}) Directive Handling}
    The C Preprocessor ({\tt cpp}) directives ({\em not} \#pragma) are handled internally using the same mechanism as
comments.  Although they are fully expanded at compile time they are reinserted back
into the unparsed source code as it is being unparsed.  Internally, the directives annotate the AST
(tree decoration) so that AST queries may use the directives at will.
Note that pragmas are a part of the language specification (grammar) and not a CPP directive.

  Note also that {\tt extern ``C'' \{\}} is also recognized so that it can be
placed around {\tt \#include} directives and other identified blocks of declarations.
Internally such declarations are explicitly marked as having extern C linkage.

\section {Pragma Handling}
   The {\tt \#pragma} is special and is not really a C Preprocessor ({\tt cpp}) directive.
It is formally part of the C and C++ language grammar, and thus we are justified in putting it 
into the AST with the rest of the language constructs (comments and directives are open for 
a degree of interpretation as to where they can be attached within the AST). Details of
this subject may be open to minor changes in future releases of ROSE.

  Pragmas are the mechanism in which C and C++ permit language
extension.  Of course, some people describe this a bit differently,
but {\tt \#pragma} is not interpreted by {\tt CPP}, and it {\it {\bf is}} interpreted by the compiler.
And it has a specific semantics since it is part of the language grammar.
The EDG documentation refers to them as pragma declarations, so they should be 
treated that way. This also is why they only really work in the grammar if 
they are declarations (since they are only permitted were common declarations 
are permitted and no where else).
% Though it seems that a declaration can go just about anywhere, so this thus
% support the idea that pragmas can go anywhere.

Note that {\tt \#pragma pack} declarations are handled in a special normalization
(see section \ref{pragma_pack_normalization}).  These pragmas are a bit different
from other pragmas and are handled as a stack-based embedded language.


% \section{Deep and Shallow Coping of the AST}
\section { Copying IR Nodes and Subtrees}

    Support is provided for a policy-based copying of the AST and subtrees
of the AST.  Flexibility and control is provided through an independent
policy mechanism that defines the copying process as shallow or deep for
different types of nodes within the AST.

% Complete documentation about this feature will be done in the near future.

  Each {\tt SgNode} object has the following public virtual member function: 

{\indent
{\mySmallFontSize

\begin{verbatim}
class SgNode {
     ...
     virtual SgNode* copy ( SgCopyHelp & help) const;
     ...
};
\end{verbatim}
}}

  Here SgCopyHelp is a virtual policy class for duplicating 
{\tt SgNode} objects and is defined as:

{\indent
{\mySmallFontSize

\begin{verbatim}
class SgCopyHelp {
   public:
     virtual SgNode* copyAst ( const SgNode *n ) = 0;
};
\end{verbatim}
}}

Two concrete classes, {\tt SgShallowCopy} and {\tt SgTreeCopy}, are provided as 
subclasses of {\tt SgCopyHelp} to configure a shallow copy (duplicating 
the current {\tt SgNode} object only) or a deep copy (duplicate the complete
subtree rooted at the current {\tt SgNode} object) respectively. The following
example illustrates how to use {\tt SgShallowCopy} and {\tt SgTreeCopy} 
to duplicate SAGE nodes and sub-trees.

{\indent
{\mySmallFontSize

\begin{verbatim}
  SgNode *orig;
  ...
  SgNode *n1 = orig->copy(SgShallowCopy::static_instance());
  SgNode *n2 = orig->copy(SgTreeCopy::static_instance());
  ... 
\end{verbatim}
}}

Here {\it n1} points to a duplicate of the {\tt SgNode} object pointed to by {\tt orig},
while {\it n2} points to a duplicate of the complete subtree rooted at {\tt orig}.
Therefore, the shallow copy {\it n1} from {\tt orig} shares all the children 
of {\tt orig}, while the deep copy {\it n2} from {\tt orig} duplicates all the children 
of {\tt orig} by recursively cloning the children objects. 
Note that the children of node {\tt orig} are determined by the tree-traversal
mechanism of ROSE. A field {\tt fp} within {\tt orig} is considered a child of {\tt orig}
only if {\tt fp} is traversed by the tree-traversal mechanism. 
For all other fields in {\tt orig}, only shallow copies are performed. 
As a result, only pointers to {\tt SgNode}s that are part of
the tree traversal rooted at {\tt orig} can be recursively cloned. 
 
To simplify the specification of shallow and deep cloning
of {\tt SgNode}s, two macros are further defined:

{\indent
{\mySmallFontSize

\begin{verbatim}
#define SgSHALLOW_COPY SgShallowCopy::static_instance()
#define SgTREE_COPY SgTreeCopy::static_instance()
\end{verbatim}
}}

The above example code, therefore, can be rewritten as:

{\indent
{\mySmallFontSize

\begin{verbatim}
  SgNode *orig;
   ...
  SgNode *n1 = orig->copy(SgSHALLOW_COPY);
  SgNode *n2 = orig->copy(SgTREE_COPY);
  ...
\end{verbatim}
}}

\section{ Template Handling in C++}

   The purpose
of this section is to lay out the details of handling C++ templates.
% Templates in C++ is quite new within ROSE and it has been 
% changing as we have interacted with other research groups (particularly Stroustrup).
Initial template handling in SAGE III represented templates as classes and function
(using generated, i.e. mangled, names) and with a flag indicating there derivation 
from a C++ template.

   ROSE allows the transformation of templated classes and
functions by generating the required specializations.  This way, all details of 
a templated class of function (or static data member) become visible to the user in
the AST and permit maximum information assumed to be required for any transformation.
No transformation occurs on the template declaration unless it's done explicitly by 
the user (this is difficult since the text string representing the template is not
formed into an AST that we can traverse).  Note that this is a result of a design
decision on the part of EDG to provide this as a default behavior and our decision 
to use it.  More recent work to get the template as an AST is underway, using some 
of the options in EDG to support this.  This later work is not robust enough to be 
the default in ROSE without a bit more work.

\subsection{C++ Constructs That Can Be Made Into Templates}
   The concept of templates does not apply to all C++ constructs and affects only a few.
The only things that can be templates are classes (including structs and likely unions),
\fixme{Check on template unions.}
functions (including member functions), and variables (static data
members).  The first two are common, but the case of templated variables perhaps requires 
an example:

{\indent
{\mySmallFontSize

\begin{verbatim}
template<typename T>
class A
   {
     public:
      // non-template data member
         int nonTemplateDataMember;

      // template data member
         T templateDataMember;

      // template static data members
         static T staticTemplateDataMember_T;
         static float staticTemplateDataMember_float;
   };
         
// This is a template static data member (SgVariableDeclaration)
template<class U> U A<U>::staticTemplateDataMember_T;

// This is a template static data member (SgVariableDeclaration)
template<class U> float A<U>::staticTemplateDataMember_float;

// template specialization for variable (was originally defined to be float!)
template<> float A<double>::staticTemplateDataMember_float;

// template specialization for variable (error: this is not possible, type mismatch)
template<> float A<double>::staticTemplateDataMember_T;
\end{verbatim}
}}

   In the case of a {\tt SgVariableDeclaration}, the information about whether or not it is a specialization
is kept with the {\tt SgVariableDeclaration}, instead of the {\tt SgInitializedName} objects that
stand for the individual variables.  Since the {\tt get\_parent()} member function returns
a pointer to the {\tt SgVariableDeclaration} from the {\tt SgInitializedName}, this
information is indirectly available from the {\tt SgInitializedName}.
% It is not clear if this is the best implementation.

  {\tt Enums}, {\tt typedefs}, {\tt namespaces}, etc. cannot appear as templated declarations.
As a result, only a few declarations contain template specific information
({\tt SgClassDeclaration}, {\tt SgFunctionDeclaration}, {\tt SgVariableDeclaration}).

\subsection{ How Templates affects the IR }
   Some IR nodes are present to support the use of templates in C++. These include:
\begin{itemize}
     \item {\tt SgTemplateParameters} \\
           Derived from SgSupport.

     \item {\tt SgTemplateArguments} \\
           Derived from SgSupport.

     \item {\tt SgTemplateDeclaration} \\
           Derived from {\tt SgDeclarationStatement}.
          \begin{itemize}
               \item Holds the template string (any comments are removed)
               \item Template name
               \item Template parameters
          \end{itemize}
     \item {\tt SgTemplateInstantiationDecl} ({\em may be renamed to SgTemplateInstantiationClassDeclaration}) \\
           Derived from {\tt SgClassDeclaration}.
          \begin{itemize}
               \item Reference to {\tt SgTemplateDeclaration}
               \item Template arguments
          \end{itemize}
     \item {\tt SgTemplateInstantiationFunctionDecl} \\
           Derived from {\tt SgFunctionDeclaration}.
          \begin{itemize}
               \item Reference to {\tt SgTemplateDeclaration}
               \item Template arguments
          \end{itemize}
     \item {\tt SgTemplateInstantiationMemberFunctionDecl} \\
           Derived from {\tt SgMemberFunctionDeclaration}.
          \begin{itemize}
               \item Reference to {\tt SgTemplateDeclaration}
               \item Template arguments
          \end{itemize}
     \item {\tt SgTemplateInstantiationDirective} \\
           This forces the explicit instantiation of the specified template when (and
           where) it appears in the source code.
%         \begin{itemize}
%              \item Reference to template declaration node (either
%                    {\tt SgTemplateInstantiationDecl}, {\tt SgTemplateInstantiationFunctionDecl},
%                    {\tt SgTemplateInstantiationMemberFunctionDecl}, or a template member data
%                    (though I have not seen code that demonstrates this and which
%                    compiles with either EDG or GNU g++).  No other declaration is 
%                    permitted (or makes sense).
%         \end{itemize}
\end{itemize}

Nodes not added include (a judgement call for now):
\begin{itemize}
     \item SgTemplateClassDeclaration
     \item SgTemplateFunctionDeclaration
     \item SgTemplateMemberFunctionDeclaration
     \item SgTemplateDataMemberDeclaration
\end{itemize}
There are many types of template declarations, at present there is an enum type
which identifies each category of template declaration.  The enum type is:

{\indent
{\mySmallFontSize

\begin{verbatim}
          enum template_type_enum 
             {
               e_template_none       = 0,
               e_template_class      = 1,
               e_template_m_class    = 2,
               e_template_function   = 3,
               e_template_m_function = 4,
               e_template_m_data     = 5
             };
\end{verbatim}
}}

A data member of this type is held in the {\tt SgTemplateDeclaration}.

% ****************************************************************************************** 
% ****************************************************************************************** 
% ****************************************************************************************** 

   We might have to distinguish between template member functions and member functions
   of template classes, so that we can exclude instantiation of template member functions
   separately from member functions of template classes (which are required for the
   definition to appear in the generated source code).  At present, this is done with a
   member function that computes this information (see the IR node documentation for more detail).

% ****************************************************************************************** 
% ****************************************************************************************** 
% ****************************************************************************************** 

\subsection{Template Specialization}
   Things that can be specialized include classes, structures, unions,
variables (static data members of templated classes), functions, and member functions.
% Template instantiations will soon (not done yet) be marked as to if they are part of
% an explicit instantiation (since those should always be output even if the templates
% instantiations are not generated and output).
   Template and template instantiations need more information stored in the IR nodes to
allow the unparser to be simplified.  We currently compute this information within
separate, post-processing, passes over the AST (see the source code in
{\tt ROSE/src/frontend/SageIII/astPostProcessing} for details).
% and setup the required template instantiation behavior (member data such as doNotInstantiate, etc.).
   Interestingly, a template specialization is not an instantiation and can co-exist in each
file and not cause linker problems (multiply defined symbols), it may cause generation of 
{\em weak symbols}.

% ************************************************************************************
% Need to test this.  Later we could
% instantiate all specializations into a separate file (a {\em template specialization file}
% and keep a list of what specializations have been put into the file so that we could
% only place specializations into the {\em template specialization file} once.
% ************************************************************************************

\subsection{ Unparsing Templates }
   The general handling of templates requires a specific sorting of the template output.
This order permits the generation of all template specializations, which allows 
each specialization to be transformed uniquely.  This is important
to the support of transformations on templates (based on template arguments).
The order of output for template handling is as follows:
\begin{enumerate}
     \item Output templates. \\
           Raw template declarations (text strings) are output at the top of the file.

     \item Output template function prototypes. \\
           Function prototypes for all specializations that we generate are required
           before the use of the template forces its instantiation.  The point is to 
           allow the specialized template function to be available for transformation.
           It can be placed anywhere in the file (typically at the end in ROSE) as long
           as a prototype has been output to prevent a full instantiation of the
           specialized template function before any use would force its instantiation by
           the back-end compiler. At that point, the template specialization generated by 
           ROSE (and perhaps transformed by the user) is not only redundant, but results 
           in an error (since the function is defined twice -- first instantiated by the 
           vendor compiler and then seen as an explicit template specialization 
           generated by ROSE).

     \item Output template function definitions. \\
           All template specializations can be now output, even if they referenced 
           templated classes for functions that would force the instantiations (the 
           reason why all prototypes must proceed the definitions of template classes 
           and functions).  These can actually appear before or after the rest of the 
           code, so \#3 and \#4 may be swapped).

     \item Output the rest of the code. \\
           This will force template instantiations of any non-specialized template classes
           or functions.  It may appear before the template functions definitions or mixed
           (interleaved) with them.

     \item Output all explicit template instantiation directives at the base of each namespace
           where they appear.  It is not clear that it is required to observe namespaces 
           since the instantiation directive could reference fully qualified type names.
           This should be sufficient to resolve type ambiguity.  
         % Note that GNU (at least) does not allow explicit template instantiation
         % directives in a class.  Not clear if it is legal C++.

\end{enumerate}

\commentout{
   It is not clear if we need to handle specializations separately (I will resolve that
after I get nonspecialized templates working), likely new IR nodes will be added for
these as well.  The general philosophy is to put as much information into the IR as possible
about the templates, their specializations, and their instantiations so that
transformations on templates can be supported.  And I am learning as I go :-).
}

\subsection{ Templates Details }
   There are several details to enumerate:
\begin{enumerate}
     \item Comments in templates are removed. They are saved in the SAGE III AST, but
           likely in incorrect positions, and not within the template 
           (before or after the template declaration).  They are not lost; since they
           are retrieved using a separate lex pass internally within ROSE.  
           % It is not clear if we should bother with editing the comments back into the
           % templates any better than just letting them be output either before of after
           % the template declaration.
           When template declarations appear in AST form, they will be placed into the
           correct positions in the generated code.

     \item Options specific to templates can be classified as follows:
          \begin{itemize}
               \item No transformations on templates. \\
                     This is the first case to got working and it is the easiest case (to
                     some extent).  Template instantiation can be handled entirely by the
                     vendor compiler, making life simple for ROSE. We also don't have to
                     generate any template specializations.
               \item Transformations on templates. \\
                     This case can be separated into two separate cases. The second is harder
                     to handle than the first).
                    \begin{itemize}
                         \item Transformations on template functions (including member functions). \\
                               This case will force transformations to happen as the
                               templated functions are instantiated (and could not happen 
                               at any earlier phase).  The instantiation can happen
                               at an earlier stage than prelinking if we force
                               auto-instantiation of templates (often triggered
                               automatically if the program in represented by a single 
                               translation unit).
                         \item Transformation of template classes. \\
                               This case is discussed in more detail below. It is a much harder
                               case, and is currently incomplete.
                         \item Transformation of template static data members. \\
                               This case is not handled yet, but should not be much trouble.
                    \end{itemize}
               \item Transformation of template specializations. \\
                     ROSE generates all template instantiations internally as template
                     specializations. As such they are no different from any other AST
                     subtree and all ROSE mechanism can be used for analysis and
                     transformation of the instantiated template.  Those instantiated 
                     templates that are transformed are marked for output in the 
                     code generation phase and output as template specializations.
                     In this approach, templates instantiated for different types
                     may be easily transformed differently.
                   % I'm not clear on how to handle this yet, IR nodes still have to be
                   % added to SAGE to handle explicit template specializations.  I will 
                   % worry about this case after we have the more basic templates working.
                   % I do not think it is much of a problem as I think it nests into the
                   % general handling of generated template instantiations (since they are
                   % generated as specializations).  However, partial specialization might
                   % get a bit tricky, though likely, through user defined transformation, 
                   % the user will get exactly what he specifies and so perhaps in a cruel 
                   % sense what he/she deserves.
          \end{itemize}

     \item Transformation of templated classes is enabled via generated specializations. \\
           This was discussed briefly above (under {\em Options specific to templates:
           item Transformation of template specializations} above).  In general,
           transformations on template classes, functions, and static data members are
           handled through the explicit generation of specializations in place of the 
           instantiations that would be generated by the back-end vendor compiler.  All 
           templates are explicitly generated as specializations in ROSE and, in principle, 
           no instantiations are required by the back-end vendor compiler.  It is not clear
           if ROSE needs to be so aggressive in eliminating template instantiations by the
           back-end vendor compiler, but doing so allows all template instantiations to be
           made available for transformation using ROSE.  For simplicity, we only output
           (within code generation) those template instantiations that are required, due 
           to transformations, and 
           allow the back-end compiler to generate as many of the required template 
           instantiations as possible.

           In order for a template to be transformed, we must save it into the 
           SAGE III AST. If it is a class template, then we only want to unparse 
           it into the final transformed code if it was modified. Otherwise its
           member functions and static members will not be defined at link time.
           Fundamentally, specialization of a class disqualifies the instantiation of its 
           member functions from the original template declaration, because the newly 
           instantiated template class becomes a distinct and separate class no longer 
           associated with the original template. The vendor 
           compiler can generate code for the new template class using the original
           template declaration or the member functions associated with the original 
           template declaration. All the functions must be generated to go
           along with the new, specialized form of the templated class, which we had to 
           specialize to permit it to be transformed.

           This potentially massive generation of all the member functions of a class
           applies only to transformations on class templates.  Transformations on member
           function templates are not affected.  They are instantiated in the prelink stage
           and seen in the SAGE III AST at that time. They can be transformed in the prelink
           stage, during, or immediately after the instantiation. This is the earliest possible
           stage where transformation on instantiated templates can be done.  A transformation
           on a templated function is handled as a transformation on each of its instantiations.
          \begin{itemize}
               \item Generation of code for transformed templated class. \\
                     If a class template is modified, then we have to unparse all of the 
                     templated member functions!  This is because an instantiated template
                     cannot force its member functions to be instantiated (unless we do
                     it explicitly, as I understand the template prelinking mechanism). Unparsing
                     the instantiated template (with a mangled name) or as a specialization
                     causes it to be considered as a new class and forces the
                     construction of all member functions. This is a slightly different concept
                     than instantiation (closer to specialization, I think, since
                     specialization is not automated and must be handled explicitly as a 
                     result). Details are discussed earlier in this section.

                     The declaration of a template class as a
                     specialization requires declaration and definition of member
                     functions of the class because the template mechanism would permit 
                     them to be different (even though this seems redundant, and even if
                     we can automate the construction of the member function by automating 
                     the declaration of specializations for all member functions and
                     static member data).

                     This mechanism needs to be controlled, so that we can control the
                     amount of code generated.  Options are:
                     \begin{itemize}
                         \item Always generate used class templates AND the member
                               functions upon which they depend.  This might seem to
                               require that we generate all possible code, though in general
                               it is only slightly less than all the member
                               functions minus the template class definition.  So maybe
                               this is a suitable option, but not in the current plan.
                         \item Generate only class template instantiations that have been 
                               transformed. Then generate
                               all the member functions upon which they depend. This is
                               the current design within ROSE.
                         \item Generate only the function template instantiations that
                               have been transformed (currently all function template
                               instantiations are generated, since we can't know in
                               advance which ones the user might wish to transform).
                     \end{itemize}
                     Note that if a template is never used in a given translation unit, 
                     then we will not instantiate it, and we can't even allow the user to 
                     see it for a possible transformation.  This is not much different
                     than existing vendor compilers that would not instantiate the template 
                     unless it was required by at least one translation unit.  It can be 
                     argued that the ability to transform templated functions and classes 
                     that are never used by an application is inherently meaningless. As is
                     the case for any vendor compiler, if the user wants to force
                     instantiation of all templated classes, functions, and static data 
                     members, then he or she can do so by including a test code 
                     that forces the explicit instantiation of every class, function,
                     static data member (or using explicit template instantiation directives).

                     If a class template has been modified then we need to make sure that all the 
                     class definition,
                     member functions, and 
                     static data members
                     are instantiated (on the next pass through the prelinker). The process should 
                     involve a call to the EDG function:
                     \begin{itemize}
                          \item {\tt void set\_instantiation\_required\_for\_template\_class\_members (a\_type\_ptr class\_type)}
                     \end{itemize}

                   % I am not currently sure how to make this happen, but it should involve the 
                   % *.ti files (I guess), and might require a modification to the EDG prelink
                   % mechanism, if it is not as simple as just generating the mangled
                   % member function names and placing them in the *.ti file (which might
                   % be all there is to it).
          \end{itemize}
\end{enumerate}

\subsection{Different Modes of Template Instantiation}
   
    We first supported only a single mode of template instantiation. Later we will
consider supporting additional modes later.
% other modes to perhaps support at a later date.  
ROSE will respond to the EDG options to control automatic template instantiation using the
option {\tt -edg:t{\it mode}},
where the mode is either:
\begin{enumerate}
   \item none (default) \\
         No template instantiation will be done.
   \item used \\
         Only templates that are used in the translation unit will be instantiated.
   \item all \\
         All possible templates will be instantiated.
   \item local \\
         Only used templates will be instantiated and they will be forced to be local to
         the file.  All instantiated functions will be declared as {\tt static}.  Note that
         {\tt static} functions and member functions are only seen by the local file scope
         (translation unit, typically the source file).
\end{enumerate}


\commentout{
\subsection{Problems with the EDG/ROSE Template Instantiation Approach}

    First lets mention that EDG does a great job and that the problems that
we have within template instantiation in ROSE are a result of trying to
convert template instantiations in EDG to be formal template specializations 
in ROSE.  This approach presents a better user interface and is substantially 
different to what EDG provides.

    For member functions EDG adds to the EDG AST the instantiations of the member functions
that are required at the end of the file.  Note that it does not declare the functions
before hand except in the instantiated class (which ROSE suppresses the output of so that
the template can be used instead and template specializations added).  Since we convert
all template member functions to specialized template member functions add the
required forward declaration for the specialized template member function after the
template declaration. This does however not allow the explicit representation of the
specialized template of the templated class (which would also have the prototype
declaration contained in it).

   We should perhaps output the specialized version of the templated class to be
uniform.  This would allow the user to do transformations on the template class
where it was instantiated with specific template arguments.  We currently look to see if
the templated class has been transformed and only then output the class (which would then
be redundant with the forward declaration of its member function!!!

  If I output all instantiations then I don't need to build forward declarations!
If I suppress class declarations then I need to output forward declarations of the member
functions that are used (so that we can output the instantiated member functions at the
end of the file).

We need at least a couple of template instantiation modes:
\begin{enumerate}
   \item Transformations on templates \\
         default should be to present an AST without the capability to transform
         templates.  However, due to EDG dropping the inline member function template
         definitions we would have to compromise and still present the member functions
         of templated classes as specializations.  Later, perhaps we can fix this.

   \item none (default) \\
         No template instantiation. This is a problem since member functions are lost as
         template declarations in the AST.  So we generate code that will not link
         (error).  If we could get the template declaration directly out of the source
         code then we could perhaps use it as a text string, but we would have to get it
         after macro expansion for safety and even then it could be a problem. In general
         if all templates are instantiated a link time then we have fewer problems.
   \item used (same as EDG) \\
         Output all instantiations that are used, this would cause link time problems with
         multiply defined symbols.
   \item all (output all template instantiations). \\
         Assume everything is transformed and output all template instantiations.
   \item local (same as EDG) \\
         Output only those template instantiations used and do so as static functions.         
\end{enumerate}
}


\commentout{
\section{Prelinking Template Instantiation}

    Prelinking is a strange thing for a source-to-source project to endure.
Basically we wish to support the analysis and/or transformation of templates
within C++.  As templates, the uninstantiated code is stored internally as a 
string, accessible, but not pleasant to deal with for analysis or transformations
(unless you parse it yourself, but that is part of the point of using ROSE).  
So any analysis and transformation of templates can happen only on the 
instantiations of the templates.

{\indent
{\mySmallFontSize

\begin{verbatim}
// this is a string within the SgTemplateDeclaration
template <class T> class X { T t; };
// specialization of template
template <> class X<float> { int t; T f; };
// variable declaration of type X<int> which is an instantiation
X<int> x;
\end{verbatim}
}}

\subsection{Why the prelinker phase}

    We want to support transformation on templates, so we have to build instantiations.
Only the instantiations are represented in the ASTs, template declarations are stored 
only as text strings.
If we were not to support any transformations on instantiated templates then we could 
avoid the prelinking and so this step will be optional (in its final form ROSE will
present this as an option, or you can obtain this trivially by just using the vendor
compiler to link, and skip the use of ROSE for this phase).  The rest of this section 
assumes that this optional behavior is in effect and that the users wants to do
transformations on the instantiated templates.

   Because users expect to link their final codes (if they do), we have to provide only
one template instantiation; globally across all files in a project.  The only time
that ROSE sees all the files in a project is at link time, so exactly as with every other
C++ compiler, we have to instantiate templates at link time (called prelinking) to avoid 
multiple definitions appearing to the linker.  ROSE only handles the prelinking required 
to instantiate the templates, not the final linking itself (which is done using the vendor 
compiler).

\subsection{Why do we compile the generated source instead of the original source within the prelink phase}
   When prelinking we must process (recompile) the ROSE generated files, because they 
contain the instantiations (as specializations) that must be seen by the prelinker to 
trigger the template instantiation mechanism to stop trying to instantiate the template
instantiation (as a class instead of as a specialization).  If we proces the original input file
then we would trigger the mechanism (in the EDG front-end) for the front-end to ask for a template to be
instantiated to satisfy what only an instantiation or a specialization could satisfy.
When a template instantiation is requested, only an instantiation (class with mangled name)
or a template specialization will satisfy it. 
Since either a template instantiation or a specialization will work, we must process
the ROSE generated code since it contains the specialization (ROSE turns all template
instantiations into template specializations to support transformations upon them).
Only the ROSE generated code contains a specialization, the original code contains 
neither.  Thus ROSE must within the prelinking stage process its own generated 
source code (not too much of a problem).

\subsection{Tutorial}

   The illustrate the prelink process we can consider an example program (file.C):

{\indent
{\mySmallFontSize

\begin{verbatim}
template < class T >
void foo ( T u )
   {
     u++;
   }

int main()
{
  foo(2);
  return 0;
}
\end{verbatim}
}}

Which contains a template function {\tt foo} and a use of the template function
instantiated with an integer. As a single file, template instantiation could be done
without prelinking using {\tt auto\_instantiation } and specifying the instantiation
mode to be either {\tt all} or {\tt used}; template instantiation at compile-time.  
But we will assume that this is part of a larger application using many files and 
present the approach using template instantiation at link time (prelinking).


After compilation, before prelinking invokes any recompilation passes,
the EDG front-end generates a {\em template instantiation} file (file.ti).
which contains the following information:

{\indent
{\mySmallFontSize

\begin{verbatim}
flg:foo__tm__2_i__FiZ1Z_v:TC
\end{verbatim}
}}

ROSE then reads this information and for each symbol included the compiler command line,
the directory where to do the compilation, and file to compile to generate that symbol.
The symbols represent possible templates that {\em could} be instantiated (if requested 
in the template {\em instantiation request} file (rose\_file.ii), discussed in more 
detail below).  ROSE generates a new {\em template instantiation} file (rose\_file.ti),
and deletes the {\em template instantiation} file generated by EDG (file.ti).  Thus ROSE
is consistant is using the {\em rose\_} prefix for all ROSE generated files.  With
the information in the {\em template instantiation} file (rose\_file.ti), for each source
file in a project consisting of multiple files, we are ready to prelink.  At this point
we have only the ROSE generated {\em template instantiation} file (rose\_file.ti) and the
ROSE generated source file (rose\_file.C).

Example {\em template instantiation} file (rose\_file.ti):

{\indent
{\mySmallFontSize

\begin{verbatim}
cmd:./translator -rose:verbose 2 --edg:restrict --edg:export -I/home/dquinlan/ROSE/NEW_ROSE/tests/nonsmoke/functional/CompileTests/A++Code -c -rose:prelink
dir:/home/dquinlan2/ROSE/LINUX-3.3.2/developersScratchSpace/Dan
fnm:/home/dquinlan2/ROSE/LINUX-3.3.2/developersScratchSpace/Dan/rose_test2005_74.C
flg:foo__tm__2_i__FiZ1Z_v:TC
\end{verbatim}
}}

In this case {\tt translator} (referenced in the command line {\tt cmd:...}) is the 
translator build using ROSE. Note that the {\tt -rose:prelink} option is used to signal 
to ROSE that some special handling is required.  This information ia available to
translators build using ROSE (see bool SgFile::isPrelinkPhase()).

   The prelink phase starts with:
\begin{enumerate}
   \item reading the object files on the command line and associating them with *.ti files.  
   \item building an internal data base of all the possible templates that could be instantiated.
   \item running the unix {\tt nm} command on each object file and saving the output.
   \item parsing the output of the saved {\tt nm} command
   \item matching undefined symbols with symbols that could be generated from instantiated templates
   \item generating the template {\em instantiation request} file (rose\_file.ii)
\end{enumerate}

The template {\em instantiation request} file (rose\_file.ii) is built by the prelinker 
to communicate to the EDG front-end what templates should be instantiated during the 
recompilation of each file.  Only the files containing the template declaration
(the first one if it could be more than one) are recompiled.  An example template 
{\em instantiation request} file (rose\_file.ii) appears as:

{\indent
{\mySmallFontSize

\begin{verbatim}
foo__tm__2_i__FiZ1Z_v
\end{verbatim}
}}

   The prelinker at this point knows:
\begin{enumerate}
   \item how to build any symbol required (all command line information is known),
   \item knows what symbols are required (from running {\tt nm})
   \item has specified what symbols should be built (templates to be instantiated)
\end{enumerate}

  The prelinker builds each requested symbol (template instantiation) by recompiling
the file associated with that symbol and the EDG front-end then reads the template 
{\em instantiation request} file (rose\_file.ii) and instantiates the template
in the AST as required.  ROSE translates the instantiated template in the AST into
a template specialization and generates the template specialization in the 
at the bottom of the file containing generated source code (a forward declaration
for the class or function is also placed at immediately after the template declaration).
This ROSE generates the file (rose\_file.C):

{\indent
{\mySmallFontSize

\begin{verbatim}
template < class T >
void foo ( T u )
   {
     u++;
   }

/* ROSE generated template specialization prototype */
template <> void foo(int y,int u);

int main()
{
  foo(2);
  return 0;
}

/* ROSE generated template specialization explicit definition */
template <> 

void foo(int u)
{
  u++;
}
\end{verbatim}
}}

\subsection{Template Instantiations Generated as Template Specializations}
   A template instantiation appears as an otherwise normal class, function, or member
function with a mangled name (e.g. {\tt void foo\_\_tm\_\_2\_i\_\_FiZ1Z\_v(i);})
In contrast a template specialization appears as a template using specific
statically defined (resolvable) template arguments (e.g. 
{\tt template <> void foo(int u)}).  ROSE internally represents all 
template instantiations as template specializations.  This is done 
because it permits a simpler recognition of the instantiated template and it's template
arguments within the AST (else parsing of mangled names would be required).  The output 
of the template as a specialization is also more satisfying (to unimportant if no
one reviews the generated source code).


\subsection{Controlling the Template Instantiation}
   Although tests are possible to test the generation of {\em all} required 
templates instantiations; it is only required for ROSE to generate template
instantiations (as specializations) for those instantiated templates that are
transformed.  ROSE internally records which templates are transformed
and only generates specializations for those cases (except for inline member
templates which are always generated as required).


\subsection{Summary of Template Instantiation Steps}
   In summary the steps associated with template instantiation are:
\begin{enumerate}
   \item initial compilation of each source file to generate: 
   \begin{enumerate}
      \item translated source file (rose\_file.C),
      \item the {\em template instantiation} file (rose\_file.ti),
      \item the object file (file.o)
   \end{enumerate}
   \item Prelink Phase:
   \begin{enumerate}
      \item reading the object files on the command line and associating them with *.ti files.  
      \item building an internal data base of all the possible templates that could be instantiated.
      \item running the Unix {\tt nm} command on each object file and saving the output.
      \item parsing the output of the saved {\tt nm} command
      \item matching undefined symbols with symbols that could be generated from instantiated templates
      \item generating the template {\em instantiation request} file (rose\_file.ii)
   \end{enumerate}
   \item call the vendor compiler's linker to link all the object files \\
         At this point all templates could have been instantiated, this behavior will
         ultimately be permitted to be controlled by the user.
\end{enumerate}

Note that EDG also uses a {\em definition list file}  I am unclear what its significance
is in the template instantiation process (it is however documented in the EDG manual).
}


\section{Compiling ROSE-generated Code Using ROSE}

    These are a few notes about parts that might be difficult if they are 
encountered in code generated by ROSE (meaning that they had to first appear in 
an applications source code and the user wanted to run the generated code through ROSE
again [I can't imagine why]).  It is a rare but interesting possibility.

   There are only a few cases where we generate code that might be a problem to compile
using ROSE. When compiling for g++ (default), ROSE generates code that will avoid specific
bugs in g++:
\begin{enumerate}
 % \item throw modifiers taking parameters (check on this detail). This is now fixed and
 %       should not be an issue.
   \item static const data members defined in the class definition (floats only)
         EDG accepts static and g++ supports const, and neither accepts what the other
         considers correct. ROSE generates code specific for the back-end and so the
         back-end must be specified in when running {\tt configure} for ROSE.  We
         don't currently support EDG as a back-end, though we support Intel C++ as a
         back-end and they use EDG, so this should work.
         \fixme{verify the details here.}
 % \item are there any others ...
\end{enumerate}


\section{Correctness of AST}

    When processing the AST, traversing it or rewriting it, it is useful to understand
why things are the way they are in the AST's implementation.  This section attempts to 
outline the properties that constitute the correctness of the AST.
\begin{enumerate}
     \item Null pointers in the AST. \\
          In general, any null valued pointer is an error in the AST.  This is 
          a policy in SAGE III, and is dramatically different from SAGE II.
          Our push for this policy has been incremental over the years and remains
          somewhat incomplete.
     \begin{enumerate}
          \item Parent pointers. \\
               Pointers to parent nodes (available through the {\tt SgNode::get\_parent()} member
               function) in the AST are set/reset after construction of the AST. As a
               result of being set within a traversal of the AST, the parents perfectly 
               match the traversal's concept of the AST {\em as a tree}.  This point is
               important since the AST included edges that make it a directed graph, and
               it is the traversal of the AST 
               that gives it its form/representation as a tree.  Thus all parent pointers
               are valid (non-null) values, except the root of the AST, which has no
               parent (and has a null valued pointer returned from 
               {\tt SgNode::get\_parent()}.  There are two possible nodes that can be
               considered a root of the AST, either the {\tt SgProject} or the 
               {\tt SgFile}; both nodes have constructors that take a translator's command
               line arguments.

          \item Function declarations. \\
               Function declarations and function prototypes are confused in the AST,
               and where a function is defined, i.e. with a function body, it appears 
               in the AST as a function declaration {\tt SgFunctionDeclarationStatement} 
               with a pointer to a function definition 
               ({\tt SgFunctionDefinitionStatement}. A function prototype can have a null 
               valued pointer returned from its {\tt get\_definition()} member function
               and is marked explicitly as a function prototype (so that the null valued
               pointer can be error checked). If the function definition is available
               in the file (not always the case) then the {\tt get\_definition()} may 
               return a valid pointer 
               to it, even for a function prototype. Thus the explicit marking of
               declarations as a prototypes is critical to its interpretation as a function
               prototype.

          \item Pointers to {\tt SgBasicBlock}. \\
               All pointers of type SgBasicBlock should be valid pointers.

          \item Other {\tt NULL} pointers \\
               A conscious attempt is made within ROSE to not communicate information
               through a null-valued pointer.  Unfortunately, this has been a 
               switch from the original design of SAGE II, which had {\tt NULL} pointers
               throughout the AST.  In general within the newer work, any NULL pointer is
               currently an error.
     \end{enumerate}

     \item What lists can be empty. \\
          SAGE III uses STL lists internally; children on many IR nodes are contained
          in such STL lists.  There are nodes where the STL lists can be empty.  These 
          nodes include:
     \begin{enumerate}
          \item SgBasicBlock
          \item SgGlobal
          \item SgExpresionList
          \item SgNamespaceDeclaration
        % \item No others that I know of at present
     \end{enumerate}

     \item Which access functions are simple and which do meaningful computation. \\
          This question will be addressed later when we can automate queries of this sort.
          In general, member functions beginning with {\tt get\_xxx} and {\tt set\_xxx}
          get or set a private data member named {\tt p\_xxx}.  Most such functions are
          trivial access functions, but some have more complex semantics.  Given that
          there are over 200 IR nodes in the SAGE III IR, and that each has numerous
          member functions, we will defer addressing this question until we can implement
          a more automated mechanism on the SAGE III source code. 
          See the Doxygen generated documentation for more details on the IR
          nodes and their member functions.
\end{enumerate}


\section{AST Normalization: \\ Subtle Ways That ROSE Output Differs \\ from the Original Source Code}

   In general, every attempt is made to preserve the look and feel of the original input
code.  Original formatting, use of C preprocessor directives 
(e.g. {\tt \#include<file.h>}), and comments are preserved within the AST and output 
in the generate code.   However, there can be minor differences between the input 
source code and the code that is generated from ROSE translators. In all cases this 
difference is due to normalizations internally within the EDG front-end.  Current 
normalizations include:

\begin{enumerate}
     \item White space differences. \\
     ROSE-generated code will appear somewhat different due to slightly different
     uses of white space within formatting of the generated code.  All attempts
     are to preserve as much of the original formatting as possible (or practical).

   % This is not fixed in ROSE and all member functions are represented identically
   % to there position inside of outside of their class within ROSE.
   %  \item Member functions defined within a class. \\
   %  Although this could be changed within EDG, we currently use EDG with
   %  a (default) internal setting which defines all member functions outside
   %  of the class declaration of which they are a member.  This normalization makes
   %  little different and is only syntactic.  Note that this in now optional for
   %  non-template classes.  The default must be changed in EDG for this to work
   %  differently. The file is host\_envir.h and the macro
   %  FRIEND\_AND\_MEMBER\_DEFINITIONS\_MAY\_BE\_MOVED\_OUT\_OF\_CLASS should be set to FALSE.
   %  Clearly, access to the EDG source code is required for this.  The default within ROSE
   %  is FALSE, changed from previous versions of ROSE (prior to 0.8.3b and earlier).

     \item Variable declarations are normalized to separated declarations. \\
     Variable declarations containing multiple names (variables to be declared) are
     normalized within the AST to form one declaration for each name (variable).  This
     simplifies program analysis since it avoids one of two ways of searching for a
     variable declaration (as a separate declaration and as a member of a list in another
     declaration). As an example:
{\indent
{\mySmallFontSize

\begin{verbatim}
     int x,y,z;
\end{verbatim}
}}
     appears in the AST (and in the unparsed [generated] code) as:
{\indent
{\mySmallFontSize

\begin{verbatim}
     int x;
     int y;
     int z;
\end{verbatim}
}}
     This {\em feature} could be changed at some point, but it has not been a
     high priority (and may be more desirable than the alternative).

     \item Typedef template arguments are expressed in terms of their base type \\
     This is not something that we can fix or change. EDG simply represents 
     at least some and maybe all template arguments with their types normalized 
     to strip away all typedefs.  Fixing this would allow 
     generation of code that is easier to verify visually.  This may receive 
     some attention in the future.

     \item Comments within templates. \\
     Comments within templates are ignored and not reproduced in the generated
     source code.  This is because the template code is held in the AST as a string
     generated by EDG, and EDG ignores the comments.  We currently output the comments 
     at either the top or bottom of the template declaration.  Later then the template
     declaration is represented as an AST, the comments will be folded into place where
     they belong.
   % This is possible to fix, but could generate incorrect code, so it has not been done.

     \item Member functions of template instantiations. \\
     Member functions of template instantiations use the same IR node as 
     templated member functions of templated classes and templated member functions of
     non-templated classes.  This is because the reason why a 
     {\tt SgTemplateInstantiationMemberFunctionDecl} exists to store the pointer to the 
     {\tt SgTemplateDeclaration} and there is only one of these, either because
     \begin{enumerate}
       \item the template declaration is of the class and the member function is declared in
          the class, or
       \item the template declaration is of a member function of a templated class and is
          defined/declared outside of the class.  In this case, the member function can 
          be for a template or non-template member function, but not both.
     \end{enumerate}

   % This is now fixed.
   % \item Handling of {\tt sizeof({\em type})}. \\
   % The {\tt sizeof({\em type})} construct used to be substituted with the value if known at
   % compile-time.  This should not be normalized now, as a result of fix to ROSE
   % 2006-04-23 (see ChangeLog for details).

     \item Calls via dereferencing of function pointers. \\
     Function calls from dereferencing pointers to functions can be represented with two 
     different forms of syntax. For example:
{\indent
{\mySmallFontSize

\begin{verbatim}
     xPtr ();
     (*xPtr) ();
\end{verbatim}
}}
     appears in the AST (and in the unparsed (generated) code) as
{\indent
{\mySmallFontSize

\begin{verbatim}
     (*xPtr)();
     (*xPtr)();
\end{verbatim}
}}

     \item C++ style cast are normalized to C style casts. \\
     EDG appears to normalize all C++ style cases to C style casts. We are
     working on the analysis to backout where C style casts could in fact
     be C++ style casts of a specific classification: const\_cast, static\_cast,
     dynamic\_cast,and reinterpret\_cast.

     \item Floating-point literal normalization \\
     Floating-point literals are internally represented in EDG as float, double, or long
     double (dependent on the type), thus the exact string representing the floating point
     literal is lost.  We have modified EDG to save the string representation (from the
     token stream) the floating-point literal, this work is recent and handles all the
     different ways that floating point literals can be expressed (even including
     hexidecimal representation of floating point literals).  The value as a float, double,
     or long double is also stored explicitly in the AST to simplify forms of analysis.
     Constant folded values are stored in the AST as well, with full unfolded constant 
     expressions output in the generated code (by default), to reproduce the original 
     source code as much as possible.

     \item Normalization of member access from a pointer. \\
     Member function access can be represented with two different forms of syntax. For example:
{\indent
{\mySmallFontSize

\begin{verbatim}
     xPtr->foo();
     (*xPtr).foo();
\end{verbatim}
}}
     appears in the AST (and in the unparsed (generated) code) as
{\indent
{\mySmallFontSize

\begin{verbatim}
     xPtr->foo();
     xPtr->foo();
\end{verbatim}
}}
     The following code is normalized differently (and somewhat inconsistently):
{\indent
{\mySmallFontSize

\begin{verbatim}
     (*xPtrPtr)->foo();
     (**xPtrPtr).foo();
\end{verbatim}
}}
     appears in the AST (and in the unparsed (generated) code) as
{\indent
{\mySmallFontSize

\begin{verbatim}
     (*(*xPtrPtr)).foo();
     (*(*xPtrPtr)).foo();
\end{verbatim}
}}
     when operators are explicitly defined by the user, as in
{\indent
{\mySmallFontSize

\begin{verbatim}
     class A
        {
          public:
               A();
               A( int *x, int y);
               int & operator[](int i);
               A *operator->() const { return Aptr; }
               A& operator*() const  { return *Aptr; }
               A*  Aptr;
               A** Aptrptr;
        };
\end{verbatim}
}}
     The following code is normalized differently (and somewhat inconsistently):
{\indent
{\mySmallFontSize

\begin{verbatim}
      A a;
      A* aptr = &a;
      A** aptrptr = &aptr;

      aptr->operator[](1);
      (*aptr)[1];
      (*aptrptr)->operator[](1);
      (*(*aptrptr))[1];

      (aptr->Aptr)->operator[](1);
      (*(aptr->Aptrptr))->operator[](1);
\end{verbatim}
}}
     and appears in the AST (and in the unparsed [generated] code) as
{\indent
{\mySmallFontSize

\begin{verbatim}
     class A a;
     class A *aptr = (&a);
     class A **aptrptr = (&aptr);
     (*aptr)[1];
     (*aptr)[1];
     (*(*aptrptr))[1];
     (*(*aptrptr))[1];
     (*aptr -> Aptr)[1];
     (*(*aptr -> Aptrptr))[1];
\end{verbatim}
}}
     \item Normalization of const ref ({\tt const \&}). \\
           Const references, such as
{\indent
{\mySmallFontSize

\begin{verbatim}
           X<A const & > x3;
\end{verbatim}
}}
     are presently normalized to be
{\indent
{\mySmallFontSize

\begin{verbatim}
           X<const A & > x3
\end{verbatim}
}}
     \item Template arguments explicitly output. \\
          Template types are output with template arguments.
     Code such as:
{\indent
{\mySmallFontSize
\begin{verbatim}
          std::string var = std::string("");
\end{verbatim}
}}
     is normalized to be
{\indent
{\mySmallFontSize
\begin{verbatim}
          std::string var = std::basic_string < char , std::char_traits< char > , std::allocator< char > > ((""));
\end{verbatim}
}}

     \item Constructor calls are really variable declarations. \\
          C++ classes can define constructors. When they do the constructors 
     are represented in the AST as a member function declaration and marked
     specifically as a constructor (conversion operators and destructors are
     also member function declarations and marked explicitly).  However,
     the call to a constructor is a bit special in C++ and does not appear
     in the AST as a member function call. It appears as a variable declaration
     within the AST fragment representing the variable declaration a
     {\tt SgConstructorInitializer} is used.  So, where a variable of a class type {\tt X} 
     is written in the code as
{\indent
{\mySmallFontSize
\begin{verbatim}
           X variable;
\end{verbatim}
}}
     the form in the AST is more similar to the code represented by
{\indent
{\mySmallFontSize
\begin{verbatim}
           X variable = X();
\end{verbatim}
}}
     Semantically the two forms of code are equivalent (since the redundant constructor calls will
     be optimized away), and so this represents a form of normalization within the AST.

     \item Redundant casts and copy constructors. \\
     The use of redundant casts are represented as nested calls to copy constructors.
     Code such as:
{\indent
{\mySmallFontSize
\begin{verbatim}
          std::string arg5 = (std::string) (std::string)std::string("");
\end{verbatim}
}}
     is normalized to be
{\indent
{\mySmallFontSize
\begin{verbatim}
          std::string arg5 = 
               std::basic_string < char , std::char_traits< char > , std::allocator< char > > 
                    (std::basic_string < char , std::char_traits< char > , std::allocator< char > > (("")));
\end{verbatim}
}}

     \item Array indexing represented as pointer arithmetic. \\
           Array indexing is translated by EDG into pointer arithmetic.  It
    is not clear if this specific sort of AST normalization is desirable.
     Code such as:
{\indent
{\mySmallFontSize
\begin{verbatim}
void foobar ( double *d1, double *d2 );
void foo()
   {
     double **array;
     int n;
     array[n] = new double[100];
     foobar(&(array[n][n]),&array[n++][n]);
   }
\end{verbatim}
}}
     is normalized to be
{\indent
{\mySmallFontSize
\begin{verbatim}
void foobar ( double *d1, double *d2 );
void foo()
   {
     double **array;
     int n;
     array[n] = new double[100];
     foobar (array[n] + n,array[n++] + n);
   }
\end{verbatim}
}}

\item Case statements always have an attached SgBasicBlock object.

\item Qualifiers are often normalized to longer names since they are computed on-the-fly
    as needed during unparsing.  The original qualified names are lost and, as a result, the 
    generated types can be excessively long and not at all similar to the original source
    code.   For example, the {\tt STL map::const\_iterator} can become:
    $ std::_Rb_tree < std::map < int , int , std::less< int > , std::allocator< std::pair<
    const int , int > > > ::key\_type , std::map < int , int , std::less< int > ,
    std::allocator< std::pair< const int , int > > > ::value_type , std::\_Select1st<
    std::map < int , int , std::less< int > , std::allocator< std::pair< const int , int >
    > > ::value\_type > , std::map < int , int , std::less< int > , std::allocator<
    std::pair< const int , int > > > ::key\_compare , std::allocator< std::pair< const int
    , int > > > ::const\_iterator $

       This problem could be fixed by computing a style alias table to permit the shortest
    type name to always be used.  Either that or we should explicitly store the lists of 
    qualified names and recompute them only where transformations have been
    done.


\item Unnamed typedefs of enums are normalized to enums. \\
     EDG appears to normalize unnamed typedefs to be enums, and the information about the 
     origin as an unnamed typedef is lost.  Since there appears to be no difference in the
     EDG AST, ROSE is unable to recover when the {\em typedef} keyword was used.  This is not
     a real problem and the semantics of the application is the same.  Without
     the name of the typedef, the typedef type can't be referenced except through its tag
     name.  However, since there are subtle ways in which the tag name is not a type name in
     C (requires the {\em struct} keyword), this could be an issue for C. I have not
     isolated a code to demonstrate this as a problem. Thus, within ROSE, code such as:
{\indent
{\mySmallFontSize
\begin{verbatim}
     typedef enum enumType { zero = 0, one, two };
\end{verbatim}
}}
     is normalized to be
{\indent
{\mySmallFontSize
\begin{verbatim}
     enum enumType { zero = 0, one, two };
\end{verbatim}
}}
     This is demonstrated in {\tt test2005\_188.C}.

\item Packing pragmas: {\em \#pragma pack} normalizations. \\
\label{pragma_pack_normalization}
     The use of packing pragmas is handled separately from other pragmas within ROSE.
     Most pragmas are strings and no special processing is done internally.  Packing
     pragmas assume a stack based semantics and allow:
{\indent
{\mySmallFontSize
\begin{verbatim}
     #pragma pack(n)      // Sets packing alignment to value n = 1,2,4,8,16, ... powers of 2
     #pragma pack(push,n) // Push previous packing alignment value and set new value to n
     #pragma pack(pop)    // Use previously pushed value of packing alignment
     #pragma pack(push)...#pragma pack(n)...#pragma pack(pop) // Alternative to #pragma pack(push,n) and #pragma pack(pop)
     #pragma pack()       // resets to packing alignment selected by compiler (default value)
\end{verbatim}
}}
     ROSE will normalize this to explicit packing pragmas for each structure (translating
     the {\em pack(push,n)} and {\em pack(pop)} to explicit values (using {\em pack(n)}).
     The reasons this is done is because this is that EDG stores the packing alignment
     values directly with the data structure and does not represent the
     pragma explicitly.
     Generated code using ROSE thus only uses {\em \#pragma pack(n)} and 
     {\em \#pragma pack()} explicitly for each structure declaration (before and 
     after each declaration, respectively). The specific placement of the 
     {\em \#pragma pack()} is also modified so that it appears immediately before and
     after the opening and closing parents for the class or structure definition.
     As an example, the following code as input:
{\indent
{\mySmallFontSize
\begin{verbatim}
#pragma pack(4)
struct A { unsigned short a; };
#pragma pack(push,8)
struct B1 { unsigned short a; };
struct B2 { unsigned short a; };
#pragma pack(pop)
struct C { unsigned short a; };
#pragma pack(push,1)
struct D { unsigned short a; };
#pragma pack(2)
struct F { unsigned short a; };
struct G { unsigned short a; };
#pragma pack(pop)
struct H { unsigned short a; };
struct I { unsigned short a; };
#pragma pack()
struct J { unsigned short a; };
\end{verbatim}
}}
will be translated (normalized) to
{\indent
{\mySmallFontSize
\begin{verbatim}
struct A 
#pragma pack(4)
{ unsigned short a; }
#pragma pack()
;
struct B1 
#pragma pack(8)
{ unsigned short a; }
#pragma pack()
;
struct B2 
#pragma pack(8)
{ unsigned short a; }
#pragma pack()
;
struct C 
#pragma pack(4)
{ unsigned short a; }
#pragma pack()
;
struct D
#pragma pack(1)
{ unsigned short a; }
#pragma pack()
;
struct F
#pragma pack(2)
{ unsigned short a; }
#pragma pack()
;
struct G
#pragma pack(2)
{ unsigned short a; }
#pragma pack()
;
struct H 
#pragma pack(4)
{ unsigned short a; }
#pragma pack()
;
#pragma pack(4)
struct I { unsigned short a; }
#pragma pack()
;
struct J { unsigned short a; };
\end{verbatim}
}}

\item Expressions in C++ {\tt typeid()} construct. \\
   Expressions within are sometimes normalized.
This is an example of input code using the typeid() operator:
{\indent
{\mySmallFontSize
\begin{verbatim}
#include <iostream>
#include <typeinfo>
using namespace std;

struct A { virtual ~A() { } };
struct B : A { };
struct C { };
struct D : C { };

void foo() {
  B bobj;
  A* ap = &bobj;
  A& ar = bobj;
  cout << "ap: " << typeid(*ap).name() << endl;
  cout << "ar: " << typeid(ar).name() << endl;
  D dobj;
  C* cp = &dobj;
  C& cr = dobj;
  cout << "cp: " << typeid(*cp).name() << endl;
  cout << "cr: " << typeid(cr).name() << endl;
  cout << "expression: " << typeid(true && false).name() << endl;
  bool t,f;
  cout << "expression: " << typeid(t && f).name() << endl;
  int less,more;
  cout << "expression: " << typeid(less < more).name() << endl;
  cout << "expression: " << typeid(less | more).name() << endl;
  cout << "expression: " << typeid(less + more).name() << endl;
\end{verbatim}
}}

This is the associated output code using the typeid() operator 
(with some reformatting)
{\indent
{\mySmallFontSize
\begin{verbatim}
#include <iostream>
#include <typeinfo>
using namespace std;

struct A { virtual inline ~A() {} };
struct B : public A {};
struct C {};
struct D : public C {};

void foo() {
  struct B bobj;
  struct A *ap = (&bobj);
  struct A &ar = bobj;
  ( *((&std::cout))<<"ap: "<<(typeid(( *ap))).name())<<std::endl;
  ( *((&std::cout))<<"ar: "<<(typeid(ar)).name())<<std::endl;
  struct D dobj;
  struct C *cp = (&dobj);
  struct C &cr = dobj;
  ( *((&std::cout))<<"cp: "<<(typeid(C )).name())<<std::endl;
  ( *((&std::cout))<<"cr: "<<(typeid(C )).name())<<std::endl;
  ( *((&std::cout))<<"expression: "<<(typeid(bool )).name())<<std::endl;
  bool t;
  bool f;
  ( *((&std::cout))<<"expression: "<<(typeid(bool )).name())<<std::endl;
  int less;
  int more;
  ( *((&std::cout))<<"expression: "<<(typeid(bool )).name())<<std::endl;
  ( *((&std::cout))<<"expression: "<<(typeid(int )).name())<<std::endl;
  ( *((&std::cout))<<"expression: "<<(typeid(int )).name())<<std::endl;
}
\end{verbatim}
}}

  Notice that not all expressions are normalized, and that the cases 
which are normalized vs. those which are not is very subtle. This 
normalization appears to be a result of the internal working of EDG 
and not the Sage III IR.  This test code can be found in {\tt test2006\_95.C}.

\end{enumerate}


\section{Non-Standard Features: \\ C++ Extensions That We Are Forced to Handle}

    Philosophically, I don't think much of language extensions. we don't add any, 
we don't think we should add any, and we would not trust ourselves to add them correctly.
That having been said, there are a few C++ extensions that are introduced by
EDG (only one that I know of) and a fair number by g++.  Because in many cases these features 
are implemented differently, we find them all worth avoiding.  However,
some applications use them, so we are somewhat forced to support them and handle the 
differences between how they are supported within both the EDG front-end and the
back-end compiler (most often GNU g++).  We list specific non-standard features of
C++ that we are forced to handle (because applications we compile mistakenly use them).

One non-standard feature that requires special handling in ROSE is the
in-class initialization of static const non-integer types. In-class initialization 
refers to code such as:
{\indent
{\mySmallFontSize
\begin{verbatim}
     class X
        {
          public:
               static const int integerValueConstant  = 42;   // Legal C++ code
               static const int integerValueConstant  = 42;   // Legal C++ code
               static const bool booleanValueConstant = true; // Legal C++ code
               static const char charValueConstant    = '\0'; // Legal C++ code

            // Illegal C++ code (non-standard, does not compile with EDG, but does with g++)
               static const double doubleValueConstant1 = 3.14; 
            // Illegal C++ code (non-standard, but compiles with EDG, and does not with g++)
               const double doubleValueConstant2 = 3.14; 
        };
\end{verbatim}
}}
           and it applies to integer-based types only (why such types are special while float
           and double are not, I don't know). However, {\tt double} is somewhat supported 
           as a non-standard extension by both EDG and GNU g++ (though in different ways).
           This is a little corner of C++ which is truly obscure, but shows up in some
           large applications at LLNL.  Since the code that works with EDG does not work
           with GNU g++ (and vice versa),
           there is no common ground.  So we assume that the code will compile using EDG
           (we have no choice) and then generate code that will compile with GNU g++.
           This means that we generate C++ code that can't be compiled with EDG, but this 
           is the mess that application developers get themselves into when they use
           non-standard features.

           The fix-up of the AST to force the generation of code suitable to GNU g++ is
           handled in the {\tt ROSE/src/frontend/SageIII/astFixup} directory.

%\item I don't know of any other non-standard features that ROSE supports.


\section{Notes on ROSE-specific Header Files }

    We borrow the header files of whatever compiler is specified as the target back-end
    compiler.  This allows the same expansion of any macros as would be expanded without
    ROSE to match the expansion that would be done with ROSE.  The mechanism for borrowing
    the header files from the target back-end compiler is somewhat messy, but fully
    automated.  There are several steps, including translation and matching the values of
    the target compiler's predefined macros, to build a set of header files that can be
    used by ROSE (by the EDG front-end) from those used by the target back-end.
    The details are handled automatically, and need not be a concern for users of ROSE.
  % This section permits us to document some of this work for a more complete internal
  % reference.
    We use the {\tt --preinclude} mechanism in EDG to force a specific generated header file 
    to be read ahead of any ROSE system header files (translated from the back-end system
    header files by the ROSE {\tt configure} mechanism).  This head file contains
    all the back-end specific macros definitions. The file name is:
    {\tt rose\_edg\_required\_macros\_and\_functions.h} and is placed in the install
    tree ({\tt <prefix>/include/<back-end compiler name>\_HEADERS/}.

\commentout{
    Note that with versions of g++ before 3.4.x, predefined macros could be found
    automatically from {\tt g++ -v -E test.C} where test.C was any trivial program.
    Since this is no longer the case for g++ 3.4, the appropriate macros are specified 
    explicitly (and of course it is a little more difficult).

    Macros can be defined either within ROSE/config/rose-g++-headerfilefixup.h
    or on the commandline directly.  Most can be defined either location.
    Macros set incorrectly by system header files (such as for {\tt \_\_flexarr}
    must be set in ROSE/config/rose-g++-headerfilefixup.h and we are dependent 
    upon this file being read before any other head files which use this macro.
    (though we can only force it to be read before any compiler specific header file
    since we copy and modify these and not the platform specific header files in
    /usr/include)\footnote{This does not appear to be a problem in practice but if we are
    required to do so in the future we could also copy and modify the platform specific
    header files (though I would like to avoid doing so to make our work as platform
    independent as possible).  But copying and modifying the target compiler's header
    files we are in a position of being guite dependent upon specific version numbers of
    back-end compilers already).}  For greatest possible safety, most predefined macros are
    given their definition withn the specificaiton of the commandline internally to EDG
    (not really a commandline, but a string handed to the EDG front-end withi a function
    call interface).

    The full commandline internally required to EDG to make it emulate g++ is:
    $ -D\_\_GNUG\_\_=\$BACK-END\_GCC\_MAJOR -D\_\_GNUC\_\_=\$BACK-END\_GCC\_MAJOR
    -D\_\_GNUC_MINOR\_\_=\$BACK-END\_GCC\_MINOR -D\_\_GNUC\_PATCHLEVEL\_\_=\$BACK-END\_GCC\_PATCHLEVEL 
    -D\_GNU\_SOURCE -D\_\_extension\_\_= -D\_\_const=const -D\_\_attribute\_\_(arg)= -D\_\_restrict= 
    -D\_\_inline= -D\_GLIBCXX\_HAVE\_STDINT\_H -D\_GLIBCXX\_EXTERN\_TEMPLATE=0 -D\_\_null=0 
    -D\_\_builtin\_expect(x,y)=(x) -D\_\_PRETTY_FUNCTION\_\_=((\_\_const char *) 0) $

    \begin{enumerate}
        \item {tt \_\_flexarr} must be redefined since it is done incorrectly in
        /usr/include/sys/cdefs.h.  It is defined first as {\tt\_\_flexarr[]}, but it should be
        {\tt \_\_flexarr[1]} to compile properly with EDG.

        \item {\tt \_\_GNUC\_\_} for GNU must be defined.

        \item {\tt \_\_GNUC\_MINOR\_\_} for GNU must be defined.

        \item {\tt \_\_GNUC\_PREREQ(maj, min)} for GNU must be defined but will be in
        /usr/include/features.h if {\tt \_\_GNUC\_\_} and {\tt \_\_GNUC\_MINOR\_\_} are defined.

        \item {\tt \_\_const} for GNU must be defined (to ``const'').

        \item {\tt \_\_null} for GNU must be defined.

        \item {\tt \_\_builtin\_expect(x,y)=(x)} for GNU must be defined (used by assert macros).

        \item {\tt \_\_PRETTY\_FUNCTION\_\_} for GNU must be defined (used as a magic
        variable within GNU gcc).

        \item {\tt \_\_extension} for GNU must be defined to be empty.

        \item {\tt \_GNU\_SOURCE} for GNU must be defined so that other important macros
        will be defined internally.

        \item {\tt \_\_GNUC\_PATCHLEVEL\_\_} for GNU is defined but I'm unclear if it is
        required.

        \item {\tt \_\_inline} for GNU must be defined to be empty (empty might not be the
        best option here).

        \item {\tt \_GLIBCXX\_HAVE\_STDINT\_H} and {\tt \_GLIBCXX\_EXTERN\_TEMPLATE=0}
        must be defined to avoid code which uses GNU extensions and would not compile with EDG
        (not legal C++ code).

        \item The following are defined in the ROSE/config/rose-g++-headerfilefixup.h: \\
             \#define \_\_FLT\_HAS\_INFINITY\_\_ 1 \\
             \#define \_\_DBL\_HAS\_INFINITY\_\_ 1 \\
             \#define \_\_LDBL\_HAS\_INFINITY\_\_ 1 \\
             \#define \_\_FLT\_HAS\_QUIET\_NAN\_\_ 1 \\
             \#define \_\_DBL\_HAS\_QUIET\_NAN\_\_ 1 \\
             \#define \_\_LDBL\_HAS\_QUIET\_NAN\_\_ 1 

    \end{enumerate}
}

\section{Comments About Declarations (Defining Declarations vs. Nondefining
    Declarations) }

    Declarations come in two kinds: those that can have a separate definition (e.g class
and function declarations) and those that cannot (e.g. enum and pragma declarations).
For example, enums have to have their definition in their declaration; there is no concept of
forward declarations of enums in C or C++.\footnote{This is in spite of the fact that they
    are implemented in many compilers. They are not part of the C or C++ language, so 
they are not implemented in ROSE.  They are, however, one of the most common language
extensions to C and C++ compilers (even certain standard following front-ends such as EDG).}

A class declaration, in C++, can have a forward declaration (even repeated forward declarations)
before the declaration that contains the class definition (the {\tt \{\}} part).  Thus 
the following code is valid C++:
{\indent
{\mySmallFontSize
\begin{verbatim}
     class X;    // forward declaration (declaration with NULL pointer to definition)
     class X {}; // defining declaration (declaration with pointer to definition)
\end{verbatim}
}}
Note that multiple forward declarations can exist, as in:
{\indent
{\mySmallFontSize
\begin{verbatim}
     class X;    // first forward declaration
     class X;    // second forward declaration
     class X {}; // defining declaration
\end{verbatim}
}}
   The first forward declaration is the {\tt firstNondefiningDeclaration} within ROSE.
All forward declarations are marked as forward declarations (see declarations modifiers
documentation, {\tt isForward()} member function). The second forward declaration is just 
another declaration and should not be referenced as a firstNondefining declaration from
any other declaration. Its defining declaration is set in the AST fix-up phase.

The following code is legal, but particularly bothersome (it now works in ROSE):
{\indent
{\mySmallFontSize
\begin{verbatim}
     void foo (struct X *ptr);  // first declaration (but not really a forward declaration)
     class X;     // first or second forward declaration (not really sure if this is the first or second)
     class X {};  // defining declaration (one one of these is allowed, in the same scope)
\end{verbatim}
}}
   In this code example, the first declaration of X appears in the function parameter
list of the forward declaration of the function foo. This is not a typical forward 
struct declaration.  We keep track of which is the 
defining declaration and which is the first nondefining declaration; the information
about which is a forward declaration is somewhat redundant.  The unparser can't just 
use the result of {\tt isForward()} since declarations can be shared. This would
result in unparsing the class definition multiple times.  Thus, we separate
the two concepts of defining and nondefining. Defining declarations are never shared
(except through the {\tt definingDeclaration} pointer); only non-defining declarations
are shared (through the {\tt firstNondefiningDeclaration} pointer).

    SAGE III contains a {\tt SgDeclarationStatement} IR node from which all declarations IR nodes
are derived (e.g. {\tt SgClassDeclaration}, {\tt SgFunctionDeclaration}, etc.).  Contained in the
{\tt SgDeclarationStatment} IR node are pointers (accessed through corresponding {\tt get\_} and
{\tt set\_} member functions [access functions]) to the first declaration (called
firstNondefiningDeclaration) and the defining declaration (called
definingDeclaration).  Both of these pointers are used internally when a pointer is
required to a declaration (so that the same first declaration can be shared)
and within the unparser (most importantly to output the definition where it
appeared in the original code).

   These pointers are initialized in the EDG/Sage interface code and are in a few cases
(redundant forward declarations where only the first one is given a proper reference to
the defining declaration), fixed-up in the ROSE/src/frontend/SageIII/AstFixes.C (AST fix-up
phase). They are handy in transformations since they simplify how one can find a
declaration and the definition if it is required.

\section{Mangled Names and Qualified Names}

     Several C++ constructions (IR nodes) have qualified names.  These
are used to specify the location of the construct within the {\em space of names}
(we have avoided calling the {\em space of names} the {\tt namespace}, since that
is a specific C++ construct) presented by the C++ program.

     Note that none of the {\tt get\_mangled()} functions are called within the EDG/Sage 
translation (I think).  At least none are called directly!

   IR nodes that contain a {\tt get\_qualified\_name()} member function are:
\begin{itemize}
     \item SgEnumDeclaration
     \item SgTypedefDeclaration
     \item SgTemplateDeclaration
     \item SgNamespaceDeclarationStatement
     \item SgClassDeclaration
     \item SgTemplateInstantiationDecl
     \item SgMemberFunctionDeclaration
     \item SgScopeStatement
     \item SgGlobal
     \item SgBasicBlock
     \item SgNamespaceDefinitionStatement
     \item SgClassDefinition
     \item SgTemplateInstantiationDefn
     \item SgNamedType
\end{itemize}

   Mangled names are a mechanism to build unique mappings to functions, classes,
and any other constructs that could be identified using a non-unique string.
Mangled names should include the qualified names of any scopes in which they
are contained.

   IR nodes that contain a {\tt get\_mangled\_name()} member function are:
\begin{itemize}
     \item SgInitializedName
     \item SgStatement (all derived classes)
\end{itemize}

\commentout{
   IR nodes that contain a {\tt get\_mangled\_qualified\_name()} member
     function are:
\begin{itemize}
     \item SgClassDeclaration
     \item SgClassDefinition
\end{itemize}
}

\commentout{
So the same function from two different namespaces would mangle to the same name!
Since Sage contains a symbol table for each scope this is not an issue.  Where a 
compiler would try to translate to C, such functions with the same name but in two
different scopes would have to be given unique names and so the mangled name would have
be be prefixed by the mangled form of the qualified name.  Where the user requires
a unique string to represent a function he/she should form that from a concatenation of
the mangled name and the mangled form of the qualified name.  At some point we will
provide higher level interfaces to Sage III to support this. At the moment
we don't want to include such functionality within what is trying to be a minimalist
interface of Sage III.
}
   Note that mangled names include parts that represents the qualified name.
The algorithm used for name mangling is best described in the actual code
where the documentation should be clear. The code for this is in the 
SgType IR nodes (and its derived IR nodes).  The codes used for the operators is
present in the function {\tt SgType::mangledNameSupport(SgName,SgUnparse\_Info)}.

\section{Passing Options to EDG and ROSE}

    By default, all command line options (except EDG or ROSE-specific options) are 
passed to the back-end compiler.  As a result the command line for the compiler
can be used with any translator built using ROSE.  This is particularly effective
in allowing large complex {\tt Makefiles} to be used by only changing the name of the 
compiler ({\tt CC} or {\tt CXX}).

    Command line options are considered EDG-specified when prefixed with 
option: {\tt -edg:xxx}, {\tt --edg:xxx}, {\tt -edg\_parameter:xxx n}, or 
{\tt --edg\_parameter:xxx n}, which then translates to {\tt -xxx}, {\tt --xxx}, 
{\tt -xxx n}, or {\tt --xxx n} (respectively) for only the command line passed to 
the EDG front-end (not passed to the back-end compiler).  These are
required to support the different types of command line arguments used in EDG.
For a complete list of the EDG options, see the EDG documentation (available 
only from EDG and covered under their license to use EDG).

     Similarly, ROSE-specific command line options are prefixed using {\tt -rose:xxx}
and only interpreted by ROSE (not passed on to EDG or the back-end compiler).  To see 
a complete list use any translator build using ROSE with the option {\tt --help}.

     All other options are passed to the back-end compiler with no processing.


\section{How to Control Language Specific Modes: C++, C, C99, UPC}

   ROSE supports a number of different modes internally (within ROSE, the SAGE III
IR, and the EDG front-end).  There are five modes supported:
\begin{enumerate}
   \item C++ mode. \\
   \begin{enumerate}
      \item C++ mode (default). \\
        This mode is used when compiling all files when no command line options are specified.
      \item C++ (strict\_warnings) mode {\tt -edg:a}. \\
        This is the mode used when compiling with the {\tt -edg:a}, violations are issued
        as warnings. Note that currently, gnu builtin functions
        are not properly defined in strict modes (so they modes should not be used).
      \item C++ (strict) mode {\tt -edg:A}. \\
        This is the mode used when compiling with the {\tt -edg:A}, violations are issued
        as errors. Note that currently, gnu builtin functions
        are not properly defined in strict modes (so they modes should not be used).
        So these strict modes are incompatable with the use of the g++ and gcc compilers
        as a back-end to ROSE.
   \end{enumerate}

   \item C mode. \\
   \begin{enumerate}
      \item ANSI C (non-strict) mode. \\
        This is the mode used when compiling with the {\tt -rose:C\_only} C89 standard 
        (works best if files have ".c" filename extension). This implies conformance with 
        the C89 ANSI standard. Also equivalent to {\tt --edg:c} option.
      \item ANSI C (strict\_warnings) mode {\tt -edg:a}. \\
        This is the mode used when compiling with the {\tt -edg:a} in addition to the 
        {\tt --edg:c} or {\tt -rose:C\_only} options (file must have ".c"
        filename extension).  This implies conformance with the C89 standard,
        violations are issued as warnings.
      \item ANSI C (strict) mode {\tt -edg:A}. \\
        This is the mode used when compiling with the {\tt -edg:A} in addition to the 
        {\tt --edg:c} or {\tt -rose:C\_only} options (file must have ".c"
        filename extension).  This implies conformance with the C89 standard,
        violations are issued as errors. 
   \end{enumerate}

   \item C99 mode. \\
   \begin{enumerate}
      \item ANSI C99 default mode. \\
        This is the mode used when compiling with the {\tt --edg:c99} (file must have ".c"
        filename extension). This implies conformance with the C99 standard. This is 
        the same as using {\tt -rose:C99\_only}.
      \item ANSI C99 {\em strict} mode. \\
        This is the mode used when compiling with the {\tt -edg:a} in addition to the 
        {\tt --edg:c99} or {\tt -rose:C99\_only} options (file must have ".c"
        filename extension).  This implies conformance with the C89 standard,
        violations are issued as errors.
   \end{enumerate}
   Note that in ANSI C99, flexible array structures can not be data members of 
   other structures. See test2005\_189.c for an example.

   \item UPC mode. \\
        This is the mode used when compiling with UPC specific modifiers, use 
        {\tt --edg:upc}.  Note that we have modified the EDG front-end to support
        this mode for both C and C++ programs.  The generated code does not 
        support calls to a UPC runtime system at present, so this is just the 
        mode required to support building the translator for C or C++ which would
        introduce the transformations required to call a UPC runtime system (such
        as has been done for OpenMP by Liao from University of Houston).

   \item K\&R C {\em strict} mode. \\
        This is the mode used when compiling with the {\tt --edg:old\_c} (file must have ".c"
        filename extension).  This option will not currently work with ROSE because 
        prototyped versions of functions are used within
        {\tt rose\_edg\_required\_macros\_and\_functions.h} and these are not allowed in EDG's
        {\tt --old\_c} mode (translated from the ROSE {\tt --edg:old\_c}.
\end{enumerate}

  Most of the time the C++ mode is sufficient for compiling either C or C++ 
applications.  Sometimes the C mode is required (then, typically {\tt -rose:C\_only} is
sufficient).  The specific K\&R strict C mode does not currently work in ROSE. But
K\&R C will compile in both the C and often C++ modes without problem. For 
C99-specific codes (relatively rare), {\tt -rose:C99\_only} is sufficient.  On rare 
occasions, a greater level of control is required and the other modes can be used.

\subsection{Strict modes can not be used with g++ and gcc compilers as back-ends to ROSE}
   Note that currently, gnu builtin functions are not properly defined in strict 
modes (so they modes should not be used).  This is a problem for strict modes for 
both C and C++.

\subsection{Use {\tt *.c} filename suffix to compile C language files}
In general most C programs can be compiled using the {\tt -rose:C\_only}
independent of their filename suffix. However,
sometimes C program files that use a non *.c suffix cannot be handled by the
{\tt -rose:C\_only} option because they contain keywords from C++ as variable 
names, etc.  In order to compile these C language programs their files must use 
a *.c (lower case {\em c}) as a filename extension (suffix).  This is an EDG
issue related to the front-end parsing and the language rules that are selected
(seemingly independent of the options specified to EDG and based partly on the
filename suffix).  Fortunately most C language programs already use the lower
case {\em c}) as a filename extension (suffix).  Test code {\tt test2006\_110.c}
demonstrates an example where the {\em *.c} suffix is required.

\commentout{

# DQ (9/12/2006): This documentation is incomplete.

\section{Low-Level AST Rewrite Interface}

   SAGE III provides a simple interface for editing, similar to that 
provided within SAGE II but consistant in interface with the other 
higher-level interfaces to the rewrite mechanisms presenting in 
\ref{AstRewrite:AstRewrite}.

      incomplete-documentation

\section{Direct Use of SAGE III to Build Code Fragments}

   Some of the examples in the ROSE Tutorial demonstrate this 
low level handling of the AST.  Significant new work is being done
to simplify the direct construction of the AST.  This work is not
available yet.  It is not discussed in this draft of the manual.
}














