\chapter{ Developer's Appendix }

\label{developersAppendix:developersAppendix}
%Added by Liao 10/17/2008
\section{Working with the ROSE SVN repositories}
\label{gettingStarted::svn}
We have two subversion repositories for ROSE: an internal one at LLNL and
an external one at SciDAC Outreach Center (using a vendor drop scheme). 
Some tips for using them are gathered in this section. 
\subsection{The External Repository}
\begin{itemize}
\item Install your svn client ($>=$1.5.1 is recommended ) with
\textit{libsvn\_ra\_dav} support
(\htmladdnormallink{http://www.webdav.org/neon}{http://www.webdav.org/neon}
and \textit{--with-ssl}) or set the right \textit{LD\_LIBRARY\_PATH} for it
(\textit{libsvn\_ra\_dav-1.so}) if you encounter the following problem:\\
 \textit{svn: Unrecognized URL scheme for
 'https://outreach.scidac.gov/svn/rose/trunk'} 
\item To check out the main trunk, type: \\
\textit{svn checkout https://outreach.scidac.gov/svn/rose/trunk rose}
\item To check out a branch, type: \\
\textit{svn checkout
https://outreach.scidac.gov/svn/rose/branches/branch\_name rose} 
\item Merge the new updates of the main trunk into your working branch. 
Conceptually, svn merge works as two step: diff two revisions and merge the different into a working copy.
So you need to know two revision numbers of the main trunk: the first is
the latest revision number of the main trunk from which your branch was
created (or most recently synchronized);
the second is usually the head revision of the main trunk.
\begin{itemize}
  \item find the revision in which your branch was created or the last
  synchronization point with the trunk:\\ 
         \textit{svn log
         https://outreach.scidac.gov/svn/rose/branches/branch\_name} 
  \item cd local work copy of your branch, do the merge (overlapped merging
  seems possible using subversion 1.5.1), assume the last synchronization
  point(or originating point) is rev 56:\\
        \textit{svn merge --dry-run -r 56:69
        https://outreach.scidac.gov/svn/rose/trunk} \\
        \textit{svn merge -r 56:69
        https://outreach.scidac.gov/svn/rose/trunk} 
   \item Solve conflicts as needed.     
   \item svn commit: please record the start and end revision numbers of
   the main trunk being merged into the log to keep track of merging.   
\end{itemize}                
\item You can check the archive of email notifications of the svn commits
from \htmladdnormallink{https://osp5.lbl.gov/pipermail/rose-commits}{https://osp5.lbl.gov/pipermail/rose-commits} 
\end{itemize}


\subsection{The Internal Repository}
Again, make sure you are using subversion $>$ 1.4.x, 1.5.1 and up is recommended.
\begin{itemize}
\item Set local subversion configuration to ignore certain files and
automatically set file attributes (e.g. binary or text files ) during committing. 
A sample config file is available in the internal ROSE subversion repository:\\
  \textit{trunk/ROSE/scripts/subversion.config}
Please save it to your own .subversion/config before committing files. 
\item List content in the repository:
  \begin{itemize}
          \item list root info: \textit{svn list
          file:///usr/casc/overture/ROSE/svn/ROSE}
          \item list all branches: \textit{svn list
          file:///usr/casc/overture/ROSE/svn/ROSE/branches}
          \item list all tags: \textit{svn list
          file:///usr/casc/overture/ROSE/svn/ROSE/tags} 
   \end{itemize}       
\item Check out something:
   \begin{itemize}
          \item the main trunk: \textit{svn co
          file:///usr/casc/overture/ROSE/svn/ROSE/trunk/ROSE rose-svn} 
          \item a tag: \textit{svn co
          file:///usr/casc/overture/ROSE/svn/ROSE/tags/tag-name}
          \item a branch: \textit{svn co
          file:///usr/casc/overture/ROSE/svn/ROSE/branches/branch-name} 
          \item Either of these can be used on a separate machine (not CASC
          or LC) with LLNL VPN access by changing \textit{file:///} to
          \textit{svn+ssh://username@hostname/}. 
    \end{itemize}      
\item Merge the contributions of a branch from the SciDAC repository to the
internal repository: assume the branch is named \textit{testonly} and the
contribution is from r4 to r5, 
\textit{sourcetree} is the working copy of the internal repository (Subversion 1.5 works better than 1.4.x):
  \begin{itemize}
          \item run make check, make dist, make distcheck on the external branch before the merge
          \item \textit{svn status} to check modified and new files. 
          \item \textit{svn add file-name} to add new files if there are
          any.
          \item \textit{svn merge --dry-run -r4:5
          https://outreach.scidac.gov/svn/rose/branches/testonly
          sourcetree} 
          \item \textit{svn merge -r4:5
          https://outreach.scidac.gov/svn/rose/branches/testonly
          sourcetree}
          \item solve any possible conflicts alone the way
           \item svn commit: please record the start and end revision numbers of
         the external branch being merged into the log to keep track of merging.   
   \end{itemize}

\item branches   
    \begin{itemize}
    \item Add a new branch based on the head of the main trunk:\\
    \textit{svn cp file:///usr/casc/overture/ROSE/svn/ROSE/trunk/ROSE file:///usr/casc/overture/ROSE/svn/ROSE/branches/branch-name}
    \item Delete a branch: \\
    \textit{svn delete file:///usr/casc/overture/ROSE/svn/ROSE/branches/branch-name}
    \end{itemize}

\item Email notification: A perl script named post-commit (under
svn/text/hooks or svn/ROSE/hooks) keeps our email recipient list.\\
          \textit{/usr/casc/overture/ROSE/svn/ROSE/hooks/post-commit}
\item To upgrade, do \textit{svn switch} to new tag URL
\item Building distributions of ROSE MUST be done with \textit{svn export}. Otherwise, .svn directories are copied into the distribution trees. 

\end{itemize}



\section{Building ROSE from the Source Code Repository Checkout {\em (for developers only)}}
\label{gettingStarted:DeveloperInstructions}

     The instructions for building ROSE from SVN are a little more complex.
A few GNU software build tools are required (not required for the user
{\em ROSE Distribution, e.g. ROSE-\VersionNumber.tar.gz}.  Required 
tools ({\bf ** Note current dependencies}):
\begin{itemize}
     \item autoconf \\
          Autoconf version 2.53 or higher is required. % Autoconf can be problematic, some
          Newer versions of Autoconf introduce experimental features that could also be problematic.  The
          Autoconf development has not been particularly good at verifying compatibility
          with previous releases of their work.  Some users have reported having to
          install version 2.53 specifically to use ROSE. 
          {\em Check the ROSE/ChangeLog for current version numbers to be used with ROSE.}
     \item automake \\
          Automake version 1.9 or higher is required. Most software projects appear to be
          less sensitive to the specific version of automake.
          {\em Check the ROSE/ChangeLog for current version numbers to be used with ROSE.}
\end{itemize}

%   Several optional tools are also useful to have for ROSE development (required to build
% documentation) these are:
% \begin{itemize}
%     \item LaTeX
%     \item Doxygen
%     \item DOT
% \end{itemize}

The {\tt ROSE/ChangeLog} details the changes between versions of ROSE and lists the
specific version numbers of all software upon which ROSE depends. Comments of this type
appear in the {\tt ChangeLog} as:
\fixme{This needs to be updated.}
{\footnotesize
\begin{verbatim}
********* TESTED with **************
(*)  automake (GNU automake) 1.6.3
(*)  autoconf (GNU Autoconf) 2.57
(*)  GNU Make version 3.79.1
(**) g++ (GCC) 3.3.6
(**) gcc (GCC) 3.3.6
(*)  doxygen 1.3.8
(*)  dot version 1.12 (Sun Aug 15 02:43:07 UTC 2004)
(*)  TeX (Web2C 7.3.1) 3.14159
(*)  Original LaTeX2HTML Version 2002 (1.62)
(*)  sqlite (requires g++ 3.3.2) 3.2.1

(*)  Optional for use of ROSE (by users), but required for internal ROSE development (by ROSE project team)
(**) Required for use of ROSE (and for all internal development)
\end{verbatim}
}

The build process for a {\em Developer Version} is:
\begin{enumerate}
     \item Checkout a NEW version from SVN: \\
     The newest work on ROSE (as of March 2008) is using SVN, instead of CVS.
     This switch to SVN means the directions for how developers use ROSE have
     changed.  This effects developers of ROSE only (or anyone with access to 
     the newer SVN repository).
     {\bf Please learn about SVN on another project before using it on ours.}

     To checkout ROSE (assumes access to repository at LLNL), type: \\
     {\tt svn checkout file:///usr/casc/overture/ROSE/svn/ROSE/trunk/ROSE svn-rose} \\
     This will checkout a copy of the source code for ROSE from the svn repository.
     Any directory name can be used for {\tt svn-rose} in the example commandline.

     \item  Update an {\em existing} version from SVN: \\
     Run {\tt svn update} from inside the ROSE directory (at the top level) to update
     an existing version of ROSE with the new changes in the SVN repository.
%    \fixme{What are the SVN rules for pruning directories?}
%    Note that the ROSE team uses a common {\tt .cvsrc} file so that reasonable options
%    to prune empty directories are used uniformly within project development.

     \item After being checked out (or updated) from SVN: \\
     Run the {\tt build} script in the top level ROSE directory to build all configure scripts
     and {\tt Makefile.in} files (using {\bf automake}).  This is the difference between the development
     environment and the distribution. This script will call the different {\bf autoconf}
     tools required to setup ROSE and also checkout other work common to multiple projects
     within CASC.

%%     \item {\bf CVS} checkout \\
%%           Type {\tt cvs -d/usr/casc/overture/ROSE/ROSE2_Repository checkout ROSE} to
%%           checkout a new version of ROSE into a new directory.  {\em The ROSE directory
%%           should not exist in the current directory.}  Checking out a new version of ROSE
%%           on top of an existing version can lead to undefined results.
%     \item Checkout a NEW version from CVS: \\
%     In your {\tt .cshrc} file set the variable {\tt CVSROOT} to 
%     {\tt /usr/casc/overture/ROSE/ROSE2\_Repository}
%     ({\tt setenv CVSROOT /usr/casc/overture/ROSE/ROSE2\_Repository})
%     and ({\tt setenv CVS\_RSH ssh}).
%     The later is to permit checkout of the {\bf acmacros} project from its separate 
%     CVS repository.  Note that this will only work from a machine on the LLNL domain.
%     Other machines will have to get the separate tarball for this project (it should
%     also be in the {\bf CVS} repository and is checked out with ROSE into the top 
%     level ROSE directory).

%     Then run {\tt source ~/.cshrc} to have this environment variable set properly.
%     Now you can run {\tt cvs} to checkout the current version from ROSE. Type
%     {\tt cvs co ROSE} or {\tt cvs checkout ROSE} to do this.

%     \item  Update an {\em existing} version from CVS: \\
%     Run {\tt cvs update} from inside the ROSE directory (at the top level) to update
%     an existing version of ROSE with the new changes in the CVS repository.
%     Note that the ROSE team uses a common {\tt .cvsrc} file so that reasonable options
%     to prune empty directories are used uniformly within project development.

%     \item After being checked out (or updated) from CVS: \\
%     Run the {\tt build} script in the top level ROSE directory to build all configure scripts
%     and {\tt Makefile.in} files (using {\bf automake}).  This is the difference between the development
%     environment and the distribution. This script will call the different {\bf autoconf}
%     tools required to setup ROSE and also checkout other work common to multiple projects
%     within CASC.

     \item Build a compile directory (for the compile tree): \\
     Make a separate directory to be the root of the compile tree. There can be many compile 
     trees if you want.

     {\em Note: Before the next step be sure you are using the correct compiler ({\bf g++}
     C++ compiler
     [see ChangLog file for current version used for development, generally any 3.x
     version]) and that you are using the correct version of {\bf autoconf} and 
     {\bf automake}.}

     \item Running configure: \\
     Type {\tt <pathToSourceTree>/configure --help} to see the different configuration options.
     {\tt <pathToSourceTree>} is meant to be the absolute or relative path to the source tree
     where the {\bf SVN} version was checked out.  After options have been selected, type
     {\tt <pathToSourceTree>/configure <selected-options>} to run the configure script.
     Running the configure script with no options is sufficient (uses default values which
     are either already set or which the configure script will figure out on your machine).
     For more on ROSE configure options, see \ref{gettingStarted:configureOptions}.

     \item Running Make after running configure: \\
     After configuration (after the configure script is finished) run {\tt make} or {\tt gmake}.
     If you have a development version then you can also make distributions by running
     {\tt make dist}.  If you want to build a new distribution {\em AND} test it, 
     run {\tt make distcheck} (make or gmake may be used interchangeably). 
     See details of running make in parallel \ref{gettingStarted:parallelMake}.

     \item Testing your new version of ROSE: \\
     Automated tests are available within the distribution of ROSE. To run these tests,
     type {\tt make check}.  Tests on a modern Intel/Linux machine currently take about 
     15 minutes to run.
   % if you have configured ROSE to reference a version of the A++/P++ library (since they include
   % tests of A++/P++ within ROSE).  Tests on a more modern Linux machine are much faster.

     \item Installing ROSE: \\
     From this point you can generate ROSE the way a user would see it (as if you had
     started with a {\em ROSE Distribution}).  Type {\tt make install} to install ROSE.
     See details of installing ROSE \ref{gettingStarted:installation}.

     \item Testing the installed version of ROSE: \\
     To test the installed version of ROSE type {\tt make installcheck}.  To test
     compilation, this forces one
     or more of the Example translators to be built using only the header files from
     the {\tt \{install\_dir\}/include} directory for their compilation.  To test linking
     ROSE translators forces the previously compiled example translator to only use the libraries
     installed in {\tt \{install\_dir\}/lib}.  This is sufficient to test the installation
     the way that users are expected to use ROSE (only from an installed version).
     A sample {\tt makefile} is generated, see \ref{gettingStarted:compilingTranslator}.

\end{enumerate}

% DQ (6/5/2008): This is the resulting lesson from the erasure
% of my ROSE directory as the result of a bad cron script.
\section{How to recover from a file-system disaster at LLNL}
   Disasters can happen (cron scripts can go very very badly).  If you 
loose files on the CASC cluster at LLNL you can get the backup from the 
night before.  It just takes a while.

   To restore from backups at LLNL: use the command: \\
{\tt restore}
\begin{enumerate}
   \item {\tt add <directory name>} \\
       This will build the list of files to be recovered.
   \item recover \\
       This will start the process to restore the files from tape.
\end{enumerate}
   This process can take a long time if you have a lot of files to recover.


\section{Generating Documentation}
   There is a standard GNU {\tt make docs} rule for building all documentation.

{\it Note to developers: To build the documentation ({\tt make docs}) you will need 
LaTeX, Doxygen and DOT to be installed (check the list of dependences in the 
{\tt ROSE/ChangeLog}). If you want to build the reference manual of Latex documentation
generated by Doxygen (not suggested) you may have to tailor your version of LaTeX to 
permit larger internal buffer sizes.  All the other LaTeX documentation, such as the
User Manual but not the Reference Manual may be built without problems using the 
default configuration for LaTeX.
}

\section{Check In Process}

{\bf NOTE: Get permission from the ROSE Development Team before you make your first check-in!}

   If you have access to the SVN repository (at LLNL) and are building the development 
version of ROSE (available only from SVN, not what we package as a ROSE distribution; 
e.g. not from a file name such as ROSE-\VersionNumber.tar.gz) then 
% the README file in the top level directory also has instructions for how to get started. 
there are a number of steps to the checkin process:
%The README\_CHECKIN file has the instructions for our standard testing process.
%This should be done before you check anything in; we don't reprint it so it is
%represented at most only once.
% {\bf NOTE: Get permission from the ROSE Development Team before you make your first check-in!}
\begin{enumerate}
   \item Make sure you are working with the latest update (run {\tt svn update} in the top
    level directory.

 % \item In {\em ROSE/configure.in} modify version number (X.Y.ZZL) at the top of the file.

   \item Run {\tt make} \&\& {\tt make docs} \&\& {\tt make check} \&\&
   {\tt make dist} \&\& 
         {\tt make distcheck} \&\& {\tt make install} \&\& {\tt make installcheck}, depending
         on how aggressively you want your changes to be tests.
   \begin{itemize}
      \item Not all tests must be run, but we will know who you are (via {\tt svn blame} 
            if the nightly test fail :-).
      \item All changes must at least compile, so that you don't hold back other
            developers who update often.
   \end{itemize}

   \item svn commit -m "<description of what you did>".

%   \item ROSE/ChangeLog
%   a new entry has to include at least the following information:
%   - new version number
%   - who checked it in
%   - comments on changes
%   - run 'cvschk' in your ROSE directory.
%     copy the output of 'cvschk' to the ChangeLog (except "Extra Files")
%     (this includes all version numbers of tools used, etc.)
%
% 4) Send an e-mail to casc-rose@llnl.gov that you are checking in a new 
%    version of ROSE. Include the new version number in this e-mail.
%
% 6) check out a fresh version of ROSE in a new directory
%   - run all tests as above, in part 2, on this checked-out version
%   - make sure all tests succeed
%
% 7) cvs rtag ROSE-X-Y-ZZL ROSE
%    - X-Y-ZZL has to be the same version number as in configure.in
%      (X, Y, Z are numbers, L is a letter, note that here '-' is used instead
%       of '.' as in configure.in)
%
% 8) send out e-mail that check-in was successful
%    - copy & paste the output of make distcheck that the new version is ready
%      for distribution in this e-mail.
%    - include your new comments from the ChangeLog
%
% regular definition of version number:
% X=Y=Z=[0-9], L=[a-z]
% X.Y.ZZL /* in configure.in */
% X-Y-ZZL /* when using cvs rtag */
%
% Note that in configure.in '.' is used to separate X,Y,ZZL where as with
% etag '-' must be used, X-Y-ZZL!

\end{enumerate}

If you do not have access to the SVN repository at LLNL, and you wish to contribute
work to the ROSE project, please make a patch.  Using the external SVN access via
LBL use {\tt svn diff} to build a patch.
Consider options: {\em --diff-cmd arg}.
DQ(7/28/2008): This section still needs to be completed!



\section {Adding New SAGE III IR Nodes (Developers Only)}
    We don't expect users to add nodes to the SAGE III Intermediate Representation (IR),
however, we need to document the process to support developers who might be extending
ROSE.  It is hoped that if you proceed to add IR nodes that you understand just
what this means (you're not extending any supported language (e.g. C++); you are only
extending the internal representation. Check with us so that we can help you and
understand what you're doing.

The SAGE III IR is now completely generated using the ROSETTA IR generator tool which 
we developed to support our work within ROSE.
The process of adding new IR nodes using ROSETTA is fairly simple: one
adds IR node definitions using a BNF syntax and provides additional
headers and implementations for customized member data and functions
when necessary. 

  There are lots of examples within the construction of the IR itself.  So you are
encouraged to look at the examples. 
% However, you can expect to hunt around a bit before you get the final generated code to
% compile and link!  
The general steps are:

\fixme{Need to cover the new Fortran support. }
\begin{enumerate}
     \item Add a new node's name into \textit{src/ROSETTA/astNodeList}
%------------- 
     \item Define the node in ROSETTA's source files under
     \textit{src/ROSETTA/src} \\
           For example, an expression node has the following line in
           \textit{src/ROSETTA/src/expression.C}:
{\indent
{\mySmallFontSize
\begin{verbatim}
          NEW_TERMINAL_MACRO (VarArgOp,"VarArgOp","VA_OP");
\end{verbatim}
}}
           This is a macro (currently) which builds an object named {\em VarArgOp} (a variable in
           ROSETTA) to be named {\em SgVarArgOp} in SAGE III, and to be referenced using an enum
           that will be called {\em V\_SgVarArgOp}.  The secondary generated enum name {\em VA\_OP}
           is historical and will be removed in a future release of ROSE.

%------------- 
     \item In the same ROSETTA source file, specify the node's SAGE class hierarchy. \\
           This is done through the specification of what looks a bit like a BNF
            production rule to define the abstract grammar. \\
{\indent
{\mySmallFontSize
\begin{verbatim}
     NEW_NONTERMINAL_MACRO (Expression,
          UnaryOp        | BinaryOp             | ExprListExp   | VarRefExp       | ClassNameRefExp |
          FunctionRefExp | MemberFunctionRefExp | ValueExp      | FunctionCallExp | SizeOfOp        |
          TypeIdOp       | ConditionalExp       | NewExp        | DeleteExp       | ThisExp         |
          RefExp         | Initializer          | VarArgStartOp | VarArgOp        | VarArgEndOp     |
          VarArgCopyOp   | VarArgStartOneOperandOp ,"Expression","ExpressionTag");
\end{verbatim} 
}}
        In this case, we added the VarArgOp IR node as an expression node in the
    abstract grammar for C++.

%------------- 
     \item Add the new node's members (fields): both data and function
     members are allowed. \\
           ROSETTA permits the addition of data fields to the class definitions for the
           new IR node. Many generic access functions will be automatically
           generated if desired. 
{\indent
{\mySmallFontSize
\begin{verbatim}
     VarArgOp.setDataPrototype  ( "$GRAMMAR_PREFIX_Expression*","operand_expr","= NULL",
				 CONSTRUCTOR_PARAMETER, BUILD_ACCESS_FUNCTIONS, DEF_TRAVERSAL, NO_DELETE);
\end{verbatim}
}}
           The new data fields are added to the new IR node.  Using the first example
           above, the new data member is of type {\tt SgExpression*}, with name
           {\tt operand\_expr}, and initialized using the source code string {\tt = NULL}.
           Additional properties that this IR node will have include:
           \begin{itemize}
                \item Its construction will take a parameter of this type and 
                      use it to initialize this member field.
                \item Access functions to {\it get} and {\it set} the member 
                      function will be automatically generated.
                \item The automatically generated AST traversal will traverse 
                      this node (i.e. it will visit its children in the AST).
                \item Have the automatically generated destructor not call 
                      delete on this field (the traversal will to that).
           \end{itemize}
           In the case of the VarArgOp, an additional data member was added.
{\indent
{\mySmallFontSize
\begin{verbatim}
     VarArgOp.setDataPrototype ( "$GRAMMAR_PREFIX_Type*", "expression_type", "= NULL",
				 CONSTRUCTOR_PARAMETER, BUILD_ACCESS_FUNCTIONS, NO_TRAVERSAL || DEF2TYPE_TRAVERSAL);
\end{verbatim} 
}}

%------------- 
    \item Most IR nodes are simpler, but SgExpression IR nodes have explicit precedence. \\
           All expression nodes have a precedence in the evaluation, but the precedence must
           be specified.  This precedence must match that of the C++ frontend.  So we 
           are not changing anything about the way that C++ evaluates expressions here!
           It is just that SAGE must have a defined value for the precedence.
           ROSETTA permits variables to be defined and edited to tailor the automatically
           generated source code for the IR.
{\indent
{\mySmallFontSize
\begin{verbatim}
           VarArgOp.editSubstitute ( "PRECEDENCE_VALUE", "16" );
\end{verbatim} 
}}
%------------- 
     \item Associate customized source code. \\
           Automatically generated source code sometimes cannot meet all
           requirements, so ROSETTA allows user to define any custom 
           code that needs to be associated with the IR node in some
           specified files. If customized code is needed, you have to
           specify the source file containing the code. 
           For example, we specify the file containing customized source
           code for {\em VarArgOp} in \textit{src/ROSETTA/src/expression.C}:
{\indent
{\mySmallFontSize
\begin{verbatim}
     VarArgOp.setFunctionPrototype ( "HEADER_VARARG_OPERATOR", "../Grammar/Expression.code" );
     VarArgOp.setDataPrototype  ( "SgExpression*", "operand_expr"   , "= NULL",
				 CONSTRUCTOR_PARAMETER, BUILD_ACCESS_FUNCTIONS, DEF_TRAVERSAL, NO_DELETE);
     VarArgOp.setDataPrototype ( "SgType*", "expression_type", "= NULL",
 CONSTRUCTOR_PARAMETER, BUILD_ACCESS_FUNCTIONS, NO_TRAVERSAL || DEF2TYPE_TRAVERSAL, NO_DELETE);
   // ...
   VarArgOp.setFunctionSource ( "SOURCE_EMPTY_POST_CONSTRUCTION_INITIALIZATION", 
                                  "Grammar/Expression.code" );
\end{verbatim} 
}}
           Pairs of special markers (such as {\em SOURCE\_VARARG\_OPERATOR}
           and {\em SOURCE\_VARARG\_END\_OPERATOR}) are used for marking the header 
           and implementation parts of the customized code. 
           For example, the marked header and implementation code portions 
           for {\em VarArgOp} in
           \textit{src/ROSETTA/Grammar/Expression.code} are:
{\indent
  {\mySmallFontSize
\begin{verbatim}
HEADER_VARARG_OPERATOR_START
   virtual unsigned int cfgIndexForEnd() const;
   virtual std::vector<VirtualCFG::CFGEdge> cfgOutEdges(unsigned int index);
   virtual std::vector<VirtualCFG::CFGEdge> cfgInEdges(unsigned int index);
HEADER_VARARG_OPERATOR_END

// ....
SOURCE_VARARG_OPERATOR_START

  SgType*
  $CLASSNAME::get_type() const
   {
     SgType* returnType = p_expression_type;
     ROSE_ASSERT(returnType != NULL);
     return returnType;
   }

  unsigned int $CLASSNAME::cfgIndexForEnd() const {
    return 1;
  }
  //....

SOURCE_VARARG_OPERATOR_END
\end{verbatim} 
 }}
           The C++ source code is extracted 
           from between the named markers (text labels) in the named file and inserted 
           into the generated source code. Using this technique, very small amounts of 
           specialized code can be tailored for each IR node, while still providing an 
           automated means of generating all the rest.  Different locations in the
           generated code can be modified with external code. Here we add the source code
           for a function.

     \item Adding the set\_type and get\_type member functions. \\
           It is not clear that this is required, but all expressions must define a
           function that can be used to describe its type (of the expression).
           It is unfortunate, but it is generally in compiling the generated source code
           that details like this are discovered.  (ROSETTA has room for improvement!)
{\indent
{\mySmallFontSize
\begin{verbatim}
     VarArgOp.setFunctionSource ( "SOURCE_SET_TYPE_DEFAULT_TYPE_EXPRESSION", 
                                       "Grammar/Expression.code" );
     VarArgOp.setFunctionSource ( "SOURCE_DEFAULT_GET_TYPE",
                                       "Grammar/Expression.code" );
\end{verbatim} 
}}

     \item Modify the EDG/SAGE connection code to have the new IR node built in the
           translation from EDG to SAGE III.  This step often requires a bit of expertise
           in working with the EDG/SAGE connection code. In general, it requires no great
           depth of knowledge of EDG.

           Two source files are usually involved: a)
           \textit{src/frontend/CxxFrontend/EDG\_SAGE\_Connection/sage\_gen\_be.C}
           which converts IL tree to SAGE III AST and is derived from EDG's 
           C++/C-generating back end \textit{cp\_gen\_be.c}; b)
           \textit{sage\_il\_to\_str.C} contains helper functions forming SAGE
           III AST from various EDG IL entries. It is derived from EDG's
           \textit{il\_to\_str.c}.  For the {\em SgVarArgOp} example, the
           following EDG-SAGE connection code is needed in
           \textit{sage\_gen\_be.C}:
{\indent
{\mySmallFontSize
\begin{verbatim}
a_SgExpression_ptr
sage_gen_expr ( an_expr_node_ptr expr, 
                a_boolean need_parens, 
    ...
              )
{
  // ...
  case eok_va_arg:
  {
   sageType = sage_gen_type(expr->type);
   sageLhs = sage_gen_expr_with_parens(operand_1,NULL);
   if (isSgAddressOfOp(sageLhs) != NULL)
     sageLhs = isSgAddressOfOp(sageLhs)->get_operand();
   else
     sageLhs = new SgPointerDerefExp(sageLhs,NULL);
  //....
   result = new SgVarArgOp(sageLhs, sageType);
   goto done_with_operation;
                       }
  }
  //.....

}
\end{verbatim} 
}}

     \item Modify the unparser to have whatever code you want generated in the final
           code generation step of the ROSE source-to-source translator.
           The source files of the unparser are located at
           \textit{src/backend/unparser}. For {\em SgVarArgOp}, it is
           unparsed by the following function in
           \textit{src/backend/unparser/CxxCodeGeneration/unparseCxx\_expressions.C}:

{\indent
{\mySmallFontSize
\begin{verbatim}

void
Unparse_ExprStmt::unparseVarArgOp(SgExpression* expr, SgUnparse_Info& info)
   {
     SgVarArgOp* varArg = isSgVarArgOp(expr);
     SgExpression* operand = varArg->get_operand_expr();
     SgType* type = varArg->get_type();
     curprint ( "va_arg(");
     unparseExpression(operand,info);
     curprint ( ",");
     unp->u_type->unparseType(type,info);
     curprint ( ")");
   }
\end{verbatim} 
}}



\end{enumerate}



\section{Separation of EDG Source Code from ROSE Distribution}

    The EDG research license restricts the distribution of their source code.
Working with EDG is still possible within an open source project such as ROSE because 
EDG permits binaries of their work to be freely distributed (protecting their source 
code).  As ROSE matured, we designed the autoconf/automake distribution mechanism
to build distributions that exclude the EDG source code and alternatively distribute
a Linux-based binary version of their code.

   All releases of ROSE, starting with 0.8.4a, are done without the EDG source code
by default.  An optional configure command line option is implemented to allow
the construction of a distribution of ROSE which includes the EDG source code
(see {\tt configure --help} for the {\tt --with-edg\_source\_code} option).

   The default options for configure will build a distribution that contains
no EDG source code (no source files or header files).  This is not a problem 
for ROSE because it can still exist as an almost entirely open source project
using only the ROSE source and the EDG binary version of the library.

  Within this default configuration, ROSE can be freely distributed on the Web
(eventually).  Importantly, this simplifies how we work with many different 
research groups and avoid the requirement for a special research license from
EDG for the use of their C and C++ front-end.  Our goal has been to simplify
the use of ROSE.

   Only the following command to configure with EDG source code is accepted:
{\indent
{\mySmallFontSize
\begin{verbatim}
     configure --with-edg_source_code=true
\end{verbatim} 
}}
This particularly restrictive syntax is used to prevent it from ever being used
by accident.  Note that the following will not work. They are equivalent to 
not having specified the option at all:
{\indent
{\mySmallFontSize
\begin{verbatim}
     configure --with-edg_source_code
     configure --with-edg_source_code=false
     configure --with-edg_source_code=True
     configure --with-edg_source_code=TRUE
     configure --with-edg_source_code=xyz
     configure 
\end{verbatim} 
}}

To see how any configuration is set up, type {\tt make testEdgSourceRule}
in the {\tt ROSE/src/frontend/CxxFrontend/EDG\_3.3/src} directory.

To build a distribution without EDG source code:
\begin{enumerate}
   \item Configure to use the EDG source code and build normally, 
   \item Then rerun configure to not use the EDG source code, and
   \item Run {\tt make dist}.
\end{enumerate}


\section{How to Deprecate ROSE Features}

    There comes a time when even the best ideas don't last
into a new version of the source code.  This section covers how to
deprecated specific functionality so that it can be removed in
later releases (typically after a couple of releases, or before
our first external release).  When using GNU compilers these mechanisms
will trigger the use of GNU attribute mechanism to permit use of such
functions in applications to be easily flagged (as warnings
output when using the GNU options {\tt -Wall}).

Both functions and data members can be deprecated, but the process if different 
for each case:
\begin{itemize}
    \item Deprecated functions and member functions. \\
          Use the macro {\tt ROSE\_DEPRECATED\_FUNCTION} after the function declaration (and before
          the closing {\bf ;}). As in:
{\indent
{\mySmallFontSize
\begin{verbatim}
       void old_great_idea_function() ROSE_DEPRECATED_FUNCTION;
\end{verbatim}
}}

    \item Deprecated data members. \\
        Use the macro {\tt ROSE\_DEPRECATED\_VARIABLE} to specify that a data members 
    or variables is to be deprecated.  This is difficult to do because data members of 
    the IR are all automatically generated and thus can't be edited in this way.  Where 
    a data member of the IR is to be deprecated, it should be specified explicitly in
    the documentation for that specific class (in the {\tt ROSE/docs/testDoxygen} directory,
    which is the staging area for all IR documentation, definitely {\em not} in the 
    {\tt ROSE/src/frontend/SageIII/docs} directory, which is frequently overwritten).  See
    details on how to document ROSE (Doxygen-Related Pages).
{\indent
{\mySmallFontSize
\begin{verbatim}
       void old_great idea_data_member ROSE_DEPRECATED_VARIABLE;
\end{verbatim} 
}}
\end{itemize}



\section{Code Style Rules for ROSE}

   I don't want to constrain anyone from being expressive, but
we have to maintain your code after you leave, so there are a few rules:
\begin{enumerate}
   \item Document your code.
         Explain every function and use variable names that clearly indicate the purpose of
         the variable. Explain what the tests are in your code (and where they are located).
   \item Write test codes to test your code (these are assembled in the {\tt ROSE/tests}
         directory (or subdirectories of {\tt ROSE/tests/roseTests}).
   \item Use assertions liberally, use boolean values arguments to 
         {\tt ROSE\_ASSERT(<expression>)}. Use of {\tt ROSE\_ASSERT(true/false)} for
         error branches is preferred.
   \item Put your code into source files (*.C) and as little as possible into header files.
   \item If you use templates, put the code into a *.C file and include that *.C file
         at the bottom of your header file.
   \item If you use a {\em for loop} and break out of the loop (using {\tt break;} 
         at some point in the iteration, then consider a {\em while loop} instead.
   \item Don't forget a default statement within switch statements.
   \item Please don't open namespaces in source files, i.e. use the fully qualified
         function name in the function definition to make the scope of the function
         as explicitly clear as possible.
   \item Think about your variable names. I too often see {\tt Node}, {\tt node}, 
         and {\tt n} in the same function.  Make your code {\em obvious} so that I 
         can understand it when I'm tired or stupid (or both).
   \item Write good code so that we don't have to debug it after you leave.
   \item Indent your code blocks.
\end{enumerate}

My rules for style are as follows. Adhere to them if you like, or don't, if you're
    appalled by them.
\begin{enumerate}
   \item Indent your code blocks (I use five spaces, but some consider this excessive).
   \item Put spaces between operators for clarity.
\end{enumerate}



\section{Things That May Happen to Your Code After You Leave}

    No one likes to have their code touched, and we would like to
avoid having to do so. We would like to have your contribution to ROSE
always work and never have to be touched.  We don't wish to pass
critical judgment on style since we want to allow many people to 
contribute to ROSE.  However, if we have to debug your code, be prepared 
that we will do a number of things to it that might offend you:
\begin{enumerate}
   \item We will add documentation where we think it is appropriate.
   \item We will add assertion tests (using ROSE\_ASSERT() macros)
         wherever we think it is appropriate.
   \item We will reformat your code if we have to understand it and the 
         formatting is a problem.  This may offend
         many people, but it will be a matter of project survival,
         so all apologies in advance.  If you fix
         anything later, your free to reformat your code as you like.  We try to change
         as little as possible of the code that is contributed.
\end{enumerate}


\section{Maintaining the ROSE Email List (casc-rose@llnl.gov)}

   There is an open email list for ROSE which can be subscribed to
automatically.  The list name is: {\bf casc-rose}.

   These are the email commands available to users of the list. To use them,
a user sends a message to Majordomo with one or more of these commands in the body of
the message. Each mailing list has a special "request" address where commands can be
sent. For example, to use the casc-rose mailing list (casc-rose@lists.llnl.gov), send
commands to casc-rose-request@lists.llnl.gov.

   It is also possible to send commands directly to majordomo@lists.llnl.gov. 
However, be sure to specify which list you want to use. With all the commands below, you 
can leave out list if you are sending to casc-rose-request@lists.llnl.gov.

\begin{itemize}
   \item subscribe list address \\
       Subscribe yourself (or address if specified) to the named list. The list may be 
       configured so that you can only subscribe yourself; ie. you can't specify an 
       address other than your own.

   \item Unsubscribe list address \\
    Unsubscribe yourself (or address if specified) from the named list. "unsubscribe *" 
    will remove you (or address) from all lists; This may not work if you have subscribed 
    using multiple addresses. The list may be configured so that you can only unsubscribe 
    yourself; ie. you can't specify an address other than your own.

   \item which address \\
    Find out which lists you (or address if specified) are on. Only lists enabled to
    supply this information will be returned to the requester.

   \item who list \\
    Find out who is on the named list. Only lists enabled to supply this information 
    will be returned to the requester.

   \item info list \\
    Retrieve the general introductory information for the named list. Only lists enabled
    to supply this information will be returned to the requester.

   \item intro list \\
    Retrieve the introductory message sent to new users. Non-subscribers may not be able
    to retrieve this.

   \item lists \\
    Show the lists served by this Majordomo server (will not show "private" lists).

   \item help \\
    Retrieve some help information on the available user commands.

   \item end \\
    Stop processing commands (useful if your email program adds a signature). 
\end{itemize}


Here are the URLs for the {\em casc-rose} email list:

Instructions on how to use a Majordomo mailing list: \\
\begin{verbatim}
    https://lists.llnl.gov/mj/user-commands.html
\end{verbatim}

Web interface for modifying a Majordomo mailing list: \\
\begin{verbatim}
    https://lists.llnl.gov/majordomo.
\end{verbatim}

Details: \\
\begin{enumerate}
   \item List name is: {\em casc-rose} not {\em casc-rose@llnl.gov}.
   \item Must be on site at LLNL.
%   \item Password is required (Tom Leher and Hichhicker's Guide).
\end{enumerate}


\section{How To Build a Binary Distribution}

   The construction of a binary distribution is done as part of making
ROSE available externally on the web to users who do not have an EDG
licence.  We make only the EDG part of ROSE available as a binary (library) 
and the rest is left as source code (just as in an all source distribution).

There are a few steps:
\begin{enumerate}
   \item Configure and build ROSE normally, using configure (use all options that you
    require in the binary distribution).  
   \item (optional) Run {\tt make dist}, this will build an {\em all source distribution} of ROSE.
   \item Rerun configure without the {\tt --with-edg\_source\_code=true} option.
   \item Run {\tt make dist}, this will build a binary distribution using the 
    binary libraries build in step one.

\end{enumerate}


\section{Avoiding Nightly Backups of Unrequired ROSE Files at LLNL}

   If your at LLNL and participating in the nightly builds and
regression testing of ROSE, then it is kind to the admin staff
to avoid having your testing directory 
{\em often many gigabytes of files} backed up nightly.

   There is a file {\tt .nsr} that you can put into
any directory that you don't need to have backed up.
The syntax of the text in the file is:
{\tt skip: .}

Additional examples are:
\begin{verbatim}
# The directives in this file are for the legato backup system
# Here we specify not to backup any of the following file types:
+skip: *.ppm *.o *.show*
\end{verbatim}

More information can be found at: \\
   www.ipnom.com/Legato-NetWorker-Commands/nsr.5.html

Thanks for saving a number of people a lot of work.


\section{Setting Up Nightly Tests}

Directions for using roseFreshTest to set up periodic regression tests:

\begin{enumerate}
   \item Get an account on the machine you are going to run the tests on.
   \item Get a scratch directory (normally /export/0/tmp.<your username>) on that
         machine.
   \item Copy (using svn cp) a stub script (scripts/roseFreshTestStub-*) to one
   with your name.
   \item Edit your new stub script as appropriate:
   \begin{enumerate}
      \item Set the versions of the different tools you want to use (compiler,
     ...).
      \item Change ROSE\_TOP to be in your scratch directory.
      \item Set ROSE\_SVNROOT to be the URL of the trunk or branch you want to
     test.
      \item Set MAILADDRS to the people you want to be sent messages about the
     progress and results of your test.
      \item MAKEFLAGS should be set for most peoples' needs, but the -j setting
     might need to be modified if you have a slower or faster computer.
      \item If you would like the copy of ROSE that you test to be checked out
     using "svn checkout" (rather than the default of "svn export"), add a
     line "SVNOP=checkout" to the stub file.
      \item The default mode of roseFreshTest is to use the most current version
     of ROSE on your branch as the one to test.  If you would like to test
     a previous version, you can set SVNVERSIONOPTION to the revision
     specification to use (one of the arguments to -r in "svn help
     checkout").
   \end{enumerate}
   \item Check your stub script in so that it will be backed up, and so that other
   people can copy from it or update it to match (infrequent) changes in the
   underlying scripts.
   \item Run "crontab -e" on the machine you will be testing on:
   \begin{enumerate}
      \item Make sure there is a line with "MAILTO=<your email>".
      \item Add new lines for each test you would like to run:
      \begin{enumerate}
         \item If other people are using the machine you are running tests on, be
               sure to coordinate the time your scripts are going to run with them.
         \item See "man crontab" for the format of the time and date specification.
         \item The command to use is (all one line):
\begin{verbatim}
           cd <your ROSE source tree>/scripts && \
           ./roseFreshTest ./roseFreshTestStub-<your stub name>.sh \
           <extra configure options>
         Where <extra configure options> are things like
         --enable-edg\_union\_struct\_debugging, --with-C\_DEBUG=...,
         --with-java, etc.
\end{verbatim}
      \end{enumerate}
   \end{enumerate}
   \item Your tests should then run on the times and dates specified.
   \item If you would ever like to run a test immediately, copy and paste the
   correct line in "crontab -e" and set the time to the next minute (note
   that the minute comes first, and the hour is in 24-hour format); ensure
   the date specification includes today's date.  Be sure to quit your
   editor -- just suspending it prevents your changes from taking effect.
\end{enumerate}

\section{Enabling PHP Support}

\begin{enumerate}
\item
Fetch and install PHP (tested with 5.2.6) from
\texttt{http://www.php.net/downloads.php}.  PHC requires a few
specific configure flags in order to be able to use PHP properly.
Fill in your choice of PHP install location where appropriate in place
of \texttt{/usr/local/php}.
\begin{verbatim}
./configure  --enable-debug --enable-embed --prefix=/usr/local/php
make && make install
\end{verbatim}

\item
Fetch and install PHC (tested with svn version r1487).  Currently only
the development release works with ROSE.
\begin{verbatim}
svn checkout http://phc.googlecode.com/svn/trunk/ phc-read-only
cd phc-read-only
touch src/generated/*                                                       
./configure --prefix=/usr/local/php --with-php=/usr/local/php
make && make install
\end{verbatim}

\item
Finally, due to an incongruence in the class hierarchies of PHC and
ROSE the following changes have to be made to the installed
\texttt{/usr/local/php/include/phc/AST\_fold.h}.  Hopefully this can be
resolved soon so that ROSE works with an unmodified upstream PHC.

\begin{verbatim}
--- src/generated/AST_fold.h    2008-07-30 10:35:32.000000000 -0700
+++ src/generated/AST_fold.h.rose       2008-08-13 15:30:37.000000000 -0700
@@ -1037,7 +1037,7 @@
                        case Nop::ID:
                                return fold_nop(dynamic_cast<Nop*>(in));
                        case Foreign::ID:
-                               return fold_foreign(dynamic_cast<Foreign*>(in));
+                               return 0;
                }
                assert(0);
        }
@@ -1271,7 +1271,7 @@
                        case Nop::ID:
                                return fold_nop(dynamic_cast<Nop*>(in));
                        case Foreign::ID:
-                               return fold_foreign(dynamic_cast<Foreign*>(in));
+                               return 0;
                        case Switch_case::ID:
                                return fold_switch_case(dynamic_cast<Switch_case*>(in));
                        case Catch::ID:
\end{verbatim}

\item
Once both packages have been installed ROSE must be configured with
the additional \texttt{--with-php=/usr/local/php} option.
\end{enumerate}



\section{Binary Analysis}

   The documentation for the binary analysis can be found in the ROSE manual at
\ref{binaryAnalysis::overview}.  However, there are a collection of details 
that we need to document about the design; so for how these details can go here.
   The design behind the support for binary analysis in ROSE has caused a number of
design meetings to discuss details.  This section is specific to the support
in ROSE for binary analysis and the development of the support in ROSE for the 
binary analysis.

\subsection{Design of the Binary AST}

This subsection is specific to the design of the binary executable file format
and specifically the representation of the binary file format in the Binary AST 
as a tree (in the graph sense) instead of as a directed graph, so that ti can be 
traversed using the mechanisms available in ROSE.

\begin{itemize}
   \item Symbols \\
Their are multiple references to symbols (as shown in the Whole Graph view of the AST with 
the binary format).  We have selected the SgAsmELFSymbolTable and the SgAsmCoffSymbolTable
instead of the SgAsmGenericSymbolTable because it points to the most derived type.
An alternative reasoning is that in stripped binariiies that require DLL support
the required symbols in the SgAsmELFSymbolTable and the SgAsmCoffSymbolTable are 
left in place to support the DLL mechanism where as all entries in the
SgAsmGenericSymbolTable are removed (get more details from Robb). 
\fixme{We should get a reference for the details of what symbols are left in stripped
       binaries and what symbols are required to support dynamic linking and where they
       are stored.}

   \item Checking the symbols in the executable using {\tt nm} \\
ROSE permits a programmable interface to the binary executable file format,
but unix utility functions provide text output of such details. For example,
use {\tt nm -D .libs/librose.so | c++filt | less } to generate a list of
all the symbols in an executable (text output).  In this case {\tt c++filt }
resolved the original names from the mangled names for executables built from 
C++ applications.  The C++ symbols appear at the bottom of the listing.

\end{itemize}

