\section{ROSE Installation}

\subsection{Requirements and Options}
\label{Requirements_Installation_Testing}
   ROSE is general software and we ultimately hope to remove any specific software
and hardware requirements.  However, our goal is to be specific about where and 
how ROSE is developed and where it is regularly tested.
% ROSE has been developed initially on the Sun workstations and later on Linux.
% Since ROSE is written in C++, it requires a C++ compiler. Some optional features
% within ROSE have some additional software requirements.

\subsubsection{Required Hardware/Operating System}
   ROSE has been developed on Linux/Intel platforms. We have not yet 
addressed significant portability issues for ROSE. But EDG has addressed
portability issues for their C++ frontend and it is available on nearly all
platforms (see {\tt www.EDG.com} for details).  ROSE is currently developed on
Linux/Intel platforms and works with all modern versions of the GNU compilers 
(3.4.x, and later). ROSE also works on both 32-bit and 64-bit architectures, as well as
with the Intel C and C++ compiler.
Future work will focus on portability to other platforms important to users.
% At a later point we will address portability of ROSE onto other platforms:
% mostly likely IBM AIX will be next (at some point).

\subsubsection{Software Requirements}
   You will require {\bf ONLY} a C++ compiler to compile ROSE; ROSE is written in C++.
% We use the GNU g++ compiler most often, but the Intel compiler (version 9.1) will also work.
Present development work is done on Intel/Linux platforms
%using the GNU g++ 3.3.x, 3.4.x, and 4.x; and the Intel compilers.  GNU compilers older
using the GNU g++ 3.4.x, and 4.x; and the Intel compilers.  
%GNU compilers older
%than 3.3.x are not supported within the ROSE project, but in some cases they might work 
%(g++ 3.2.x is likely to work, but g++ 2.96 will not work).

   ROSE users may either obtain a free research license from EDG and hence ROSE with EDG source code,
or alternatively, obtain ROSE that contains a binary version of the EDG work.
The latter is limited to specific platforms and versions of compilers. See EDG (www.edg.com) for details 
and limitations on how their software may be used. \\

\commentout{
   We suggest that ROSE users obtain a free research license from EDG until we 
regularly distribute binary versions of the libedg.so library (their software).
Email us for details on our distributions of ROSE that contain a binary version
of the EDG work (and that completely hide the EDG interface). These are limited
to specific platforms and versions of compilers. See EDG (www.edg.com) for details 
and limitations on how their software may be used.
}

{\bf Use of Required Software:} \\
  The following software is required in order to build and use ROSE:
\begin{itemize}
   \item {\bf ROSE} : \\ 
     There are two versions of ROSE supported: the 
     {\it Distribution Version} for users (typical) and the {\it Development Version}
     (intended only for ROSE development team), which is what is found in the ROSE software
     repository and has additional software requirements (autoconf, automake, Doxygen, LaTeX,
     etc.).  The exact requirements are listed in the {\tt ROSE/ChangeLog} (including version 
     numbers for each release of ROSE).
     \begin{itemize}
     \item {\bf Distribution Version} \\
       Provided as a tared and compressed file in the form,
       ROSE-\VersionNumber.tar.gz.  This is the most typical way that users will
       see and work with ROSE.  Instructions are also located in the ROSE/README file.
     \item {\bf Development Version} \\
       Only available directly from the Subversion (SVN) repository. The details of building this
       version are located in the Appendix \ref{gettingStarted:DeveloperInstructions}.
     \end{itemize}  
%     ROSE is delivered as either a development version (svn) or
%     as a distribution for users. The (typical) user version comes either with EDG source or binary distribution.
   \item {\bf g++} : version $>=$ 3.4.x  \\
     In order to use OpenMP or gFortran g++ $>=$ 4.2.x is required.
   \item {\bf BOOST} : version $>=$ 1.35.0  \\
     Visit \htmladdnormallink{www.boost.org}{http://www.boost.org/}
     for more details about BOOST and \htmladdnormallink{www.boost.org/users/download}{http://www.boost.org/users/download/}
     for download and installation instructions.
   \item {\bf Autoconf} : version $>=$ 2.59 . Needed \emph{ONLY} for development version.  \\
     Autoconf is an extensible package of M4 macros that produce shell scripts to automatically configure software source code packages. 
   \item {\bf Automake} : version $>=$ 1.96 . Needed \emph{ONLY} for development version.  \\
     Automake is a tool for automatically generating `Makefile.in' files compliant with the GNU Coding Standards.
\end{itemize}

To simplify the descriptions of the build process, we define:
\begin{itemize} 
   \item {\bf Source Tree} \\
         Location of source code directory (there is only one source tree).
   \item {\bf Compile Tree} \\
         Location of compiled object code and executables. There can be many compile trees
         representing either different: configure options, compilers used to build ROSE
         and ROSE translators, compilers specified as backends for ROSE (to compile ROSE
         generated code), or architectures.
\end{itemize}

We {\em strongly} recommend that the {\bf Source Tree} and the {\bf Compile Tree} be
different. This avoids many potential problems with the {\em make clean} rules.
Note that the {\bf Compile Tree} will be the same as the {\bf Source Tree}
if the user has {\em not} explicitly generated a separate directory in which to
run {\em configure} and compile ROSE. If the {\bf Source Tree} and {\bf Compile Tree} 
are the same, then there is only one combined {\bf Source/Compile Tree}.
Alternatively, numerous different {\bf Compile Trees} can be used from 
a single {\bf Source Tree}.  More than one {\bf Compile Tree} allows
ROSE to be generated on different platforms from a single source 
(either a generated distribution or checked out from SVN).  ROSE is developed 
and tested internally using separate {\bf Compile Trees}. \\


{\bf Use of Optional Software:} \\
%ROSE does not presently require any special software, but more
More functionality within ROSE is available if one has additional (freely available) software installed:
\begin{itemize}
   \item {\bf Doxygen} : \\
          Most ROSE documentation is generated using LaTex and Doxygen, thus
          Doxygen is required for ROSE developers that want to regenerate the ROSE
          documentation. This is not required for ROSE users, since all documentation is 
          included in the ROSE distribution. Visit {\it www.doxygen.org} for details and to
          download software.  There are no ROSE-specific configure options to use Doxygen;
          it must only be available within your path.
   \item {\bf LaTeX} : \\
          LaTeX is used for a significant portion of the ROSE documentation. LaTex is 
          included on most Unix systems. There are no ROSE specific configure options 
          to use LaTeX; it must only be available within your path.
   \item {\bf DOT (GraphViz)} : \\
          ROSE uses DOT for generating graphs of ASTs, Control Flow, etc.
          DOT is also used internally by Doxygen.
          Visit {\it www.graphviz.org} for details and to download software.
          An example showing the use of the DOT to build graphs is in the ROSE Tutorial.
          There are no ROSE-specific configure options to use dot; it must only be 
          available within your path.
   \item {\bf SQLite} : \\
          ROSE users can store persistent data across separate compilation of files by
          storing information in an SQLite database.  This is used by several features
          in ROSE (call-graph generation, for example) and may be used directly by the
          user for storage of user-defined analysis data.  Such database support is
          one way to handle global analysis (the other way is to build the whole
          application AST). Visit {\it www.sqlite.com} for details and to download
          software. %Details are in section \ref{gettingStarted:Database}.
          An example showing the use of the ROSE database mechanism is in the ROSE
          Tutorial. Use of SQLite requires special ROSE configuration options (so that
          the SQLite library can be added to the link line at compile time).  See ROSE
          configuration options for more details ({\tt configure --help}).
   \item {\bf mpicc} : \\
          mpicc is a compiler for MPI development. If ROSE is configures with MPI
          enabled, one can utilize features in ROSE that allow for distributed
          parallel AST traversals.
\end{itemize}

% \subsection{Installation and Testing }

%\subsection{How to Build ROSE}

\commentout{
%   This section addresses the building and installation of ROSE.  
There are two versions of ROSE supported: the 
{\it Distribution Version} for users (typical) and the {\it Development Version}
(intended only for ROSE development team), which is what is found in the ROSE software
repository and has additional software requirements (autoconf, automake, Doxygen, LaTeX,
etc.).  The exact requirements are listed in the {\tt ROSE/ChangeLog} (including version 
numbers for each release of ROSE).
\begin{itemize}
   \item {\bf Distribution Version} \\
         Provided as a tared and compressed file in the form,
         ROSE-\VersionNumber.tar.gz.  This is the most typical way that users will
         see and work with ROSE.  Instructions are also located in the ROSE/README file.

   \item {\bf Development Version} \\
         Only available directly from the Subversion (SVN) repository. The details of building this
         version are located in the Appendix \ref{gettingStarted:DeveloperInstructions}.
\end{itemize}  
To simplify the descriptions of the build process, we define:
\begin{itemize} 
   \item {\bf Source Tree} \\
         Location of source code directory (there is only one source tree).
   \item {\bf Compile Tree} \\
         Location of compiled object code and executables. There can be many compile trees
         representing either different: configure options, compilers used to build ROSE
         and ROSE translators, compilers specified as backends for ROSE (to compile ROSE
         generated code), or architectures.
   \item {\bf Install Tree} \\
         Location of compiled object code when copied to a clean installation directory.
         This is commonly used in order to refer to a project from outside, i.e. to 
         use the provided interfaces. 
\end{itemize}

We {\em strongly} recommend that the {\bf Source Tree} and the {\bf Compile Tree} be
different. This avoids many potential problems with the {\em make clean} rules.
Note that the {\bf Compile Tree} will be the same as the {\bf Source Tree}
if the user has {\em not} explicitly generated a separate directory in which to
run {\em configure} and compile ROSE. If the {\bf Source Tree} and {\bf Compile Tree} 
are the same, then there is only one combined {\bf Source/Compile Tree}.
Alternatively, numerous different {\bf Compile Trees} can be used from 
a single {\bf Source Tree}.  More than one {\bf Compile Tree} allows
ROSE to be generated on different platforms from a single source 
(either a generated distribution or checked out from SVN).  ROSE is developed 
and tested internally using separate {\bf Compile Trees}.
}

\commentout{
\subsection{Summary of Build Process}
\label{gettingStarted:SummaryInstructions}
   More specific information is available in the subsections below 
(subsection \ref{gettingStarted:UserInstructions} and 
subsection \ref{gettingStarted:DeveloperInstructions}), but in summary the steps are:
% The general steps are (but use the more details instructions below, depending 
% on the version of ROSE that you have):
\begin{itemize}
   \item Type {\tt \<Source Tree\>/configure;} \\
         This can be done in a separate directory or in the source directory.
         We suggest using a separate directory, but either will work. The 
         {\bf Source Tree} may be specified with either an absolute or relative 
         path.
   \item type {\tt make} \\ to build the source code (you may also use the parallel make
         option (e.g. {\tt make -j4} to run make using 4 processes).
   \item Type {\tt make check} \\ to run internal tests to test your build.
   \item Type {\tt make install} \\ to install the build library for more convenient use.
         (default is {\tt /usr/local}, use {\tt --prefix=`pwd'} to specify the current directory).
   \item Type {\tt make documentation} \\ to build both html and LaTeX documentation.
\end{itemize}
}

\subsection{Building BOOST}
\label{gettingStarted:BOOST}

The following is a quick guide on how to install BOOST. For more details please 
refer to \htmladdnormallink{www.boost.org}{http://www.boost.org/}:

\begin{enumerate}
     \item Download BOOST. \\
       Download BOOST at \htmladdnormallink{www.boost.org/users/download}{http://www.boost.org/users/download/}.
     \item Untar BOOST. \\
       Type {\tt tar -zxf BOOST-[VersionNumber].tar.gz} to untar the BOOST distribution.
     \item Create a separate compile tree. \\
           Type {\tt mkdir compileTree} to build a location for the object files and
           documentation (use any name you like for this directory, e.g. BOOST\_BUILD).
     \item Create a separate install tree. \\
           Type {\tt mkdir installTree} to create a location for the install filesto reside (e.g. BOOST\_INSTALL).
     \item Change directory to the new compile tree directory. \\
           Type {\tt cd compileTree; }. This changes the current directory to the newly
           created directory.
     \item Run the {\tt configure} script. \\
           Type {\tt \{AbsoluteOrRelativePathToSourceTree\}/configure --prefix=[installTree]} 
           to run the BOOST {\tt configure} script.  The path to the configure script 
           may be either relative or absolute. The prefix option specifies the installation directory (e.g. BOOST\_INSTALL).
     \item Run {\tt make}. \\
           Type {\tt make} to build all the source files. 
     \item Run {\tt make install}. \\
           Type {\tt make install} to copy all build files into the install directory. BOOST is now available in your 
           installTree (e.g. BOOST\_INSTALL) to be used by ROSE.
\end{enumerate}          

\subsection{Building ROSE From a Distribution (ROSE-\VersionNumber.tar.gz)}
\label{gettingStarted:UserInstructions}
   The process for building ROSE from a ROSE {\em Distribution Version} is the same as for
    most standard software distributions (e.g those using autoconf tools):
\begin{enumerate}
     \item Untar ROSE. \\
           Type {\tt tar -zxf ROSE-\VersionNumber.tar.gz} to untar the ROSE distribution.
     \item Build a separate compile tree. \\
           Type {\tt mkdir compileTree} to build a location for the object files and
           documentation (use any name you like for this directory).
     \item Change directory to the new compile tree directory. \\
           Type {\tt cd compileTree; }. This changes the current directory to the newly
           created directory.
     \item Run the {\tt configure} script. \\
           Type {\tt \{AbsoluteOrRelativePath\}/configure --prefix=`pwd` --with-boost=[BOOST\_installTree]} 
           to run the ROSE {\tt configure} script.  The path to the configure script 
           may be either relative or absolute. The prefix option on the configure 
           command line is only required if you run {\tt make install} (suggested), because 
           the default location for installation is {\tt \//usr\//local} and most users don't
           have permission to write to that directory. This is common to all projects that
           use autoconf.  ROSE follows the GNU Makefile Standards as a result of using
           autoconf and automake tools for its build system. As of ROSE-0.8.9a, the
           default setting for the install directory (prefix) is the build tree.
           For more on ROSE configure options, see section \ref{gettingStarted:configureOptions}.
     \item Run {\tt make}. \\
           Type {\tt make} to build all the source files. See details of running 
           {\tt make} in parallel in section \ref{gettingStarted:parallelMake}.
     \item To test ROSE (optional). \\
           Type {\tt make check} to test the ROSE library against a collection of test codes.
           See details of running make in parallel \ref{gettingStarted:parallelMake}.
     \item To install ROSE, type {\tt make install}. \\
           Installation is optional, but suggested. Users can simplify their use of ROSE 
           by using it from an installed version of ROSE.  This permits compilation 
           using a single include directory and the specification of only two libraries.
           See details of installing ROSE in section \ref{gettingStarted:installation}.
     \item Testing the installation of ROSE (optional). \\
           To test the installation and the location where ROSE is installed, against a
           collection of test codes (the application examples in {\tt ROSE/tutorial}), 
           type {\tt make installcheck}.
         % To optionally test the installation of ROSE \\
         % Type {\tt make installcheck} to test the location where ROSE is installed
         % against a simple test code. 
           A sample {\tt makefile} is generated; see section \ref{gettingStarted:compilingTranslator}.
\end{enumerate}

\subsection{ROSE Configure Options}
\label{gettingStarted:configureOptions}
     A few example configure options are:
\begin{itemize}
   \item {\tt ../ROSE/configure --with-boost=[BOOST\_installTree]} \\
         This will configure ROSE to be compiled in the current directory (separate from
         the {\bf Source Tree}).  The installation (from {\tt make install}) will be placed in
         {\tt /usr/local}.  Most users don't have permission to write to this directory,
         so we suggest always including the {\em prefix option} (e.g. {\tt --prefix=`pwd`}).
   \item {\tt ../ROSE/configure --prefix=`pwd` --with-boost=[BOOST\_installTree]} \\
         Configure in the current directory so that installation will also happen in the
         current  directory (a {\tt install} subdirectory will be built).
   \item {\tt ../ROSE/configure --with-CXX\_DEBUG=-g --with-C\_DEBUG=-g 
              --with-CXX\_WARNINGS=-Wall --prefix=`pwd`--with-boost=[BOOST\_installTree]} \\
         Configure as above, but with debugging and warnings turned on ({\tt -Wall} is
         specific to the gnu compilers).
   \item {\tt ../ROSE/configure --with-SQLite=/home/dquinlan/SQLite/sqliteCompileTree --prefix=`pwd` --with-boost=[BOOST\_installTree]} \\
         Configure as above, but permit use of SQLite database for storage of analysis
         results between compilation of separate files (one type of support in ROSE for
         global analysis).
   \item {\tt ../ROSE/configure --prefix=`pwd` --with-mpi --with-gcc-omp --with-boost=[BOOST\_installTree]} \\
         Configure as above, but with MPI and OpenMP support for ROSE to run AST traversals
         in parallel (distributed and shared memory).
   \item {\tt ../ROSE/configure --prefix=`pwd` --with-binarysql --with-boost=[BOOST\_installTree]} \\
         The binarysql flag allows ROSE to read a binary file previously stored as a sql file (e.g. fetched from
         IDA Pro). 
   \item {\tt ../ROSE/configure --prefix=`pwd` --with-javaport=yes SWIG=swig --with-boost=[BOOST\_installTree]} \\
         This allows ROSE to be build with javaport, a support that connects ROSE to Java via SWIG.
         The Eclipse plug-in to ROSE is based on this work.
   \item Additional Examples \\ 
         More detailed documentation on configure options can be found by typing 
         {\tt configure --help}, or see figure
         \ref{gettingStarted:configureHelpOptionOutputPart1} for complete
         listing.
\end{itemize}

%\fixme{It is suggested by TID that we move the figures to this location.
%       Checkout if this is possible.}

Output of {\tt configure --help} is detailed in figures 
\ref{gettingStarted:configureHelpOptionOutputPart1} (Part 1) and
\ref{gettingStarted:configureHelpOptionOutputPart2} (Part 2):
{\indent
{\mySmallestFontSize
% Do this when processing latex to generate non-html (not using latex2html)
\begin{latexonly}
\begin{figure}[tb]
\begin{center}
\begin{tabular}{|c|} \hline
     {\tt configure --help} Option Output (Part 1)
\\\hline\hline
   \lstinputlisting{roseConfigureOptions.aa}
\\\hline
\end{tabular}
\end{center}
\caption{ Example output from configure --help in {\tt ROSE} directory (Part 1). }
\end{figure}
\end{latexonly}

% Do this when processing latex to build html (using latex2html)
\begin{htmlonly}
   \verbatiminput{roseConfigureOptions.aa}
   \vspace{0.5in}
   Example output from configure --help in {\tt ROSE} directory (Part 1).
\end{htmlonly}
\label{gettingStarted:configureHelpOptionOutputPart1}
%end of scope in font size
}
% End of scope in indentation
}

% Output of {\tt configure --help} (Part 2):
{\indent
{\mySmallestFontSize
% Do this when processing latex to generate non-html (not using latex2html)
\begin{latexonly}
\begin{figure}[tb]
\begin{center}
\begin{tabular}{|c|} \hline
     {\tt configure --help} Option Output (Part 2)
\\\hline\hline
   \lstinputlisting{roseConfigureOptions.ab}
\\\hline
\end{tabular}
\end{center}
\caption{ Example output from configure --help in {\tt ROSE} directory (Part 2). }
\end{figure}
\end{latexonly}

% Do this when processing latex to build html (using latex2html)
\begin{htmlonly}
   \verbatiminput{roseConfigureOptions.ab}
   \vspace{0.5in}
   Example output from configure --help in {\tt ROSE} directory (Part 2).
\end{htmlonly}
\label{gettingStarted:configureHelpOptionOutputPart2}
%end of scope in font size
}
% End of scope in indentation
}


\subsection{Running {\em GNU Make} in Parallel}
\label{gettingStarted:parallelMake}
     ROSE uses general {\tt Makefiles} and is not dependent on {\em GNU Make}.
     However, {\em GNU Make} has an option to permit compilation in parallel and 
     we support this. Thus you may use {\tt make} with the {\tt -j<n>} option if 
     you want to run {\tt make} in parallel (a good value for {\tt n} is typically 
     twice the number of processors in your computer).  We have paid special 
     attention to the design of the ROSE {\em makefiles} to permit parallel 
     {\tt make} to work; we also use it regularly within development work.

\subsection{Installing ROSE}
\label{gettingStarted:installation}
     Installation (using {\tt make install}) is optional, but suggested. Users can
simplify their use of ROSE by using it from an installed version of ROSE.  This permits
compilation using a single include directory and the specification of only two libraries, 
as in:
% {\footnotesize
\begin{verbatim}
     g++ -I{\<install dir\>/include} -o executable executable.C 
         -L{\<install dir\>/lib} -lrose -ledg -lm $(RT_LIBS)
\end{verbatim}
% }
See the example makefile in \\
{\tt ROSE/exampleTranslators/documentedExamples/simpleTranslatorExamples/exampleMakefile} \\
in Section \ref{gettingStarted:compilingTranslator}
for exact details of building a translator on your machine (setup by configure and tested
by {\tt make installcheck}).  Note that the tutorial example codes are also tested
by {\tt make installcheck} and the {\tt example\_makefile} there can also serve as
an example.

     {\bf autoconf} uses {\tt /usr/local} as the default location for all 
installations. Only {\em root} has write privileges to that directory, so you
will likely get an error if you have not overridden the default value with a new
location.  To change the location, you need to have used the 
{\tt --prefix=\{install\_dir\}} to run the {\tt configure} script.  You can 
rerun the {\tt configure} script without rebuilding ROSE.

%\fixme{ROSE-0.8.9a now sets the default install path (prefix) to the build tree. The
%    documentation does not yet really reflect this detail.  Also since the build process
%    writes to the install directory we have to explain this in the documentation.}

\subsection{Testing ROSE}
     A set of test programs is available.  %More details are in section \ref{testing}.
% written about this in later versions of the manual.  
Type {\tt make check} to run your build version of ROSE using these test codes.  
Several years of contributed bug reports and internal test codes have been accumulated 
in the {\tt ROSE/tests} directory.

\subsection{Getting Help}
     You may use the following mailing list to ask for help from the ROSE development 
team: {\it casc-rose *dot* llnl *dot* gov}.
