
\chapter{Binary Analysis: Support for the Analysis of Binary Executables}

\label{binaryAnalysis::overview}

\section{Introduction}

   ROSE supports the disassemble and analysis of binary executables for x86 and AMR
instruction sets.  ROSE implements this support as part of general research work
to support combining analysis for source code and analysis for binaries.  

\section{The Binary AST}

\subsection{The Binary Executable Format}

    ROSE handles Linux and Windows binary formats; thus ELF format for Linux and PE, NE,
LE, DOS formats for Windows.  The details of each format are represented in IR nodes in
the AST (using structures common to the representation of such low level data).  About
60 IR nodes have been added to ROSE to support the binary executable formats; this
support allows the analysis of any Linux or Windows, OS\/2, or DOS binary.

The binary executable file format can be analyized seperately from the instructions
using the command line option: {\tt -rose:read\_executable\_file\_format\_only}.  this
allows graphs generated using the ROSE visualization mechanisms to be easily restricted 
(in size) to the just the IR nodes specific to the binary executable file format.
\fixme{We need an example of the binary executable format AST.}

\subsection{Instruction Disassembly}

    ROSE has its own disassembler; a recursive disassembler that is well suited to
details of variable length instruction set handling and data stored in the instruction
stream.  All details of the instructions, and the roperands and operator expression trees,
etc. are stored in the binary AST as seperate IR nodes.
\fixme{We need an example of the AST for a few instructions.}

\section{Binary Analysis}

   A number of binary analysis passes are provided, most are a part of the Compass
framework for software analysis.  See the Compass manual for more details on supported
binary analysis.

   The ROSE tutorial shows a number of binary analysis passes over both the binary
instructions and the executable file format.

\section{Compass as a Binary Analysis Tool}

   Compass is a tool framework for building software analysis tools using rules; Compass
reports violations of the rules in the evaluation of the software.  Compass is a
reletively simple application built on top of ROSE.  Most of the complexity and code 
within Compass is that it includes a large collection to rules, each rule has its
own implementation of an arbitrary test over the source code or the binary.  Rules
(checkers) may be defined over the AST or any other graph built within ROSE to store 
program analysis information. See the Compass manual for more details on supported
binary analysis.

\section{Usage}
     See the ROSE Tutorial for an example.



