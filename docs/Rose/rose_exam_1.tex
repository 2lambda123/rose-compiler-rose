\chapter{Getting Started}

  This chapter is about getting started.  It is assumed that
you already are familiar with C++ and some of the simplest parts
of the Standard Template Language (STL). You will also need web 
access to the doxygen generated documentation for ROSE (availalbe
at {\em http://www.roseCompiler.org}.

% *******************************************************
%                  Source Code Analysis
% *******************************************************
\section{Source Code Analysis}
   This subsection is resticted to analysis of the input code
and generating information about what is in the input source code.

\subsection{Function Detection}
   Write the ROSE based analysis tool to detect functions in the input source code
and report the location in the file and the number of lines of code in each function.

Start using:
\begin{verbatim}
#include "rose.h"
int main( int argc, char * argv[] )
   {
  // Generate the ROSE AST.
     SgProject* project = frontend(argc,argv);

     std::vector<SgNode*> functions = NodeQuery::querySubTree(sageProject,V_SgFunctionDeclaration);
     for (std::vector<SgNode*>::iterator i = functions.begin(); i != functions.end(); i++)
        {
          printf ("function name = %s \n",*i->get_name().c_str());
        }

     return 0;
   }

\end{verbatim}

% *******************************************************
%               Source Code Transformation
% *******************************************************
\section{Source Code Transformations}
   This subsection deals with transformations on the 
input code and the generation of new code.

\subsection{Identity Translator}
   Write the code required to use ROSE to define a translator
that takes in any input C/C++/Fortran code and outputs the 
same code.

Start using:
\begin{verbatim}
#include "rose.h"
int main( int argc, char * argv[] )
   {
  // Generate the ROSE AST.
     SgProject* project = frontend(argc,argv);

  // AST consistency tests (optional for users, but this enforces more of our tests)
     AstTests::runAllTests(project);

  // regenerate the source code and call the vendor 
  // compiler, only backend error code is reported.
     return backend(project);
   }
\end{verbatim}

\subsection{Trivial Translator}
   Write the translator that will change all instances of the literal
{\tt 1} to {\tt 2} in any input source code.  This should be done only
in the input source file and not header files.




% *******************************************************
%                  Binary Code Analysis
% *******************************************************
\section{Binary Code Analysis}
   This subsection deals with analysis of the binary and the
output of information about the binary.

\subsection{Function Detection}
   Generate a translator that identifies all of the function in the binary.
{\em (Hint, this requires detecting only the SgAsmFunction IR nodes in the binary AST.)}
Report the function name, the number of instructions, and the starting and ending location
in the executable file and in memory if it were loaded for execution.

Start using the example from the source code analysis and substitute:
\begin{verbatim}
     std::vector<SgNode*> functions = NodeQuery::querySubTree(sageProject,V_SgAsmFunctionDeclaration);
\end{verbatim}



% *******************************************************
%               Binary Code Transformation
% *******************************************************
\section{Binary Code Transformations}
   This subsection tests your ability to rewrite parts of the binary
AST and regenerate the binary.  Note that depending on the transformation
the resulting binary may or may not be executable.

\subsection{NOP Translation}
   Rewrite sequences of single byte x86 NOP instructions to use the newer 
Intel x86 multi-byte NOP instructions.  Verify that the resulting binary 
is executable. Use only the space provided by the sequence of single byte NOP 
instructions else your generated executable will may not run properly.
{\em Hint: replace each instruction with the appropriate NOP instruction 
using the build interface function to NOP construction.}




\commentout{
Start using:
\begin{verbatim}
#include "rose.h"
int main( int argc, char * argv[] )
   {
  // Generate the ROSE AST.
     SgProject* project = frontend(argc,argv);

  // AST consistency tests (optional for users, but this enforces more of our tests)
     AstTests::runAllTests(project);

#if 0
  // This function will reset the source position information to match the generated code.
     project->resetSourcePositionToGeneratedCode();
#endif

#if 0
  // Output an optional graph of the AST (just the tree, when active)
     printf ("Generating a dot file... (ROSE Release Note: turn off output of dot files before committing code) \n");
     generateDOT ( *project );
#endif

#if 0
  // Output an optional graph of the AST (the whole graph, of bounded complexity, when active)
     const int MAX_NUMBER_OF_IR_NODES_TO_GRAPH_FOR_WHOLE_GRAPH = 8000;
     generateAstGraph(project,MAX_NUMBER_OF_IR_NODES_TO_GRAPH_FOR_WHOLE_GRAPH);
#endif

  // regenerate the source code and call the vendor 
  // compiler, only backend error code is reported.
     return backend(project);
   }

\end{verbatim}
}
