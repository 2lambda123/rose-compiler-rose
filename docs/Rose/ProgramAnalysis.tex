\chapter{Program Analysis}

\label{ProgramAnalysis}

% \newcommand{\DatabaseExampleDirectory}{@top_srcdir@/ExamplePreprocessors/DatabaseTutorials} %??
% \newcommand{\DatabaseExampleDirectory}{/home/thuerey1/rose/dbtest/examples}
% \newcommand{\DatabaseExampleDirectory}{/home/dquinlan/ROSE/NEW_ROSE/ExamplePreprocessors/DocumentedExamples/DataBaseExamples}

Program analysis is an important part of required support for sophisticated 
transformations. This work is currently incomplete and is
the subject of significant current research work.  Specific support for global analysis 
is provided via a database mechanism provided within ROSE and as part of work
in merging multiple ASTs from different files to hold the AST from a whole project
(many files) in memory at one time.

Note that ROSE supports binary analysis, chapter \ref{binaryAnalysis::overview}
contains details that are specific to program analysis on binaries (executables,
libraries, object files, etc.).
A goal of ROSE is to unify much of the program analysis for source code and binaries, 
but there are details that are presented separately.

\section{General Program Analysis}

   General program analysis is a critical piece of the work to
provide optimization capabilities to ROSE to support DOE applications.
This work generally lags behind the compiler construction issues
and robustness required to handle large scale DOE applications.

\subsection{Call Graph Analysis}

   Global {\it call graphs} are available, examples are in the ROSE Tutorial.
ROSE supports two modes of building the call graph, with and without SGLite 
database support (controled from the ROSE configure commandline). The use of SQLite
database support permits information to be accumulated into a named database file
over the compilation of many files as required to support large multi-file projects
(even separated over many directories).

   The call graph uses the new graph IR nodes to support handling large scale 
graphs in ROSE. {\em Need to comment more on this.}

   The mechanisms for filtering functions from the call graph is a subject
of ongoing work to reduce the size of the call graph for more useful analysis
and presentation.

Note:  The Call Graph currently has a mechanism for inclusion/exclusion of
paths/files locations of functions so that functions can be filtered from the
Call Graph.  This mechanism is currently hard coded in the test code and need
to be controled from the command line in the future.  ROSe included commandline
support for specification of (see {\tt --help} option for details):
\begin{itemize}
   \item include paths
   \item exclude paths
   \item include files
   \item exclude files
\end{itemize}

At present we can include paths for 
the locations of functions to be included in the Call Graph. The specification of
paths and files to include/exclude should be controled from the commandline.


\subsection{C++ Class Hierarchy Graph Analysis}

   Class hierarchy graphs are available, examples are in the
ROSE Tutorial.

\subsection{Control Flow Graphs}

   Control graphs exist in two forms, one that is closely tied to
the AST and another that is separate from the AST.  See the ROSE
Tutorial for examples of how to use these.

\subsection{Dependence Analysis}

   Complete use-def chains are available, the ROSE Tutorial shows examples of
how to access this information.

\subsection{Alias Analysis}

   A linear alias analysis is provided, we need to write more about this.


\subsection{Open Analysis}

   The Open Analysis project provides a connection to ROSE and permits
the use of their pointer analysis with ROSE.  More details on Open Analysis
(and a reference) later.

\subsection{More Program Analysis}

   Current work and collaborations will hopefully support an significant 
expansion of the program analysis supported within ROSE.  We are working with
a number of groups on pointer analysis, abstract interpretation, etc.

\section{Database Support for Global Analysis}
\label{RoseExamples:RoseTutorial:Database}

   The purpose of database support in ROSE is to provide a persistent
place for the accumulation of analysis results.  The database used within ROSE
is the publicly available SQLite relational database.  Work has
been done to provide a simple and extensible interface to SQLite.  The 
demonstration and testing of the ROSE database mechanism has been supported 
through the construction of the call graph and class hierarchy graphs. These
are discussed in subsequent subsections.

See chapter on {\em Getting Started} \ref{gettingStarted:Database}
for details of SQLite installation and configuration.  Previous work
supported MySQL, but this was overly complex.

\commentout{
% Older instructions written by Nils to describe installation and use of SQL Data Base
\subsection{Database Setup}
\label{sec:DatabaseRequirements}

ROSE uses a \emph{MySQL} database to store global information over
multiple runs of a preprocessor. If there is a MySQL database
running at your workplace, you can skip the installation instructions. Only make
sure the connection parameters like hostname and login are correct.
For the persistent storage of global information over multiple preprocessor runs
, the MySQL database, and
it's C++ interface \emph{MySQL++} should be installed before installing
ROSE. Simply download the current production release of MySQL and
the latest version of the MySQL++ API from the download section
of http://www.mysql.org , then configure and install first MySQL then
MySQL++. Under Unix, both can easlily installed locally into a home directory
without requiring root access. Once everything is in place, make sure the MySQL 
daemon is running, by e.g. executing ''{\tt ~/local/share/mysql/mysql.server start}''
if MySQL was installed into {\tt ~/local}.
\fixme{this isn't really a tutorial...?}
\fixme{How to configure ROSE parameters for database?}
} % commented out

\commentout{
Commands required to install MySQL and set it up for use:
\begin{verbatim}

% MySQL database must be used with 3.2.2 (not 3.3.2 until the correct patches are available).

% Type:
% patch -p 1 < ../mysql++-gcc-3.0.patch.gz
% patch -p 1 < ../mysql++-gcc-3.2.patch.gz
% patch -p 1 < ../mysql++gcc-3.2.2.patch.gz

% cd MySQL/mysql-4.0.17
% configure --prefix=`pwd`
% make -j2

% make install must be run in another directory
% cd ..
% mkdir MySQL_Install
% cd mysql-4.0.17
% configure --prefix=../MySQL_Install
% make install
% cd ../mysql++-1.7.9

% Rerun automake to rebuild Makefile.in files after runing patch (AM v1.4, AC v2.1 needed)
% /usr/bin/automake -a
% /usr/bin/autoconf

% ./configure --prefix=/usr/casc/overture/MySQL/MySQL_Install
%             --with-mysql=/usr/casc/overture/MySQL/MySQL_Install
%             --with-mysql-lib=/usr/casc/overture/MySQL/MySQL_Install
% make
% make install

% Run setup script
% cd MySQL/mysql-4.0.17/scripts
% ./mysql_install_db

% Start daemon (first)
%      We might want to run this in Qing's machines to simplify the access 
%      to the database through any other machies.
% cd /usr/casc/overture/MySQL/mysql-4.0.17 ; /usr/casc/overture/MySQL/mysql-4.0.17/bin/mysqld_safe &

% Set the password for access to the database
% /usr/casc/overture/MySQL/mysql-4.0.17/bin/mysqladmin -u root password 'rose-dq'

% Add a user (so everyone does not have to be root)
%     DELETE * FROM user WHERE user = '';
%          Removes default SQL configuration so that username is required
%     GRANT ALL ON db.* TO roseuser@'%.llnl.gov' IDENTIFIED BY 'rosepwd';
%          Adds new user; works from any llnl.gov domain and requires password
%     FLUSH PRIVILEGES;
%          Forces MySQL to reload privledged information (user table in mysql)
%
% /usr/casc/overture/MySQL/MySQL_Install/bin/mysql -u root -prose-dq mysql -e "DELETE FROM user WHERE user = ''; GRANT ALL ON db.* TO dquinlan@'%.llnl.gov' IDENTIFIED BY 'rosepwd'; FLUSH PRIVILEGES; SELECT * FROM user;"

% cd to ROSE Compile Tree
%     set --with-MySQL=/usr/casc/overture/MySQL/MySQL_Install (specify loaction of MySQL)
%     set --with-MySQL_username=roseuser (consistant with roseuser@'%.llnl.gov' specificed above)
%     set --with-MySQL_password=rosepwd (consistant with IDENTIFIED BY 'rosepwd' specified above)
%     set --with-MySQL_database_name=database-dq (choose unique name for database)
%
%  /home/dquinlan/ROSE/NEW_ROSE/configure --with-GNU_HEADERS --with-FLTK_include=/home/thuerey1/local/include --with-FLTK_libs=/home/thuerey1/local/lib --with-GraphViz_include=/home/thuerey1/local/include/graphviz --with-GraphViz_libs=/home/thuerey1/local/lib/graphviz --with-MySQL=/usr/casc/overture/MySQL/MySQL_Install --with-MySQL_username=roseuser --with-MySQL_password=rosepwd --with-MySQL_database_name=database-dq --with-CXX_DEBUG=-g --with-C_DEBUG=-g --with-CXX_WARNINGS=-Wall --enable-dq-developer-tests --with-AxxPxx=/home/dquinlan/ROSE/A++P++Install --prefix=/home/dquinlan2/ROSE/LINUX-3.3.2-FULL_OPTIONS

% How to login to mysql (directly):
% mysql -h localhost --user root -p

% Currently this only works from a different machine (not tux49)
% mysql -h tux49 --user roseuser -prosepwd
% --or--
% mysql -h tux49 -u roseuser -prosepwd

% How to view the user table in mysql
% USE mysql;
% SELECT * FROM user;
% --or--
% SHOW DATABASES;
% USE <database name>;
% SHOW TABLES;

% Run configure using command:
% /home/dquinlan/ROSE/NEW_ROSE/configure --with-GNU_HEADERS --with-FLTK_include=/home/thuerey1/local/include --with-FLTK_libs=/home/thuerey1/local/lib --with-GraphViz_include=/home/thuerey1/local/include/graphviz --with-GraphViz_libs=/home/thuerey1/local/lib/graphviz --with-MySQL=/usr/casc/overture/MySQL/MySQL_Install --with-MySQL_username=dquinlan --with-MySQL_server=tux49 --with-MySQL_password=rosepwd --with-MySQL_database_name=database-dq --with-CXX_DEBUG=-g --with-C_DEBUG=-g --with-CXX_WARNINGS=-Wall --enable-dq-developer-tests --with-AxxPxx=/home/dquinlan/ROSE/A++P++Install --prefix=/home/dquinlan2/ROSE/LINUX-3.3.2-FULL_OPTIONS

% Things to Install Tomorrow (with Nils):
%    Boost
%    graphviz

\end{verbatim}
} % commented out

\commentout{
% Older instructions written by Nils to describe installation and use of SQL Data Base
Script to help install MySQL:
\begin{verbatim}

# configure and make
./configure --prefix=/home/thuerey1/local
make
make install
cd scripts
./mysql_server_db

# start mysql daemon
cd /home/thuerey1/local/ ; /home/thuerey1/local//bin/mysqld_safe &
# set first password to 'rose'
/home/thuerey1/local//bin/mysqladmin -u root -h tux32.llnl.gov password 'rose'


# stop again
cd ~/local/share/mysql/
./mysql.server stop


# mysql++ install
# for 3.2.2 :
# untar... then, cd <dir>
patch -p1 < ../mysql++-gcc-3.0.patch
patch -p1 < ../mysql++-gcc-3.2.patch
patch -p1 < ../mysql++-gcc-3.2.2.patch
# AM v1.4, AC v2.1 needed - AM v1.5, AC v2.5 as from overture bin's do not work...
automake -a
autoconf
# end 3.2.2

./configure --prefix=/home/thuerey1/local/ --with-mysql=/home/thuerey1/local/ --with-mysql-lib=/home/thuerey1/local/lib/mysql/
- patch connection.cc to ..."(char *)mysql_info(&mysql);"

# create test tables (start mysql) then
/home/thuerey1/local/bin/mysql
use test;
create table testtab ( id int NOT NULL, text varchar(200), dat date, primary key (id) ) comment='test table';
show tables;
insert into testtab values (1, 'bla', '2003-08-08' );
select * from testtab;
insert into testtab (id,text,dat) values (2, 'bla', '2003-08-08' );

# create new database
./mysqladmin -u root create rose

# better config for multiple users from different computers:
# create roseuser(rosepwd) that can connect from any llnl.gov computer
# better use safer root pwd in this case
/home/thuerey1/local/bin/mysqladmin -u root password '<PWD>'
~/local/bin/mysql -u root -p<PWD> -e "GRANT ALL ON db.* TO roseuser@'%.llnl.gov' IDENTIFIED BY 'rosepwd'; FLUSH PRIVILEGES;"

\end{verbatim}
} % commented out


\subsection{Making a Connection To the Database and Table Creation}
\label{sec:TutDB:Connection}

% Do this when processing latex to generate non-html (not using latex2html)
\begin{latexonly}
\begin{figure}[tb]
\begin{center}
\begin{tabular}{|c|} \hline
     Database Connection Example
\\\hline\hline
\lstinputlisting{\DatabaseExampleDirectory/customtable_example.C}
\\\hline
\end{tabular}
\end{center}
\caption{ \label{fig:TutDB:Connection} Source code for the database connection example. }
\end{figure}
\end{latexonly}


Figure~\ref{fig:TutDB:Connection} shows the listing of a program that connects
to the ROSE database, creates a custom table, and performs some simple 
SQL queries. In the main function, at line 12, a \emph{GlobalDatabaseConnection} object is created
and is used to connect to the database in line 13. When the initialization
succeeds, the database connection and ROSE database are ready for use.

Line 16 creates a \emph{TableAccess} object. This object can be used to perform 
SQL queries like SELECT, INSERT or MODIFY on a given table in the database. The TableAccess
object is templated by a \emph{RowdataInterface} object that defines the structure 
of the table. For this example program, a RowdataInterface object for a test table
is created in line 6 and 7. Here, two macros are called that handle the definition of
the RowdataInterface class and all standard member functions. The general syntax
is {\tt CREATE\_TABLE\_[n]( [tablename], [column-1-datatype], [column-1-name], 
[column-2-datatype], [column-2-name], ... [column-n-datatype], [column-n-name] );},
where the ''[...]'' represents values to be filled in, such as the name of the table. As column datatype, all standard
C-datatypes as {\em bool,char,short,long,float,double etc.} are valid. The resulting
RowdataInterface class will contain standard functions to retrieve information about the table or
its columns. An instance of this class has all private member variables to store
the data of a single row of the table. Furthermore it has {\tt get\_[column-X-name]()} functions together with 
the corresponding {\tt set\_[column-X-name]([value])} functions to modify the values. By 
convention, tables used in ROSE will have one column more than specified, hence, $n+1$ in total.
The first column, which is always added, is a column of type {\tt int} with the name
{\em id}. This is used to easily identify all rows of a table. RowdataInterface classes
used as template argument with a TableAccess class are required to have an {\em id}-column.

The class created by \emph{CREATE\_TABLE} will be called ''[tablename]Rowdata,'' where ''[tablename]''
is the first argument for the \emph{CREATE\_TABLE} macro-call.  The \emph{DEFINE\_TABLE} call
is necessary to define global and static member variables of the RowdataInterface class. It has
to be called once in a project, e.g. in the source file containing the main function, with exactly 
the same parameters as the \emph{CREATE\_TABLE} call.
Thus, lines 6 and 7 together with lines 16 and 17 define the test table as having three columns:
an integer ''id'' column, a ''name'' column storing a string and finally a third column ''number''
storing a double precision floating point number. The {\tt initialize} call in line 17
will ensure the table exists and create it if necessary.

The next two statements at line 20 and 21 create a rowdata object that
stores all fields of a single row of the test table. The constructor has
initial arguments for all of the fields of a row. In this case ''name'' and
$1.0$ are used to initialize the field's name and number, which were specified
in lines 6 and 7. The first argument \emph{UNKNOWNID} is used to set the value
of the row-id to the default value $0$, which means that the id is not yet
properly initialized. $0$ is never used as an id for table rows; the lowest 
possible valid id is $1$. Note that the insert function initializes
the id of the row, as insert will create a new row in table that has a valid id.

In lines 24 and 25 a SQL query is performed, which selects all rows where the 
number column is equal to $1.0$. The string passed to the \emph{select} 
member function call contains the conditional expression (the \emph{WHERE}
clause) of an SQL statement. Hence, the single equals sign is an SQL
equality test, and not, for example, an assignment. The selected rows of the table
are returned as a vector of rowdata objects. As in line 21, a row matching 
the select condition was inserted into the table, so at least one row should
be returned (if the example program was executed multiple times without
deleting the test table, entries from previous runs may be returned as well).
Assuming that the example program is run for the first time, the SQL
query should return the inserted row, and the first object in the results vector
should be identical to the inserted one. Lines 27 and 28 modify the name 
and number fields in memory. The modify call in line 29 then updates
the database, by changing the existing row in the table and making the changes
persistent. Line 32 is an exemplary call to delete a row of the table -- the 
deletion uses the id of a row, so all other fields do not have to contain the
same values as the row stored in the database.

The insert statement in line 35 simply inserts the row just deleted into the 
table again, leaving the test table in a different state. Hence, executing
the example program multiple times should fill the test table with multiple rows.
In line 37, the connection to the database is closed. Try to add a call to
{\tt GlobalDatabase::DEBUG\_dump()} before the shutdown function call, and run
it multiple times to see how the automatic id assignment works.


\subsection{Working With the Predefined Tables}
\label{sec:TutDB:RoseTables}

% Do this when processing latex to generate non-html (not using latex2html)
\begin{latexonly}
\begin{figure}[tb] 
\begin{center} 
\begin{tabular}{|c|} \hline
     Table Creation Example
\\\hline\hline
\lstinputlisting{\DatabaseExampleDirectory/rosedb_example.C}
\\\hline
\end{tabular} \end{center}
\caption{ \label{fig:TutDB:RoseTables} Source code for the predefined tables example. }
\end{figure} 
\end{latexonly}

While the first database example worked on a self-defined table, this tutorial will 
explain how to use one of the tables that are predefined for usage within ROSE.
Its source code is shown in Figure~\ref{fig:TutDB:RoseTables}. These
tables are easier to use because their structure is already defined in the {\tt TableDefinition.h}
file. Lines 6 and 7 define the tables used for storing information about projects and files
in ROSE using the macros \emph{DEFINE\_TABLE\_PROJECTS} and \emph{DEFINE\_TABLE\_FILES}. 
These macros call the corresponding macros from the previous example to define the structure
of these tables. 

The easiest way to use these tables is the \emph{CREATE\_TABLE} macro.
The first parameter is a GlobalDatabaseConnection object, the second one is the name of the
table. Hence, line 17 will initialize the projects table, and create an instance of
the projectsTableAccess object having the same name as the table, ''projects.'' Line 18
initializes the files table in the same way. Now two instances of the TableAccess class
for the projectsRowdata and the filesRowdata objects are declared in the main scope, and 
are ready to be used. 

The example program performs an initialization to retrieve the ids for the 
project and the file currently processed, which is usually needed for 
a traversal. Lines 21 and 22 set values for project and file name, although
these values might normally be retrieved from the corresponding SgProject
and SgFile nodes \fixme{project name? parameter?}. As all projects work on 
the single ROSE database and share the same tables for function and
data, each of these tables has a \emph{projectId} column to specify to which 
project each row belongs. Thus, one of the first tasks a preprocessor using the
database will do is to enable these ids to select or insert rows.

The \emph{TableAccess::retrieveCreateByColumn} function is used for this purpose.
It tries to identify an entry using a unique name, and creates that entry
if it does not yet exist, or retrieves the id of the existing entry otherwise. 
The function takes a pointer to a rowdata object, the name of the column to use,
and the unique name of the row as arguments (see line 25). So in this case
the ''name'' column and the string ''testProject'' are used. As with the 
normal insert function from the first example, the retrieveCreateByColumn function
sets the id field of the rowdata to the correct value. A new variable storing
this project id is created in line 27. For the file id, the procedure is almost
the same -- with the exception that the project id is also passed to the
function call in line 32. For most other ids other than the project id,
the project id is used to retrieve the row for the desired project. If 
a project id is passed to the retrieveCreateByColumn function, it assumes 
the table has a ''projectId'' column, which has to match the given value.

Instead of working with these ids, the example program just prints these
values to stdout, and quits. The ids will remain the same over multiple runs
of the program. Try changing the file or project ids, to force new entries
to be created.


\subsection{Working With Database Graphs}
\label{sec:TutDB:DBGraphs}

% Do this when processing latex to generate non-html (not using latex2html)
\begin{latexonly} 
\begin{figure}[tb] 
\begin{center} 
\begin{tabular}{|c|} \hline
    	Database Graph Example
\\\hline\hline
\lstinputlisting{\DatabaseExampleDirectory/databasegraph_example.C}
\\\hline
\end{tabular} \end{center}
\caption{ \label{fig:TutDB:DatabaseGraph} Source code for the database graph example }
\end{figure} 
\end{latexonly}

The following tutorial program will use the ROSE tables to build a graph for
a user-defined table. Each execution of the program will enlarge the test
graph by adding three nodes and edges to them from a random node in the graph.

\subsection{A Simple Callgraph Traversal}
\label{sec:TutDB:SimpleCG}

% Do this when processing latex to generate non-html (not using latex2html)
\begin{latexonly} 
\begin{figure}[tb] 
\begin{center} 
\begin{tabular}{|c|} \hline
     Simple Callgaph Example
\\\hline\hline
\lstinputlisting{\DatabaseExampleDirectory/simplecallgraph_example.C}
\\\hline
\end{tabular} \end{center}
\caption{ \label{fig:TutDB:SimpleCallgraph} Source code for the simple callgraph example }
\end{figure}
\end{latexonly}

The last database example tutorial will show how to use the database graph features
explained in the previous example in combination with a AST-traversal to
build a simple callgraph.  










