\chapter{AST Rewrite Mechanism}
\label{AstRewrite:AstRewrite}
\textcolor{red}{\bf WARNING: AST Rewrite Mechanism is not robust. Please use SageInterface functions instead! See ROSE tutorial's AST construction Chapter for details.}

% Purpose:
%\begin{itemize}
%   \item A. Section overview
%   \item B. Why you should read this manual
%   \item C. How to use this manual
%   \item D. Terminology
%   \item E. Overview of library
%   \item F. Program control
%   \item G. Error messages
%   \item H. Section summary
%\end{itemize}
%\begin{center}
%*********************  \newline
%\end{center}
%\vspace{0.25in}

\commentout{
Visionary part not to be included in documentation yet!

     Semantics:
       1) The newest AST is always traversed (rewrite rules are always applied to the
          newest possible AST).
           A) Transformations on header files are seen by the intermediate files in
              subsequent transformations.  In order to make this work we have to write out
              the whole AST.
                a) Optimizations could be done if no header files are transformed, since then
                   the original header files are the same.

    Guiding (problematic) example:
       1) header file (A.h):
             class A { int x; };
          Source file:
             int main () { A a; a.x = 0; }
          Transformation 1:
             Add ``int y;'' to class A (transformation on header file foo.h)
          Transformation 2:
             Add ``a.y = 0;'' to main function

          Transformation 2 depends upon transformation 1 and the intermediate source code
          which specifies transformation 2 requires the updated header file (defined after
          transformation 1).

    Things to do:
        1) Add unparse whole AST to SgFile.
        2) Unparse whole AST and generate new header files with include statements in the
           source file.

}


% Quote:
% (e.g.) The successful con artist is a mirror of the time and place where he works.

   The Abstract Syntax Tree (AST) Rewrite Mechanism permits modifications 
to the AST.  To effect changes to the input source code, modifications to the 
AST are done by a ROSE translator; and new version of the source code is produced.
% to effect changes to the input source code output as generated source code.  
Although analysis is possible by only reading the AST, transformations (and changes in the
output code from the input code) can only be accomplished by rewriting portions of the AST.
The AST is the single intermediate form manipulated by the preprocessor.  All changes
are eventually output as modifications to the input source code after being processed
through the intermediate form.

% \section{Introduction}
% \label{AstRewrite:introduction}

   The material in this chapter builds on material presented in the previous two
chapters; Writing a Source-to-Source Preprocessor (chapter 
\ref{preprocessorDesign:preprocessorDesign}) and AST Processing 
(chapter \ref{AstProcessing:astProcessing}). This chapter presents the required
AST Rewrite Traversal and the simple interface functions to the {\tt AST\_Rewrite} class.  
A section is included that 
demonstrates code that rewrites the AST for any input code. More complex examples are
possible but each uses the AST Rewrite Mechanism in a similar way.  The ROSE Tutorial 
documents a few more interesting examples.% \ref{RoseExamples:RoseExamples}.

\section{Introduction}

   The rewrite mechanism in ROSE contains four different levels of interface
within its design.  Table \ref{tab:RewriteInterface} shows the different levels of
the interface design for the ROSE rewrite mechanism.  Each level consists of
simple tree editing operations ({\tt insert()}, {\tt replace()}, and 
{\tt remove()}) that can operate on statements within the AST.

\begin{table}[htb]
{\begin{center}
    \renewcommand{\arraystretch}{1.25}

\begin{tabular}{|c|c|c|l|}
\hline
\multirow{3}{35mm}{Relative Positioning (contains state)} 
     & \multirow{3}{15mm}{String-Based} 
          & \multirow{3}{35mm}{High Level Interface (level 4)} 
               & {insert(SgNode*,string,scope,location)} \\
     & {} & {} & {replace(SgNode*,string,scope,location)} \\
     & {} & {} & {remove(SgNode*)} \\\hline
\multirow{9}{35mm}{Absolute Positioning (contains no state)} 
     & \multirow{3}{15mm}{String-Based} 
          & \multirow{3}{35mm}{Mid Level Interface (level 3)} 
          & {insert(SgNode*,string,location)} \\ 
     & {} & {} & {replace(SgNode*,string,location)} \\
     & {} & {} & {remove(SgNode*)} \\\cline{2-4}
     & \multirow{6}{15mm}{SgNode*} 
          & \multirow{3}{35mm}{Low Level Interface (level 2)} 
          & {insert(SgNode*,SgNode*)} \\ 
     & {} & {} & {replace()} \\
     & {} & {} & {remove(SgNode*)} \\\cline{3-4}
     & {} & \multirow{3}{35mm}{SAGE III Interface (level 1)} 
          & {insert(SgNode*,SgNode*)} \\ 
     & {} & {} & {replace(SgNode*,SgNode*)} \\ 
     & {} & {} & {remove(SgNode*)} \\\hline 
\end{tabular}

 \end{center}
}
  \caption{Different levels of the ROSE Rewrite mechanism.}
  \label{tab:RewriteInterface}
\end{table}

% Many attempts to input source code from files
% \verbatiminput{\AstRewriteExampleDirectory/testExample1.C}
% \verbatiminput{/home/dquinlan/ROSE/NEW_ROSE/AST_RewriteMechanism/testExample1.C}
% \verbatimfiles{/home/dquinlan/ROSE/NEW_ROSE/AST_RewriteMechanism/testExample1.C}
% \verbatimlisting{/home/dquinlan/ROSE/NEW_ROSE/AST_RewriteMechanism/testExample1.C}
% \begin{listing}
%     testExample1.C
% \end{listing}
% \listinginput{5}{/home/dquinlan/ROSE/NEW_ROSE/AST_RewriteMechanism/testExample1.C}

\section{Multiple Interfaces to Rewrite Mechanism}

   There are four different levels of interfaces in the rewrite mechanism
because there are many different program transformations requirements.  Each level
builds on the lower level, and the highest level interface is the most sophisticated 
internally. Each interface has only three functions: {\tt insert()}, {\tt replace()}, 
and {\tt remove()}.

\subsection{SAGE III Rewrite Interface}
   This lowest possible level of interface is implemented as member functions on the
SgNode objects. It is used internally to implement the higher level interfaces (including
the Low Level Rewrite Interface.  Uniformly,
operations of {\tt insert()}, {\tt replace()}, and {\tt remove()} apply only to SAGE III 
objects representing containers (SAGE III objects that have containers internally, 
such as SgGlobal, SgBasicBlock, etc.).  Strings cannot be specified at this level of
interface; only subtrees of the AST may be specified.  New AST fragments must be
built separately and may be inserted or used to replace existing AST subtrees in the AST.
Operations using this interface have the following properties:
\begin{itemize}
   \item Operations performed on collections only.
   \item Operations are immediate executed.
   \item Operations are local on the specified node of the AST.
   \item Operations do not take attached comments or preprocessor directives into account. \\
    This can lead to unexpected results (e.g. removing or moving {\tt \#include} directives
    by accident).
\end{itemize}

\subsection{Low Level Rewrite Interface}
    This interface is similar to the SAGE III Rewrite Interface except that operations
are performed on any statement and not on the containers that store the statement lists.
The domain of the operations -- on the statements instead of on the parent nodes of the 
statements -- is the most significant difference between the two interfaces.  An additional
feature includes support for repositioning attached comments/directives from removed
nodes to their surrounding nodes to preserve them within {\tt replace()} and 
{\tt remove()} operations.  Additional support is provided for marking inserted statements
as transformations within the Sg\_File\_Info objects.
Operations using this interface have the following properties:
\begin{itemize}
   \item Attached comments/directives are relocated.
   \item Inserted AST fragments are marked within the Sg\_File\_Info objects.
   \item Operations are immediate.
   \item Operations are local on the specified node of the AST.
\end{itemize}

\subsection{Mid Level Rewrite Interface}
    This interface builds on the low-level interface and adds the string interface,
which permits simpler specification of transformations.
Operations using this interface have the following properties:
\begin{itemize}
   \item Strings used to specify transformations.
   \item Operations are immediate.
   \item Operations are local on the specified node of the AST.
\end{itemize}

\subsection{High Level Rewrite Interface}
    This interface presents the same string based rewrite mechanism as the mid-level
interface but adds additional capabilities.
This interface is the most flexible rewrite interface within ROSE.  Although it 
must be used within a traversal to operate on the AST, it provides a mechanism to express
more sophisticated transformations with less complexity due to its support of relative 
positioning of transformation strings within the AST (relative to the current node within
a traversal). 

    The high-level rewrite mechanism uses the same three functions as the other rewrite
interfaces, but with an expanded range of enum values to specify the intended scope
and the location in that scope.  The scope is specified using the ScopeIdentifierEnum
type defined in the HighLevelCollectionTypedefs class. These enum values are:
\begin{itemize}
   \item unknownScope
   \item LocalScope 
   \item ParentScope 
   \item NestedLoopScope
   \item NestedConditionalScope
   \item FunctionScope
   \item FileScope
   \item GlobalScope
   \item Preamble
\end{itemize}
The position in any scope is specified by the PlacementPositionEnum
type, which is defined in the HighLevelCollectionTypedefs class. 
These enum values are:
\begin{itemize}
     \item PreamblePositionInScope
     \item TopOfScope
     \item TopOfIncludeRegion
     \item BottomOfIncludeRegion
     \item BeforeCurrentPosition
     \item ReplaceCurrentPosition
     \item AfterCurrentPosition
     \item BottomOfScope
\end{itemize}

Function prototypes of interface functions:
{\indent
{\mySmallFontSize
\begin{verbatim}
   void insert (SgNode*, string ,HighLevelCollectionTypedefs::ScopeIdentifierEnum,HighLevelCollectionTypedefs::PlacementPositionEnum);
\end{verbatim}
}}
Example of how to use specific insertion of transformation into the AST (required traversal not shown):
{\indent
{\mySmallFontSize
\begin{verbatim}
   insert (astNode, ``int x;'' ,HighLevelCollectionTypedefs::FunctionScope,HighLevelCollectionTypedefs::TopOfScope);
\end{verbatim}
}}
Operations using this interface have the following properties:
\begin{itemize}
   \item Adds relative positioning to the specification of transformations.
   \item Requires traversal for operation on the AST.
   \item Operations are delayed and occur durring the required traversal, all operations
    are completed by the end of the traversal.
   \item Operations occur on AST nodes along a path defined by the chain from the current
    input node to the operator to the root node of the AST (SgProject).
\end{itemize}



\subsection{Advantages and Disadvantages of Rewrite Interfaces}

   Each interface builds upon the lower level interfaces and each 
has some advantages and disadvantages. Table 
\ref{tab:RewriteInterfaceAdvantagesDisadvantages} lists the
major features and requirements associated with each.  The
high-level interface (Level 4) presents the most sophisticated 
features, but only works as part of a traversal of the AST.  The 
mid-level interface is the lowest level interface that permits 
the specification of transformations as strings.  The low-level interface
is useful when AST fragments are built directly using the SAGE III
classes through their constructors (a somewhat tedious process).
The low level interface preserves the original interfaces adopted 
from SAGE II.

\begin{table}[htb]
{\begin{center}
    \renewcommand{\arraystretch}{1.25}

\begin{tabular}{|c|c|c|c|c|}
\hline
 {Interface:Features} & {Contains State}  & {Positioning} & {String} & {Traversal} \\\hline
     Level 1 & No State &  Absolute  & AST Subtree & Not Used    \\
     Level 2 & No State &  Absolute  & AST Subtree & Not Used    \\
     Level 3 & No State &  Absolute  &    String   & Not Used    \\
     Level 4 &  State   &  Relative  &    String   & Required    \\
\hline
\end{tabular}

 \end{center}
}
  \caption{Advantages and disadvantages of different level interfaces 
           within the ROSE Rewrite Mechanism.}
  \label{tab:RewriteInterfaceAdvantagesDisadvantages}
\end{table}

\section{Generation of Input for Transformation Operators}
    Providing operators to {\tt insert()}, {\tt replace()}, {\tt remove()}
solves only part of the problem of simplifying transformations.  The other part 
of the problem is generating the input to the transformation operators.  Both 
{\tt insert()} and {\tt replace()} require input, either as an AST fragment or
as a string containing source code.  This section presents the pros and cons of the
specification of transformations as strings.

\subsection{Use of Strings to Specify Transformations}

    The mid-level and high-level rewrite interfaces introduce the use of
strings to specify transformations.  Using strings to specify transformations 
attempts to define a simple mechanism for a non-compiler audience 
to express moderately complex transformations.  The alternative is to build the
AST fragments to be inserted directly using SAGE III and the constructors
for its objects.  In general, the direct construction of AST fragments
is exceedingly tedious, and while aspects can be automated, the most
extreme example of this automation is the AST constructions from source 
code strings.  A disadvantage is that the generation of the AST fragment
from strings is slower, but it is only a compile-time issue.

\subsection{Using SAGE III Directly to Specify Transformations}

   It is possible to build AST fragments directly using SAGE III and insert these into
the AST.  This alternative to the use of strings is more complex and is only
briefly referenced in this section.

   The constructors for each of the SAGE III objects form the user interface required to
build up the AST fragments.  The documentation for these appear in the reference chapter
of this manual.

A few notes:
\begin{enumerate}
 \item Use the {\tt Sg\_File\_Info* Sg\_File\_Info::generateDefaultFileInfoForTransformationNode();} 
static member function to generate the Sg\_File\_Info object required for each of the
constructor calls.  This marks the IR nodes as being part of a transformation and signals
that they should be output within code generation (unparsing).
\end{enumerate}


\section{AST Rewrite Traversal of the High-Level Interface}

    The AST Rewrite Mechanism uses a traversal of the AST, similar to the
design of a traversal using the AST Processing 
(Chapter \ref{AstProcessing:astProcessing}) part of ROSE.
The example code \ref{AstRewrite:example1} specifically shows an {\bf AstTopDownBottomUpProcessing}
\ref{AstProcessing:AstTopDownBottomUpProcessing}
traversal of the AST.  Using conditional compilation,
the example code shows the somewhat trivial changes required to convert
a read-only AST traversal into a read-write AST rewrite operation. In this
example the AST traversal is converted to be ready for rewrite operations, but
no rewrite operations are shown. The purpose of this example is only to show the
modifications to an existing traversal that are required to use the AST rewrite
mechanism.

   The specialized AST rewrite traversal is internally derived from the ASTProcessing
{\bf TopDownBottomUp} traversal (processing) but adds additional operations
in recording the local context of source code position (in the inherited attribute)
and performs additional operations on the way back up the AST (on the synthesized attribute).

% \vspace{0.5in}

% A note to the reader about a difference between the two versions
% of the documentation that we generate.
\begin{latexonly}
% This listing appears in B\&W within the html version of the documentation.
\end{latexonly}
\begin{htmlonly}
% This listing appears in color within the generated latex postscript document.
\end{htmlonly}
% This reference doesn't seem to work properly.
% See link to postscript documentation at \ref{Rose:postscriptVersionOfUserManual}.

% DQ (9/12/2006): If we make this a figure then it will overflow the page.
% maby we should split this figure into two pieces (as is done in the ROSE tutorial for
% some of the longer examples there.
% \begin{figure}[tb]
{\indent
{\mySmallFontSize
\label{AstRewrite:exampleShowingModificationsToUseAstRewriteMechanism}
% Do this when processing latex to generate non-html (not using latex2html)
\begin{latexonly}
   \lstinputlisting{\AstRewriteExampleDirectory/astRewriteExample1.C}
\end{latexonly}
% Do this when processing latex to build html (using latex2html)
\begin{htmlonly}
   \verbatiminput{\AstRewriteExampleDirectory/astRewriteExample1.C}
\end{htmlonly}
%end of scope in font size
}
% End of scope in indentation
}
% \caption{ Example of now to use AST Rewrite Mechanism. }
% \label{AST_Code}
% \end{figure}

\fixme{This should be a figure that fits onto a single page.}

\commentout{
\begin{latexonly}
\begin{figure}[tb]
\begin{center}
\begin{tabular}{|c|} \hline
     Example Showing AST Rewrite Traversal
\\\hline\hline

% \lstinputlisting{/home/dquinlan/ROSE/NEW_ROSE/AST_RewriteMechanism/testExample1.C}
\lstinputlisting{\AstRewriteExampleDirectory/astRewriteExample1.C}

\\\hline
\end{tabular}
\end{center}
\caption{ Example of now to use AST Rewrite Mechanism. }
\label{AST_Code}
\end{figure}
\end{latexonly}
}

   This example shows the setup required to use the AST Rewrite Mechanism.
The next section shows how to add new code to the AST.  The {\tt main()}
function is as in example of how to use a traversal (see chapter 
\ref{preprocessorDesign:AstTraversalPreprocessor}).

   Note that the differences between the traversal required for use
with the AST Rewrite Mechanism is different from the traversals associated
with \ref{AstProcessing:AstTopDownBottomUpProcessing}.  The exact differences
are enabled and disabled in the example
\ref{AstRewrite:exampleShowingModificationsToUseAstRewriteMechanism}
by setting the macro {\tt USE\_REWRITE\_MECHANISM} to zero (0) or one (1).

   The differences between traversals using
{\tt AstTopDownBottomUpProcessing<InheritedAttrbute,SynthesizedAttribute>}
and traversals using the AST Rewrite Mechanism (
{\tt AST\_Rewrite::RewriteTreeTraversal<InheritedAttrbute,SynthesizedAttribute>})
are both required to use the AST Rewrite Mechanism. They are:
\begin{enumerate}
     \item InheritedAttributes must derive from {\tt AST\_Rewrite::InheritedAttribute}.
     \item Must define constructor {\tt InheritedAttribute::InheritedAttribute(SgNode* astNode)}.
     \item Must define copy constructor: \\
              {\tt InheritedAttribute::InheritedAttribute(const InheritedAttribute \& X, SgNode* astNode)}.
     \item SynthesizedAttribute must derive from {\tt AST\_Rewrite::SynthesizedAttribute}
     \item Must derive new traversal from \\ 
              {\tt AST\_Rewrite::RewriteTreeTraversal<InheritedAttrbute,SynthesizedAttribute>})
           instead of \\
              {\tt AstTopDownBottomUpProcessing<InheritedAttrbute,SynthesizedAttribute>}.
\end{enumerate}


\section{Examples}
    This section presents several examples using the different interfaces
to specify simple transformations.  

\commentout{
\subsection{SAGE III Interface Example}

\fixme{Incomplete documentation.}

\subsection{Low-Level Interface Example}

\fixme{Incomplete documentation.}

\subsection{Mid-Level Interface Example}

\fixme{Incomplete documentation.}

\subsection{High-Level Interface Example}

\fixme{Incomplete documentation.}
}

\subsection{String Specification of Source Code}
    Both the mid-level and high-level interfaces use strings to
specify source code. The examples below show how to specify 
the strings.
\subsubsection{Specification of Source Code}
    Specification of source code is straight forward.
However, quoted strings must be escaped and strings spanning
more then one line must use the string continuation character ("$\backslash$").
\begin{itemize}
   \item {\tt MiddleLevelRewrite::insert(statement,"int newVariable;",locationInScope);}
   \item {\tt MiddleLevelRewrite::insert(statement,"timer($\backslash$"functionName$\backslash$");",locationInScope);}
   \item {\tt MiddleLevelRewrite::insert(statement,\\ "/* Starting Comment */
    $\backslash$n $\backslash$\\ int y; { int y; } for (y=0; y < 10; y++){z = 1;}
    $\backslash$n $\backslash$\\ /* Ending Comment */$\backslash$n",locationInScope);}
\end{itemize}

\subsubsection{Specification of CPP Directives}
    Specification of CPP directives as strings is as one would expect
except that where quotes ("") appear in the string they must be escaped
($\backslash$"$\backslash$") to remain persistent in the input string.
\begin{itemize}
   \item {\tt MiddleLevelRewrite::insert(statement,"\#define TEST",locationInScope);}
   \item {\tt MiddleLevelRewrite::insert(statement,"\#include<foo.h>",locationInScope);}
   \item {\tt MiddleLevelRewrite::insert(statement,"\#include $\backslash$"foo.h$\backslash$"",locationInScope);}
\end{itemize}

\subsubsection{Specification of Comments}
    Specification of comments are similar.
\begin{itemize}
   \item {\tt MiddleLevelRewrite::insert(statement,"/* C style comment test */",locationInScope);}
   \item {\tt MiddleLevelRewrite::insert(statement,"// C++ comment test ",locationInScope);}
\end{itemize}

\subsubsection{Specification of Macros}
    The specification of macros is similar to CPP directives
except that longer macros often have line continuation and formatting.
We show how to preserve this in the example macro definition below.
Transformation involving the use of a macro is more complex if the
macro call is to be preserved in the final transformation 
(left unexpanded in the generation of the AST fragment with the 
rewrite mechanism).

\paragraph{Macro Definition:}
   A macro definition is similar to a CPP directive. The long example is taken from the
Tuning Analysis Utilities (TAU) project which instruments code with similar macro calls.
\begin{itemize}
   \item {\tt MiddleLevelRewrite::insert(statement,"\#include<foo.h>",locationInScope);}
   \item {\tt MiddleLevelRewrite::insert(statement,"\#include $\backslash$"foo.h$\backslash$"",locationInScope);}
   \item {\tt MiddleLevelRewrite::insert(statement,"\#define PRINT\_MACRO(name) name;",locationInScope);}
   \item {\tt MiddleLevelRewrite::insert(statement,\\
"$\backslash$n$\backslash$\\
\#ifdef USE\_ROSE$\backslash$n$\backslash$\\
// If using a translator built using ROSE process the simpler tauProtos.h header  $\backslash$n$\backslash$\\
// file instead of the more complex TAU.h header file (until ROSE is more robust) $\backslash$n$\backslash$\\
   \#include $\backslash$"tauProtos.h$\backslash$"$\backslash$n$\backslash$n\\
// This macro definition could be placed into the tauProtos.h header file $\backslash$n$\backslash$\\
   \#define TAU\_PROFILE(name, type, group) $\backslash$$\backslash$$\backslash$n$\backslash$\\
        static TauGroup\_t tau\_gr = group; $\backslash$$\backslash$$\backslash$n$\backslash$\\
        static FunctionInfo tauFI(name, type, tau\_gr, \#group); $\backslash$$\backslash$$\backslash$n$\backslash$\\
        Profiler tauFP(\&tauFI, tau\_gr); $\backslash$n$\backslash$\\
\#else$\backslash$n$\backslash$\\
   \#include $\backslash$"TAU.h$\backslash$"$\backslash$n$\backslash$\\
\#endif"$\backslash$$\backslash$\\
,locationInScope);}
\end{itemize}

\paragraph{Macro Use:}
   This example of macro use shows how to leave the macro unexpanded in the
AST fragment (which is generated to be patched into the application's AST).
\begin{itemize}
   \item {\tt MiddleLevelRewrite::insert(statement, \\
                 MiddleLevelRewrite::postponeMacroExpansion("PRINT\_MACRO($\backslash$"Hello World!$\backslash$")"),locationInScope);} \\
   \item {\tt MiddleLevelRewrite::insert(statement, \\
                 MiddleLevelRewrite::postponeMacroExpansion("TAU\_PROFILE($\backslash$"main$\backslash$", \\
                 $\backslash$"$\backslash$",TAU\_USER)"),locationInScope);}
\end{itemize}


\commentout{
\section{Rewriting the AST}

   Modifications to the AST involve the specification of new source code
(as a string) and a relative or absolute position within the AST.  
Since the specification of new source code occurs within a traversal 
(at any node of the AST), the position specified is relative to the 
current node within the traversal.

The new source code can be added before or after any location in the AST.
The specification of a position for new source code requires the
specification of a scope and a position within the scope.
New source code can also replace existing subtrees of the AST (internally
this is handled by insertion of new code and removal of old subtrees).

The scope is specified relative to the current position in the AST (during a traversal 
of the AST) using either:
\begin{enumerate}
     \item Local Scope (default)
     \item Parent Scope \\
           The specification of an optional parameter permits backing up the tree though
           multiple nestings (parent of parent scope, applied recursively up to the 
           Global Scope).
     \item Global Scope
\end{enumerate}

Within any specific scope, new source code may be positioned at either:
\begin{enumerate}
     \item Top of scope \\
           This applies only to AST scope statements: \fixme{Complete the list of scope statements.}
          \begin{itemize}
               \item Basic Blocks
               \item {\tt for} Statements
               \item {\tt while} statements
          \end{itemize}
     \item Before current statement
     \item Replace current statement
     \item After current statement
     \item Bottom of scope \\
           This applies to all of the same statements (locations) listed above 
           for top of scope.  In case a {\tt return} statement exists the new
           code is placed before the {\tt return} statement.
\end{enumerate}

Within the global scope additional locations can be specified:
\begin{enumerate}
     \item Top of {\tt \#include} region
     \item Bottom of {\tt \#include} region
\end{enumerate}


Information about the position of new source code (relative or absolute scope and position
in scope) and is used together with a string (C++ {\tt std::string} type) containing the new
source code. This data in used in the single user interface function of the 
{\tt AST\_Rewrite} class \ref{AstRewrite:AstRewriteInterfaceFunction}.

{\indent
{\mySmallFontSize

\label{AstRewrite:AstRewriteInterfaceFunction}

% Do this when processing latex to generate non-html (not using latex2html)
\begin{latexonly}
   \lstinputlisting{\AstRewriteExampleDirectory/astRewriteInterfaceFunction.codeFragment}
\end{latexonly}

% Do this when processing latex to build html (using latex2html)
\begin{htmlonly}
   \verbatiminput{\AstRewriteExampleDirectory/astRewriteInterfaceFunction.codeFragment}
\end{htmlonly}

%end of scope in font size
}
% End of scope in indentation
}


The function defined \ref{AstRewrite:AstRewriteInterfaceFunction} takes the following
parameters:
\begin{enumerate}
     \item Synthesized attribute (defined and used within the {\tt
           evaluateRewriteSynthesizedAttribute()} function.
     \item A C++ {\tt std::string} object representing new source 
           code to add to the AST \label{inputString}.
     \item Inherited attribute (defined and used within the {\tt
           evaluateRewriteSynthesizedAttribute()} function.
     \item Relative scope (selected values from 
           {\tt AST\_Rewrite::ScopeIdentifierEnum}.
     \item Position in scope selected values from 
           {\tt AST\_Rewrite::PlacementPositionEnum} inputRelativeLocation,
     \item Type of source string from
           {\tt AST\_Rewrite::SourceCodeStringClassificationEnum}
     \item Specification of source code in new scope ({\tt boolean} value)
           (equivalent to adding opening and closing braces ({\tt \{} and {\tt \}}) to the beginning and end 
           of the input string \ref{inputString}, respectively.
\end{enumerate}
}

\section{Example Using AST Rewrite}

    This section demonstrates a simple example using the AST Rewrite Mechanism.
The input code \ref{AstRewrite:ExampleInputCode} contains the variable declaration statement {\tt int x;}
which example preprocessor {\tt testRewritePermutations} (a testcode in the 
{\bf ROSE/tests/roseTests/astRewriteTests} directory) will use to place additional variable
declarations in all possible relative/absolute positions.

{\indent
{\mySmallFontSize

\label{AstRewrite:ExampleInputCode}

% Do this when processing latex to generate non-html (not using latex2html)
\begin{latexonly}
   \lstinputlisting{\AstRewriteExampleDirectory/inputRewrite1.C}
\end{latexonly}

% Do this when processing latex to build html (using latex2html)
\begin{htmlonly}
   \verbatiminput{\AstRewriteExampleDirectory/inputRewrite1.C}
\end{htmlonly}

%end of scope in font size
}
% End of scope in indentation
}

The new variable declarations contain, as a substring of the variable name,
the relative scope and location in that scope (relative to the target 
declaration {\tt int x;}.  The output of processing this input file is a new 
code \ref{AstRewrite:ExampleOutputCode} with many added
declarations, one for each possible relative/absolute position possible
(relative to the declaration: {\tt int x;}).

{\indent
{\mySmallFontSize

\label{AstRewrite:ExampleOutputCode}

% Do this when processing latex to generate non-html (not using latex2html)
\begin{latexonly}
   \lstinputlisting{\AstRewriteExampleDirectory/exampleRewrite1.C}
\end{latexonly}

% Do this when processing latex to build html (using latex2html)
\begin{htmlonly}
   \verbatiminput{\AstRewriteExampleDirectory/exampleRewrite1.C}
\end{htmlonly}

%end of scope in font size
}
% End of scope in indentation
}

\section{Limitations (Known Bugs)}

   There are several types of statements the AST rewrite mechanism can not 
currently process.  This section enumerates these and explains why each is difficult 
or not currently possible. Note that some appear unable to be handled,
while others will only require special handling that is not yet implemented.
\begin{enumerate}
     \item Why we have to skip SgCaseOptionStmt statements. \\
        Example of code in generated intermediate file for a SgCaseOptionStmt:
{\indent
{\mySmallFontSize
\begin{verbatim}
             int GlobalScopePreambleStart;
             int GlobalScopePreambleEnd;
             int CurrentLocationTopOfScopeStart;
             int CurrentLocationTopOfScopeEnd;
             int CurrentLocationBeforeStart;
             int CurrentLocationBeforeEnd;
             int CurrentLocationReplaceStart;
             case 0:{y++;break;}
             int CurrentLocationReplaceEnd;
             int CurrentLocationAfterStart;
             int CurrentLocationAfterEnd;
             int CurrentLocationBottomOfScopeStart;
             int CurrentLocationBottomOfScopeEnd;
\end{verbatim}
}}
        The problem is that marker declarations that appear after the SgCaseOptionStmt 
        are included in the scope of the SgCaseOptionStmt while those that appear 
        before it are not in the same scope.

     \item SgDefaultOptionStmt (see reason \#1 above).
     \item SgCtorInitializerList \\
        This case would require special handling to be generated in the 
        intermediate file, and it would require special handling isolated 
        from the AST.  This case can probably be handled in the future with extra work.
     \item SgFunctionParameterList (see reason \#3 above).
     \item SgClassDefinition \\
        Since the SgClassDefinition is so structurally tied to the SgClassDeclaration, 
        it makes more sense to process the SgClassDeclaration associated with the 
        SgClassDefinition instead of the SgClassDefinition directly.  Presently the 
        processing of the SgClassDefinition is not supported through any indirect 
        processing of the SgClassDeclaration, this could be implemented in the future.
     \item SgGlobal \\
        This case is not implemented. It would require special handling, but it might be 
        implemented in the future.
     \item SgBasicBlock used in a SgForStatement \\
        Because of the declaration of the {\tt for} loop (C language construct) index
        variable, this case would 
        require special handling. This case could be implemented in the future.
     \item SgBasicBlock used in a SgFunctionDefinition \\
        Because of the declaration of the function parameter variable, this case would 
        require special handling. This case could be implemented in the future.
     \item SgBasicBlock used in a SgSwitchStatement \\
        Example of code in generated intermediate file for a SgBasicBlock used in 
        SgSwitchStatement:
{\indent
{\mySmallFontSize
\begin{verbatim}
             int main()
             { /* local stack #0 */
             int x;
             int y;
             switch(x)
             { /* local stack #1 */ 
             int GlobalScopePreambleStart;
             int GlobalScopePreambleEnd;
             int CurrentLocationTopOfScopeStart;
             int CurrentLocationTopOfScopeEnd;
             int CurrentLocationBeforeStart;
             int CurrentLocationBeforeEnd;
             int CurrentLocationReplaceStart;
             {case 0:{y++;break;}default:{y++;break;}}
             int CurrentLocationReplaceEnd;
             int CurrentLocationAfterStart;
             int CurrentLocationAfterEnd;
             int CurrentLocationBottomOfScopeStart;
             int CurrentLocationBottomOfScopeEnd;
             /* Reference marker variables to avoid compiler warnings */
                };     };
\end{verbatim}
}}
        This is more difficult because the declaration markers must appear after 
        the {\tt "\{ /* local stack \#1 */"} but then the statement 
        {\tt "{case 0:{y++;break;}default:{y++;break;}}"}
        cannot appear after a switch.  It is probably impossible to fix this case due to the
        design and constraints of the C++ language (design and limitations of the switch statement).
        This is not a serious problem; it just means that the whole switch statement must
        be operated upon instead of the block within the switch statement separately (not
        a serious limitation).
\end{enumerate}






