\documentclass{article}
\usepackage{listings}

\begin{document}
\author{Jeremiah Willcock}
\title{Finite differencing documentation}
\maketitle

The implementation of the finite differencing optimization in ROSE is in
\texttt{src/midend/programTransformation/finiteDifferencing}.  Finite
differencing is an optimization which uses relationships between values of
the same expression at different program points to simplify computations of
the expression~\cite{paige_koenig}.  In particular, for a given expression,
it adds a cache variable to contain the value of the expression.  Every use
of the expression is replaced by a reference to the cache variable, and
every modification of the value of the expression is prefixed by a
corresponding update to the cache variable.  This optimization is somewhat
stronger than partial redundancy elimination, and so it can add extra
computations of the expression.  Beyond this removal of redundant
computations, however, is the main goal of finite differencing: simplifying
the updates to the cache variable based on knowledge of its previous value.
In particular, finite differencing's goal is to be a generalized form of
strength reduction.

The implementation of finite differencing in ROSE is in direct
correspondence to the algorithm given above.  For a given expression, a
cache variable is created, and all uses of the expression are replaced by
the cache variable.  All updates to variables used in the expression are
also found.  Currently these updates must be assignments (including
combined assignment operators such as \lstinline{+=}) or the \lstinline{++}
or \lstinline{--} operators.  For each update, the value of the expression
after the update is expressed using the state of the program before the
update (for example, \lstinline{n * x} after the update \lstinline{x = x + 3}
is expressed as \lstinline{n * (x + 3)} using the pre-update value of
\lstinline{x}).  Then, a set of rewrite rules is used to attempt to
simplify the update.  For example, \lstinline{cache = n * (x + 3}) is rewritten to
\lstinline{cache = n * x + n * 3}.  Since the value of the cache variable before
the update is known to be correct, the example can be rewritten further to
\lstinline{cache = cache + n * 3} or \lstinline{cache += n * 3}.  This
set of rewrites has changed a multiplication by a variable into a
multiplication by a constant and an addition, which are simpler operations.

A simple profitability analysis which applies finite differencing only as
strength reduction for multiplications in simple loops is also provided,
and this is the main wrapper used for finite differencing in practice.
However, the core algorithm is extensible to encompass arbitrary uses of
finite differencing, include its use on higher-level, user-defined
abstractions.

\section*{Rewrite system used for finite differencing}

As stated earlier, finite differencing uses a set of rewrite rules to
simplify expressions algebraically.  A custom rewrite engine was developed
for this, but a specialized language such as Stratego could be used in the
future.  In this system, a rewrite rule is expressed as an object which
takes a Sage node as input and possibly rewrites it.  This object returns
\lstinline{true} if it could rewrite the node and \lstinline{false} if it
could not.  The node must be updated in place.  The rewrite rule object
does not need to search for nested redexes (expressions to be rewritten),
since this is handled at a higher level of the rewrite system.  The
standard kind of rewrite rule used for algebraic rewriting is a
pattern-action rule, which is an implementation of a rewrite rule which
accepts two patterns.  The first pattern (the match pattern) is matched to
the input node, and a set of variable bindings is produced if it matches.
These variable bindings are then substituted into the second pattern (the
action) to produce the result of the rewrite rule.  This is a relatively
simple system, and it is likely to be replaced by something more
sophisticated in the future.

\bibliography{docs}
\bibliographystyle{abbrv}

\end{document}
