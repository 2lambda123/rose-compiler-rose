\chapter{Dependence analysis and Polyhedric Model}

\label{polyhedric:polyhedric}

	This chapter reffers to the project PolyhedralDependenceAnalysis.

	\section{What can I find in this project ?}
	
	Each subsection is related to a subdirectory in the project.
	
		\subsection{RoseToFada}
	
	The project's subdirectory RoseToFada contains a translator from Rose IR (SageIII) to FadaLib IR.\\
	FADAlib \cite{FADAweb} is an implementation of Fuzzy Array Dependence Analysis \cite{FADA}. FADA is an 
analysis for non static control programs (or irregular programs).\\
	Here, we reduce the scope of the transformation to static control programs \cite{Feautrier} to fit with 
following part. It means that support of conditionals and while-loops is not implement (but can be).\\
	Also, the translation don't address structure and complex-types.\\
	
	So, the subset of C that is address by this analysis contains only:
\begin{itemize}
	\item for-loop: for (i = a; i <= b; i++)
	\item scalar variable
	\item array reference with index that are affine application: a[i + 2*j][3*i + 5]
\end{itemize}
	
		\subsection{PMDAtoMDA}
	
	Pseudo-multidimensional array are frequently used, but this representation is out of the scope of FADAlib
(array index are not an affine application: a[i * n + j]).\\
	The component PMDAtoMDA can traverse a function in Rose IR and add an attribute to all pseudo-multidimensional
array. This attribute can generate (in Rose IR) each component of the correspondant access vector.\\
	But, that no magic and highly constraint: we need tab[i * n * m + j * n + k] instead of tab[i][j][k] for an
array of dimension [X, m, n] (i, j, k can be any affine form of the iterators).
	
		\subsection{RoseToPPL}
		
	Parma Polyhedral Library \cite{PPL} is a C++, object-oriented polyhedron manipulation library.\\
	This component generate a polyhedral representation of dependences from FadaLib analysis.
	
		\subsection{Schedule}
		
	It's an experimental implementation of the Farkas Multipliers Method to obtains the space (polyhedron) of 
valid schedule with respect to the dependances found by the Fuzzy Array Dependence Analysis.

	\section{What I need to compile this project with Rose ?}

	Pseudo-multidimensional array related part has no extra-dependences.
	
	\subsection{RoseToFada}
	
		FadaLib \cite{FADAweb} is needed and depend from PipLib (Parametric Integer Programming Library)
\cite{PIP}.
		You need to add following options to configure script:
\begin{verbatim}
--with-pip=PATH
--with-fada=PATH
--enable-fadalib=yes
\end{verbatim}
	
	\subsection{RoseToPPL and Schedule}

		Theses components depend from RoseToFada.\\
		Both use Parma Polyhedral Library \cite{PPL} to represent polyhedrons.\\
		You need to add following options to configure script:
\begin{verbatim}
--with-ppl=PATH
--enable-ppl=yes
\end{verbatim}
		

	\section{And now some maths !}

\begin{verbatim}
More about "polyhedral representation of dependences":
	Definitions:
		- Global variables: scalar variables from global scope (ex: array size), theirs noted G
		- Statement:
			* an iteration domain D, define by a polyhedron 
				(ie: {z | f(z) >= 0} where f is an affine form (N pow(n) -> N) )
			* Some 'read' and 'write' variables
		- Dependance:
			* 2 statements (that can be the same)
				- with D1 and D2 there respective iteration domain
			* a dependence function f: D1 x G -> D2 an affine application
			
	The polyhedral representation of this dependence is a polyhedron P in D1 x D2 x G:
		- z in P => z in f & z in D1 & z in D2
\end{verbatim}
