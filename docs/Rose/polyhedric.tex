\chapter{Polyhedral Model}
\label{polyhedric:polyhedric}

	This chapter reffers to the project PolyhedralModel.

	\section{What can I find in this project ?}
	\label{polyhedric:insideproject}
	
	The project come with a doxygen documentation:
	
	\$ cd projects/PolyhedralModel/docs \\
	\$ doxygen doxy.conf
	
	\section{What I need to compile this project with Rose ?}
	\label{polyhedric:getstarted}

	To enable the Polyhedral Model project, you need PPL (Parma Polyhedra Library) (May 2011: version 0.11.2). The configure options are: \\
	\emph{--enable-ppl --with-ppl=[path to PPL]}\\
	
	Code generation is one of the main future of this project. It needs the library Cloog (Chunky Loop Generator) (May 2011: version 0.16.2). The configure options are: \\
	\emph{--enable-cloog --with-cloog=[path to Cloog]}\\
	
	Some other works depend on ScopLib and Candl, they can be enable by the same kind of options.

	\section{And now a few maths !}
	\label{polyhedric:maths}

More about "polyhedral representation of dependences".

		\subsection{Definitions and Notations}
		\label{polyhedric:maths:defs}

			\subsubsection{Integer Polyhedron}
			\label{polyhedric:maths:defs:polyhedron}

A integer polyhedron (which would be the only polyhedron that we will consider) is a set of integers points.

$\forall n \in \mathbb{N}, \mathbb{P} = \{\bar{z} \in \mathbb{Z}^{n} | \mathcal{P}(\bar{z}) \geq 0 \}$ where:
\begin{itemize}
	\item $n$ is the dimension of the polyhedron
	\item $\mathcal{P}: \mathbb{Z}^{n} \rightarrow \mathbb{Z}^{m}$ is an affine application
	\item $m$ is the number of intersected half-space
\end{itemize}
($\bar{z} \geq 0$ means $\forall i, z_{i} \geq 0$)

			\subsubsection{Notation}
			\label{polyhedric:maths:defs:notation}

\begin{itemize}
	\item $ \Box $ is the vector concatenation operator;
	\item $ 1_{i} $ is a vector with all component equal to zero but the $i^{th}$;
	\item $ Row_{i}^{\mathbb{P}}(\bar{z}) \geq 0 $: the $i^{th}$ inequalitie that define the polyhedron $\mathbb{P}$.\\
	$Row_{i}^{\mathbb{P}}(\bar{z}) = \mathcal{P}(\bar{z}).1_{i}$ where $\mathcal{P}$ is the affine application associate
	to $\mathbb{P}$.
\end{itemize}
		\subsection{Program}
		\label{polyhedric:maths:}

All considered programs are on the scope of Affine Control Loop (ACL), follow some notation:
\begin{itemize}
	\item $\mathbb{G}$ the set of Global variables: scalars from global scope (ex: array size).\\
	We will note $\bar{g}$ an element of $\mathbb{G}$.
	\item A set of statement $\mathbb{S}$, define by:
	\begin{itemize}
		\item An iteration domain: $\mathbb{D} = \{\bar{z} | f(\bar{z}, \bar{g}) \geq 0\}$ where f is an affine
				form $f: \mathbb{Z}^{n} \times \mathbb{G} \rightarrow \mathbb{Z}$ where n is the number
				of iterators.
		\item Two set of variables: read and write.\\
		A variable can be a scalar or an array element. In second case, the access vector to this element need to
		be an affine application $f: \mathbb{D} \times \mathbb{G} \rightarrow \mathbb{Z}^{dim(array)}$.\\
		NB: for a pseudo-multi-dimensionnal array, index needs to be 
		$\sum_{i=1}^{dim(array)} f_{i} \times \prod_{j=1}^{i} size(dim_{j})$
	\end{itemize}
\end{itemize}

		\subsection{Dependence Graph}

We consider the graph of dependence $\Gamma$, in this directed graph, nodes are statement and vertex are dependence.
Vertex are qualified with the conditions of the dependence.\\
Given two statements $S_{1}$ and $S_{2}$ with $\mathbb{D}_{1}$ and $\mathbb{D}_{2}$ theirs iterations domains.
A vertex from $S_{1}$ to $S_{2}$ will be qualified by:
\begin{itemize}
	\item a condition $q_{1 \rightarrow 2}(\bar{z}, \bar{g}) \geq 0$ where $q_{1 \rightarrow 2}$ is an affine application
		$q_{1 \rightarrow 2}: \mathbb{D}_{1} \times \mathbb{G} \rightarrow \mathbb{Z}^{n}$ where is the number of
		sub-conditions.
	\item an affine application $f_{1 \rightarrow 2} : \mathbb{D}_{1} \rightarrow \mathbb{D}_{2}$
\end{itemize}
This mean that the statement $S_{1}$ at iteration point $\bar{z} \in \mathbb{D}_{1}$ depends on $S_{2}$ at iteration point
$f_{1 \rightarrow 2}(\bar{z}) \in \mathbb{D}_{1}$ if $q_{1 \rightarrow 2}(\bar{z}, \bar{g}) \geq 0$.\\

		\subsection{Polyhedron associate to a dependence}

An affine application $f: \mathbb{E}_{1} \rightarrow \mathbb{E}_{2}$ can be represent as a polyhedron
$\mathbb{P} \subset \mathbb{E}_{1} \times \mathbb{E}_{2}$.\\
Given $\mathbb{P}_{1 \rightarrow 2} \subset \mathbb{Z}^{n_{1} + n_{2} + n_{g}}$ where $n_{1}$ and $n_{2}$ and $n_{g}$ are
respectively the number of iterators of $S_{1}$ and $S_{2}$, and the number of globals.\\

$\mathbb{P}_{1 \rightarrow 2}$ will be the polyhedron associate to the dependence between $S_{1}$ and $S_{2}$ iff:
%\begin{center}
%$\forall (\bar{z_{1}}, \bar{z_{2}}, \bar{g}) \in \mathbb{D}_{1} \times \mathbb{D}_{2} \times \mathbb{G}$
%
%( $q_{1 \rightarrow 2}(\bar{z_{1}}, \bar{g}) \geq 0 \wedge \bar{z_{2}} = f_{1 \rightarrow 2}(\bar{z_{1}})$ )
%
%$\Leftrightarrow$
%
%$(\bar{z_{1}}, \bar{z_{2}}, \bar{g}) \in \mathbb{P}_{1 \rightarrow 2}$
%\end{center}
%An other version of $\mathbb{P}_{1 \rightarrow 2}$, more restrictive, that consider also the domains (this one is use ):
\begin{center}
$\forall (\bar{z_{1}}, \bar{z_{2}}, \bar{g}) \in \mathbb{Z}^{n_{1} + n_{2} + n_{g}}$

($\bar{z_{1}} \in \mathbb{D}_{1} \wedge \bar{z_{2}} \in \mathbb{D}_{2} \wedge \bar{g} \in \mathbb{G} \wedge q_{1 \rightarrow 2}(\bar{z_{1}}, \bar{g}) \geq 0 \wedge \bar{z_{2}} = f_{1 \rightarrow 2}(\bar{z_{1}})$ )

$\Leftrightarrow$

$ \bar{z_{1}} \,\Box\, \bar{z_{2}} \,\Box\, \bar{g} \in \mathbb{P}_{1 \rightarrow 2}$
\end{center}

		\subsection{Example}
		\label{polyhedric:maths:example}

\begin{figure}[!h]	
\begin{lstlisting}[language=C, numbers=left]
for (i = 0; i < n; i++) {
    s[i] = 0
    for (j = 0; j < n; j++) {
        s[i] = s[i] + a[i][j] * x[j]
    }
}
\end{lstlisting}
\caption{A sample program}
\label{polyhedric:sampleprogram}
\end{figure}

			\subsubsection{Program}

Consider the small program of figure \ref{polyhedric:sampleprogram}. It contains a reference to one global variable
$\mathbb{G} = \{ n \}$ and we have two statement $S_{1}$ and $S_{2}$, respectively at line 2 and 4, with associated domains:

$\mathbb{D}_{1} = \{ i | 0 \leq i \leq n - 1 \}$ 

$\mathbb{D}_{2} = \{ i, j | 0 \leq i, j \leq n - 1 \}$

			\subsubsection{Dependences}

A dependence analysis (from FADAlib for example) would give us:
\begin{itemize}
	\item $S_{1}$ has no dependence (it never read a variable)
	\item $S_{2}$ has dependence at iteration point $( i, j )$:
	\begin{itemize}
		\item on $S_{1}$ at iteration point $( i )$: if $j = 0$ 
		\item on $S_{2}$ at iteration point $( i, j - 1 )$: if $j \geq 1$ 
	\end{itemize}
\end{itemize}

So, the dependence graph is composed from two nodes $\{S_{1}, S_{2}\}$ and two edges:
\begin{itemize}
	\item $S_{2} \rightarrow S_{1}$
	\begin{itemize}
		\item $q_{S_{2} \rightarrow S_{1}}(i, j, n) = (j, -j)$
		\item $f_{S_{2} \rightarrow S_{1}}(i, j) = (i)$
	\end{itemize}
	\item $S_{2} \rightarrow S_{2}$
	\begin{itemize}
		\item $q_{S_{2} \rightarrow S_{2}}(i, j, n) = j - 1$
		\item $f_{S_{2} \rightarrow S_{2}}(i, j) = (i, j - 1)$
	\end{itemize}
\end{itemize}

			\subsubsection{Polyhedron}

We have two dependances, then two polyhedrons 
\begin{itemize}
	\item $\mathbb{P}_{S_{2} \rightarrow S_{1}}$ associate with the following affine application:\\
	$\mathcal{P}_{S_{2} \rightarrow S_{1}}: \mathbb{Z}^{2 + 1 + 1} \rightarrow  \mathbb{Z}^{4 + 2 + 2 + 2} $\\
	$$\mathcal{P}_{S_{2} \rightarrow S_{1}}(i, j, i', n) = 
	\begin{pmatrix}
		1  & 0  & 0  & 0 \\
		-1 & 0  & 0  & 1 \\
		0  & 1  & 0  & 0 \\
		0  & -1 & 0  & 1 \\
		0  & 0  & 1  & 0 \\
		0  & 0  & -1 & 1 \\
		0  & 1  & 0  & 0 \\
		0  & -1 & 0  & 0 \\
		1  & 0  & -1 & 0 \\
		-1 & 0  & 1  & 0 \\
	\end{pmatrix}
	\times
	\begin{pmatrix}
		i  \\
		j  \\
		i' \\
		n  \\
	\end{pmatrix}
	+
	\begin{pmatrix}
		0 \\
		1 \\
		0 \\
		1 \\
		0 \\
		1 \\
		0 \\
		0 \\
		0 \\
		0 \\
	\end{pmatrix}$$
	\item $\mathbb{P}_{S_{2} \rightarrow S_{2}}$ associate with the following affine application:\\
	$\mathcal{P}_{S_{2} \rightarrow S_{2}}: \mathbb{Z}^{2 + 2 + 1} \rightarrow  \mathbb{Z}^{4 + 4 + 1 + 4} $\\
	$$\mathcal{P}_{S_{2} \rightarrow S_{2}}(i, j, i', j', n) = 
	\begin{pmatrix}
		1  & 0  & 0  & 0  & 0  \\
		-1 & 0  & 0  & 0  & 1  \\
		0  & 1  & 0  & 0  & 0  \\
		0  & -1 & 0  & 0  & 1  \\
		0  & 0  & 1  & 0  & 0  \\
		0  & 0  & -1 & 0  & 1  \\
		0  & 0  & 0  & 1  & 0  \\
		0  & 0  & 0  & -1 & 1  \\
		0  & 1  & 0  & 0  & 0  \\
		1  & 0  & -1 & 0  & 0  \\
		-1 & 0  & 1  & 0  & 0  \\
		0  & 1  & 0  & -1 & 0  \\
		0  & -1 & 0  & 1  & 0  \\
	\end{pmatrix}
	\times
	\begin{pmatrix}
		i  \\
		j  \\
		i' \\
		j' \\
		n  \\
	\end{pmatrix}
	+
	\begin{pmatrix}
		0  \\
		-1 \\
		0  \\
		-1 \\
		0  \\
		-1 \\
		0  \\
		-1 \\
		-1 \\
		0  \\
		0  \\
		-1 \\
		1  \\
	\end{pmatrix}$$
\end{itemize}


	\section{Generating affine schedule with polyhedric model}
	\label{polyhedric:schedule}

		\subsection{What is this ?}

	Says  that an operation is the execution of statement $S \in \mathbb{S}$ at an iteration point 
$\bar{z} \in \mathbb{D}_{S}$, we will note it $<\!S, \bar{z}\!>$.\\
	A schedule $\Theta$ will associate a date to an operation $<\!S, \bar{z}\!>$. An affine schedule is a set (one
by statement) of affine applications $\Theta_{S}: \mathbb{D}_{S} \times \mathbb{G} \rightarrow \mathbb{Z}^{n}$ where $n$
is the dimension of the schedule (date).

	We look for 1-dimensional schedules only. In this case:
$$\Theta_{S}(\bar{z}, \bar{g}) = \bar{\alpha_{S}} . \bar{z} + \bar{\beta_{S}} . \bar{g} + \kappa_{S}$$

		\subsection{Generating the set of Valid Schedule}
		\label{polyhedric:schedule:vss}

			\subsubsection{Valid Schedule ?}
			\label{polyhedric:schedule:vss:vss}
			
	A valid schedule is a schedule that respect the dependences of a program. So, one of such schedule need to respect:
	$$ \forall (S_{1}, S_{2}) \in \mathbb{S}, \; \mathbb{P}_{S_{2} \rightarrow S_{1}} \neq \emptyset $$
	$$ \implies $$
	$$ \forall \, \bar{z_{2}} \,\Box\, \bar{z_{1}} \,\Box\, \bar{g} \in \mathbb{P}_{S_{2} \rightarrow S_{1}}, \; 
	\Theta_{S_{1}}(\bar{z_{1}}, \bar{g}) + \Delta_{S_{1}} \leq \Theta_{S_{2}}(\bar{z_{2}, \bar{g}}) $$
	$$ \implies $$
	$$ \forall \, \bar{z_{2}} \,\Box\, \bar{z_{1}} \,\Box\, \bar{g} \in \mathbb{P}_{S_{2} \rightarrow S_{1}}, \; 
	\Theta_{S_{2}}(\bar{z_{2}, \bar{g}}) - \Theta_{S_{1}}(\bar{z_{1}}, \bar{g}) - \Delta_{S_{1}} \geq 0 $$
	where:
\begin{itemize}
	\item $ \mathbb{P}_{S_{2} \rightarrow S_{1}} $ is the polyhedron associated to the dependence from $S_{2}$ to $S_{1}$,
	empty if this dependence do not exist;
	\item $ \Delta_{S_{1}} $ is the latence of the $S_{1}$ statement execution, considered equal to 1 in following
	devellopement.
\end{itemize}

			\subsubsection{Farkas algorithm}
			\label{polyhedric:schedule:vss:farkas}

	\paragraph{Theorem:} Affine Form of Farkas Lemma

\begin{quote}	
	Let $\mathbb{P}$ be a nonempty polyhedron defined by $p$ affine inequalities:
	$$ \bar{z} \in \mathbb{P} \iff \forall k \in [\![ 1, p ]\!], a_{k}.\bar{z} + b_{k} \geq 0 $$
	
	Then an affine form $\psi$ is non negative everywhere in $\mathbb{P}$ iff it is a positive affine combination:
	
	$$ \forall \bar{z} \in \mathbb{P}, \psi(\bar{z}) \geq 0 $$
	$$ \iff $$
	$$ \forall k \in [\![ 0, p ]\!], \; \exists \lambda_{k} \geq 0 \;\text{such as}\;
	\forall \bar{z} \in \mathbb{P}, \; \psi(\bar{z}) = \lambda_{0} + \sum_{k = 1}^{p} \lambda_{k}.(a_{k}.\bar{z} + b_{k}) $$
\end{quote}

	With the valid schedule definition and the above theorem, we have: 
	$$ \forall (S_{1}, S_{2}) \in \mathbb{S}, \; \mathbb{P}_{S_{2} \rightarrow S_{1}} \neq \emptyset $$
	$$ \implies $$
	$$ \forall k \in [\![ 0, p ]\!], \; \exists \lambda_{k} \geq 0 \;\text{such as}\;
	\forall \; \bar{z_{2}} \,\Box\, \bar{z_{1}} \,\Box\, \bar{g} \in \mathbb{P}_{S_{2} \rightarrow S_{1}}, \; $$
	$$ \Theta_{S_{2}}(\bar{z_{2}, \bar{g}}) - \Theta_{S_{1}}(\bar{z_{1}}, \bar{g}) - 1 =
	\lambda_{0} + \sum_{k = 1}^{p} \lambda_{k} . Row_{k}^{\mathbb{P}}(\bar{z_{2}} \,\Box\, \bar{z_{1}} \,\Box\, \bar{g}) $$

			\subsubsection{Example}
			\label{polyhedric:schedule:vss:example}

	We still consider the small program of figure \ref{polyhedric:sampleprogram}.\\
	Firstly, we need to remove from polyhedron some redondante constraints:
$$\mathcal{P}_{S_{2} \rightarrow S_{1}}(i, j, i', n) = 
	\begin{pmatrix}
		1  & 0  & 0  & 0 \\
		-1 & 0  & 0  & 1 \\
		0  & 1  & 0  & 0 \\
		0  & -1 & 0  & 0 \\
		1  & 0  & -1 & 0 \\
		-1 & 0  & 1  & 0 \\
	\end{pmatrix}
	\times
	\begin{pmatrix}
		i  \\
		j  \\
		i' \\
		n  \\
	\end{pmatrix}
	+
	\begin{pmatrix}
		0  \\
		-1 \\
		0  \\
		0  \\
		0  \\
		0  \\
	\end{pmatrix}$$
$$\mathcal{P}_{S_{2} \rightarrow S_{2}}(i, j, i', j', n) = 
	\begin{pmatrix}
		1  & 0  & 0  & 0  & 0  \\
		-1 & 0  & 0  & 0  & 1  \\
		0  & -1 & 0  & 0  & 1  \\
		0  & 1  & 0  & 0  & 0  \\
		1  & 0  & -1 & 0  & 0  \\
		-1 & 0  & 1  & 0  & 0  \\
		0  & 1  & 0  & -1 & 0  \\
		0  & -1 & 0  & 1  & 0  \\
	\end{pmatrix}
	\times
	\begin{pmatrix}
		i  \\
		j  \\
		i' \\
		j' \\
		n  \\
	\end{pmatrix}
	+
	\begin{pmatrix}
		0  \\
		-1 \\
		-1 \\
		-1 \\
		0  \\
		0  \\
		-1 \\
		1  \\
	\end{pmatrix}$$
	
	We also consider the following expressions for $\Theta_{S_{1}}$ and $\Theta_{S_{2}}$:
$$\Theta_{S_{1}}(i, n) = \alpha_{S_{1}}^{1} \times i + \beta_{S_{1}}^{1} \times n + \kappa_{S_{1}}$$
$$\Theta_{S_{2}}(i, j, n) = \alpha_{S_{2}}^{1} \times i + \alpha_{S_{2}}^{2} \times j + \beta_{S_{2}}^{1} \times n + \kappa_{S_{2}}$$

	The application of Farkas Lemma give us:
\begin{itemize}
	\item for $S_{2} \rightarrow S_{1}$:
	$$ \Theta_{S_{2}}(i, j, n) - \Theta_{S_{1}}(i', n) - 1 =
	\lambda_{0}^{S_{2} \rightarrow S_{1}} + \sum_{k = 1}^{6} \lambda_{k}^{S_{2} \rightarrow S_{1}} . Row_{k}^{\mathbb{P}}(i, j, i', n) $$
	$$ \left\{
	\begin{array}{l}
	\begin{array}{rcl}
			\alpha_{S_{2}}^{1} & = &
				\lambda_{1}^{S_{2} \rightarrow S_{1}}
				- \lambda_{2}^{S_{2} \rightarrow S_{1}}
				+ \lambda_{5}^{S_{2} \rightarrow S_{1}}
				- \lambda_{6}^{S_{2} \rightarrow S_{1}}\\
			\alpha_{S_{2}}^{2} & = &
				\lambda_{3}^{S_{2} \rightarrow S_{1}}
				- \lambda_{4}^{S_{2} \rightarrow S_{1}} \\
			- \alpha_{S_{1}}^{1} & = &
				\lambda_{6}^{S_{2} \rightarrow S_{1}}
				- \lambda_{5}^{S_{2} \rightarrow S_{1}} \\
			\beta_{S_{2}}^{1} - \beta_{S_{1}}^{1} & = &
				\lambda_{2}^{S_{2} \rightarrow S_{1}} \\
			\kappa_{S_{1}} - \kappa_{S_{2}} - 1 & = &
				\lambda_{0}^{S_{2} \rightarrow S_{1}}
				- \lambda_{2}^{S_{2} \rightarrow S_{1}}
		\end{array}\\
		\forall k \in [\![ 0, 6 ]\!], \quad \lambda_{k}^{S_{2} \rightarrow S_{1}} \geq 0
	\end{array}
	\right. $$
	\item for $S_{2} \rightarrow S_{2}$:
	$$ \Theta_{S_{2}}(i, j, n) - \Theta_{S_{2}}(i', j', n) - 1 =
	\lambda_{0}^{S_{2} \rightarrow S_{2}} + \sum_{k = 1}^{6} \lambda_{k}^{S_{2} \rightarrow S_{2}} . Row_{k}^{\mathbb{P}}(i, j, i', j', n) $$
	$$ \left\{
	\begin{array}{l}
		\begin{array}{rcl}
			 \alpha_{S_{2}}^{1} & = &
				\lambda_{1}^{S_{2} \rightarrow S_{2}}
				- \lambda_{2}^{S_{2} \rightarrow S_{2}}
				+ \lambda_{5}^{S_{2} \rightarrow S_{2}}
				- \lambda_{6}^{S_{2} \rightarrow S_{2}} \\
			\alpha_{S_{2}}^{2} & = &
				- \lambda_{3}^{S_{2} \rightarrow S_{2}}
				+ \lambda_{4}^{S_{2} \rightarrow S_{2}}
				+ \lambda_{7}^{S_{2} \rightarrow S_{2}}
				- \lambda_{8}^{S_{2} \rightarrow S_{2}} \\
			- \alpha_{S_{2}}^{1} & = &
				- \lambda_{5}^{S_{2} \rightarrow S_{2}}
				+ \lambda_{6}^{S_{2} \rightarrow S_{2}} \\
			- \alpha_{S_{2}}^{2} & = &
				- \lambda_{7}^{S_{2} \rightarrow S_{2}}
				+ \lambda_{8}^{S_{2} \rightarrow S_{2}} \\
			0 & = & \lambda_{2}^{S_{2} \rightarrow S_{2}} 
				+ \lambda_{3}^{S_{2} \rightarrow S_{2}}  \\
			-1 & = & \lambda_{0}^{S_{2} \rightarrow S_{2}} 
				- \lambda_{2}^{S_{2} \rightarrow S_{2}} 
				- \lambda_{3}^{S_{2} \rightarrow S_{2}} 
				- \lambda_{4}^{S_{2} \rightarrow S_{2}}
				- \lambda_{7}^{S_{2} \rightarrow S_{2}} 
				+ \lambda_{8}^{S_{2} \rightarrow S_{2}}
		\end{array}\\
		\forall k \in [\![ 0, 8 ]\!], \quad \lambda_{k}^{S_{2} \rightarrow S_{2}} \geq 0
	\end{array}
	\right. $$
\end{itemize}

			\subsubsection{Dimension of the Valid Schedule space}


